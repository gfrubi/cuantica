\chapter{Ecuaci'on de onda relativista para part'iculas de spin 0. La
ecuaci'on de Klein-Gordon y sus aplicaciones}

La descripci'on de fen'omenos a altas energ'ias requiere la
investigaci'on de ecuaciones de onda relativistas. Esto significa
ecuaciones de onda que sean invariantes bajo transformaciones de Lorentz. La
transici'on de una descripci'on no-relativista a una relativista
implica que varios conceptos de la teor'ia no-relativista tienen que ser
reinvestigados, en particular:

\begin{enumerate}
\item Coordenadas espaciales y temporales deben ser tratadas igualmente en
la teor'ia.

\item A energ'ias altas (relativistas, $E>2mc^2$) ocurren procesos de creaci'on
y aniquilaci'on de pares, usualmente en la forma de creaci'on de pares
part'icula-antipart'icula. Luego, a energ'ias relativistas la
conservaci'on del n'umero de part'iculas no es m'as una
suposici'on v'alida
\item Desde un punto de vista m'as pr'actico, se espera que la inclusi'on de
efectos relativistas en la descripci'on cu'antica de las part'iculas implique
correcciones a las predicciones obtenidas con la teor'ia norelativista. Por
ejemplo a los niveles de energ'ia del 'atomo de Hidr'ogeno. En este
caso, la rapidez del electr'on en el estado fundamental puede ser
estimado como
\begin{align}
v  & =\sqrt{\frac{Ke^{2}}{mr}}\\
& =\sqrt{K\left(  \frac{e}{m}\right)  ^{2}\frac{m}{r}}\\
& =\sqrt{9\times10^{9}\times\left(  1.7\times10^{11}\right)  ^{2}\times
\frac{9.1\times10^{-31}}{0.5\times10^{-10}}}\\
& =2.\,\allowbreak175\,7\times10^{6}\left(  \frac{m}{s}\right)  ,
\end{align}
de modo que
\begin{equation}
\frac{v}{c}=\frac{2.\,\allowbreak175\,7\times10^{6}}{3\times10^{8}}\sim
10^{-3}.
\end{equation}

A partir de esta estimaci'on es posible esperar que las correcciones
relativistas sean del orden de $0,1\%$.


\end{enumerate}

\section{Notaci'on}

Introducimos el tensor m'etrico $g_{\mu \nu }$ (componentes covariantes)
como 
\begin{equation}
g_{\mu \nu }=\left[ 
\begin{array}{cccc}
1 & 0 & 0 & 0 \\ 
0 & -1 & 0 & 0 \\ 
0 & 0 & -1 & 0 \\ 
0 & 0 & 0 & -1
\end{array}
\right] .
\end{equation}
Las componentes contravariantes surgen de la relaci'on $g^{\mu \sigma
}g_{\sigma \nu }=\delta _{\nu }^{\mu }$, en donde $\delta _{\nu }^{\mu }$\
es la delta de Kronecker. Para relatividad especial, 
\begin{equation}
g^{\mu \nu }=\left[ 
\begin{array}{cccc}
1 & 0 & 0 & 0 \\ 
0 & -1 & 0 & 0 \\ 
0 & 0 & -1 & 0 \\ 
0 & 0 & 0 & -1
\end{array}
\right] .
\end{equation}
Las coordenadas se denotan por $x^{\mu }=\left(
x^{0},x^{1},x^{2},x^{3}\right) =\left( ct,x,y,z\right) $, en tanto que el
vector (cuadrivector) momentum es $p^{\mu }=\left(
p^{0},p^{1},p^{2},p^{3}\right) =\left( E/c,p_{x},p_{y},p_{z}\right) $.
Letras griegas indican 'indices desde 0 a 3, en tanto que letras latinas
denotan 'indices de 1 a 3.

El \emph{operador momentum} se define por 
\begin{equation}
\hat{p}_{\mu }\equiv i\hbar \partial _{\mu }=i\hbar \left( \frac{%
\partial }{c\partial t},\frac{\partial }{\partial x},\frac{\partial }{%
\partial y},\frac{\partial }{\partial z}\right) =i\hbar \left( \frac{%
\partial }{c\partial t},\nabla \right) .
\end{equation}
Sus componentes contravariantes son $\hat{p}^{\mu }=i\hbar \partial ^{\mu
}=i\hbar g^{\mu \nu }\partial _{\nu }=i\hbar \left( \partial /c\partial
t,-\nabla \right) $, de modo que podemos escribir 
\begin{eqnarray}
\hat{p}^{\mu }\hat{p}_{\mu } &=&-\hbar ^{2}g^{\mu \nu }\partial _{\mu
}\partial _{\nu }=-\hbar ^{2}\partial ^{\mu }\partial _{\mu } \notag \\
&=&-\hbar ^{2}\square =-\hbar ^{2}\left( \frac{\partial ^{2}}{%
c^{2}\partial t^{2}}-\nabla ^{2}\right) . \label{Mom-Operator}
\end{eqnarray}

Finalmente, la relaci'on de conmutaci'on entre momentum y
posici'on est'a ahora dada por 
\begin{equation}
\left[ \hat{p}^{\mu },x^{\nu }\right] _{-}=i\hbar g^{\mu \nu }.
\label{conmuta}
\end{equation}

\section{La ecuaci'on de Klein-Gordon}

De la Mec'anica Cu'antica elemental sabemos que la ecuaci'on de
Schr\"{o}dinger 
\begin{equation}
i\hbar \frac{\partial \psi }{\partial t}=\left[ -\frac{\hbar ^{2}}{2m}%
\nabla ^{2}+V\left( \vec{x}\right) \right] \psi \left( \vec{x},t\right)
\end{equation}
corresponde a la relaci'on no-relativista de energ'ia en forma
operacional, 
\begin{equation}
\hat{E}=\frac{\hat{p}^{2}}{2m}+\hat{V}\left( \vec{x}\right) ,
\end{equation}
en donde 
\begin{equation}
\hat{E}=i\hbar \frac{\partial }{\partial t},\qquad \hat{p}=-i\hbar \nabla
\end{equation}
son los operadores de energ'ia y momentum, respectivamente. En orden
de obtener una ecuaci'on de onda relativista, comenzamos considerando
part'iculas libres con la relaci'on relativista entre momentum y
energ'ia 
\begin{equation}
p^{\mu }p_{\mu }=\frac{E^{2}}{c^{2}}-\vec{p}\cdot \vec{p}=m^{2}c^{2}.
\end{equation}
Ahora reemplazamos el momentum $p^{\mu }$\ por el operador momentum $%
\hat{p}^{\mu }=i\hbar \partial ^{\mu }=i\hbar \left( \hat{p}^{0},{\bf %
\hat{p}}\right) $. Obtenemos as'i la \emph{ecuaci'on de Klein-Gordon}
para part'iculas libres, 
\begin{equation}
\hat{p}^{\mu }\hat{p}_{\mu }\psi =m^{2}c^{2}\psi . \label{KG1}
\end{equation}
Aqu'i $m$ es la masa (de reposo) de la part'icula y $c$ la velocidad
de la luz en el vac'io. Con la ayuda de (\ref{Mom-Operator}), podemos
escribir esta ecuaci'on en la forma 
\begin{equation}
\left( \square +\frac{m^{2}c^{2}}{\hbar ^{2}}\right) \psi =0, \label{KG2}
\end{equation}
o bien, 
\begin{equation}
\left( \frac{\partial ^{2}}{c^{2}\partial t^{2}}-\nabla ^{2}+\frac{m^{2}c^{2}%
}{\hbar ^{2}}\right) \psi =0.
\end{equation}
Verificamos inmediatamente la covariancia de Lorentz de la ecuaci'on de
Klein-Gordon, dado que $\hat{p}^{\mu }\hat{p}_{\mu }$\ es invariante de
Lorentz. Tambi'en reconocemos a (\ref{KG2}) como la ecuaci'on de onda
cl'asica incluyendo el \emph{t'ermino de masa} $m^{2}c^{2}/\hbar
^{2} $. Soluciones libres son de la forma 
\begin{equation}
\psi =\psi _{0}\exp \left( -\frac{i}{\hbar }p_{\mu }x^{\mu }\right) =\psi
_{0}\exp \left[ \frac{i}{\hbar }\left( \vec{p}\cdot \vec{x}%
-Et\right) \right] . \label{FreeSol}
\end{equation}
De hecho, insertando (\ref{FreeSol}) en (\ref{KG1}) nos lleva a la
condici'on 
\begin{equation}
E=\pm c\sqrt{m^{2}c^{2}+p^{2}}.
\end{equation}
Luego, existen soluciones tanto para $E=+c\sqrt{m^{2}c^{2}+p^{2}}$ positivo
como negativo, $E=-c\sqrt{m^{2}c^{2}+p^{2}}$. Veremos m'as adelante que
las soluciones con energ'ia negativa est'an f'isicamente conectadas
con antipart'iculas. Dado que las antipart'iculas pueden de hecho ser
observadas en la naturaleza, hemos ya obtenido una indicaci'on del valor
de extender la teor'ia no-relativista.

A continuaci'on construimos la cuadricorriente $j_{\mu }$ conectada con (%
\ref{KG1}). En analog'ia a nuestras consideraciones concernientes a la
ecuaci'on de Schr\"{o}dinger, esperamos una ley de conservaci'on para $%
j_{\mu }$. Partimos de (\ref{KG1}), escrita en la forma 
\begin{equation}
\left( \hat{p}^{\mu }\hat{p}_{\mu }-m^{2}c^{2}\right) \psi =0
\end{equation}
y tomamos el conjugado complejo de esta ecuaci'on, esto es, 
\begin{equation}
\left( \hat{p}^{\mu }\hat{p}_{\mu }-m^{2}c^{2}\right) \psi ^{\ast }=0.
\end{equation}
Multiplicando ambas ecuaciones por la izquierda, la primera por $\psi ^{\ast
}$ y la segunda por $\psi $, y calculando la diferencia de las dos
ecuaciones resultantes encontramos 
\begin{equation}
\psi ^{\ast }\left( \hat{p}^{\mu }\hat{p}_{\mu }-m^{2}c^{2}\right) \psi
-\psi \left( \hat{p}^{\mu }\hat{p}_{\mu }-m^{2}c^{2}\right) \psi ^{\ast }=0,
\end{equation}
o bien, 
\begin{equation}
\psi ^{\ast }\left( \hbar ^{2}\partial ^{\mu }\partial _{\mu
}+m^{2}c^{2}\right) \psi -\psi \left( \hbar ^{2}\partial ^{\mu }\partial
_{\mu }+m^{2}c^{2}\right) \psi ^{\ast }=0.
\end{equation}



\section{El tensor momentum-energ'ia del campo de Klein-Gordon}

Demostraremos ahora el procedimiento del formalismo de Lagrange para el
campo de Klein-Gordon. En este caso tratamos con dos campos de onda $\psi $
y $\psi ^{\ast }$ (para part'iculas cargadas). Ahora, la \emph{densidad
lagrangeana para el campo de Klein-Gordon} es de la forma 
\begin{equation}
\mathcal{L}\left( \psi ,\psi ^{\ast },\partial _{\mu }\psi ,\partial _{\mu
}\psi ^{\ast }\right) =\frac{\hbar ^{2}}{2m}\left( g^{\mu \nu }\partial
_{\mu }\psi ^{\ast }\partial _{\nu }\psi -\frac{m^{2}c^{2}}{\hbar ^{2}}%
\psi ^{\ast }\psi \right) . \label{LagKG}
\end{equation}
La constante $\hbar ^{2}/2m$ se escoge de manera que $\int \mathcal{L}%
d^{3}x$ tenga dimensiones de energ'ia. La demostraci'on de que esta
expresi'on es la densidad lagrangeana correcta para el campo de
Klein-Gordon est'a dada al mostrar que, con ayuda de las ecuaciones de
Euler-Lagrange 
\begin{equation}
\frac{\partial }{\partial x^{\mu }}\frac{\partial \mathcal{L}}{\partial \psi
_{\sigma ,\mu }}-\frac{\partial \mathcal{L}}{\partial \psi _{\sigma }}=0,
\end{equation}
uno obtiene las ecuaciones de Klein-Gordon para $\psi $ y $\psi ^{\ast }$.
Calculamos: 
\begin{eqnarray*}
\frac{\partial \mathcal{L}}{\partial \psi _{,\alpha }} &=&\frac{\hbar ^{2}%
}{2m}g^{\mu \nu }\psi _{,\mu }^{\ast }\frac{\partial \psi _{,\nu }}{\partial
\psi _{,\alpha }}=\frac{\hbar ^{2}}{2m}g^{\mu \nu }\psi _{,\mu }^{\ast
}\delta _{\nu }^{\alpha }, \\
\frac{\partial \mathcal{L}}{\partial \psi _{,\alpha }} &=&\frac{\hbar ^{2}%
}{2m}g^{\mu \alpha }\psi _{,\mu }^{\ast }=\frac{\hbar ^{2}}{2m}\partial
^{\alpha }\psi ^{\ast },
\end{eqnarray*}
\begin{equation}
\frac{\partial \mathcal{L}}{\partial \psi }=-\frac{mc^{2}}{2}\psi ^{\ast }.
\end{equation}
Luego, 
\begin{eqnarray*}
\frac{\hbar ^{2}}{2m}\partial _{\alpha }\partial ^{\alpha }\psi ^{\ast }+%
\frac{mc^{2}}{2}\psi ^{\ast } &=&0, \\
\partial ^{\alpha }\partial _{\alpha }\psi ^{\ast }+\frac{m^{2}c^{2}}{%
\hbar ^{2}}\psi ^{\ast } &=&0
\end{eqnarray*}


\subsection{Densidad Lagrangeana y tensor de momentum-energ'ia
para una part'icula de Klein-Gordon en un campo electromagn'etico}

Una extensi'on natural de la densidad lagrangeana (\ref{LagKG}) para el
sistema acoplado del campo de Maxwell y el campo de Klein-Gordon es 
\begin{equation}
\mathcal{L}=-\frac{1}{4}F_{\mu \nu }F^{\mu \nu }+\frac{1}{2m}\left[ g^{\mu
\nu }\left( -i\hbar \partial _{\mu }-\frac{e}{c}A_{\mu }\right) \psi
^{\ast }\left( i\hbar \partial _{\nu }-\frac{e}{c}A_{\nu }\right) \psi
-m^{2}c^{2}\psi ^{\ast }\psi \right] , \label{LagKGM}
\end{equation}
con $F_{\mu \nu }\equiv \partial _{\mu }A_{\nu }-\partial _{\nu }A_{\mu }$.
Para aplicar las ecuaciones de Euler-Lagrange, primero escribimos (\ref
{LagKGM}) en una forma m'as conveniente para el c'alculo de las
derivadas necesarias: 
\begin{eqnarray*}
\mathcal{L} &=&-\frac{1}{4}\left( \partial _{\mu }A_{\nu }-\partial _{\nu
}A_{\mu }\right) \left( \partial ^{\mu }A^{\nu }-\partial ^{\nu }A^{\mu
}\right) +\frac{1}{2m}g^{\mu \nu }\left[ \left( -i\hbar \partial _{\mu
}\psi ^{\ast }-\frac{e}{c}A_{\mu }\psi ^{\ast }\right) \right. \\
&&\times \left. \left( i\hbar \partial _{\nu }\psi -\frac{e}{c}A_{\nu
}\psi \right) -\frac{1}{4}g_{\mu \nu }m^{2}c^{2}\psi ^{\ast }\psi \right] ,
\\
\mathcal{L} &=&-\frac{1}{4}\left[ \left( \partial _{\mu }A_{\nu }\right)
\partial ^{\mu }A^{\nu }-\left( \partial _{\mu }A_{\nu }\right) \partial
^{\nu }A^{\mu }-\left( \partial _{\nu }A_{\mu }\right) \partial ^{\mu
}A^{\nu }+\left( \partial _{\nu }A_{\mu }\right) \partial ^{\nu }A^{\mu }%
\right] +\frac{1}{2m}g^{\mu \nu }\times \\
&&\times \left[ -i\hbar \left( \partial _{\mu }\psi ^{\ast }\right) \left(
i\hbar \partial _{\nu }\psi -\frac{e}{c}A_{\nu }\psi \right) -\frac{e}{c}%
A_{\mu }\psi ^{\ast }\left( i\hbar \partial _{\nu }\psi -\frac{e}{c}A_{\nu
}\psi \right) -\frac{1}{4}g_{\mu \nu }m^{2}c^{2}\psi ^{\ast }\psi \right] ,
\\
\mathcal{L} &=&-\frac{1}{2}\left[ \left( \partial _{\mu }A_{\nu }\right)
\partial ^{\mu }A^{\nu }-\left( \partial _{\nu }A_{\mu }\right) \partial
^{\mu }A^{\nu }\right] +\frac{1}{2m}g^{\mu \nu }\left[ \hbar ^{2}\left(
\partial _{\mu }\psi ^{\ast }\right) \partial _{\nu }\psi +i\frac{e\hbar }{%
c}\left( \partial _{\mu }\psi ^{\ast }\right) A_{\nu }\psi \right. \\
&&\left. -i\frac{e\hbar }{c}A_{\mu }\psi ^{\ast }\partial _{\nu }\psi
+\left( \frac{e}{c}\right) ^{2}A_{\mu }\psi ^{\ast }A_{\nu }\psi -\frac{1}{4}%
g_{\mu \nu }m^{2}c^{2}\psi ^{\ast }\psi \right] ,
\end{eqnarray*}
o bien, 
\begin{eqnarray}
\mathcal{L} &=&-\frac{1}{2}Y^{\nu \rho }g^{\mu \sigma }A_{\rho ,\sigma
}\left( A_{\nu ,\mu }-A_{\mu ,\nu }\right) +\frac{1}{2m}g^{\mu \nu }\left[
\hbar ^{2}\psi _{,\mu }^{\ast }\psi _{,\nu }+i\frac{e\hbar }{c}\psi
_{,\mu }^{\ast }A_{\nu }\psi +\right. \notag \\
&&\left. -i\frac{e\hbar }{c}A_{\mu }\psi ^{\ast }\psi _{,\nu }+\left( 
\frac{e}{c}\right) ^{2}A_{\mu }A_{\nu }\psi ^{\ast }\psi -\frac{1}{4}g_{\mu
\nu }m^{2}c^{2}\psi ^{\ast }\psi \right] . \label{LagKGM2}
\end{eqnarray}
Calculamos ahora las derivadas: 
\begin{eqnarray*}
\frac{\partial \mathcal{L}}{\partial \psi ^{\ast }} &=&\frac{1}{2m}g^{\alpha
\beta }\left[ -i\frac{e\hbar }{c}A_{\alpha }\psi _{,\beta }+\left( \frac{e%
}{c}\right) ^{2}A_{\alpha }A_{\beta }\psi -\frac{1}{4}g_{\alpha \beta
}m^{2}c^{2}\psi \right] , \\
\frac{\partial \mathcal{L}}{\partial \psi _{,\alpha }^{\ast }} &=&\frac{1}{2m%
}g^{\alpha \beta }\left( \hbar ^{2}\psi _{,\beta }+i\frac{e\hbar }{c}%
A_{\beta }\psi \right) , \\
\frac{\partial }{\partial x^{\alpha }}\frac{\partial \mathcal{L}}{\partial
\psi _{,\alpha }^{\ast }} &=&\frac{1}{2m}g^{\alpha \beta }\left[ \hbar
^{2}\psi _{,\beta ,\alpha }+i\frac{e\hbar }{c}\partial _{\alpha }\left(
A_{\beta }\psi \right) \right] ,
\end{eqnarray*}
y las insertamos en las ecuaciones de Euler-Lagrange (\ref{Eu-La}): 
\begin{eqnarray*}
-i\frac{e\hbar }{c}g^{\alpha \beta }A_{\alpha }\psi _{,\beta }+\left( 
\frac{e}{c}\right) ^{2}g^{\alpha \beta }A_{\alpha }A_{\beta }\psi
-m^{2}c^{2}\psi -\hbar ^{2}g^{\alpha \beta }\psi _{,\beta ,\alpha }-i\frac{%
e\hbar }{c}g^{\alpha \beta }\partial _{\alpha }\left( A_{\beta }\psi
\right) &=&0, \\
-\hbar ^{2}g^{\alpha \beta }\psi _{,\beta ,\alpha }-i\frac{e\hbar }{c}%
g^{\alpha \beta }\partial _{\alpha }\left( A_{\beta }\psi \right) -i\frac{%
e\hbar }{c}g^{\alpha \beta }A_{\alpha }\psi _{,\beta }+\left( \frac{e}{c}%
\right) ^{2}g^{\alpha \beta }A_{\alpha }A_{\beta }\psi -m^{2}c^{2}\psi &=&0,
\\
-\hbar ^{2}\partial ^{\alpha }\partial _{\alpha }\psi -i\frac{e\hbar }{c}%
\partial ^{\alpha }\left( A_{\alpha }\psi \right) -i\frac{e\hbar }{c}%
A^{\alpha }\partial _{\alpha }\psi +\left( \frac{e}{c}\right) ^{2}A^{\alpha
}A_{\alpha }\psi -m^{2}c^{2}\psi &=&0,
\end{eqnarray*}
para obtener 
\begin{equation}
\left( i\hbar \partial ^{\alpha }-\frac{e}{c}A^{\alpha }\right) \left(
i\hbar \partial _{\alpha }-\frac{e}{c}A_{\alpha }\right) \psi
-m^{2}c^{2}\psi =0.
\end{equation}
De (\ref{LagKGM2}) podemos obtener de manera an'aloga las ecuaciones de
Maxwell: 
\begin{eqnarray*}
\frac{\partial \mathcal{L}}{\partial A_{\alpha }} &=&\frac{1}{2m}g^{\mu \nu }%
\left[ i\frac{e\hbar }{c}\psi _{,\mu }^{\ast }\delta _{\nu }^{\alpha }\psi
-i\frac{e\hbar }{c}\delta _{\mu }^{\alpha }\psi ^{\ast }\psi _{,\nu
}+\left( \frac{e}{c}\right) ^{2}\left( \delta _{\mu }^{\alpha }A_{\nu
}+A_{\mu }\delta _{\nu }^{\alpha }\right) \psi ^{\ast }\psi \right] , \\
\frac{\partial \mathcal{L}}{\partial A_{\alpha }} &=&\frac{1}{2m}\left[ i%
\frac{e\hbar }{c}g^{\mu \alpha }\psi _{,\mu }^{\ast }\psi -i\frac{e\hbar 
}{c}g^{\alpha \nu }\psi ^{\ast }\psi _{,\nu }+\left( \frac{e}{c}\right)
^{2}\left( g^{\alpha \nu }A_{\nu }+g^{\mu \alpha }A_{\mu }\right) \psi
^{\ast }\psi \right] , \\
\frac{\partial \mathcal{L}}{\partial A_{\alpha }} &=&\frac{1}{2m}\left[ i%
\frac{e\hbar }{c}\psi \partial ^{\alpha }\psi ^{\ast }-i\frac{e\hbar }{c}%
\psi ^{\ast }\partial ^{\alpha }\psi +2\left( \frac{e}{c}\right)
^{2}A^{\alpha }\psi ^{\ast }\psi \right]
\end{eqnarray*}
\begin{eqnarray*}
\frac{\partial \mathcal{L}}{\partial A_{\alpha ,\beta }} &=&-\frac{1}{2}%
g^{\nu \rho }g^{\mu \sigma }\frac{\partial A_{\rho ,\sigma }}{\partial
A_{\alpha ,\beta }}\left( A_{\nu ,\mu }-A_{\mu ,\nu }\right) -\frac{1}{2}%
g^{\nu \rho }g^{\mu \sigma }A_{\rho ,\sigma }\left( \frac{\partial A_{\nu
,\mu }}{\partial A_{\alpha ,\beta }}-\frac{\partial A_{\mu ,\nu }}{\partial
A_{\alpha ,\beta }}\right) , \\
\frac{\partial \mathcal{L}}{\partial A_{\alpha ,\beta }} &=&-\frac{1}{2}%
g^{\nu \rho }g^{\mu \sigma }\delta _{\rho }^{\alpha }\delta _{\sigma
}^{\beta }\left( A_{\nu ,\mu }-A_{\mu ,\nu }\right) -\frac{1}{2}Y^{\nu \rho
}g^{\mu \sigma }A_{\rho ,\sigma }\left( \delta _{\nu }^{\alpha }\delta _{\mu
}^{\beta }-\delta _{\mu }^{\alpha }\delta _{\nu }^{\beta }\right) , \\
\frac{\partial \mathcal{L}}{\partial A_{\alpha ,\beta }} &=&-\frac{1}{2}%
g^{\nu \alpha }g^{\mu \beta }\left( A_{\nu ,\mu }-A_{\mu ,\nu }\right) -%
\frac{1}{2}A_{\rho ,\sigma }\left( g^{\alpha \rho }g^{\beta \sigma
}-g^{\beta \rho }g^{\alpha \sigma }\right) , \\
\frac{\partial \mathcal{L}}{\partial A_{\alpha ,\beta }} &=&-\frac{1}{2}%
\left( g^{\nu \alpha }g^{\mu \beta }A_{\nu ,\mu }-g^{\nu \alpha }g^{\mu
\beta }A_{\mu ,\nu }\right) -\frac{1}{2}\left( g^{\alpha \rho }g^{\beta
\sigma }A_{\rho ,\sigma }-g^{\beta \rho }g^{\alpha \sigma }A_{\rho ,\sigma
}\right) , \\
\frac{\partial \mathcal{L}}{\partial A_{\alpha ,\beta }} &=&-\frac{1}{2}%
\left( \partial ^{\beta }A^{\alpha }-\partial ^{\alpha }A^{\beta }\right) -%
\frac{1}{2}\left( \partial ^{\beta }A^{\alpha }-\partial ^{\alpha }A^{\beta
}\right) , \\
\frac{\partial \mathcal{L}}{\partial A_{\alpha ,\beta }} &=&\partial
^{\alpha }A^{\beta }-\partial ^{\beta }A^{\alpha }.
\end{eqnarray*}
Finalmente, encontramos 
\begin{equation}
\frac{1}{2m}\left[ i\frac{e\hbar }{c}\psi \partial ^{\alpha }\psi ^{\ast
}-i\frac{e\hbar }{c}\psi ^{\ast }\partial ^{\alpha }\psi +2\left( \frac{e}{%
c}\right) ^{2}A^{\alpha }\psi ^{\ast }\psi \right] -\partial _{\beta }\left(
\partial ^{\alpha }A^{\beta }-\partial ^{\beta }A^{\alpha }\right) =0,
\end{equation}
o bien, 
\begin{eqnarray*}
\partial _{\mu }\left( \partial ^{\mu }A^{\nu }-\partial ^{\nu }A^{\mu
}\right) &=&\frac{1}{2m}\left[ i\frac{e\hbar }{c}\psi ^{\ast }\partial
^{\nu }\psi -i\frac{e\hbar }{c}\psi \partial ^{\nu }\psi ^{\ast }-2\left( 
\frac{e}{c}\right) ^{2}A^{\nu }\psi ^{\ast }\psi \right] , \\
\partial _{\mu }\left( \partial ^{\mu }A^{\nu }-\partial ^{\nu }A^{\mu
}\right) &=&i\frac{e\hbar }{2mc}\left( \psi ^{\ast }\partial ^{\nu }\psi
-\psi \partial ^{\nu }\psi ^{\ast }\right) -\frac{e^{2}}{mc^{2}}A^{\nu }\psi
^{\ast }\psi .
\end{eqnarray*}
Definiendo 
\begin{equation}
j^{\nu }\equiv \frac{1}{4\pi }\left[ \frac{ie\hbar }{2m}\left( \psi ^{\ast
}\partial ^{\nu }\psi -\psi \partial ^{\nu }\psi ^{\ast }\right) -\frac{e^{2}%
}{mc}A^{\nu }\psi ^{\ast }\psi \right] , \label{JotaNu}
\end{equation}
podemos escribir las ecuaciones de Maxwell en su forma usual, 
\begin{equation}
\partial _{\mu }F^{\mu \nu }=\frac{4\pi }{c}j^{\nu }. \label{Maxwell}
\end{equation}


\section{Energ'ia, Momentum y Carga Total del Campo de Klein-Gordon.}

\begin{equation}
\psi =\sum_{\vec{p}_{n}}\left\{ a_{\vec{p}_{n}}\varphi _{\vec{p}_{n}\left(
+\right) }+b_{\vec{p}_{n}}^{\ast }\varphi _{\vec{p}_{n}\left( +\right)
}^{\ast }\right\}   \label{Solucion General}
\end{equation}%
es la soluci'on general de la ecuaci'on de campo. Las funciones $%
\varphi _{\vec{p}_{n}\left( \pm \right) }$ sabemos est'an dadas por:%
\begin{eqnarray}
\varphi _{\vec{p}_{n}\left( +\right) } &=&\sqrt{\frac{mc^{2}}{L^{3}E_{p_{n}}}%
}e^{\frac{i}{\hbar }\left( \vec{p}_{n}\cdot \vec{x}-E_{p_{n}}t\right) }
\label{Phi(+-)} \\
\varphi _{\vec{p}_{n}\left( +\right) }^{\ast } &=&\varphi _{\vec{p}%
_{n}\left( -\right) }=\sqrt{\frac{mc^{2}}{L^{3}E_{p_{n}}}}e^{\frac{i}{\hbar }%
\left( -\vec{p}_{n}\cdot \vec{x}+E_{p_{n}}t\right) },
\end{eqnarray}%
con:%
\begin{equation}
E_{p_{n}}=c\sqrt{\vec{p}_{n}^{2}+m^{2}c^{2}}.  \label{Epn}
\end{equation}


En general, sabemos que en la teor'ia de campos relativista la energ%
'ia y el momentum total del campo en consideraci'on est'an dados
respectivamente por:%
\begin{equation}
H=\int T_{0}^{\quad 0}d^{3}x,  \label{Energia Total}
\end{equation}%
y:%
\begin{equation}
p_{i}=-\frac{1}{c}\int T_{i}^{\quad 0}d^{3}x,  \label{Momentum Total}
\end{equation}%
donde:%
\begin{equation}
T_{\mu }^{\quad \nu }=\frac{\partial \psi }{\partial x^{\mu }}\frac{\partial
{\cal L}}{\partial \left( \partial \psi /\partial x^{\nu }\right) }+\frac{%
\partial \psi ^{\ast }}{\partial x^{\mu }}\frac{\partial {\cal L}}{\partial
\left( \partial \psi ^{\ast }/\partial x^{\nu }\right) }-{\cal L}g_{\mu
}^{\quad \nu }  \label{Tensor Energia-Momentum}
\end{equation}%
es el tensor de energ'ia-momentum, y ${\cal L}$ es la densidad
Lagrangeana del campo.

Para el campo de K-G, la densidad Lagrangeana viene dada por:%
\begin{eqnarray}
{\cal L}\left( \psi ,\psi ^{\ast },\frac{\partial \psi }{\partial x^{\mu }},%
\frac{\partial \psi ^{\ast }}{\partial x^{\nu }}\right) &=&\frac{\hbar ^{2}}{%
2m}\left\{ g^{\mu \nu }\frac{\partial \psi ^{\ast }}{\partial x^{\mu }}\frac{%
\partial \psi }{\partial x^{\nu }}-\frac{m^{2}c^{2}}{\hbar ^{2}}\psi ^{\ast
}\psi \right\}  \notag \\
&=&\frac{\hbar ^{2}}{2m}\left\{ g^{0\nu }\frac{\partial \psi ^{\ast }}{%
\partial x^{0}}\frac{\partial \psi }{\partial x^{\nu }}+g^{i\nu }\frac{%
\partial \psi ^{\ast }}{\partial x^{i}}\frac{\partial \psi }{\partial x^{\nu
}}-\frac{m^{2}c^{2}}{\hbar ^{2}}\psi ^{\ast }\psi \right\}  \notag \\
&=&\frac{\hbar ^{2}}{2m}\left\{ g^{00}\frac{\partial \psi ^{\ast }}{\partial
x^{0}}\frac{\partial \psi }{\partial x^{0}}+g^{ii}\frac{\partial \psi ^{\ast
}}{\partial x^{i}}\frac{\partial \psi }{\partial x^{i}}-\frac{m^{2}c^{2}}{%
\hbar ^{2}}\psi ^{\ast }\psi \right\}  \notag \\
&=&\frac{\hbar ^{2}}{2m}\left\{ \frac{1}{c^{2}}\frac{\partial \psi ^{\ast }}{%
\partial t}\frac{\partial \psi }{\partial t}-\frac{\partial \psi ^{\ast }}{%
\partial x^{i}}\frac{\partial \psi }{\partial x^{i}}-\frac{m^{2}c^{2}}{\hbar
^{2}}\psi ^{\ast }\psi \right\}  \notag \\
&=&\frac{\hbar ^{2}}{2m}\left\{ \frac{1}{c^{2}}\dot{\psi}^{\ast }\dot{\psi}%
-\partial _{i}\psi ^{\ast }\partial _{i}\psi -\frac{m^{2}c^{2}}{\hbar ^{2}}%
\psi ^{\ast }\psi \right\} ,  \label{Densidad Lagrangeana Campo K-G}
\end{eqnarray}%
as'i, reemplazando las componentes adecuadas del tensor de energ'i%
a-momentum en (\ref{Energia Total}) y (\ref{Momentum Total}); y usando (\ref%
{Densidad Lagrangeana Campo K-G}), se obtiene que:%
\begin{eqnarray}
H &=&\int \left\{ \frac{\partial \psi }{\partial x^{0}}\frac{\partial 
{\cal L}}{\partial \left( \partial \psi /\partial x^{0}\right) }+\frac{%
\partial \psi ^{\ast }}{\partial x^{0}}\frac{\partial {\cal L}}{\partial
\left( \partial \psi ^{\ast }/\partial x^{0}\right) }-{\cal L}g_{0}^{\quad
0}\right\} d^{3}x  \notag \\
&=&\int \left\{ \dot{\psi}\frac{\partial {\cal L}}{\partial \dot{\psi}}+\dot{%
\psi}^{\ast }\frac{\partial {\cal L}}{\partial \dot{\psi}^{\ast }}-{\cal L}%
\right\} d^{3}x  \notag \\
&=&\int \left\{ \dot{\psi}\left( \frac{\hbar ^{2}}{2mc^{2}}\dot{\psi}^{\ast
}\right) +\dot{\psi}^{\ast }\left( \frac{\hbar ^{2}}{2mc^{2}}\dot{\psi}%
\right) -\frac{\hbar ^{2}}{2m}\left( \frac{1}{c^{2}}\dot{\psi}^{\ast }\dot{%
\psi}-\partial _{i}\psi ^{\ast }\partial _{i}\psi -\frac{m^{2}c^{2}}{\hbar
^{2}}\psi ^{\ast }\psi \right) \right\} d^{3}x  \notag \\
&=&\int \left\{ \frac{\hbar ^{2}}{2mc^{2}}\dot{\psi}^{\ast }\dot{\psi}+\frac{%
\hbar ^{2}}{2mc^{2}}\dot{\psi}^{\ast }\dot{\psi}-\frac{\hbar ^{2}}{2mc^{2}}%
\dot{\psi}^{\ast }\dot{\psi}+\frac{\hbar ^{2}}{2m}\partial _{i}\psi ^{\ast
}\partial _{i}\psi +\frac{\hbar ^{2}}{2m}\frac{m^{2}c^{2}}{\hbar ^{2}}\psi
^{\ast }\psi \right\} d^{3}x  \notag \\
&=&\frac{\hbar ^{2}}{2m}\int \left\{ \frac{1}{c^{2}}\dot{\psi}^{\ast }\dot{%
\psi}+\partial _{i}\psi ^{\ast }\partial _{i}\psi +\frac{m^{2}c^{2}}{\hbar
^{2}}\psi ^{\ast }\psi \right\} d^{3}x,  \label{Energia Total Campo K-G}
\end{eqnarray}%
y:%
\begin{eqnarray}
p_{i} &=&-\frac{1}{c}\int \left\{ \frac{\partial \psi }{\partial x^{i}}\frac{%
\partial {\cal L}}{\partial \left( \partial \psi /\partial x^{0}\right) }+%
\frac{\partial \psi ^{\ast }}{\partial x^{i}}\frac{\partial {\cal L}}{%
\partial \left( \partial \psi ^{\ast }/\partial x^{0}\right) }-{\cal L}%
g_{i}^{\quad 0}\right\} d^{3}x  \notag \\
&=&-\frac{1}{c^{2}}\int \left\{ \frac{\partial \psi }{\partial x^{i}}\frac{%
\partial {\cal L}}{\partial \dot{\psi}}+\frac{\partial \psi ^{\ast }}{%
\partial x^{i}}\frac{\partial {\cal L}}{\partial \dot{\psi}^{\ast }}\right\}
d^{3}x  \notag \\
&=&-\frac{\hbar ^{2}}{2mc^{2}}\int \left\{ \left( \partial _{i}\psi \right)
\dot{\psi}^{\ast }+\left( \partial _{i}\psi ^{\ast }\right) \dot{\psi}%
\right\} d^{3}x  \label{Momentum Total Campo K-G}
\end{eqnarray}%
son respectivamente la energ'ia y el momentum total del campo de
Klein-Gordon.

Por otro lado, la carga total se obtiene integrando la densidad de carga
definida por:%
\begin{equation}
\rho =i\frac{\hbar e}{2mc^{2}}\left( \psi ^{\ast }\frac{\partial \psi }{%
\partial t}-\psi \frac{\partial \psi ^{\ast }}{\partial t}\right)
\label{Densida de Carga Campo K-G}
\end{equation}%
sobre todo el volumen de la caja. Es decir:%
\begin{eqnarray}
Q &=&\int \rho d^{3}x  \notag \\
&=&i\frac{\hbar e}{2mc^{2}}\int \left\{ \psi ^{\ast }\dot{\psi}-\psi \dot{%
\psi}^{\ast }\right\} d^{3}x.  \label{Carga Total Campo K-G}
\end{eqnarray}

Ahora, reemplazando (\ref{Phi(+-)}) en la soluci'on general (\ref%
{Solucion General}), encontramos:%
\begin{eqnarray*}
\psi &=&\sum_{\vec{p}_{n}}\left( \frac{mc^{2}}{L^{3}E_{p_{n}}}\right) ^{%
\frac{1}{2}}\left\{ a_{\vec{p}_{n}}e^{\frac{i}{\hbar }\left( \vec{p}%
_{n}\cdot \vec{x}-E_{p_{n}}t\right) }+b_{\vec{p}_{n}}^{\ast }e^{\frac{i}{%
\hbar }\left( -\vec{p}_{n}\cdot \vec{x}+E_{p_{n}}t\right) }\right\} \\
&=&\left( \frac{mc^{2}}{L^{3}}\right) ^{\frac{1}{2}}\sum_{\vec{p}_{n}}\left(
E_{p_{n}}\right) ^{-\frac{1}{2}}\left\{ a_{\vec{p}_{n}}e^{\frac{i}{\hbar }%
\vec{p}_{n}\cdot \vec{x}}e^{-\frac{i}{\hbar }E_{p_{n}}t}+b_{\vec{p}%
_{n}}^{\ast }e^{-\frac{i}{\hbar }\vec{p}_{n}\cdot \vec{x}}e^{\frac{i}{\hbar }%
E_{p_{n}}t}\right\} .
\end{eqnarray*}%
Es directo que:%
\begin{equation}
\psi ^{\ast }=\left( \frac{mc^{2}}{L^{3}}\right) ^{\frac{1}{2}}\sum_{\vec{p}%
_{n}}\left( E_{p_{n}}\right) ^{-\frac{1}{2}}\left\{ a_{\vec{p}_{n}}^{\ast
}e^{-\frac{i}{\hbar }\vec{p}_{n}\cdot \vec{x}}e^{\frac{i}{\hbar }%
E_{p_{n}}t}+b_{\vec{p}_{n}}e^{\frac{i}{\hbar }\vec{p}_{n}\cdot \vec{x}}e^{-%
\frac{i}{\hbar }E_{p_{n}}t}\right\} .
\end{equation}

Ahora, calcularemos las respectivas derivadas involucradas en (\ref{Energia
Total Campo K-G}), (\ref{Momentum Total Campo K-G}) y (\ref{Carga Total
Campo K-G}). Estas son respectivamente:%
\begin{eqnarray*}
\dot{\psi} &=&\frac{\partial }{\partial t}\left[ \left( \frac{mc^{2}}{L^{3}}%
\right) ^{\frac{1}{2}}\sum_{\vec{p}_{n}}\left( E_{p_{n}}\right) ^{-\frac{1}{2%
}}\left\{ a_{\vec{p}_{n}}e^{\frac{i}{\hbar }\vec{p}_{n}\cdot \vec{x}}e^{-%
\frac{i}{\hbar }E_{p_{n}}t}+b_{\vec{p}_{n}}^{\ast }e^{-\frac{i}{\hbar }\vec{p%
}_{n}\cdot \vec{x}}e^{\frac{i}{\hbar }E_{p_{n}}t}\right\} \right]  \\
&=&\left( \frac{mc^{2}}{L^{3}}\right) ^{\frac{1}{2}}\sum_{\vec{p}_{n}}\left(
E_{p_{n}}\right) ^{-\frac{1}{2}}\left\{ a_{\vec{p}_{n}}e^{\frac{i}{\hbar }%
\vec{p}_{n}\cdot \vec{x}}\frac{\partial }{\partial t}\left( e^{-\frac{i}{%
\hbar }E_{p_{n}}t}\right) +b_{\vec{p}_{n}}^{\ast }e^{-\frac{i}{\hbar }\vec{p}%
_{n}\cdot \vec{x}}\frac{\partial }{\partial t}\left( e^{\frac{i}{\hbar }%
E_{p_{n}}t}\right) \right\}  \\
&=&\left( \frac{mc^{2}}{L^{3}}\right) ^{\frac{1}{2}}\sum_{\vec{p}_{n}}\left(
E_{p_{n}}\right) ^{-\frac{1}{2}}e^{\frac{i}{\hbar }\vec{p}_{n}\cdot \vec{x}%
}\left\{ a_{\vec{p}_{n}}e^{\frac{i}{\hbar }\vec{p}_{n}\cdot \vec{x}}\left( -%
\frac{i}{\hbar }E_{p_{n}}\right) e^{-\frac{i}{\hbar }E_{p_{n}}t}+b_{\vec{p}%
_{n}}^{\ast }e^{-\frac{i}{\hbar }\vec{p}_{n}\cdot \vec{x}}\left( \frac{i}{%
\hbar }E_{p_{n}}\right) e^{\frac{i}{\hbar }E_{p_{n}}t}\right\}  \\
&=&-\frac{i}{\hbar }\left( \frac{mc^{2}}{L^{3}}\right) ^{\frac{1}{2}}\sum_{%
\vec{p}_{n}}\left( E_{p_{n}}\right) ^{\frac{1}{2}}\left\{ a_{\vec{p}_{n}}e^{%
\frac{i}{\hbar }\vec{p}_{n}\cdot \vec{x}}e^{-\frac{i}{\hbar }E_{p_{n}}t}-b_{%
\vec{p}_{n}}^{\ast }e^{-\frac{i}{\hbar }\vec{p}_{n}\cdot \vec{x}}e^{\frac{i}{%
\hbar }E_{p_{n}}t}\right\} ,
\end{eqnarray*}%
\begin{eqnarray*}
\dot{\psi}^{\ast } &=&\left( \dot{\psi}\right) ^{\ast } \\
&=&\left( -\frac{i}{\hbar }\left( \frac{mc^{2}}{L^{3}}\right) ^{\frac{1}{2}%
}\sum_{\vec{p}_{n}}\left( E_{p_{n}}\right) ^{\frac{1}{2}}\left\{ a_{\vec{p}%
_{n}}e^{\frac{i}{\hbar }\vec{p}_{n}\cdot \vec{x}}e^{-\frac{i}{\hbar }%
E_{p_{n}}t}-b_{\vec{p}_{n}}^{\ast }e^{-\frac{i}{\hbar }\vec{p}_{n}\cdot \vec{%
x}}e^{\frac{i}{\hbar }E_{p_{n}}t}\right\} \right) ^{\ast } \\
&=&\frac{i}{\hbar }\left( \frac{mc^{2}}{L^{3}}\right) ^{\frac{1}{2}}\sum_{%
\vec{p}_{n}}\left( E_{p_{n}}\right) ^{\frac{1}{2}}\left\{ a_{\vec{p}%
_{n}}^{\ast }e^{-\frac{i}{\hbar }\vec{p}_{n}\cdot \vec{x}}e^{\frac{i}{\hbar }%
E_{p_{n}}t}-b_{\vec{p}_{n}}e^{\frac{i}{\hbar }\vec{p}_{n}\cdot \vec{x}}e^{-%
\frac{i}{\hbar }E_{p_{n}}t}\right\} ,
\end{eqnarray*}%
\begin{eqnarray*}
\partial _{i}\psi  &=&\partial _{i}\left[ \left( \frac{mc^{2}}{L^{3}}\right)
^{\frac{1}{2}}\sum_{\vec{p}_{n}}\left( E_{p_{n}}\right) ^{-\frac{1}{2}%
}\left\{ a_{\vec{p}_{n}}e^{\frac{i}{\hbar }\vec{p}_{n}\cdot \vec{x}}e^{-%
\frac{i}{\hbar }E_{p_{n}}t}+b_{\vec{p}_{n}}^{\ast }e^{-\frac{i}{\hbar }\vec{p%
}_{n}\cdot \vec{x}}e^{\frac{i}{\hbar }E_{p_{n}}t}\right\} \right]  \\
&=&\left( \frac{mc^{2}}{L^{3}}\right) ^{\frac{1}{2}}\sum_{\vec{p}_{n}}\left(
E_{p_{n}}\right) ^{-\frac{1}{2}}\left\{ a_{\vec{p}_{n}}\partial _{i}\left(
e^{\frac{i}{\hbar }\vec{p}_{n}\cdot \vec{x}}\right) e^{-\frac{i}{\hbar }%
E_{p_{n}}t}+b_{\vec{p}_{n}}^{\ast }\partial _{i}\left( e^{-\frac{i}{\hbar }%
\vec{p}_{n}\cdot \vec{x}}\right) e^{\frac{i}{\hbar }E_{p_{n}}t}\right\}  \\
&=&\left( \frac{mc^{2}}{L^{3}}\right) ^{\frac{1}{2}}\sum_{\vec{p}_{n}}\left(
E_{p_{n}}\right) ^{-\frac{1}{2}}\left\{ a_{\vec{p}_{n}}\left( \frac{i}{\hbar 
}\vec{p}_{n}\right) e^{\frac{i}{\hbar }\vec{p}_{n}\cdot \vec{x}}e^{-\frac{i}{%
\hbar }E_{p_{n}}t}+b_{\vec{p}_{n}}^{\ast }\left( -\frac{i}{\hbar }\vec{p}%
_{n}\right) e^{-\frac{i}{\hbar }\vec{p}_{n}\cdot \vec{x}}e^{\frac{i}{\hbar }%
E_{p_{n}}t}\right\}  \\
&=&\frac{i}{\hbar }\left( \frac{mc^{2}}{L^{3}}\right) ^{\frac{1}{2}}\sum_{%
\vec{p}_{n}}\left( E_{p_{n}}\right) ^{-\frac{1}{2}}\vec{p}_{n}\left\{ a_{%
\vec{p}_{n}}e^{\frac{i}{\hbar }\vec{p}_{n}\cdot \vec{x}}e^{-\frac{i}{\hbar }%
E_{p_{n}}t}-b_{\vec{p}_{n}}^{\ast }e^{-\frac{i}{\hbar }\vec{p}_{n}\cdot \vec{%
x}}e^{\frac{i}{\hbar }E_{p_{n}}t}\right\} ,
\end{eqnarray*}%
\begin{eqnarray*}
\partial _{i}\psi ^{\ast } &=&\left( \partial _{i}\psi \right) ^{\ast } \\
&=&\left( \frac{i}{\hbar }\left( \frac{mc^{2}}{L^{3}}\right) ^{\frac{1}{2}%
}\sum_{\vec{p}_{n}}\left( E_{p_{n}}\right) ^{-\frac{1}{2}}\vec{p}_{n}\left\{
a_{\vec{p}_{n}}e^{\frac{i}{\hbar }\vec{p}_{n}\cdot \vec{x}}e^{-\frac{i}{%
\hbar }E_{p_{n}}t}-b_{\vec{p}_{n}}^{\ast }e^{-\frac{i}{\hbar }\vec{p}%
_{n}\cdot \vec{x}}e^{\frac{i}{\hbar }E_{p_{n}}t}\right\} \right) ^{\ast } \\
&=&-\frac{i}{\hbar }\left( \frac{mc^{2}}{L^{3}}\right) ^{\frac{1}{2}}\sum_{%
\vec{p}_{n}}\left( E_{p_{n}}\right) ^{-\frac{1}{2}}\vec{p}_{n}\left\{ a_{%
\vec{p}_{n}}^{\ast }e^{-\frac{i}{\hbar }\vec{p}_{n}\cdot \vec{x}}e^{\frac{i}{%
\hbar }E_{p_{n}}t}-b_{\vec{p}_{n}}e^{\frac{i}{\hbar }\vec{p}_{n}\cdot \vec{x}%
}e^{-\frac{i}{\hbar }E_{p_{n}}t}\right\} .
\end{eqnarray*}%
Finalmente, reemplazando los resultados anteriores (donde corresponda) en (%
\ref{Energia Total Campo K-G}), (\ref{Momentum Total Campo K-G}) y (\ref%
{Carga Total Campo K-G}), obtenemos:%
\begin{eqnarray*}
H &=&\frac{\hbar ^{2}}{2m}\int \left\{ \frac{1}{c^{2}}\dot{\psi}^{\ast }\dot{%
\psi}+\partial _{i}\psi ^{\ast }\partial _{i}\psi +\frac{m^{2}c^{2}}{\hbar
^{2}}\psi ^{\ast }\psi \right\} d^{3}x \\
&=&\frac{\hbar ^{2}}{2m}\int d^{3}x\left\{ 
\begin{array}{c}
\frac{m}{\hbar ^{2}L^{3}}\sum_{\vec{p}_{n},\vec{p}_{n}^{\prime }}\left(
E_{p_{n}}E_{p_{n}^{\prime }}\right) ^{\frac{1}{2}}\left[ 
\begin{array}{c}
\left( a_{\vec{p}_{n}}^{\ast }e^{-\frac{i}{\hbar }\vec{p}_{n}\cdot \vec{x}%
}e^{\frac{i}{\hbar }E_{p_{n}}t}-b_{\vec{p}_{n}}e^{\frac{i}{\hbar }\vec{p}%
_{n}\cdot \vec{x}}e^{-\frac{i}{\hbar }E_{p_{n}}t}\right)  \\ 
\times \left( a_{\vec{p}_{n}^{\prime }}e^{\frac{i}{\hbar }\vec{p}%
_{n}^{\prime }\cdot \vec{x}}e^{-\frac{i}{\hbar }E_{p_{n}^{\prime }}t}-b_{%
\vec{p}_{n}^{\prime }}^{\ast }e^{-\frac{i}{\hbar }\vec{p}_{n}^{\prime }\cdot 
\vec{x}}e^{\frac{i}{\hbar }E_{p_{n}^{\prime }}t}\right) 
\end{array}%
\right]  \\ 
+\frac{mc^{2}}{\hbar ^{2}L^{3}}\sum_{\vec{p}_{n},\vec{p}_{n}^{\prime
}}\left( E_{p_{n}}E_{p_{n}^{\prime }}\right) ^{-\frac{1}{2}}\left( \vec{p}%
_{n}\cdot \vec{p}_{n}^{\prime }\right) \left[ 
\begin{array}{c}
\left( a_{\vec{p}_{n}}^{\ast }e^{-\frac{i}{\hbar }\vec{p}_{n}\cdot \vec{x}%
}e^{\frac{i}{\hbar }E_{p_{n}}t}-b_{\vec{p}_{n}}e^{\frac{i}{\hbar }\vec{p}%
_{n}\cdot \vec{x}}e^{-\frac{i}{\hbar }E_{p_{n}}t}\right)  \\ 
\times \left( a_{\vec{p}_{n}^{\prime }}e^{\frac{i}{\hbar }\vec{p}%
_{n}^{\prime }\cdot \vec{x}}e^{-\frac{i}{\hbar }E_{p_{n}^{\prime }}t}-b_{%
\vec{p}_{n}^{\prime }}^{\ast }e^{-\frac{i}{\hbar }\vec{p}_{n}^{\prime }\cdot 
\vec{x}}e^{\frac{i}{\hbar }E_{p_{n}^{\prime }}t}\right) 
\end{array}%
\right]  \\ 
+\frac{m^{3}c^{4}}{\hbar ^{2}L^{3}}\sum_{\vec{p}_{n},\vec{p}_{n}^{\prime
}}\left( E_{p_{n}}E_{p_{n}^{\prime }}\right) ^{-\frac{1}{2}}\left[ 
\begin{array}{c}
\left( a_{\vec{p}_{n}}^{\ast }e^{-\frac{i}{\hbar }\vec{p}_{n}\cdot \vec{x}%
}e^{\frac{i}{\hbar }E_{p_{n}}t}+b_{\vec{p}_{n}}e^{\frac{i}{\hbar }\vec{p}%
_{n}\cdot \vec{x}}e^{-\frac{i}{\hbar }E_{p_{n}}t}\right)  \\ 
\times \left( a_{\vec{p}_{n}^{\prime }}e^{\frac{i}{\hbar }\vec{p}%
_{n}^{\prime }\cdot \vec{x}}e^{-\frac{i}{\hbar }E_{p_{n}^{\prime }}t}+b_{%
\vec{p}_{n}^{\prime }}^{\ast }e^{-\frac{i}{\hbar }\vec{p}_{n}^{\prime }\cdot 
\vec{x}}e^{\frac{i}{\hbar }E_{p_{n}^{\prime }}t}\right) 
\end{array}%
\right] 
\end{array}%
\right\}  \\
&=&\sum_{\vec{p}_{n},\vec{p}_{n}^{\prime }}\frac{\left(
E_{p_{n}}E_{p_{n}^{\prime }}\right) ^{-\frac{1}{2}}}{2L^{3}}\int
d^{3}x\left\{ 
\begin{array}{c}
E_{p_{n}}E_{p_{n}^{\prime }}\left( 
\begin{array}{c}
a_{\vec{p}_{n}}^{\ast }a_{\vec{p}_{n}^{\prime }}e^{-\frac{i}{\hbar }\left( 
\vec{p}_{n}-\vec{p}_{n}^{\prime }\right) \cdot \vec{x}}e^{\frac{i}{\hbar }%
\left( E_{p_{n}}-E_{p_{n}^{\prime }}\right) t}-a_{\vec{p}_{n}}^{\ast }b_{%
\vec{p}_{n}^{\prime }}^{\ast }e^{-\frac{i}{\hbar }\left( \vec{p}_{n}+\vec{p}%
_{n}^{\prime }\right) \cdot \vec{x}}e^{\frac{i}{\hbar }\left(
E_{p_{n}}+E_{p_{n}^{\prime }}\right) t} \\ 
-b_{\vec{p}_{n}}a_{\vec{p}_{n}^{\prime }}e^{\frac{i}{\hbar }\left( \vec{p}%
_{n}+\vec{p}_{n}^{\prime }\right) \cdot \vec{x}}e^{-\frac{i}{\hbar }\left(
E_{p_{n}}+E_{p_{n}^{\prime }}\right) t}+b_{\vec{p}_{n}}b_{\vec{p}%
_{n}^{\prime }}^{\ast }e^{\frac{i}{\hbar }\left( \vec{p}_{n}-\vec{p}%
_{n}^{\prime }\right) \cdot \vec{x}}e^{-\frac{i}{\hbar }\left(
E_{p_{n}}-E_{p_{n}^{\prime }}\right) t}%
\end{array}%
\right)  \\ 
+\frac{\vec{p}_{n}\cdot \vec{p}_{n}^{\prime }}{c^{-2}}\left( 
\begin{array}{c}
a_{\vec{p}_{n}}^{\ast }a_{\vec{p}_{n}^{\prime }}e^{-\frac{i}{\hbar }\left( 
\vec{p}_{n}-\vec{p}_{n}^{\prime }\right) \cdot \vec{x}}e^{\frac{i}{\hbar }%
\left( E_{p_{n}}-E_{p_{n}^{\prime }}\right) t}-a_{\vec{p}_{n}}^{\ast }b_{%
\vec{p}_{n}^{\prime }}^{\ast }e^{-\frac{i}{\hbar }\left( \vec{p}_{n}+\vec{p}%
_{n}^{\prime }\right) \cdot \vec{x}}e^{\frac{i}{\hbar }\left(
E_{p_{n}}+E_{p_{n}^{\prime }}\right) t} \\ 
-b_{\vec{p}_{n}}a_{\vec{p}_{n}^{\prime }}e^{\frac{i}{\hbar }\left( \vec{p}%
_{n}+\vec{p}_{n}^{\prime }\right) \cdot \vec{x}}e^{-\frac{i}{\hbar }\left(
E_{p_{n}}+E_{p_{n}^{\prime }}\right) t}+b_{\vec{p}_{n}}b_{\vec{p}%
_{n}^{\prime }}^{\ast }e^{\frac{i}{\hbar }\left( \vec{p}_{n}-\vec{p}%
_{n}^{\prime }\right) \cdot \vec{x}}e^{-\frac{i}{\hbar }\left(
E_{p_{n}}-E_{p_{n}^{\prime }}\right) t}%
\end{array}%
\right)  \\ 
+m^{2}c^{4}\left( 
\begin{array}{c}
a_{\vec{p}_{n}}^{\ast }a_{\vec{p}_{n}^{\prime }}e^{-\frac{i}{\hbar }\left( 
\vec{p}_{n}-\vec{p}_{n}^{\prime }\right) \cdot \vec{x}}e^{\frac{i}{\hbar }%
\left( E_{p_{n}}-E_{p_{n}^{\prime }}\right) t}+a_{\vec{p}_{n}}^{\ast }b_{%
\vec{p}_{n}^{\prime }}^{\ast }e^{-\frac{i}{\hbar }\left( \vec{p}_{n}+\vec{p}%
_{n}^{\prime }\right) \cdot \vec{x}}e^{\frac{i}{\hbar }\left(
E_{p_{n}}+E_{p_{n}^{\prime }}\right) t} \\ 
+b_{\vec{p}_{n}}a_{\vec{p}_{n}^{\prime }}e^{\frac{i}{\hbar }\left( \vec{p}%
_{n}+\vec{p}_{n}^{\prime }\right) \cdot \vec{x}}e^{-\frac{i}{\hbar }\left(
E_{p_{n}}+E_{p_{n}^{\prime }}\right) t}+b_{\vec{p}_{n}}b_{\vec{p}%
_{n}^{\prime }}^{\ast }e^{\frac{i}{\hbar }\left( \vec{p}_{n}-\vec{p}%
_{n}^{\prime }\right) \cdot \vec{x}}e^{-\frac{i}{\hbar }\left(
E_{p_{n}}-E_{p_{n}^{\prime }}\right) t}%
\end{array}%
\right) 
\end{array}%
\right\}  \\
&=&\frac{1}{2L^{3}}\sum_{\vec{p}_{n},\vec{p}_{n}^{\prime }}\left(
E_{p_{n}}E_{p_{n}^{\prime }}\right) ^{-\frac{1}{2}}\int d^{3}x\left\{ 
\begin{array}{c}
\left[ E_{p_{n}}E_{p_{n}^{\prime }}+c^{2}\left( \vec{p}_{n}\cdot \vec{p}%
_{n}^{\prime }\right) +m^{2}c^{4}\right] a_{\vec{p}_{n}}^{\ast }a_{\vec{p}%
_{n}^{\prime }}e^{\frac{i}{\hbar }\left( E_{p_{n}}-E_{p_{n}^{\prime
}}\right) t}e^{-\frac{i}{\hbar }\left( \vec{p}_{n}-\vec{p}_{n}^{\prime
}\right) \cdot \vec{x}} \\ 
-\left[ E_{p_{n}}E_{p_{n}^{\prime }}+c^{2}\left( \vec{p}_{n}\cdot \vec{p}%
_{n}^{\prime }\right) -m^{2}c^{4}\right] a_{\vec{p}_{n}}^{\ast }b_{\vec{p}%
_{n}^{\prime }}^{\ast }e^{\frac{i}{\hbar }\left( E_{p_{n}}+E_{p_{n}^{\prime
}}\right) t}e^{-\frac{i}{\hbar }\left( \vec{p}_{n}+\vec{p}_{n}^{\prime
}\right) \cdot \vec{x}} \\ 
-\left[ E_{p_{n}}E_{p_{n}^{\prime }}+c^{2}\left( \vec{p}_{n}\cdot \vec{p}%
_{n}^{\prime }\right) -m^{2}c^{4}\right] b_{\vec{p}_{n}}a_{\vec{p}%
_{n}^{\prime }}e^{-\frac{i}{\hbar }\left( E_{p_{n}}+E_{p_{n}^{\prime
}}\right) t}e^{\frac{i}{\hbar }\left( \vec{p}_{n}+\vec{p}_{n}^{\prime
}\right) \cdot \vec{x}} \\ 
+\left[ E_{p_{n}}E_{p_{n}^{\prime }}+c^{2}\left( \vec{p}_{n}\cdot \vec{p}%
_{n}^{\prime }\right) +m^{2}c^{4}\right] b_{\vec{p}_{n}}b_{\vec{p}%
_{n}^{\prime }}^{\ast }e^{-\frac{i}{\hbar }\left( E_{p_{n}}-E_{p_{n}^{\prime
}}\right) t}e^{\frac{i}{\hbar }\left( \vec{p}_{n}-\vec{p}_{n}^{\prime
}\right) \cdot \vec{x}}%
\end{array}%
\right\}  \\
&=&\frac{1}{2}\sum_{\vec{p}_{n},\vec{p}_{n}^{\prime }}\left(
E_{p_{n}}E_{p_{n}^{\prime }}\right) ^{-\frac{1}{2}}\left\{ 
\begin{array}{c}
\left[ E_{p_{n}}E_{p_{n}^{\prime }}+c^{2}\left( \vec{p}_{n}\cdot \vec{p}%
_{n}^{\prime }\right) +m^{2}c^{4}\right] a_{\vec{p}_{n}}^{\ast }a_{\vec{p}%
_{n}^{\prime }}e^{\frac{i}{\hbar }\left( E_{p_{n}}-E_{p_{n}^{\prime
}}\right) t}\delta _{\vec{p}_{n},\vec{p}_{n}^{\prime }} \\ 
-\left[ E_{p_{n}}E_{p_{n}^{\prime }}+c^{2}\left( \vec{p}_{n}\cdot \vec{p}%
_{n}^{\prime }\right) -m^{2}c^{4}\right] a_{\vec{p}_{n}}^{\ast }b_{\vec{p}%
_{n}^{\prime }}^{\ast }e^{\frac{i}{\hbar }\left( E_{p_{n}}+E_{p_{n}^{\prime
}}\right) t}\delta _{\vec{p}_{n},-\vec{p}_{n}^{\prime }} \\ 
-\left[ E_{p_{n}}E_{p_{n}^{\prime }}+c^{2}\left( \vec{p}_{n}\cdot \vec{p}%
_{n}^{\prime }\right) -m^{2}c^{4}\right] b_{\vec{p}_{n}}a_{\vec{p}%
_{n}^{\prime }}e^{-\frac{i}{\hbar }\left( E_{p_{n}}+E_{p_{n}^{\prime
}}\right) t}\delta _{\vec{p}_{n},-\vec{p}_{n}^{\prime }} \\ 
+\left[ E_{p_{n}}E_{p_{n}^{\prime }}+c^{2}\left( \vec{p}_{n}\cdot \vec{p}%
_{n}^{\prime }\right) +m^{2}c^{4}\right] b_{\vec{p}_{n}}b_{\vec{p}%
_{n}^{\prime }}^{\ast }e^{-\frac{i}{\hbar }\left( E_{p_{n}}-E_{p_{n}^{\prime
}}\right) t}\delta _{\vec{p}_{n},\vec{p}_{n}^{\prime }}%
\end{array}%
\right\}  \\
&=&\frac{1}{2}\sum_{\vec{p}_{n},\vec{p}_{n}^{\prime }}\left\{ 
\begin{array}{c}
\left( E_{p_{n}}E_{p_{n}}\right) ^{-\frac{1}{2}}\left[
E_{p_{n}}E_{p_{n}}+c^{2}\left( \vec{p}_{n}\cdot \vec{p}_{n}\right)
+m^{2}c^{4}\right] a_{\vec{p}_{n}}^{\ast }a_{\vec{p}_{n}}e^{\frac{i}{\hbar }%
\left( E_{p_{n}}-E_{p_{n}}\right) t} \\ 
-\left( E_{p_{n}}E_{-p_{n}}\right) ^{-\frac{1}{2}}\left[
E_{p_{n}}E_{-p_{n}}+c^{2}\left( \vec{p}_{n}\cdot \left( -\vec{p}_{n}\right)
\right) -m^{2}c^{4}\right] a_{\vec{p}_{n}}^{\ast }b_{-\vec{p}_{n}}^{\ast }e^{%
\frac{i}{\hbar }\left( E_{p_{n}}+E_{-p_{n}}\right) t} \\ 
-\left( E_{p_{n}}E_{-p_{n}}\right) ^{-\frac{1}{2}}\left[
E_{p_{n}}E_{-p_{n}}+c^{2}\left( \vec{p}_{n}\cdot \left( -\vec{p}_{n}\right)
\right) -m^{2}c^{4}\right] b_{\vec{p}_{n}}a_{-\vec{p}_{n}}e^{-\frac{i}{\hbar 
}\left( E_{p_{n}}+E_{-p_{n}}\right) t} \\ 
+\left( E_{p_{n}}E_{p_{n}}\right) ^{-\frac{1}{2}}\left[
E_{p_{n}}E_{p_{n}}+c^{2}\left( \vec{p}_{n}\cdot \vec{p}_{n}\right)
+m^{2}c^{4}\right] b_{\vec{p}_{n}}b_{\vec{p}_{n}}^{\ast }e^{-\frac{i}{\hbar }%
\left( E_{p_{n}}-E_{p_{n}}\right) t}%
\end{array}%
\right\}  \\
&=&\sum_{\vec{p}_{n}}\frac{1}{2E_{p_{n}}}\left\{ 
\begin{array}{c}
\left( E_{p_{n}}^{2}+c^{2}\vec{p}_{n}^{2}+m^{2}c^{4}\right) \left( a_{\vec{p}%
_{n}}^{\ast }a_{\vec{p}_{n}}+b_{\vec{p}_{n}}b_{\vec{p}_{n}}^{\ast }\right) 
\\ 
-\left( E_{p_{n}}^{2}-c^{2}\vec{p}_{n}^{2}-m^{2}c^{4}\right) \left( a_{\vec{p%
}_{n}}^{\ast }b_{-\vec{p}_{n}}^{\ast }e^{2\frac{i}{\hbar }E_{p_{n}}t}+b_{%
\vec{p}_{n}}a_{-\vec{p}_{n}}e^{-2\frac{i}{\hbar }E_{p_{n}}t}\right) 
\end{array}%
\right\} 
\end{eqnarray*}%
pero, de (\ref{Epn}) vemos que:%
\begin{equation}
c^{2}\vec{p}_{n}^{2}+m^{2}c^{4}=E_{p_{n}}^{2},
\end{equation}%
por lo tanto, la energ'ia total del campo es\footnote{%
Notar que si $\left\{ a_{\vec{p}_{n}}^{\ast },a_{\vec{p}_{n}},b_{\vec{p}%
_{n}},b_{\vec{p}_{n}}^{\ast }\right\} $ fueran operadores que satisfacen las
reglas de conmutaci'on:%
\begin{eqnarray*}
\left[ \hat{a}_{\vec{p}_{n}},\hat{a}_{\vec{p}_{n}}^{\dagger }\right]  &=&1,
\\
\left[ \hat{a}_{\vec{p}_{n}},\hat{a}_{\vec{p}_{n}}\right]  &=&\left[ \hat{a}%
_{\vec{p}_{n}}^{\dagger },\hat{a}_{\vec{p}_{n}}^{\dagger }\right] =0, \\
\left[ \hat{b}_{\vec{p}_{n}},\hat{b}_{\vec{p}_{n}}^{\dagger }\right]  &=&1,
\\
\left[ \hat{b}_{\vec{p}_{n}},\hat{b}_{\vec{p}_{n}}\right]  &=&\left[ \hat{b}%
_{\vec{p}_{n}}^{\dagger },\hat{b}_{\vec{p}_{n}}^{\dagger }\right] =0, \\
\left[ \hat{a}_{\vec{p}_{n}},\hat{b}_{\vec{p}_{n}^{\prime }}\right]  &=&%
\left[ \hat{a}_{\vec{p}_{n}}^{\dagger },\hat{b}_{\vec{p}_{n}^{\prime
}}^{\dagger }\right] =0,
\end{eqnarray*}%
entonces la energ'ia total ser'ia el Operador Hamiltoniano que tiene
la forma:%
\begin{eqnarray*}
\hat{H} &=&\sum_{\vec{p}_{n}}\left( \hat{a}_{\vec{p}_{n}}^{\dagger }\hat{a}_{%
\vec{p}_{n}}+\hat{b}_{\vec{p}_{n}}\hat{b}_{\vec{p}_{n}}^{\dagger }\right)  \\
&=&\sum_{\vec{p}_{n}}E_{p_{n}}\left( \hat{a}_{\vec{p}_{n}}^{\dagger }\hat{a}%
_{\vec{p}_{n}}+\hat{b}_{\vec{p}_{n}}^{\dagger }\hat{b}_{\vec{p}_{n}}\right)
+\sum_{\vec{p}_{n}}E_{p_{n}},
\end{eqnarray*}%
donde el 'ultimo t'ermino del lado derecho corresponder'ia a la
energ'ia del \textit{vac'io.}}:%
\begin{eqnarray*}
H &=&\sum_{\vec{p}_{n}}\frac{2E_{p_{n}}^{2}}{2E_{p_{n}}}\left( a_{\vec{p}%
_{n}}^{\ast }a_{\vec{p}_{n}}+b_{\vec{p}_{n}}b_{\vec{p}_{n}}^{\ast }\right) 
\\
&=&\sum_{\vec{p}_{n}}E_{p_{n}}\left( a_{\vec{p}_{n}}^{\ast }a_{\vec{p}%
_{n}}+b_{\vec{p}_{n}}^{\ast }b_{\vec{p}_{n}}\right) .
\end{eqnarray*}

An'alogamente, para el momentum total del campo encontramos que\footnote{%
Nuevamente, considerando los coeficientes de la expansi'on en serie como
operadores, vamoa a encontrar que el operador momentum del campo es:%
\begin{eqnarray*}
\overrightarrow{\hat{p}} &=&\sum_{\vec{p}_{n}}\vec{p}_{n}\left\{ \hat{a}_{%
\vec{p}_{n}}^{\dagger }\hat{a}_{\vec{p}_{n}}+\hat{b}_{\vec{p}_{n}}^{\dagger }%
\hat{b}_{\vec{p}_{n}}\right\} +\sum_{\vec{p}_{n}}\vec{p}_{n} \\
&=&\sum_{\vec{p}_{n}}\vec{p}_{n}\left\{ \hat{a}_{\vec{p}_{n}}^{\dagger }\hat{%
a}_{\vec{p}_{n}}+\hat{b}_{\vec{p}_{n}}^{\dagger }\hat{b}_{\vec{p}%
_{n}}\right\} ,
\end{eqnarray*}%
ya que el 'ultimo t'ermino del lado derecho se anula debido a que $%
\vec{p}_{n}$ es impar.}:%
\begin{eqnarray*}
\vec{p} &=&-\frac{\hbar ^{2}}{2mc^{2}}\int \left\{ \left( \partial _{i}\psi
\right) \dot{\psi}^{\ast }+\left( \partial _{i}\psi ^{\ast }\right) \dot{\psi%
}\right\} d^{3}x \\
&=&-\frac{\hbar ^{2}}{2mc^{2}}\int d^{3}x\left\{ 
\begin{array}{c}
-\frac{mc^{2}}{\hbar ^{2}L^{3}}\sum_{\vec{p}_{n},\vec{p}_{n}^{\prime
}}\left( \frac{E_{p_{n}^{\prime }}}{E_{p_{n}}}\right) ^{\frac{1}{2}}\vec{p}%
_{n}\left[ 
\begin{array}{c}
\left( a_{\vec{p}_{n}}e^{\frac{i}{\hbar }\vec{p}_{n}\cdot \vec{x}}e^{-\frac{i%
}{\hbar }E_{p_{n}}t}-b_{\vec{p}_{n}}^{\ast }e^{-\frac{i}{\hbar }\vec{p}%
_{n}\cdot \vec{x}}e^{\frac{i}{\hbar }E_{p_{n}}t}\right)  \\ 
\times \left( a_{\vec{p}_{n}^{\prime }}^{\ast }e^{-\frac{i}{\hbar }\vec{p}%
_{n}^{\prime }\cdot \vec{x}}e^{\frac{i}{\hbar }E_{p_{n}^{\prime }}t}-b_{\vec{%
p}_{n}^{\prime }}e^{\frac{i}{\hbar }\vec{p}_{n}^{\prime }\cdot \vec{x}}e^{-%
\frac{i}{\hbar }E_{p_{n}^{\prime }}t}\right) 
\end{array}%
\right]  \\ 
-\frac{mc^{2}}{\hbar ^{2}L^{3}}\sum_{\vec{p}_{n},\vec{p}_{n}^{\prime
}}\left( \frac{E_{p_{n}}}{E_{p_{n}^{\prime }}}\right) ^{\frac{1}{2}}\vec{p}%
_{n}^{\prime }\left[ 
\begin{array}{c}
\left( a_{\vec{p}_{n}^{\prime }}^{\ast }e^{-\frac{i}{\hbar }\vec{p}%
_{n}^{\prime }\cdot \vec{x}}e^{\frac{i}{\hbar }E_{p_{n}^{\prime }}t}-b_{\vec{%
p}_{n}^{\prime }}e^{\frac{i}{\hbar }\vec{p}_{n}^{\prime }\cdot \vec{x}}e^{-%
\frac{i}{\hbar }E_{p_{n}^{\prime }}t}\right)  \\ 
\times \left( a_{\vec{p}_{n}}e^{\frac{i}{\hbar }\vec{p}_{n}\cdot \vec{x}}e^{-%
\frac{i}{\hbar }E_{p_{n}}t}-b_{\vec{p}_{n}}^{\ast }e^{-\frac{i}{\hbar }\vec{p%
}_{n}\cdot \vec{x}}e^{\frac{i}{\hbar }E_{p_{n}}t}\right) 
\end{array}%
\right] 
\end{array}%
\right\}  \\
&=&\frac{1}{2L^{3}}\sum_{\vec{p}_{n},\vec{p}_{n}^{\prime }}\int
d^{3}x\left\{ 
\begin{array}{c}
\left( \frac{E_{p_{n}^{\prime }}}{E_{p_{n}}}\right) ^{\frac{1}{2}}\vec{p}%
_{n}\left( 
\begin{array}{c}
a_{\vec{p}_{n}}a_{\vec{p}_{n}^{\prime }}^{\ast }e^{\frac{i}{\hbar }\left( 
\vec{p}_{n}-\vec{p}_{n}^{\prime }\right) \cdot \vec{x}}e^{-\frac{i}{\hbar }%
\left( E_{p_{n}}-E_{p_{n}^{\prime }}\right) t}-a_{\vec{p}_{n}}b_{\vec{p}%
_{n}^{\prime }}e^{\frac{i}{\hbar }\left( \vec{p}_{n}+\vec{p}_{n}^{\prime
}\right) \cdot \vec{x}}e^{-\frac{i}{\hbar }\left( E_{p_{n}}+E_{p_{n}^{\prime
}}\right) t} \\ 
-b_{\vec{p}_{n}}^{\ast }a_{\vec{p}_{n}^{\prime }}^{\ast }e^{-\frac{i}{\hbar }%
\left( \vec{p}_{n}+\vec{p}_{n}^{\prime }\right) \cdot \vec{x}}e^{\frac{i}{%
\hbar }\left( E_{p_{n}}+E_{p_{n}^{\prime }}\right) t}+b_{\vec{p}_{n}}^{\ast
}b_{\vec{p}_{n}^{\prime }}e^{-\frac{i}{\hbar }\left( \vec{p}_{n}-\vec{p}%
_{n}^{\prime }\right) \cdot \vec{x}}e^{\frac{i}{\hbar }\left(
E_{p_{n}}-E_{p_{n}^{\prime }}\right) t}%
\end{array}%
\right)  \\ 
+\left( \frac{E_{p_{n}}}{E_{p_{n}^{\prime }}}\right) ^{\frac{1}{2}}\vec{p}%
_{n}^{\prime }\left( 
\begin{array}{c}
a_{\vec{p}_{n}^{\prime }}^{\ast }a_{\vec{p}_{n}}e^{\frac{i}{\hbar }\left( 
\vec{p}_{n}-\vec{p}_{n}^{\prime }\right) \cdot \vec{x}}e^{-\frac{i}{\hbar }%
\left( E_{p_{n}}-E_{p_{n}^{\prime }}\right) t}-a_{\vec{p}_{n}^{\prime
}}^{\ast }b_{\vec{p}_{n}}^{\ast }e^{-\frac{i}{\hbar }\left( \vec{p}_{n}+\vec{%
p}_{n}^{\prime }\right) \cdot \vec{x}}e^{\frac{i}{\hbar }\left(
E_{p_{n}}+E_{p_{n}^{\prime }}\right) t} \\ 
-b_{\vec{p}_{n}^{\prime }}a_{\vec{p}_{n}}e^{\frac{i}{\hbar }\left( \vec{p}%
_{n}+\vec{p}_{n}^{\prime }\right) \cdot \vec{x}}e^{-\frac{i}{\hbar }\left(
E_{p_{n}}+E_{p_{n}^{\prime }}\right) t}+b_{\vec{p}_{n}^{\prime }}b_{\vec{p}%
_{n}}^{\ast }e^{-\frac{i}{\hbar }\left( \vec{p}_{n}-\vec{p}_{n}^{\prime
}\right) \cdot \vec{x}}e^{\frac{i}{\hbar }\left( E_{p_{n}}-E_{p_{n}^{\prime
}}\right) t}%
\end{array}%
\right) 
\end{array}%
\right\}  \\
&=&\frac{1}{2L^{3}}\sum_{\vec{p}_{n},\vec{p}_{n}^{\prime }}\int
d^{3}x\left\{ 
\begin{array}{c}
\left[ \left( \frac{E_{p_{n}^{\prime }}}{E_{p_{n}}}\right) ^{\frac{1}{2}}%
\vec{p}_{n}a_{\vec{p}_{n}}a_{\vec{p}_{n}^{\prime }}^{\ast }+\left( \frac{%
E_{p_{n}}}{E_{p_{n}^{\prime }}}\right) ^{\frac{1}{2}}\vec{p}_{n}^{\prime }a_{%
\vec{p}_{n}^{\prime }}^{\ast }a_{\vec{p}_{n}}\right] e^{-\frac{i}{\hbar }%
\left( E_{p_{n}}-E_{p_{n}^{\prime }}\right) t}e^{\frac{i}{\hbar }\left( \vec{%
p}_{n}-\vec{p}_{n}^{\prime }\right) \cdot \vec{x}} \\ 
-\left[ \left( \frac{E_{p_{n}^{\prime }}}{E_{p_{n}}}\right) ^{\frac{1}{2}}%
\vec{p}_{n}a_{\vec{p}_{n}}b_{\vec{p}_{n}^{\prime }}+\left( \frac{E_{p_{n}}}{%
E_{p_{n}^{\prime }}}\right) ^{\frac{1}{2}}\vec{p}_{n}^{\prime }b_{\vec{p}%
_{n}^{\prime }}a_{\vec{p}_{n}}\right] e^{-\frac{i}{\hbar }\left(
E_{p_{n}}+E_{p_{n}^{\prime }}\right) t}e^{\frac{i}{\hbar }\left( \vec{p}_{n}+%
\vec{p}_{n}^{\prime }\right) \cdot \vec{x}} \\ 
-\left[ \left( \frac{E_{p_{n}^{\prime }}}{E_{p_{n}}}\right) ^{\frac{1}{2}}%
\vec{p}_{n}b_{\vec{p}_{n}}^{\ast }a_{\vec{p}_{n}^{\prime }}^{\ast }+\left( 
\frac{E_{p_{n}}}{E_{p_{n}^{\prime }}}\right) ^{\frac{1}{2}}\vec{p}%
_{n}^{\prime }a_{\vec{p}_{n}^{\prime }}^{\ast }b_{\vec{p}_{n}}^{\ast }\right]
e^{\frac{i}{\hbar }\left( E_{p_{n}}+E_{p_{n}^{\prime }}\right) t}e^{-\frac{i%
}{\hbar }\left( \vec{p}_{n}+\vec{p}_{n}^{\prime }\right) \cdot \vec{x}} \\ 
+\left[ \left( \frac{E_{p_{n}^{\prime }}}{E_{p_{n}}}\right) ^{\frac{1}{2}}%
\vec{p}_{n}b_{\vec{p}_{n}}^{\ast }b_{\vec{p}_{n}^{\prime }}+\left( \frac{%
E_{p_{n}}}{E_{p_{n}^{\prime }}}\right) ^{\frac{1}{2}}\vec{p}_{n}^{\prime }b_{%
\vec{p}_{n}^{\prime }}b_{\vec{p}_{n}}^{\ast }\right] e^{\frac{i}{\hbar }%
\left( E_{p_{n}}-E_{p_{n}^{\prime }}\right) t}e^{-\frac{i}{\hbar }\left( 
\vec{p}_{n}-\vec{p}_{n}^{\prime }\right) \cdot \vec{x}}%
\end{array}%
\right\}  \\
&=&\frac{1}{2L^{3}}\sum_{\vec{p}_{n},\vec{p}_{n}^{\prime }}\left\{ 
\begin{array}{c}
\left[ \left( \frac{E_{p_{n}^{\prime }}}{E_{p_{n}}}\right) ^{\frac{1}{2}}%
\vec{p}_{n}a_{\vec{p}_{n}}a_{\vec{p}_{n}^{\prime }}^{\ast }+\left( \frac{%
E_{p_{n}}}{E_{p_{n}^{\prime }}}\right) ^{\frac{1}{2}}\vec{p}_{n}^{\prime }a_{%
\vec{p}_{n}^{\prime }}^{\ast }a_{\vec{p}_{n}}\right] e^{-\frac{i}{\hbar }%
\left( E_{p_{n}}-E_{p_{n}^{\prime }}\right) t}L^{3}\delta _{\vec{p}_{n},\vec{%
p}_{n}^{\prime }} \\ 
-\left[ \left( \frac{E_{p_{n}^{\prime }}}{E_{p_{n}}}\right) ^{\frac{1}{2}}%
\vec{p}_{n}a_{\vec{p}_{n}}b_{\vec{p}_{n}^{\prime }}+\left( \frac{E_{p_{n}}}{%
E_{p_{n}^{\prime }}}\right) ^{\frac{1}{2}}\vec{p}_{n}^{\prime }b_{\vec{p}%
_{n}^{\prime }}a_{\vec{p}_{n}}\right] e^{-\frac{i}{\hbar }\left(
E_{p_{n}}+E_{p_{n}^{\prime }}\right) t}L^{3}\delta _{\vec{p}_{n},-\vec{p}%
_{n}^{\prime }} \\ 
-\left[ \left( \frac{E_{p_{n}^{\prime }}}{E_{p_{n}}}\right) ^{\frac{1}{2}}%
\vec{p}_{n}b_{\vec{p}_{n}}^{\ast }a_{\vec{p}_{n}^{\prime }}^{\ast }+\left( 
\frac{E_{p_{n}}}{E_{p_{n}^{\prime }}}\right) ^{\frac{1}{2}}\vec{p}%
_{n}^{\prime }a_{\vec{p}_{n}^{\prime }}^{\ast }b_{\vec{p}_{n}}^{\ast }\right]
e^{\frac{i}{\hbar }\left( E_{p_{n}}+E_{p_{n}^{\prime }}\right) t}L^{3}\delta
_{\vec{p}_{n},-\vec{p}_{n}^{\prime }} \\ 
+\left[ \left( \frac{E_{p_{n}^{\prime }}}{E_{p_{n}}}\right) ^{\frac{1}{2}}%
\vec{p}_{n}b_{\vec{p}_{n}}^{\ast }b_{\vec{p}_{n}^{\prime }}+\left( \frac{%
E_{p_{n}}}{E_{p_{n}^{\prime }}}\right) ^{\frac{1}{2}}\vec{p}_{n}^{\prime }b_{%
\vec{p}_{n}^{\prime }}b_{\vec{p}_{n}}^{\ast }\right] e^{\frac{i}{\hbar }%
\left( E_{p_{n}}-E_{p_{n}^{\prime }}\right) t}L^{3}\delta _{\vec{p}_{n},\vec{%
p}_{n}^{\prime }}%
\end{array}%
\right\}  \\
&=&\frac{1}{2}\sum_{\vec{p}_{n}}\left\{ 
\begin{array}{c}
\left[ \left( \frac{E_{p_{n}}}{E_{p_{n}}}\right) ^{\frac{1}{2}}\vec{p}_{n}a_{%
\vec{p}_{n}}a_{\vec{p}_{n}}^{\ast }+\left( \frac{E_{p_{n}}}{E_{p_{n}}}%
\right) ^{\frac{1}{2}}\vec{p}_{n}a_{\vec{p}_{n}}^{\ast }a_{\vec{p}_{n}}%
\right] e^{-\frac{i}{\hbar }\left( E_{p_{n}}-E_{p_{n}}\right) t} \\ 
-\left[ \left( \frac{E_{-p_{n}}}{E_{p_{n}}}\right) ^{\frac{1}{2}}\vec{p}%
_{n}a_{\vec{p}_{n}}b_{-\vec{p}_{n}}+\left( \frac{E_{p_{n}}}{E_{-p_{n}}}%
\right) ^{\frac{1}{2}}\left( -\vec{p}_{n}\right) b_{-\vec{p}_{n}}a_{\vec{p}%
_{n}}\right] e^{-\frac{i}{\hbar }\left( E_{p_{n}}+E_{-p_{n}}\right) t} \\ 
-\left[ \left( \frac{E_{-p_{n}}}{E_{p_{n}}}\right) ^{\frac{1}{2}}\vec{p}%
_{n}b_{\vec{p}_{n}}^{\ast }a_{-\vec{p}_{n}}^{\ast }+\left( \frac{E_{p_{n}}}{%
E_{-p_{n}}}\right) ^{\frac{1}{2}}\left( -\vec{p}_{n}\right) a_{-\vec{p}%
_{n}}^{\ast }b_{\vec{p}_{n}}^{\ast }\right] e^{\frac{i}{\hbar }\left(
E_{p_{n}}+E_{-p_{n}}\right) t} \\ 
+\left[ \left( \frac{E_{p_{n}}}{E_{p_{n}}}\right) ^{\frac{1}{2}}\vec{p}%
_{n}b_{\vec{p}_{n}}^{\ast }b_{\vec{p}_{n}}+\left( \frac{E_{p_{n}}}{E_{p_{n}}}%
\right) ^{\frac{1}{2}}\vec{p}_{n}b_{\vec{p}_{n}}b_{\vec{p}_{n}}^{\ast }%
\right] e^{\frac{i}{\hbar }\left( E_{p_{n}}-E_{p_{n}}\right) t}%
\end{array}%
\right\}  \\
&=&\frac{1}{2}\sum_{\vec{p}_{n}}\vec{p}_{n}\left\{ a_{\vec{p}_{n}}a_{\vec{p}%
_{n}}^{\ast }+a_{\vec{p}_{n}}^{\ast }a_{\vec{p}_{n}}-\left[ a_{\vec{p}%
_{n}}b_{-\vec{p}_{n}}-b_{-\vec{p}_{n}}a_{\vec{p}_{n}}\right] e^{-2\frac{i}{%
\hbar }E_{p_{n}}t}-\left[ b_{\vec{p}_{n}}^{\ast }a_{-\vec{p}_{n}}^{\ast
}-a_{-\vec{p}_{n}}^{\ast }b_{\vec{p}_{n}}^{\ast }\right] e^{2\frac{i}{\hbar }%
E_{p_{n}}t}+b_{\vec{p}_{n}}^{\ast }b_{\vec{p}_{n}}+b_{\vec{p}_{n}}b_{\vec{p}%
_{n}}^{\ast }\right\}  \\
&=&\sum_{\vec{p}_{n}}\vec{p}_{n}\left\{ a_{\vec{p}_{n}}^{\ast }a_{\vec{p}%
_{n}}+b_{\vec{p}_{n}}^{\ast }b_{\vec{p}_{n}}\right\} .
\end{eqnarray*}

Y para la carga total:%
\begin{eqnarray*}
Q &=&i\frac{\hbar e}{2mc^{2}}\int \left\{ \psi ^{\ast }\dot{\psi}-\psi \dot{%
\psi}^{\ast }\right\} d^{3}x \\
&=&i\frac{\hbar e}{2mc^{2}}\int d^{3}x\left\{ 
\begin{array}{c}
-i\frac{mc^{2}}{\hbar L^{3}}\sum_{\vec{p}_{n},\vec{p}_{n}^{\prime }}\left( 
\frac{E_{p_{n}^{\prime }}}{E_{p_{n}}}\right) ^{\frac{1}{2}}\left[ 
\begin{array}{c}
\left( a_{\vec{p}_{n}}^{\ast }e^{-\frac{i}{\hbar }\vec{p}_{n}\cdot \vec{x}%
}e^{\frac{i}{\hbar }E_{p_{n}}t}+b_{\vec{p}_{n}}e^{\frac{i}{\hbar }\vec{p}%
_{n}\cdot \vec{x}}e^{-\frac{i}{\hbar }E_{p_{n}}t}\right) \\ 
\times \left( a_{\vec{p}_{n}^{\prime }}e^{\frac{i}{\hbar }\vec{p}%
_{n}^{\prime }\cdot \vec{x}}e^{-\frac{i}{\hbar }E_{p_{n}^{\prime }}t}-b_{%
\vec{p}_{n}^{\prime }}^{\ast }e^{-\frac{i}{\hbar }\vec{p}_{n}^{\prime }\cdot 
\vec{x}}e^{\frac{i}{\hbar }E_{p_{n}^{\prime }}t}\right)%
\end{array}%
\right] \\ 
-i\frac{mc^{2}}{\hbar L^{3}}\sum_{\vec{p}_{n},\vec{p}_{n}^{\prime }}\left( 
\frac{E_{p_{n}}}{E_{p_{n}^{\prime }}}\right) ^{\frac{1}{2}}\left[ 
\begin{array}{c}
\left( a_{\vec{p}_{n}^{\prime }}e^{\frac{i}{\hbar }\vec{p}_{n}^{\prime
}\cdot \vec{x}}e^{-\frac{i}{\hbar }E_{p_{n}^{\prime }}t}+b_{\vec{p}%
_{n}^{\prime }}^{\ast }e^{-\frac{i}{\hbar }\vec{p}_{n}^{\prime }\cdot \vec{x}%
}e^{\frac{i}{\hbar }E_{p_{n}^{\prime }}t}\right) \\ 
\times \left( a_{\vec{p}_{n}}^{\ast }e^{-\frac{i}{\hbar }\vec{p}_{n}\cdot 
\vec{x}}e^{\frac{i}{\hbar }E_{p_{n}}t}-b_{\vec{p}_{n}}e^{\frac{i}{\hbar }%
\vec{p}_{n}\cdot \vec{x}}e^{-\frac{i}{\hbar }E_{p_{n}}t}\right)%
\end{array}%
\right]%
\end{array}%
\right\} \\
&=&\frac{e}{2L^{3}}\sum_{\vec{p}_{n},\vec{p}_{n}^{\prime }}\int
d^{3}x\left\{ 
\begin{array}{c}
\left( \frac{E_{p_{n}^{\prime }}}{E_{p_{n}}}\right) ^{\frac{1}{2}}\left( 
\begin{array}{c}
a_{\vec{p}_{n}}^{\ast }a_{\vec{p}_{n}^{\prime }}e^{-\frac{i}{\hbar }\left( 
\vec{p}_{n}-\vec{p}_{n}^{\prime }\right) \cdot \vec{x}}e^{\frac{i}{\hbar }%
\left( E_{p_{n}}-E_{p_{n}^{\prime }}\right) t}-a_{\vec{p}_{n}}^{\ast }b_{%
\vec{p}_{n}^{\prime }}^{\ast }e^{-\frac{i}{\hbar }\left( \vec{p}_{n}+\vec{p}%
_{n}^{\prime }\right) \cdot \vec{x}}e^{\frac{i}{\hbar }\left(
E_{p_{n}}+E_{p_{n}^{\prime }}\right) t} \\ 
+b_{\vec{p}_{n}}a_{\vec{p}_{n}^{\prime }}e^{\frac{i}{\hbar }\left( \vec{p}%
_{n}+\vec{p}_{n}^{\prime }\right) \cdot \vec{x}}e^{-\frac{i}{\hbar }\left(
E_{p_{n}}+E_{p_{n}^{\prime }}\right) t}-b_{\vec{p}_{n}}b_{\vec{p}%
_{n}^{\prime }}^{\ast }e^{\frac{i}{\hbar }\left( \vec{p}_{n}-\vec{p}%
_{n}^{\prime }\right) \cdot \vec{x}}e^{-\frac{i}{\hbar }\left(
E_{p_{n}}-E_{p_{n}^{\prime }}\right) t}%
\end{array}%
\right) \\ 
+\left( \frac{E_{p_{n}}}{E_{p_{n}^{\prime }}}\right) ^{\frac{1}{2}}\left( 
\begin{array}{c}
a_{\vec{p}_{n}^{\prime }}a_{\vec{p}_{n}}^{\ast }e^{-\frac{i}{\hbar }\left( 
\vec{p}_{n}-\vec{p}_{n}^{\prime }\right) \cdot \vec{x}}e^{\frac{i}{\hbar }%
\left( E_{p_{n}}-E_{p_{n}^{\prime }}\right) t}-a_{\vec{p}_{n}^{\prime }}b_{%
\vec{p}_{n}}e^{\frac{i}{\hbar }\left( \vec{p}_{n}+\vec{p}_{n}^{\prime
}\right) \cdot \vec{x}}e^{-\frac{i}{\hbar }\left( E_{p_{n}}+E_{p_{n}^{\prime
}}\right) t} \\ 
+b_{\vec{p}_{n}^{\prime }}^{\ast }a_{\vec{p}_{n}}^{\ast }e^{-\frac{i}{\hbar }%
\left( \vec{p}_{n}+\vec{p}_{n}^{\prime }\right) \cdot \vec{x}}e^{\frac{i}{%
\hbar }\left( E_{p_{n}}+E_{p_{n}^{\prime }}\right) t}-b_{\vec{p}_{n}^{\prime
}}^{\ast }b_{\vec{p}_{n}}e^{\frac{i}{\hbar }\left( \vec{p}_{n}-\vec{p}%
_{n}^{\prime }\right) \cdot \vec{x}}e^{-\frac{i}{\hbar }\left(
E_{p_{n}}-E_{p_{n}^{\prime }}\right) t}%
\end{array}%
\right)%
\end{array}%
\right\} \\
&=&\frac{e}{2L^{3}}\sum_{\vec{p}_{n},\vec{p}_{n}^{\prime }}\int
d^{3}x\left\{ 
\begin{array}{c}
\left[ \left( \frac{E_{p_{n}^{\prime }}}{E_{p_{n}}}\right) ^{\frac{1}{2}}a_{%
\vec{p}_{n}}^{\ast }a_{\vec{p}_{n}^{\prime }}+\left( \frac{E_{p_{n}}}{%
E_{p_{n}^{\prime }}}\right) ^{\frac{1}{2}}a_{\vec{p}_{n}^{\prime }}a_{\vec{p}%
_{n}}^{\ast }\right] e^{\frac{i}{\hbar }\left( E_{p_{n}}-E_{p_{n}^{\prime
}}\right) t}e^{-\frac{i}{\hbar }\left( \vec{p}_{n}-\vec{p}_{n}^{\prime
}\right) \cdot \vec{x}} \\ 
+\left[ \left( \frac{E_{p_{n}}}{E_{p_{n}^{\prime }}}\right) ^{\frac{1}{2}}b_{%
\vec{p}_{n}^{\prime }}^{\ast }a_{\vec{p}_{n}}^{\ast }-\left( \frac{%
E_{p_{n}^{\prime }}}{E_{p_{n}}}\right) ^{\frac{1}{2}}a_{\vec{p}_{n}}^{\ast
}b_{\vec{p}_{n}^{\prime }}^{\ast }\right] e^{\frac{i}{\hbar }\left(
E_{p_{n}}+E_{p_{n}^{\prime }}\right) t}e^{-\frac{i}{\hbar }\left( \vec{p}%
_{n}+\vec{p}_{n}^{\prime }\right) \cdot \vec{x}} \\ 
+\left[ \left( \frac{E_{p_{n}^{\prime }}}{E_{p_{n}}}\right) ^{\frac{1}{2}}b_{%
\vec{p}_{n}}a_{\vec{p}_{n}^{\prime }}-\left( \frac{E_{p_{n}}}{%
E_{p_{n}^{\prime }}}\right) ^{\frac{1}{2}}a_{\vec{p}_{n}^{\prime }}b_{\vec{p}%
_{n}}\right] e^{-\frac{i}{\hbar }\left( E_{p_{n}}+E_{p_{n}^{\prime }}\right)
t}e^{\frac{i}{\hbar }\left( \vec{p}_{n}+\vec{p}_{n}^{\prime }\right) \cdot 
\vec{x}} \\ 
-\left[ \left( \frac{E_{p_{n}^{\prime }}}{E_{p_{n}}}\right) ^{\frac{1}{2}}b_{%
\vec{p}_{n}}b_{\vec{p}_{n}^{\prime }}^{\ast }+\left( \frac{E_{p_{n}}}{%
E_{p_{n}^{\prime }}}\right) ^{\frac{1}{2}}b_{\vec{p}_{n}^{\prime }}^{\ast
}b_{\vec{p}_{n}}\right] e^{-\frac{i}{\hbar }\left(
E_{p_{n}}-E_{p_{n}^{\prime }}\right) t}e^{\frac{i}{\hbar }\left( \vec{p}_{n}-%
\vec{p}_{n}^{\prime }\right) \cdot \vec{x}}%
\end{array}%
\right\} \\
&=&\frac{e}{2L^{3}}\sum_{\vec{p}_{n},\vec{p}_{n}^{\prime }}\left\{ 
\begin{array}{c}
\left[ \left( \frac{E_{p_{n}^{\prime }}}{E_{p_{n}}}\right) ^{\frac{1}{2}}a_{%
\vec{p}_{n}}^{\ast }a_{\vec{p}_{n}^{\prime }}+\left( \frac{E_{p_{n}}}{%
E_{p_{n}^{\prime }}}\right) ^{\frac{1}{2}}a_{\vec{p}_{n}^{\prime }}a_{\vec{p}%
_{n}}^{\ast }\right] e^{\frac{i}{\hbar }\left( E_{p_{n}}-E_{p_{n}^{\prime
}}\right) t}L^{3}\delta _{\vec{p}_{n},\vec{p}_{n}^{\prime }} \\ 
+\left[ \left( \frac{E_{p_{n}}}{E_{p_{n}^{\prime }}}\right) ^{\frac{1}{2}}b_{%
\vec{p}_{n}^{\prime }}^{\ast }a_{\vec{p}_{n}}^{\ast }-\left( \frac{%
E_{p_{n}^{\prime }}}{E_{p_{n}}}\right) ^{\frac{1}{2}}a_{\vec{p}_{n}}^{\ast
}b_{\vec{p}_{n}^{\prime }}^{\ast }\right] e^{\frac{i}{\hbar }\left(
E_{p_{n}}+E_{p_{n}^{\prime }}\right) t}L^{3}\delta _{\vec{p}_{n},-\vec{p}%
_{n}^{\prime }} \\ 
+\left[ \left( \frac{E_{p_{n}^{\prime }}}{E_{p_{n}}}\right) ^{\frac{1}{2}}b_{%
\vec{p}_{n}}a_{\vec{p}_{n}^{\prime }}-\left( \frac{E_{p_{n}}}{%
E_{p_{n}^{\prime }}}\right) ^{\frac{1}{2}}a_{\vec{p}_{n}^{\prime }}b_{\vec{p}%
_{n}}\right] e^{-\frac{i}{\hbar }\left( E_{p_{n}}+E_{p_{n}^{\prime }}\right)
t}L^{3}\delta _{\vec{p}_{n},-\vec{p}_{n}^{\prime }} \\ 
-\left[ \left( \frac{E_{p_{n}^{\prime }}}{E_{p_{n}}}\right) ^{\frac{1}{2}}b_{%
\vec{p}_{n}}b_{\vec{p}_{n}^{\prime }}^{\ast }+\left( \frac{E_{p_{n}}}{%
E_{p_{n}^{\prime }}}\right) ^{\frac{1}{2}}b_{\vec{p}_{n}^{\prime }}^{\ast
}b_{\vec{p}_{n}}\right] e^{-\frac{i}{\hbar }\left(
E_{p_{n}}-E_{p_{n}^{\prime }}\right) t}L^{3}\delta _{\vec{p}_{n},\vec{p}%
_{n}^{\prime }}%
\end{array}%
\right\} \\
&=&\frac{e}{2}\sum_{\vec{p}_{n}}\left\{ 
\begin{array}{c}
\left[ \left( \frac{E_{p_{n}}}{E_{p_{n}}}\right) ^{\frac{1}{2}}a_{\vec{p}%
_{n}}^{\ast }a_{\vec{p}_{n}}+\left( \frac{E_{p_{n}}}{E_{p_{n}}}\right) ^{%
\frac{1}{2}}a_{\vec{p}_{n}}a_{\vec{p}_{n}}^{\ast }\right] e^{\frac{i}{\hbar }%
\left( E_{p_{n}}-E_{p_{n}}\right) t} \\ 
+\left[ \left( \frac{E_{p_{n}}}{E_{-p_{n}}}\right) ^{\frac{1}{2}}b_{-\vec{p}%
_{n}}^{\ast }a_{\vec{p}_{n}}^{\ast }-\left( \frac{E_{-p_{n}}}{E_{p_{n}}}%
\right) ^{\frac{1}{2}}a_{\vec{p}_{n}}^{\ast }b_{-\vec{p}_{n}}^{\ast }\right]
e^{\frac{i}{\hbar }\left( E_{p_{n}}+E_{-p_{n}}\right) t} \\ 
+\left[ \left( \frac{E_{-p_{n}}}{E_{p_{n}}}\right) ^{\frac{1}{2}}b_{\vec{p}%
_{n}}a_{-\vec{p}_{n}}-\left( \frac{E_{p_{n}}}{E_{-p_{n}}}\right) ^{\frac{1}{2%
}}a_{-\vec{p}_{n}}b_{\vec{p}_{n}}\right] e^{-\frac{i}{\hbar }\left(
E_{p_{n}}+E_{-p_{n}}\right) t} \\ 
-\left[ \left( \frac{E_{p_{n}}}{E_{p_{n}}}\right) ^{\frac{1}{2}}b_{\vec{p}%
_{n}}b_{\vec{p}_{n}}^{\ast }+\left( \frac{E_{p_{n}}}{E_{p_{n}}}\right) ^{%
\frac{1}{2}}b_{\vec{p}_{n}}^{\ast }b_{\vec{p}_{n}}\right] e^{-\frac{i}{\hbar 
}\left( E_{p_{n}}-E_{p_{n}}\right) t}%
\end{array}%
\right\} \\
&=&\frac{e}{2}\sum_{\vec{p}_{n}}\left\{ a_{\vec{p}_{n}}^{\ast }a_{\vec{p}%
_{n}}+a_{\vec{p}_{n}}a_{\vec{p}_{n}}^{\ast }+\left[ b_{-\vec{p}_{n}}^{\ast
}a_{\vec{p}_{n}}^{\ast }-a_{\vec{p}_{n}}^{\ast }b_{-\vec{p}_{n}}^{\ast }%
\right] e^{2\frac{i}{\hbar }E_{p_{n}}t}+\left[ b_{\vec{p}_{n}}a_{-\vec{p}%
_{n}}-a_{-\vec{p}_{n}}b_{\vec{p}_{n}}\right] e^{-2\frac{i}{\hbar }%
E_{p_{n}}t}-b_{\vec{p}_{n}}b_{\vec{p}_{n}}^{\ast }-b_{\vec{p}_{n}}^{\ast }b_{%
\vec{p}_{n}}\right\} \\
&=&e\sum_{\vec{p}_{n}}\left\{ a_{\vec{p}_{n}}^{\ast }a_{\vec{p}_{n}}-b_{\vec{%
p}_{n}}^{\ast }b_{\vec{p}_{n}}\right\} .
\end{eqnarray*}
\subsection{Cuadricorriente:}%

\begin{align*}
\left(  \hbar^{2}D_{\mu}D^{\mu}+m^{2}c^{2}\right)  \psi & =0\\
\left(  \hbar^{2}D_{\mu}D^{\mu}+m^{2}c^{2}\right)  \psi^{\ast}  & =0
\end{align*}
\begin{align*}
\psi^{\ast}\left(  \hbar^{2}D_{\mu}D^{\mu}+m^{2}c^{2}\right)  \psi & =0\\
\psi\left(  \hbar^{2}D_{\mu}D^{\mu}+m^{2}c^{2}\right)  \psi^{\ast}  & =0
\end{align*}
\begin{equation}
\psi^{\ast}\left(  \hbar^{2}D_{\mu}D^{\mu}+m^{2}c^{2}\right)  \psi
-\psi\left(  \hbar^{2}D_{\mu}D^{\mu}+m^{2}c^{2}\right)  \psi^{\ast}=0
\end{equation}
y usando el truco de Oelker,
\begin{equation}
\hbar^{2}D_{\mu}\left(  \psi^{\ast}D^{\mu}\psi-\psi D^{\mu}\psi^{\ast}\right)
=0
\end{equation}
Por eso, si definimos una densidad de corriente para la carga,
\begin{equation}
j^{\mu}=\frac{i\hbar e}{2m}\left(  \psi^{\ast}D^{\mu}\psi-\psi D^{\mu}%
\psi^{\ast}\right)
\end{equation}
Tendr'iamos
\begin{equation}
D_{\mu}j^{\mu}=0
\end{equation}
Si expresamos $D^{\mu}$ en forma expl'icita,
\begin{equation}
j^{\mu}=\frac{i\hbar e}{2m}\left(  \psi^{\ast}\partial^{\mu}\psi
-\psi\partial^{\mu}\psi^{\ast}\right)  -\frac{e^{2}}{mc}A^{\mu}\psi^{\ast
}\psi
\end{equation}
y en particular,
\begin{equation}
\rho_{\pm}=\frac{1}{c}j^{0^{\prime}}=\frac{1}{c}\frac{i\hbar e}{2m}\left(
\psi_{\pm}^{\ast}D^{0}\psi_{\pm}-\psi_{\pm}D^{0}\psi_{\pm}^{\ast}\right)
\end{equation}


\begin{itemize}
\item \textbf{Si consideramos part'iculas libres, }Entonces,
\begin{align*}
\psi_{\pm}  & =A_{\pm}e^{\frac{i}{\hbar}\left(  -p_{i}x^{i}\mp E_{p}t\right)
}\\
\psi_{\pm}^{\ast}  & =A_{\pm}e^{-\frac{i}{\hbar}\left(  -p_{i}x^{i}\mp
E_{p}t\right)  }%
\end{align*}
y no hay campos de gauge. Entonces,
\begin{align*}
\rho_{\pm}  & =\frac{i\hbar e}{2mc}\left(  \psi_{\pm}^{\ast}\partial
^{0}\psi_{\pm}-\psi_{\pm}\partial^{0}\psi_{\pm}^{\ast}\right) \\
& =\frac{i\hbar e}{2mc^{2}}\left(  \psi_{\pm}^{\ast}\partial_{t}\psi_{\pm
}-\psi_{\pm}\partial_{t}\psi_{\pm}^{\ast}\right) \\
& =\pm\frac{e}{mc^{2}}E_{p}\psi_{\pm}^{\ast}\psi_{\pm}%
\end{align*}
Entonces, llegamos a la conclusi'on que $\psi_{+}$ significa cargas $+e$, y
masa $m$; y que $\psi_{-}$ significa cargas $-e$ y masa $m$. La
soluci'on siempre ser\'{a} una superposici'on lineal de \emph{ambos}
tipos de estados.
\end{itemize}

\section{L'imite No Relativista de la ecuaci'on de Klein-Gordon}

Hacemos el ansatz
\begin{equation}
\psi\left(  x^{\mu}\right)  =\varphi\left(  x^{i},t\right)  e^{-\frac{i}%
{\hbar}mc^{2}t}%
\end{equation}
Tenemos que la energ'ia cin\'{e}tica es
\begin{equation}
E_{c}=E-mc^{2}%
\end{equation}
Si el caso es no relativista, sabemos que $E_{c}$ es peque\~{n}a.
\begin{equation}
E_{c}\ll mc^{2}%
\end{equation}
y adem\'{a}s, por alguna extra\~{n}a raz'on$\left(
\text{\textquestiondown ?}\right)  $,
\begin{equation}
\left|  i\hbar\frac{\partial\varphi}{\partial t}\right|  \approx E_{c}%
\varphi\ll mc^{2}\varphi
\end{equation}
y si calculamos las derivadas, se tiene que\newline%
\begin{align*}
\partial_{t}\psi & =\left[  \partial_{t}\varphi\left(  x^{i},t\right)
-\frac{i}{\hbar}mc^{2}\varphi\left(  x^{i},t\right)  \right]  e^{-\frac
{i}{\hbar}mc^{2}t}\\
\partial_{t}^{2}\psi & =\partial_{t}\left\{  \left[  \partial_{t}%
\varphi\left(  x^{i},t\right)  -\frac{i}{\hbar}mc^{2}\varphi\left(
x^{i},t\right)  \right]  e^{-\frac{i}{\hbar}mc^{2}t}\right\} \\
& =-\left(  -\partial_{t}^{2}\varphi\left(  x^{i},t\right)  +\frac{2i}{\hbar
}mc^{2}\partial_{t}\varphi+\frac{1}{\hbar^{2}}m^{2}c^{4}%
\varphi\right)  e^{-\frac{i}{\hbar}mc^{2}t}%
\end{align*}
Al parecer, se usa el argumento extra\~{n}o para decir que
\begin{equation}
\partial_{t}^{2}\psi\approx-\left(  \frac{2i}{\hbar}mc^{2}\partial
_{t}\varphi+\frac{1}{\hbar^{2}}m^{2}c^{4}\varphi\right)  e^{-\frac
{i}{\hbar}mc^{2}t}%
\end{equation}
Si insertamos esto dentro de la ecuaci'on de Klein-Gordon, llegamos a la
ecuaci'on de Schroedinger para una part'icula libre sin spin,
\begin{equation}
i\hbar\partial_{t}\varphi=\hbar^{2}\frac{\partial_{i}\partial^{i}\varphi
}{2m}=-\frac{\hbar^{2}}{2m}\nabla^{2}\varphi
\end{equation}


\subsection{Part'iculas libres de Spin-0}

Vamos a suponer que nuestra part'icula, de carga $\ \pm e$, est\'{a}
encerrada en un cubo de potencial, de arista $L$, o condiciones
peri'odicas.
\begin{equation}
\psi_{n\left(  \pm\right)  }=A_{n\left(  \pm\right)  }e^{\frac{i}{\hbar
}\left(  -p_{in}x^{i}\mp E_{pn}t\right)  }%
\end{equation}
\begin{equation}
p_{in}=\frac{2\pi}{L}n_{i}\text{, }n_{i}=\text{entero negativo.}%
\end{equation}
\begin{equation}
E_{pn}=c\sqrt{p_{n}^{2}+m^{2}c^{2}}%
\end{equation}
\begin{equation}
\pm e=\int_{L^{3}}\rho_{\pm}d^{3}x
\end{equation}
\begin{equation}
\rho_{\pm}=\pm\frac{e}{mc^{2}}E_{p}\psi_{\pm}^{\ast}\psi_{\pm}%
\end{equation}
\begin{align*}
\pm e  & =\pm\int_{L^{3}}\frac{e}{mc^{2}}E_{pn}\psi_{\pm n}^{\ast}%
\psi_{\pm n}d^{3}x\\
1  & =\frac{1}{mc^{2}}\int_{L^{3}}E_{pn}\left|  A_{n\left(  \pm\right)
}\right|  ^{2}d^{3}x\\
mc^{2}  & =E_{pn}\left|  A_{n\left(  \pm\right)  }\right|  ^{2}L^{3}%
\end{align*}
\begin{equation}
\left|  A_{n\left(  \pm\right)  }\right|  =\sqrt{\frac{mc^{2}}{E_{pn}%
L^{3}}}%
\end{equation}
\begin{align*}
\psi_{n\left(  \pm\right)  }  & =\sqrt{\frac{mc^{2}}{E_{pn}L^{3}}}%
e^{\frac{i}{\hbar}\left(  -p_{in}x^{i}\mp E_{pn}t\right)  }\\
\psi_{n\left(  +\right)  }  & =\sqrt{\frac{mc^{2}}{E_{pn}L^{3}}}%
e^{\frac{i}{\hbar}\left(  -p_{in}x^{i}-E_{pn}t\right)  }\\
\psi_{n\left(  -\right)  }  & =\sqrt{\frac{mc^{2}}{E_{pn}L^{3}}}%
e^{\frac{i}{\hbar}\left(  -p_{in}x^{i}+E_{pn}t\right)  }%
\end{align*}
Entonces, para part'iculas positivas y negativas, se tiene
respectivamente,
\begin{align*}
\psi_{\left(  +\right)  }  & =\sum_{n}a_{n}\psi_{n\left(  +\right)  }=\sum
_{n}a_{n}\sqrt{\frac{mc^{2}}{L^{3}}}\frac{1}{\sqrt{E_{pn}}}e^{\frac
{i}{\hbar}\left(  -p_{in}x^{i}-E_{pn}t\right)  }\\
\psi_{\left(  -\right)  }  & =\sum_{n}b_{n}\psi_{n\left(  -\right)  }=\sum
_{n}b_{n}\sqrt{\frac{mc^{2}}{L^{3}}}\frac{1}{\sqrt{E_{pn}}}e^{\frac
{i}{\hbar}\left(  -p_{in}x^{i}+E_{pn}t\right)  }%
\end{align*}
Ahora bien, si una part'icula es neutra, se ve de la definici'on de
densidad de carga que la funci'on de onda debe ser real;
\begin{equation}
\psi=\psi^{\ast}%
\end{equation}
Si uno, por una extra\~{n}a raz'on$\left(  \text{\textquestiondown ?}%
\right)  $, supone que una part'icula neutra puede ser descrita por
\begin{equation}
\psi_{n\left(  0\right)  }=\frac{1}{\sqrt{2}}\left(  \psi_{n\left(  +\right)
}\left(  \mathbf{p}_{n}\right)  +\psi_{n\left(  -\right)  }\left(
-\mathbf{p}_{n}\right)  \right)
\end{equation}
y de ah'i,
\begin{equation}
\psi_{n\left(  0\right)  }=2\sqrt{\frac{mc^{2}}{2E_{pn}L^{3}}}\cos\left(
\frac{-p_{in}x^{i}-E_{pn}t}{\hbar}\right)
\end{equation}
y claro,
\begin{equation}
\rho=0
\end{equation}


\subsection{La ecuaci'on de Klein-Gordon en la forma de una ecuaci'on de
Schroedinger.}

En la tarea 1, probamos que, usando las matrices de Pauli, que la ecuaci'on
de Klein - Gordon puede ser escrita en la forma de una ecuaci'on de
Schroedinger:
\begin{equation}
i\hbar\frac{\partial\mathbf{\Psi}}{\partial t}=H\mathbf{\Psi}%
\end{equation}
Con
\begin{equation}
\mathbf{\Psi}=\left(
\begin{array}
[c]{c}%
\varphi\\
\chi
\end{array}
\right)
\end{equation}
Con
\begin{equation}
\varphi=\frac{1}{\sqrt{2}}\left(  \psi+i\frac{\hbar}{mc^{2}}\frac{\partial
\psi}{\partial t}\right)  \text{, }\chi=\frac{1}{\sqrt{2}}\left(  \psi
-i\frac{\hbar}{mc^{2}}\frac{\partial\psi}{\partial t}\right)
\end{equation}
y
\begin{equation}
H=-\frac{\hbar^{2}}{2m}\left(  \sigma_{3}+i\sigma_{2}\right)  \nabla
^{2}+mc^{2}\sigma_{3}%
\end{equation}
Bueno, a partir de eso, es f\'{a}cil ver que tenemos dos soluciones, para
energ'ia positiva y negativa.

Ojo:
\begin{equation}
\psi=\frac{1}{\sqrt{2}}\left(  \varphi+\chi\right)
\end{equation}


\subsubsection{Energ'ia Positiva:}%

\begin{equation}
E=+E_{p}%
\end{equation}
\begin{equation}
\mathbf{\Psi}^{+}\left(  \mathbf{p}\right)  =A_{\left(  +\right)  }\left(
\begin{array}
[c]{c}%
\varphi_{0}^{\left(  +\right)  }\\
\chi_{0}^{\left(  +\right)  }%
\end{array}
\right)  e^{\frac{i}{\hbar}\left(  -p_{i}x^{i}-E_{p}t\right)  }=\left(
\begin{array}
[c]{c}%
\varphi^{\left(  +\right)  }\left(  \mathbf{p}\right) \\
\chi^{\left(  +\right)  }\left(  \mathbf{p}\right)
\end{array}
\right)
\end{equation}
con
\begin{equation}
\left(
\begin{array}
[c]{c}%
\varphi_{0}^{\left(  +\right)  }\\
\chi_{0}^{\left(  +\right)  }%
\end{array}
\right)  =\left(
\begin{array}
[c]{c}%
mc^{2}+E_{p}\\
mc^{2}-E_{p}%
\end{array}
\right)
\end{equation}
\begin{equation}
A_{\left(  +\right)  }=\frac{1}{\sqrt{E_{p}L^{3}}}\frac{1}{\sqrt{4mc^{2}}}%
\end{equation}


\subsubsection{Energ'ia Negativa.}%

\begin{equation}
E=-E_{p}%
\end{equation}
\begin{equation}
\mathbf{\Psi}^{-}\left(  \mathbf{p}\right)  =A_{\left(  -\right)  }\left(
\begin{array}
[c]{c}%
\varphi_{0}^{\left(  -\right)  }\\
\chi_{0}^{-}%
\end{array}
\right)  e^{\frac{i}{\hbar}\left(  -p_{i}x^{i}+E_{p}t\right)  }=\left(
\begin{array}
[c]{c}%
\varphi^{\left(  -\right)  }\left(  \mathbf{p}\right) \\
\chi^{\left(  -\right)  }\left(  \mathbf{p}\right)
\end{array}
\right)
\end{equation}
con
\begin{equation}
\left(
\begin{array}
[c]{c}%
\varphi_{0}^{\left(  -\right)  }\\
\chi_{0}^{\left(  -\right)  }%
\end{array}
\right)  =\left(
\begin{array}
[c]{c}%
mc^{2}-E_{p}\\
mc^{2}+E_{p}%
\end{array}
\right)
\end{equation}
\begin{equation}
A_{\left(  -\right)  }=A_{\left(  +\right)  }=\frac{1}{\sqrt{E_{p}L^{3}}}%
\frac{1}{\sqrt{4mc^{2}}}%
\end{equation}


\subsubsection{Conjugaci'on de Carga.}%

\begin{align*}
\Psi^{\left(  -\right)  }\left(  -\mathbf{p}\right)   & =\left(
\begin{array}
[c]{c}%
\varphi^{\left(  -\right)  }\left(  -\mathbf{p}\right) \\
\chi^{\left(  -\right)  }\left(  -\mathbf{p}\right)
\end{array}
\right) \\
& =\left(
\begin{array}
[c]{c}%
\chi^{\left(  +\right)  \ast}\left(  \mathbf{p}\right) \\
\varphi^{\left(  +\right)  \ast}\left(  \mathbf{p}\right)
\end{array}
\right) \\
& =\hat{\sigma}_{1}\Psi^{\left(  +\right)  }\left(  \mathbf{p}\right)
\end{align*}
Si el estado
\begin{equation}
\Psi=\left(
\begin{array}
[c]{c}%
\varphi\\
\chi
\end{array}
\right)
\end{equation}
representa a una part'icula de carga positiva, entonces,
\begin{equation}
\hat{C}\Psi\hat{C}^{-1}=\Psi_{c}=\hat{\sigma}_{1}\Psi^{\ast}=\left(
\begin{array}
[c]{c}%
\chi^{\ast}\\
\varphi^{\ast}%
\end{array}
\right)
\end{equation}
Llamamos a $\Psi_{c}$ el estado de carga conjugado de $\Psi$.

La conjugaci'on de carga implica las siguientes transformaciones:
\begin{align*}
\varphi_{0}^{\left(  +\right)  }  & \rightarrow\chi_{0}^{\left(  -\right)  }\\
\chi_{0}^{\left(  +\right)  }  & \rightarrow\varphi_{0}^{\left(  -\right)  }\\
\mathbf{p}  & \rightarrow-\mathbf{p}\\
+E_{p}  & \rightarrow-E_{p}%
\end{align*}
Si el estado $\Psi$ describe a una part'icula, el $\Psi_{c}$ a una antipart'icula.

\subsubsection{Paridad.}

Si la part'icula es neutra, es su propia antipart'icula.
\begin{equation}
\Psi_{c}=\sigma_{1}\Psi^{\ast}=\alpha\Psi
\end{equation}
El factor $\alpha$ debe ser real. Esto se puede deducir a partir de
\begin{equation}
\psi=\frac{1}{\sqrt{2}}\left(  \varphi+\chi\right)
\end{equation}
si la part'icula es neutra, $\psi$ es real, y por lo tanto,
\begin{equation}
\operatorname{Im}\left(  \varphi\right)  =-\operatorname{Im}\left(
\chi\right)
\end{equation}
pero $\psi_{c}$ tambi\'{e}n debe serlo.
\begin{equation}
\operatorname{Im}\left(  \alpha\varphi\right)  =-\operatorname{Im}\left(
\alpha\chi\right)
\end{equation}
De esas dos ecuaciones es directo demostrar que $\alpha$ es real.

Adem\'{a}s, como
\begin{equation}
\left(  \Psi_{c}\right)  _{c}=\Psi
\end{equation}
\begin{equation}
\Psi_{c}=\alpha\Psi
\end{equation}
\begin{align*}
\left(  \alpha\Psi\right)  _{c}  & =\Psi\\
\alpha\Psi_{c}  & =\Psi\\
\alpha^{2}  & =1
\end{align*}
\begin{equation}
\alpha=\pm1
\end{equation}
As'i, existen dos tipos de part'iculas:

\begin{enumerate}
\item Part'iculas neutras de spin-0 con \textbf{paridad de carga
positiva}, $\alpha=+1$.
\begin{equation}
\Psi_{c}=\Psi
\end{equation}
\begin{equation}
\Psi=\left(
\begin{array}
[c]{c}%
\varphi\\
\chi
\end{array}
\right)
\end{equation}
\begin{equation}
\hat{C}\Psi\hat{C}^{-1}=\Psi_{c}=\hat{\sigma}_{1}\Psi^{\ast}=\left(
\begin{array}
[c]{c}%
\chi^{\ast}\\
\varphi^{\ast}%
\end{array}
\right)
\end{equation}
\begin{equation}
\varphi^{\ast}=\chi
\end{equation}


\item Part'iculas neutras de spin-0 con \textbf{paridad de carga
negativa}, $\alpha=-1$.
\begin{equation}
\Psi_{c}=-\Psi
\end{equation}
\begin{equation}
\Psi=\left(
\begin{array}
[c]{c}%
\varphi\\
\chi
\end{array}
\right)
\end{equation}
\begin{equation}
\hat{C}\Psi\hat{C}^{-1}=\Psi_{c}=\hat{\sigma}_{1}\Psi^{\ast}=\left(
\begin{array}
[c]{c}%
\chi^{\ast}\\
\varphi^{\ast}%
\end{array}
\right)  =-\left(
\begin{array}
[c]{c}%
\varphi\\
\chi
\end{array}
\right)
\end{equation}
\begin{equation}
\varphi^{\ast}=-\chi
\end{equation}

\end{enumerate}

\subsection{Interacci'on con el campo.}

Siempre usamos el anzats
\begin{equation}
\psi\left(  t,x\right)  =\psi\left(  x\right)  e^{-\frac{i}{\hbar}\varepsilon
t}%
\end{equation}
Analizamos s'olo casos de potencial electrost\'{a}tico... por ahora
\begin{equation}
e\phi\left(  r\right)  =U\left(  r\right)
\end{equation}


\begin{itemize}
\item \textbf{Densidad de Carga}:
\begin{equation}
j^{\mu}=\frac{i\hbar e}{2m}\left(  \psi^{\ast}\partial^{\mu}\psi
-\psi\partial^{\mu}\psi^{\ast}\right)  -\frac{e^{2}}{mc}A^{\mu}\psi^{\ast
}\psi
\end{equation}
\begin{equation}
j^{0}=\frac{e}{mc}\left(  \varepsilon-eA^{0}\right)  \psi^{\ast}%
\psi\left(  r\right)
\end{equation}
\begin{equation}
\rho=\frac{e}{mc^{2}}\left(  \varepsilon-eA^{0}\right)  \psi^{\ast}%
\psi\left(  r\right)
\end{equation}
La densidad es positiva o negativa dependiendo del nivel de energ'ia en el
que se encuentre.

\item \textbf{Soluci'on de la Ecuaci'on, para potenciales }$U\left(
x\right)  $:
\begin{equation}
\left(  \hbar^{2}D_{\mu}D^{\mu}+m^{2}c^{2}\right)  \psi=0
\end{equation}
\begin{equation}
D_{\mu}=\partial_{\mu}+i\frac{e}{c\hbar}A_{\mu}%
\end{equation}
y entonces, la ecuaci'on K-G toma la forma:
\begin{equation}
\left(  \partial_{\mu}\partial^{\mu}+i\frac{e}{c\hbar}\partial_{\mu}A^{\mu
}+i\frac{e}{c\hbar}A_{\mu}\partial^{\mu}-\frac{e^{2}}{c^{2}\hbar^{2}}A_{\mu
}A^{\mu}+\frac{m^{2}c^{2}}{\hbar^{2}}\right)  \psi=0
\end{equation}
y si
\begin{equation}
A^{0}=\frac{U\left(  x\right)  }{e}%
\end{equation}
llegamos, tras algo de macanaca,
\begin{equation}
\left(  \frac{1}{c^{2}}\partial_{t}^{2}+\partial_{i}\partial^{i}+i2\frac
{1}{c^{2}\hbar}U\partial_{t}-\frac{1}{c^{2}\hbar^{2}}U^{2}+\frac{m%
^{2}c^{2}}{\hbar^{2}}\right)  \psi=0
\end{equation}
Si ahora usamos
\begin{align*}
\psi=\psi\left(  x\right)  e^{-\frac{i}{\hbar}\varepsilon t}\\
\partial_{t}\psi=-\frac{i}{\hbar}\varepsilon\psi\\
\partial_{t}^{2}\psi=-\frac{\varepsilon^{2}}{\hbar^{2}}\psi
\end{align*}
llegamos a la ecuaci'on de Helmholtz,
\begin{equation}
\left(  \nabla^{2}+k^{2}\left(  x\right)  \right)  \psi\left(  x\right)  =0
\end{equation}
con
\begin{equation}
k^{2}\left(  x\right)  =\frac{\left(  \varepsilon-U\left(  x\right)  \right)
^{2}-m^{2}c^{4}}{c^{2}\hbar^{2}}%
\end{equation}


\item \textbf{Separaci'on de la parte angular y radial para simetr'ias
esf\'{e}ricas}:
\begin{equation}
\left(  \nabla^{2}+k^{2}\left(  x\right)  \right)  \psi\left(  x\right)  =0
\end{equation}
El Laplaciano, en coordenadas esf\'{e}ricas, es
\begin{equation}
\nabla^{2}=\frac{1}{r^{2}}\partial_{r}\left(  r^{2}\partial_{r}\right)
+\frac{1}{r^{2}\operatorname{sen}\theta}\partial_{\theta}\left(
\operatorname{sen}\theta\partial_{\theta}\right)  +\frac{1}{r^{2}%
\operatorname{sen}^{2}\theta}\partial_{\phi}^{2}%
\end{equation}
y si usamos la tradicional separaci'on de variables,
\begin{equation}
\psi\left(  x\right)  =u\left(  r\right)  Y\left(  \theta,\phi\right)
\end{equation}
uno puede demostrar f\'{a}cilmente que
\begin{equation}
\frac{r^{2}}{\psi\left(  x\right)  }\nabla^{2}\psi\left(  x\right)  =\frac
{1}{u}\partial_{r}r^{2}\partial_{r}u+\frac{1}{Y}\left[  \frac{1}%
{\operatorname{sen}\theta}\partial_{\theta}\left(  \operatorname{sen}%
\theta\partial_{\theta}\right)  Y+\frac{1}{\operatorname{sen}^{2}\theta
}\partial_{\phi}^{2}Y\right]
\end{equation}
y de la ecuaci'on de Helmholtz, uno tiene que
\begin{align*}
\left(  \nabla^{2}+k^{2}\right)  \psi\left(  x\right)   & =0\\
\frac{r^{2}}{\psi\left(  x\right)  }\nabla^{2}\psi\left(  x\right)
+r^{2}k^{2}  & =0
\end{align*}
y as'i,
\begin{equation}
\frac{1}{u}\partial_{r}r^{2}\partial_{r}u+r^{2}k^{2}\left(  r\right)
=-\frac{1}{Y}\left[  \frac{1}{\operatorname{sen}\theta}\partial_{\theta
}\left(  \operatorname{sen}\theta\partial_{\theta}\right)  Y+\frac
{1}{\operatorname{sen}^{2}\theta}\partial_{\phi}^{2}Y\right]  =\lambda
\end{equation}
Para la parte radial,
\begin{equation}
\left\{  \partial_{r}r^{2}\partial_{r}+r^{2}k^{2}\left(  r\right)
-\lambda\right\}  u=0
\end{equation}
y para la angular,
\begin{equation}
\left[  \frac{1}{\operatorname{sen}\theta}\partial_{\theta}\left(
\operatorname{sen}\theta\partial_{\theta}\right)  +\frac{1}{\operatorname{sen}%
^{2}\theta}\partial_{\phi}^{2}+\lambda\right]  Y=0
\end{equation}
La soluci'on de esta ecuaci'on son los arm'onicos esf\'{e}ricos
$Y_{lm}$ con $\lambda=l\left(  l+1\right)  $, $l=0,1,2,...$, $m=\pm0,\pm
1,\pm2,...$. Entonces,
\begin{equation}
\left\{  \partial_{r}r^{2}\partial_{r}+r^{2}k^{2}\left(  r\right)  -l\left(
l+1\right)  \right\}  u=0
\end{equation}
Un ansatz com\'{u}n es:
\begin{equation}
u\left(  r\right)  =\frac{R\left(  r\right)  }{r}%
\end{equation}
con lo que tenemos
\begin{equation}
\left(  \partial_{r}^{2}-\frac{l\left(  l+1\right)  }{r^{2}}+k^{2}\right)
R_{l}=0
\end{equation}

\end{itemize}

\subsubsection{\'{A}tomo pi'onico $\left(  \text{mes'on }\pi^{-}\right)
$ con n\'{u}cleo puntual.}

Si hacemos
\begin{equation}
e\phi=U\left(  r\right)  =-\frac{Ze}{r}%
\end{equation}
y llamamos a la combinaci'on
\begin{equation}
\alpha=\frac{e^{2}}{\hbar c}%
\end{equation}
constante de estructura fina, llegamos, expandiendo lo que hay que expandir,
a
\begin{equation}
\left(  \partial_{r}^{2}-\frac{l\left(  l+1\right)  -Z^{2}\alpha^{2}}{r^{2}%
}+\frac{2Z\alpha\varepsilon}{\hbar c}\frac{1}{r}-\frac{m^{2}%
c^{4}-\varepsilon^{2}}{c^{2}\hbar^{2}}\right)  R_{l}=0
\end{equation}
Calcularemos las energ'ias y las funciones de onda en el rango
$-mc^{2}<\varepsilon<mc^{2}$. $\left(  \text{\textquestiondown ?}%
\right)  $

\textbf{Cambios de variable}:%

\begin{align*}
\beta & =2\sqrt{\frac{m^{2}c^{4}-\varepsilon^{2}}{c^{2}\hbar^{2}}}\\
\rho & =\beta r\\
\mu & =\sqrt{\left(  l+\frac{1}{2}\right)  ^{2}-\left(  Z\alpha\right)  ^{2}%
}\\
\lambda & =\frac{2Z\alpha\varepsilon}{\hbar c\beta}%
\end{align*}
llegamos a
\begin{equation}
\left(  \partial_{\rho}^{2}-\frac{\mu^{2}-\frac{1}{4}}{\rho^{2}}+\frac
{\lambda}{\rho}-\frac{1}{4}\right)  R_{l}\left(  \rho\right)  =0
\end{equation}


\textbf{L'imites para justificar el anzats}:

\begin{itemize}
\item En el l'imite cuando $\rho\rightarrow\infty$,
\begin{equation}
\left(  \partial_{\rho}^{2}-\frac{1}{4}\right)  R_{l}\left(  \rho\right)  =0
\end{equation}
\begin{equation}
R_{l}\left(  \rho\right)  _{\rho\rightarrow\infty}=ae^{-\rho/2}+be^{\rho/2}%
\end{equation}
y para que no diverja,
\begin{equation}
R_{l}\left(  \rho\right)  _{\rho\rightarrow\infty}=ae^{-\rho/2}%
\end{equation}


\item En el l'imite cuando $\rho\rightarrow0$,
\begin{equation}
\left(  \partial_{\rho}^{2}-\frac{\mu^{2}-\frac{1}{4}}{\rho^{2}}\right)
R_{l}\left(  \rho\right)  =0
\end{equation}
Si suponemos
\begin{equation}
R_{l}\left(  \rho\right)  =a\rho^{\nu}%
\end{equation}
\begin{align*}
a\nu\left(  \nu-1\right)  \rho^{\nu-2}-a\left(  \mu^{2}-\frac{1}{4}\right)
\rho^{\nu-2}  & =0\\
\nu\left(  \nu-1\right)  \rho^{\nu-2}-\left(  \mu^{2}-\frac{1}{4}\right)
\rho^{\nu-2}  & =0\\
\nu^{2}-\nu-\left(  \mu^{2}-\frac{1}{4}\right)   & =0
\end{align*}
\begin{equation}
\nu_{\pm}=\frac{1}{2}\pm\sqrt{\frac{1}{4}+\mu^{2}-\frac{1}{4}}=\frac{1}{2}%
\pm\mu
\end{equation}
Ahora bien, consideremos la funci'on $u\left(  r\right)  $:
\begin{equation}
u=\frac{R}{r}=a\frac{\rho^{\nu}}{r}=a\beta\rho^{\nu-1}%
\end{equation}
Si nos fijamos en la soluci'on particular $l=0$,
\begin{equation}
\mu=\sqrt{\frac{1}{4}-\left(  Z\alpha\right)  ^{2}}\Rightarrow\nu_{\pm}%
=\frac{1}{2}\pm\sqrt{\frac{1}{4}-\left(  Z\alpha\right)  ^{2}}%
\end{equation}
\begin{equation}
u=a\beta\rho^{\nu-1}=a\beta\rho^{\pm\sqrt{\frac{1}{4}-\left(  Z\alpha\right)
^{2}}-\frac{1}{2}}%
\end{equation}
y si consideramos el signo menos en $\nu_{\pm}$, se podr'ia dar el caso
que la soluci'on diverja en $\rho=0$. Si tomamos $+$, puede normalizarse.
Para $l=0$, la funci'on de onda es real si $Z\alpha<\frac{1}{2}$.
\end{itemize}

Haremos el siguiente anzats:
\begin{equation}
R_{l}=N\rho^{\frac{1}{2}+\mu}e^{-\rho/2}f\left(  \rho\right)
\end{equation}
Reemplazando en
\begin{equation}
\left(  d_{\rho}^{2}-\frac{\mu^{2}-\frac{1}{4}}{\rho^{2}}+\frac{\lambda}{\rho
}-\frac{1}{4}\right)  R_{l}\left(  \rho\right)  =0
\end{equation}
se supone que se llega a
\begin{equation}
\frac{d^{2}f}{d\rho^{2}}+\left(  \frac{2\mu+1}{\rho}-1\right)  \frac{df}%
{d\rho}-\frac{\mu+\frac{1}{2}-\lambda}{\rho}f=0
\end{equation}


Cambios de variable:
\begin{align*}
2\mu+1  & =c\\
\mu+\frac{1}{2}-\lambda & =a
\end{align*}
\begin{equation}
\frac{d^{2}f}{d\rho^{2}}+\left(  \frac{c}{\rho}-1\right)  \frac{df}{d\rho
}-\frac{a}{\rho}f=0
\end{equation}


Podemos, para tratar de solucionar esta ecuaci'on, de expresar a $f$ en
series:
\begin{equation}
f=\sum_{n=0}^{\infty}a_{n}\rho^{n}%
\end{equation}
\begin{align*}
d_{\rho}f  & =\sum_{n=0}^{\infty}a_{n}n\rho^{n-1}=\sum_{n=1}^{\infty}%
a_{n}n\rho^{n-1}\\
d_{\rho}^{2}f  & =\sum_{n=0}^{\infty}a_{n}n\left(  n-1\right)  \rho^{n-2}%
=\sum_{n=2}^{\infty}a_{n}n\left(  n-1\right)  \rho^{n-2}%
\end{align*}
\begin{align*}
\sum_{n=2}^{\infty}a_{n}n\left(  n-1\right)  \rho^{n-2}+\left(  \frac{c}{\rho
}-1\right)  \sum_{n=1}^{\infty}a_{n}n\rho^{n-1}-\frac{a}{\rho}\sum
_{n=0}^{\infty}a_{n}\rho^{n}  & =0\\
\sum_{n=2}^{\infty}a_{n}n\left(  n-1\right)  \rho^{n-2}+\frac{c}{\rho}%
\sum_{n=1}^{\infty}a_{n}n\rho^{n-1}-\sum_{n=1}^{\infty}a_{n}n\rho^{n-1}%
-\frac{a}{\rho}\sum_{n=0}^{\infty}a_{n}\rho^{n}  & =0\\
\sum_{n=2}^{\infty}a_{n}n\left(  n-1\right)  \rho^{n-2}+c\sum_{n=1}^{\infty
}a_{n}n\rho^{n-2}-\sum_{n=1}^{\infty}a_{n}n\rho^{n-1}-a\sum_{n=0}^{\infty
}a_{n}\rho^{n-1}  & =0
\end{align*}
\begin{align*}
m  & =n-2\\
p  & =n-1
\end{align*}
\begin{align*}
\sum_{m=0}^{\infty}a_{m+2}\left(  m+2\right)  \left(  m+1\right)  \rho
^{m}+c\sum_{m=-1}^{\infty}a_{m+2}\left(  m+2\right)  \rho^{m}-\sum
_{p=0}^{\infty}a_{p+1}\left(  p+1\right)  \rho^{p}-a\sum_{p=-1}^{\infty
}a_{p+1}\rho^{p}  & =0\\
\sum_{m=0}^{\infty}a_{m+2}\left(  m+2\right)  \left(  m+1\right)  \rho
^{m}+c\sum_{m=-1}^{\infty}a_{m+2}\left(  m+2\right)  \rho^{m}-\sum
_{m=0}^{\infty}a_{m+1}\left(  m+1\right)  \rho^{m}-a\sum_{m=-1}^{\infty
}a_{m+1}\rho^{m}  & =0\\
a_{m+2}\left(  m+2\right)  \left(  m+1\right)  +ca_{m+2}\left(  m+2\right)
-a_{m+1}\left(  m+1\right)  -aa_{m+1}  & =0
\end{align*}
\begin{equation}
a_{m+2}\left(  m+2\right)  \left(  m+1\right)  +ca_{m+2}\left(  m+2\right)
=a_{m+1}\left(  m+1\right)  +aa_{m+1}%
\end{equation}
\begin{equation}
n=m+2
\end{equation}
\begin{align*}
a_{n}n\left(  n-1\right)  +ca_{n}n  & =a_{n-1}\left(  n-1\right)  +aa_{n-1}\\
a_{n}n\left[  n+c-1\right]   & =a_{n-1}\left[  n+a-1\right] \\
a_{n}  & =a_{n-1}\frac{1}{n}\frac{n+a-1}{n+c-1}%
\end{align*}
\begin{align*}
a_{1}  & =a_{0}\frac{a}{c}\\
a_{2}  & =a_{1}\frac{\left[  2+a-1\right]  }{2\left[  2+c-1\right]  }\\
& =\frac{1}{2}a_{1}\frac{a+1}{c+1}\\
& =\frac{1}{2}a_{0}\frac{a}{c}\frac{a+1}{c+1}\\
a_{3}  & =\frac{1}{2}\frac{1}{3}a_{0}\frac{a}{c}\frac{a+1}{c+1}\frac{a+2}%
{c+2}\\
a_{n}  & =a_{0}\frac{1}{n!}\frac{\left(  a+n-1\right)  !}{\left(
c+n-1\right)  !}%
\end{align*}
\begin{equation}
f=a_{0}\sum_{n=0}^{\infty}\frac{1}{n!}\frac{\left(  a+n-1\right)  !}{\left(
c+n-1\right)  !}\rho^{n}%
\end{equation}
Y esta es la funci'on hipergeom\'{e}trica confluyente $_{1}F_{1}\left(
a,c;\rho\right)  $. Pero esta funci'on diverge para $\rho\rightarrow\infty$
como
\begin{equation}
_{1}F_{1}\left(  a,c;\rho\rightarrow\infty\right)  \rightarrow\frac
{\Gamma\left(  c\right)  }{\Gamma\left(  a\right)  }\rho^{a-c}e^{\rho}%
\end{equation}
Entonces, para que la normalizaci'on sea posible, debe existir alg\'{u}n
valor de $n$ en donde la serie se corte. Para ello, pedimos que $a$ sea un
n\'{u}mero entero negativo. Entonces, debe existir alg\'{u}n $n^{\prime}$ tal
que
\begin{equation}
a+n^{\prime}=0
\end{equation}
y entonces, todos los t\'{e}rminos $a_{p}$ con $p>n^{\prime}$, ser\'{a}n cero,
pues
\begin{equation}
a_{n}=a_{0}\frac{1}{n!}\frac{\left(  a+n-1\right)  !}{\left(  c+n-1\right)  !}%
\end{equation}


Ahora bien, hab'iamos definido
\begin{equation}
\mu+\frac{1}{2}-\lambda=a
\end{equation}
pero $a=-n^{\prime}$, as'i que
\begin{equation}
\mu+\frac{1}{2}-\lambda+n^{\prime}=0
\end{equation}
Pero resulta que
\begin{equation}
\lambda=\frac{2Z\alpha\varepsilon}{\hbar c\beta}%
\end{equation}
y entonces,
\begin{equation}
\mu+\frac{1}{2}+n^{\prime}=\frac{2Z\alpha\varepsilon}{\hbar c\beta}%
\end{equation}
Reemplazando
\begin{equation}
\beta=2\sqrt{\frac{m^{2}c^{4}-\varepsilon^{2}}{c^{2}\hbar^{2}}}%
\end{equation}
\begin{equation}
\mu=\sqrt{\left(  l+\frac{1}{2}\right)  ^{2}-\left(  Z\alpha\right)  ^{2}}%
\end{equation}
Se llega, tras un poco de macanaca, a:
\begin{equation}
mc^{2}\left(  1+\frac{Z^{2}\alpha^{2}}{\left(  \sqrt{\left(  l+\frac{1}%
{2}\right)  ^{2}-\left(  Z\alpha\right)  ^{2}}+\frac{1}{2}+n^{\prime}\right)
^{2}}\right)  ^{-1/2}=\varepsilon_{n^{\prime}l}%
\end{equation}
Si llamamos al n\'{u}mero cu\'{a}ntico principal
\begin{equation}
n=n^{\prime}+l+1
\end{equation}
\begin{equation}
mc^{2}\left(  1+\frac{Z^{2}\alpha^{2}}{\left(  \sqrt{\left(  l+\frac{1}%
{2}\right)  ^{2}-\left(  Z\alpha\right)  ^{2}}+\frac{1}{2}+n-l-1\right)  ^{2}%
}\right)  ^{-1/2}=\varepsilon_{n^{\prime}l}%
\end{equation}
Si expandimos en series,
\begin{equation}
mc^{2}\left(  1-\frac{Z^{2}\alpha^{2}}{2n^{2}}-\frac{Z\alpha^{4}}{2n^{4}%
}\left(  \frac{n}{l+\frac{1}{2}}-\frac{3}{4}\right)  +...\right)
^{-1/2}=\varepsilon_{n^{\prime}l}%
\end{equation}
Energ'ia de reposo, eigenvalor de Schroedinger, y correcci'on relativista.

\subsubsection{Soluci'on para un potencial esf\'{e}rico cuadrado:}%

\begin{equation}
V\left(  x\right)  =\left\{
\begin{array}
[c]{l}%
-V_{0}\text{ para }r\leq R\\
0\text{ \ \ \ \ \ para }r>R
\end{array}
\right.
\end{equation}
\begin{equation}
k\left(  x\right)  =\frac{1}{c\hbar}\sqrt{\left(  \varepsilon-V\left(
x\right)  \right)  ^{2}-m^{2}c^{4}}%
\end{equation}
\begin{align*}
k_{i}\left(  r\leq R\right)   & =\frac{1}{c\hbar}\sqrt{\left(  \varepsilon
+V_{0}\right)  ^{2}-m^{2}c^{4}}\\
k_{o}\left(  r>R\right)   & =\frac{1}{c\hbar}\sqrt{\varepsilon^{2}-m%
^{2}c^{4}}%
\end{align*}
Usamos
\begin{equation}
\left\{  \partial_{r}r^{2}\partial_{r}+r^{2}k^{2}\left(  r\right)  -l\left(
l+1\right)  \right\}  u=0
\end{equation}
y la substituci'on
\begin{equation}
\rho=k\left(  r\right)  r
\end{equation}
para llegar a la ecuaci'on diferencial de Bessel:
\begin{equation}
\left(  2\rho\frac{d}{d\rho}+\rho^{2}\frac{d^{2}}{d\rho^{2}}-l\left(
l+1\right)  +\rho^{2}\right)  u\left(  \rho\right)  =0
\end{equation}
La soluci'on para $r\leq R$ es
\begin{equation}
u\left(  \rho\right)  =Nj_{l}\left(  \rho\right)  =Nj_{l}\left(
k_{i}r\right)
\end{equation}
(ESTO ES MUY RARO. \textquestiondown SON LAS ESPECIALES? \textquestiondown Y
LA SUMATORIA?) en donde no hemos usado las funciones de Neumann $n_{l}(\rho)$ debido a que divergen en cero.

Las funciones de Bessel pueden ser calculadas en forma muy f\'{a}cil, a partir
de
\begin{align*}
j_{0}\left(  \rho\right)   & =\frac{\operatorname{sen}\rho}{\rho}\\
j_{1}\left(  \rho\right)   & =\frac{\operatorname{sen}\rho}{\rho^{2}}%
-\frac{\cos\rho}{\rho}%
\end{align*}
y usando la relaci'on de recurrencia
\begin{align*}
f_{n-1}+f_{n+1}  & =\left(  2n+1\right)  \rho^{-1}f_{n}\\
\left(  2n+1\right)  \frac{df_{n}}{d\rho}  & =nf_{n-1}-\left(  n+1\right)
f_{n+1}%
\end{align*}
relaci'on que cumplen las funciones de Bessel, de Neumann y de Hankel:
\begin{align*}
h_{n}^{\left(  1\right)  }\left(  \rho\right)   & =j_{n}\left(  \rho\right)
+in_{n}\left(  \rho\right) \\
h_{n}^{\left(  2\right)  }\left(  \rho\right)   & =j_{n}\left(  \rho\right)
-in_{n}\left(  \rho\right)
\end{align*}
Ahora, si nos preocupamos de la regi'on externa, y hacemos la
sustituci'on $\rho=ikr$ en vez de $kr$, recordando que $k$ es constante,
llegamos a
\begin{equation}
\frac{d^{2}u}{d\rho^{2}}+2\rho^{-1}\frac{du}{d\rho}-\left(  1+\frac{l\left(
l+1\right)  }{\rho^{2}}\right)  u=0
\end{equation}
A lo que no hay que llegar. Si no hacemos eso, sino que consideramos la
ecuaci'on diferencial de Bessel,
\begin{equation}
\left(  2\rho\frac{d}{d\rho}+\rho^{2}\frac{d^{2}}{d\rho^{2}}-l\left(
l+1\right)  +\rho^{2}\right)  u\left(  \rho\right)  =0
\end{equation}
podemos reescribirla como
\begin{equation}
2\rho^{-1}\frac{du}{d\rho}+\frac{d^{2}u}{d\rho^{2}}+\left[  1-\frac{l\left(
l+1\right)  }{\rho^{2}}\right]  u=0
\end{equation}
que si no fuera por el exponente $-1$ ser'ia lo del libro. La soluci'on
general de esa ecuaci'on es
\begin{equation}
h_{l}^{\left(  1\right)  }\left(  \rho\right)  =j_{l}\left(  \rho\right)
+in_{l}\left(  \rho\right)
\end{equation}
y no se usa $h_{l}^{\left(  2\right)  }$, pues, \textexclamdown diverge en
$\rho=\infty$!! Para un argumento imaginario, la funci'on de Hankel de
primer tipo est\'{a} dada por
\begin{align*}
h_{0}^{\left(  1\right)  }\left(  i\beta r\right)   & =-\frac{1}{\beta
r}e^{-\beta r}\\
h_{1}^{\left(  1\right)  }\left(  i\beta r\right)   & =i\left(  \frac{1}{\beta
r}+\frac{1}{\beta^{2}r^{2}}\right)  e^{-\beta r}\\
h_{2}^{\left(  1\right)  }\left(  i\beta r\right)   & =i\left(  \frac{1}{\beta
r}+\frac{3}{\beta^{2}r^{2}}+\frac{3}{\beta^{3}r^{3}}\right)  e^{-\beta r}%
\end{align*}
y pa' mas, podemos usar la relaci'on de recurrencia.

Para determinar el eigenvalor de energ'ia $\varepsilon$, requerimos que
sea iguales la funci'on y su derivada en $r=R$, as'i que
\begin{equation}
\frac{1}{u_{i}}\frac{du_{i}}{dr}=\frac{1}{u_{o}}\frac{du_{o}}{dr}\text{ en
}r=R\text{.}%
\end{equation}
para que las constantes de normalizaci'on se cancelen. As'i que se debe
cumplir
\begin{equation}
\frac{j_{l}^{\prime}\left(  k_{i}R\right)  }{j_{l}\left(  k_{i}R\right)
}=\frac{h_{l}^{\left(  1\right)  \prime}\left(  ik_{o}R\right)  }%
{h_{l}^{\left(  1\right)  }\left(  ik_{o}R\right)  }%
\end{equation}
para el nivel base, $l=0$,
\begin{equation}
k_{i}\cot\left(  k_{i}R\right)  =-k_{o}%
\end{equation}
e iterando uno puede obtener m\'{a}s ecuaciones trascendentes de donde
despejar $\varepsilon$.

\subsection{Gauge:}

Bajo la transformaci'on
\begin{equation}
A_{\mu}\rightarrow A_{\mu^{\prime}}=A_{\mu}+\partial_{\mu}\chi\left(
x\right)
\end{equation}
se puede ver, reemplazando en la ecuaci'on K-G, que
\begin{equation}
\psi\left(  x\right)  \rightarrow\psi^{\prime}\left(  x\right)  =e^{\frac
{ie}{\hbar c}\chi\left(  x\right)  }\psi\left(  x\right)
\end{equation}

\section{Problemas.}

\begin{enumerate}
\item El campo real de Klein-Gordon es descrito por la densidad Hamiltoniana
\begin{equation}
\mathcal{H}=\frac{1}{2}\left[ c^{2}\pi ^{2}(x) +\left( {\bf %
\nabla }\phi \right) ^{2}+\mu ^{2}\phi ^{2}\right]
\end{equation}

 Use las relaciones de conmutaci'on
\begin{eqnarray*}
\left[ \phi _{r}\left( \vec{x},t\right) ,\pi _{s}\left( \vec{x}%
^{\prime },t\right) \right] &=&i\hbar \delta _{rs}\delta \left( \vec{x}%
-\vec{x}^{\prime }\right) \\
\left[ \phi _{r}\left( \vec{x},t\right) ,\phi _{s}\left( \vec{x}%
^{\prime },t\right) \right] &=&0 \\
\left[ \pi _{r}\left( \vec{x},t\right) ,\pi _{s}\left( \vec{x}^{\prime
},t\right) \right] &=&0
\end{eqnarray*}
para demostrar que
\begin{eqnarray*}
\left[ H,\phi (x) \right] &=&-i\hbar c^{2}\pi (x) 
\\
\left[ H,\pi (x) \right] &=&i\hbar \left( \mu ^{2}-{\bf %
\nabla }^{2}\right) \phi (x) 
\end{eqnarray*}
donde $H$ es el Hamiltoniano del campo.

\textbf{Soluci'on}:

Sabemos que
\begin{equation}
H=\int d^{3}\vec{x}\mathcal{H}
\end{equation}
con
\begin{equation}
\mathcal{H}=\frac{1}{2}\left[ c^{2}\pi ^{2}-\left( \partial _{s}\phi \right)
\left( \partial ^{s}\phi \right) +\mu ^{2}\phi ^{2}\right] 
\end{equation}
Entonces,
\begin{eqnarray*}
\left[ H,\phi \left( \vec{x}\right) \right] &=&\left[ \int d^{3}{\bf x%
}\frac{1}{2}\left[ c^{2}\pi ^{2}-\left( \partial _{s}\phi \right) \left(
\partial ^{s}\phi \right) +\mu ^{2}\phi ^{2}\right] ,\phi \left( \vec{x}%
^{\prime }\right) \right] \\
&=&\left[ \int d^{3}\vec{x}\frac{1}{2}\left[ c^{2}\pi ^{2}-\left(
\partial _{s}\phi \right) \left( \partial ^{s}\phi \right) +\mu ^{2}\phi ^{2}%
\right] ,\phi \left( \vec{x}^{\prime }\right) \right] \\
&=&\frac{1}{2}c^{2}\left[ \int d^{3}\vec{x}\pi ^{2},\phi \left( \vec{x}%
^{\prime }\right) \right] -\frac{1}{2}\left[ \int d^{3}\vec{x}\left(
\partial _{s}\phi \right) \left( \partial ^{s}\phi \right) ,\phi \left( 
\vec{x}^{\prime }\right) \right] + \\
&&+\frac{1}{2}\mu ^{2}\left[ \int d^{3}\vec{x}\phi ,\phi ^{2}\left( 
\vec{x}^{\prime }\right) \right] \\
&=&\frac{1}{2}c^{2}\left[ \int d^{3}\vec{x}\pi ^{2},\phi \left( \vec{x}%
^{\prime }\right) \right] -\frac{1}{2}\left[ \int d^{3}\vec{x}\left(
\partial _{s}\phi \right) \left( \partial ^{s}\phi \right) ,\phi \left( 
\vec{x}^{\prime }\right) \right] \\
&=&\frac{1}{2}c^{2}\int d^{3}\vec{x}\left[ \pi ^{2},\phi \left( \vec{x}%
^{\prime }\right) \right] -\frac{1}{2}\int d^{3}\vec{x}\left[ \left(
\partial _{s}\phi \right) \left( \partial ^{s}\phi \right) ,\phi \left( 
\vec{x}^{\prime }\right) \right] 
\end{eqnarray*}
Como
\begin{equation}
\left[ AB,C\right] =A\left[ B,C\right] +\left[ A,C\right] B
\end{equation}
entonces
\begin{eqnarray*}
\left[ H,\phi \left( \vec{x}\right) \right] &=&\frac{1}{2}c^{2}\int d^{3}%
\vec{x}\left[ \pi ,\phi \left( \vec{x}^{\prime }\right) \right] \pi +%
\frac{1}{2}c^{2}\int d^{3}\vec{x}\pi \left[ \pi ,\phi \left( \vec{x}%
^{\prime }\right) \right] + \\
&&-\frac{1}{2}\int d^{3}\vec{x}\left( \partial _{s}\phi \right) \left[
\left( \partial ^{s}\phi \right) ,\phi \left( \vec{x}^{\prime }\right) %
\right] -\frac{1}{2}\int d^{3}\vec{x}\left[ \left( \partial _{s}\phi
\right) ,\phi \left( \vec{x}^{\prime }\right) \right] \left( \partial
^{s}\phi \right) \\
&=&c^{2}\int d^{3}\vec{x}\left[ \pi ,\phi \left( \vec{x}^{\prime
}\right) \right] \pi -\int d^{3}\vec{x}\left[ \partial _{s}\phi ,\phi
\left( \vec{x}^{\prime }\right) \right] \partial ^{s}\phi 
\end{eqnarray*}
Como
\begin{eqnarray*}
\left[ \phi _{r}\left( \vec{x},t\right) ,\pi _{s}\left( \vec{x}%
^{\prime },t\right) \right] &=&i\hbar \delta _{rs}\delta \left( \vec{x}%
-\vec{x}^{\prime }\right) \\
\left[ \phi _{r}\left( \vec{x},t\right) ,\phi _{s}\left( \vec{x}%
^{\prime },t\right) \right] &=&0 \\
\left[ \pi _{r}\left( \vec{x},t\right) ,\pi _{s}\left( \vec{x}^{\prime
},t\right) \right] &=&0
\end{eqnarray*}
entonces
\begin{eqnarray*}
\left[ H,\phi \left( \vec{x}^{\prime }\right) \right] &=&c^{2}\int d^{3}%
\vec{x}\left[ \pi \left( \vec{x}\right) ,\phi \left( \vec{x}%
^{\prime }\right) \right] \pi \left( \vec{x}\right) -\int d^{3}\vec{x}%
\left[ \partial _{s}\phi \left( \vec{x}\right) ,\phi \left( \vec{x}%
^{\prime }\right) \right] \partial ^{s}\phi \left( \vec{x}\right) \\
&=&-ic^{2}\hbar \int d^{3}\vec{x}\delta \left( \vec{x}-\vec{x}%
^{\prime }\right) \pi \left( \vec{x}\right) -\int d^{3}\vec{x}\left[
\partial _{s}\phi \left( \vec{x}\right) ,\phi \left( \vec{x}^{\prime
}\right) \right] \partial ^{s}\phi \left( \vec{x}\right) \\
&=&-ic^{2}\hbar \pi \left( \vec{x}^{\prime }\right) -\int d^{3}{\bf x%
}\left[ \partial _{s}\phi \left( \vec{x}\right) ,\phi \left( \vec{x}%
^{\prime }\right) \right] \partial ^{s}\phi \left( \vec{x}\right) 
\end{eqnarray*}
Ahora bien, recordemos que
\begin{equation}
\left[ A\left( \vec{x},t\right) ,B\left( {\bf y},t\right) \right]
=i\hbar \left( \frac{\delta A\left( \vec{x},t\right) }{\delta \phi
^{A}\left( {\bf z},t\right) }\frac{\delta B\left( {\bf y},t\right) }{%
\delta \pi _{A}\left( {\bf z},t\right) }-\frac{\delta A\left( \vec{x}%
,t\right) }{\delta \pi _{A}\left( {\bf z},t\right) }\frac{\delta B\left( 
{\bf y},t\right) }{\delta \phi ^{A}\left( {\bf z},t\right) }\right) 
\end{equation}
en donde $\delta $ representa la derivada funcional.

As'i,
\begin{eqnarray*}
\left[ \partial _{s}\phi \left( \vec{x}\right) ,\phi \left( \vec{x}%
^{\prime }\right) \right] &=&i\hbar \left( \frac{\delta \partial _{s}\phi
\left( \vec{x},t\right) }{\delta \phi \left( {\bf z},t\right) }\frac{%
\delta \phi \left( \vec{x}^{\prime },t\right) }{\delta \pi \left( {\bf %
z},t\right) }-\frac{\delta \partial _{s}\phi \left( \vec{x},t\right) }{%
\delta \pi \left( {\bf z},t\right) }\frac{\delta \phi \left( \vec{x}%
^{\prime },t\right) }{\delta \phi \left( {\bf z},t\right) }\right) \\
&=&-i\hbar \frac{\delta \partial _{s}\phi \left( \vec{x},t\right) }{%
\delta \pi \left( {\bf z},t\right) }\frac{\delta \phi \left( \vec{x}%
^{\prime },t\right) }{\delta \phi \left( {\bf z},t\right) } \\
&=&-i\hbar \frac{\partial _{s}\delta \phi \left( \vec{x},t\right) }{%
\delta \pi \left( {\bf z},t\right) }\frac{\delta \phi \left( \vec{x}%
^{\prime },t\right) }{\delta \phi \left( {\bf z},t\right) } \\
&=&0
\end{eqnarray*}
y por lo tanto,
\begin{equation}
\left[ H,\phi \left( \vec{x}\right) \right] =-i\hbar c^{2}\pi \left( 
\vec{x}\right) 
\end{equation}

De la misma forma,
\begin{eqnarray*}
\left[ H\left( \vec{x}\right) ,\pi \left( \vec{x}^{\prime }\right) %
\right] &=&\frac{1}{2}c^{2}\left[ \int d^{3}\vec{x}\pi ^{2}\left( 
\vec{x}\right) ,\pi \left( \vec{x}^{\prime }\right) \right] -\frac{1}{2%
}\left[ \int d^{3}\vec{x}\left( \partial _{s}\phi \left( \vec{x}%
\right) \right) \left( \partial ^{s}\phi \left( \vec{x}\right) \right)
,\pi \left( \vec{x}^{\prime }\right) \right] +\frac{1}{2}\mu ^{2}\left[
\int d^{3}\vec{x}\phi ^{2}\left( \vec{x}\right) ,\pi \left( \vec{x}%
^{\prime }\right) \right] \\
&=&-\frac{1}{2}\int d^{3}\vec{x}\left[ \left( \partial _{s}\phi \left( 
\vec{x}\right) \right) \left( \partial ^{s}\phi \left( \vec{x}\right)
\right) ,\pi \left( \vec{x}^{\prime }\right) \right] +\frac{1}{2}\mu
^{2}\int d^{3}\vec{x}\left[ \phi ^{2}\left( \vec{x}\right) ,\pi \left( 
\vec{x}^{\prime }\right) \right] \\
&=&-\frac{1}{2}\int d^{3}\vec{x}\left( \partial _{s}\phi \left( \vec{x}%
\right) \right) \left[ \partial ^{s}\phi \left( \vec{x}\right) ,\pi
\left( \vec{x}^{\prime }\right) \right] -\frac{1}{2}\int d^{3}\vec{x}%
\left[ \partial _{s}\phi \left( \vec{x}\right) ,\pi \left( \vec{x}%
^{\prime }\right) \right] \partial ^{s}\phi \left( \vec{x}\right) + \\
&&+\frac{1}{2}\mu ^{2}\int d^{3}\vec{x}\phi \left( \vec{x}\right) %
\left[ \phi \left( \vec{x}\right) ,\pi \left( \vec{x}^{\prime }\right) %
\right] +\frac{1}{2}\mu ^{2}\int d^{3}\vec{x}\left[ \phi \left( \vec{x}%
\right) ,\pi \left( \vec{x}^{\prime }\right) \right] \phi \left( {\bf x%
}\right) \\
&=&-\int d^{3}\vec{x}\left[ \partial _{s}\phi \left( \vec{x}\right)
,\pi \left( \vec{x}^{\prime }\right) \right] \partial ^{s}\phi \left( 
\vec{x}\right) +\mu ^{2}\int d^{3}\vec{x}\left[ \phi \left( \vec{x}%
\right) ,\pi \left( \vec{x}^{\prime }\right) \right] \phi \left( {\bf x%
}\right) \\
&=&-\int d^{3}\vec{x}\left[ \partial _{s}\phi \left( \vec{x}\right)
,\pi \left( \vec{x}^{\prime }\right) \right] \partial ^{s}\phi \left( 
\vec{x}\right) +i\mu ^{2}\hbar \int d^{3}\vec{x}\delta \left( 
\vec{x}-\vec{x}^{\prime }\right) \phi \left( \vec{x}\right) \\
&=&-\int d^{3}\vec{x}\left[ \partial _{s}\phi \left( \vec{x}\right)
,\pi \left( \vec{x}^{\prime }\right) \right] \partial ^{s}\phi \left( 
\vec{x}\right) +i\mu ^{2}\hbar \phi \left( \vec{x}^{\prime }\right) 
\end{eqnarray*}
y como
\begin{eqnarray*}
\left[ \partial _{s}\phi \left( \vec{x}\right) ,\pi \left( \vec{x}%
^{\prime }\right) \right] &=&i\hbar \left( \frac{\delta \partial _{s}\phi
\left( \vec{x},t\right) }{\delta \phi \left( {\bf z},t\right) }\frac{%
\delta \pi \left( \vec{x}^{\prime },t\right) }{\delta \pi \left( {\bf z%
},t\right) }-\frac{\delta \partial _{s}\phi \left( \vec{x},t\right) }{%
\delta \pi \left( {\bf z},t\right) }\frac{\delta \pi \left( \vec{x}%
^{\prime },t\right) }{\delta \phi \left( {\bf z},t\right) }\right) \\
&=&i\hbar \left( \frac{\partial _{s}\delta \phi \left( \vec{x},t\right) 
}{\delta \phi \left( {\bf z},t\right) }\frac{\delta \pi \left( \vec{x}%
^{\prime },t\right) }{\delta \pi \left( {\bf z},t\right) }\right) \\
&=&i\hbar \left( \partial _{s}\left[ \delta \left( \vec{x}-{\bf z}%
\right) \right] \delta \left( \vec{x}^{\prime }-{\bf z}\right) \right) 
\\
&=&i\hbar \partial _{s}\left[ \delta \left( \vec{x}-\vec{x}^{\prime
}\right) \right] \\
&=&-i\hbar \partial _{s^{\prime }}\left[ \delta \left( \vec{x}^{\prime
}-\vec{x}\right) \right] \\
&=&-i\hbar \delta \left( \vec{x}^{\prime }-\vec{x}\right) \partial
_{s^{\prime }}
\end{eqnarray*}
entonces
\begin{eqnarray*}
\left[ H\left( \vec{x}\right) ,\pi \left( \vec{x}^{\prime }\right) %
\right] &=&i\hbar \int d^{3}\vec{x}\delta \left( \vec{x}^{\prime }-%
\vec{x}\right) \partial _{s^{\prime }}\partial ^{s}\phi \left( \vec{x}%
\right) +i\mu ^{2}\hbar \phi \left( \vec{x}^{\prime }\right) \\
&=&i\hbar \partial _{s^{\prime }}\partial ^{s^{\prime }}\phi \left( 
\vec{x}^{\prime }\right) +i\mu ^{2}\hbar \phi \left( \vec{x}^{\prime
}\right) \\
&=&i\hbar \left( \mu ^{2}+\partial _{s^{\prime }}\partial ^{s^{\prime
}}\right) \phi \left( \vec{x}^{\prime }\right) \\
&=&i\hbar \left( \mu ^{2}-\nabla ^{2}\right) \phi \left( \vec{x}%
^{\prime }\right) 
\end{eqnarray*}
\begin{equation}
\left[ H,\pi \left( \vec{x}\right) \right] =i\hbar \left( \mu
^{2}-\nabla ^{2}\right) \phi \left( \vec{x}\right) 
\end{equation}

\item De este resultado, y de las ecuaciones de movimiento de Heisemberg
para los operadores $\phi (x) $ y $\pi (x) $,
muestre que
\begin{eqnarray*}
\dot{\phi}(x) &=&c^{2}\pi (x) \\
\left( \square +\mu ^{2}\right) \phi (x) &=&0
\end{eqnarray*}

\textbf{Soluci'on}:

En efecto, como
\begin{eqnarray*}
i\hbar \frac{d\phi }{dt} &=&-\left[ H,\phi \right] \\
i\hbar \frac{d\pi }{dt} &=&-\left[ H,\pi \right] 
\end{eqnarray*}
entonces
\begin{eqnarray*}
i\hbar \frac{d\phi }{dt} &=&-i\hbar c^{2}\pi \left( \vec{x}\right) \\
i\hbar \frac{d\pi }{dt} &=&i\hbar \left( \mu ^{2}-\nabla ^{2}\right)
\phi 
\end{eqnarray*}
y
\begin{eqnarray*}
\dot{\phi} &=&-c^{2}\pi \\
\dot{\pi} &=&\left( \mu ^{2}-\nabla ^{2}\right) \phi 
\end{eqnarray*}
Y como de la primera ecuaci'on
\begin{equation}
\dot{\pi}=-\frac{1}{c^{2}}\ddot{\phi}
\end{equation}
entonces
\begin{eqnarray*}
-\frac{1}{c^{2}}\ddot{\phi} &=&\mu ^{2}\phi -\nabla ^{2}\phi \\
\left( \square +\mu ^{2}\right) \phi &=&0
\end{eqnarray*}
\end{enumerate}

Muestre que la densidad Lagrangiana
\begin{equation}
\mathcal{L}=-\frac{1}{2}\left[ \partial _{\alpha }\phi _{\beta }\right] %
\left[ \partial ^{\alpha }\phi ^{\beta }\right] +\frac{1}{2}\left[ \partial
_{\alpha }\phi ^{\alpha }\right] ^{2}+\frac{1}{2}\mu ^{2}\phi _{\alpha
}(x) \phi ^{\alpha }(x) 
\end{equation}

para el campo vectorial real $\phi ^{\alpha }(x) $ nos dirige a

las ecuaciones de campo
\begin{equation}
\left[ \eta _{\alpha \beta }\left( \square +\mu ^{2}\right) -\partial
_{\alpha }\partial _{\beta }\right] \phi ^{\beta }(x) =0
\end{equation}

\textbf{Soluci'on}:

Las ecuaciones de campo son
\begin{equation}
\frac{\partial \mathcal{L}}{\partial \phi ^{\gamma }}-\partial _{\lambda }%
\frac{\partial \mathcal{L}}{\partial \partial _{\lambda }\phi ^{\gamma }}=0
\end{equation}
Escribiremos la densidad lagrangiana de una forma m'as c'omoda:
\begin{equation}
\mathcal{L}=-\frac{1}{2}\eta ^{\mu \alpha }\eta _{\nu \beta }\left[ \partial
_{\mu }\phi ^{\nu }\right] \left[ \partial _{\alpha }\phi ^{\beta }\right] +%
\frac{1}{2}\left[ \partial _{\alpha }\phi ^{\alpha }\right] ^{2}+\frac{1}{2}%
\mu ^{2}\eta _{\mu \nu }\phi ^{\mu }\phi ^{\nu }
\end{equation}
y entonces
\begin{eqnarray*}
\frac{\partial \mathcal{L}}{\partial \phi ^{\gamma }} &=&\frac{1}{2}\mu
^{2}\eta _{\mu \nu }\frac{\partial }{\partial \phi ^{\gamma }}\left( \phi
^{\mu }\phi ^{\nu }\right) \\
&=&\frac{1}{2}\mu ^{2}\eta _{\mu \nu }\left( \delta _{\gamma }^{\mu }\phi
^{\nu }+\phi ^{\mu }\delta _{\gamma }^{\nu }\right) \\
&=&\frac{1}{2}\mu ^{2}\left( \eta _{\mu \nu }\delta _{\gamma }^{\mu }\phi
^{\nu }+\eta _{\mu \nu }\phi ^{\mu }\delta _{\gamma }^{\nu }\right) \\
&=&\frac{1}{2}\mu ^{2}\left( \eta _{\gamma \nu }\phi ^{\nu }+\eta _{\mu
\gamma }\phi ^{\mu }\right) \\
&=&\mu ^{2}\phi _{\gamma }
\end{eqnarray*}
y
\begin{eqnarray*}
\frac{\partial \mathcal{L}}{\partial \partial _{\lambda }\phi ^{\gamma }} &=&%
\frac{\partial }{\partial \partial _{\lambda }\phi ^{\gamma }}\left( -\frac{1%
}{2}\eta ^{\mu \alpha }\eta _{\nu \beta }\left[ \partial _{\mu }\phi ^{\nu }%
\right] \left[ \partial _{\alpha }\phi ^{\beta }\right] +\frac{1}{2}\left[
\partial _{\alpha }\phi ^{\alpha }\right] ^{2}\right) \\
&=&\left( -\frac{1}{2}\eta ^{\mu \alpha }\eta _{\nu \beta }\frac{\partial }{%
\partial \partial _{\lambda }\phi ^{\gamma }}\left[ \partial _{\mu }\phi
^{\nu }\right] \left[ \partial _{\alpha }\phi ^{\beta }\right] -\frac{1}{2}%
\eta ^{\mu \alpha }\eta _{\nu \beta }\left[ \partial _{\mu }\phi ^{\nu }%
\right] \frac{\partial }{\partial \partial _{\lambda }\phi ^{\gamma }}\left[
\partial _{\alpha }\phi ^{\beta }\right] +\frac{1}{2}\frac{\partial }{%
\partial \partial _{\lambda }\phi ^{\gamma }}\left[ \delta _{\alpha }^{\beta
}\partial _{\beta }\phi ^{\alpha }\right] ^{2}\right) \\
&=&\left( -\frac{1}{2}\eta ^{\mu \alpha }\eta _{\nu \beta }\delta _{\mu
}^{\lambda }\delta _{\gamma }^{\nu }\partial _{\alpha }\phi ^{\beta }-\frac{1%
}{2}\eta ^{\mu \alpha }\eta _{\nu \beta }\partial _{\mu }\phi ^{\nu }\delta
_{\alpha }^{\lambda }\delta _{\gamma }^{\beta }+\frac{1}{2}\frac{\partial }{%
\partial \partial _{\lambda }\phi ^{\gamma }}\left[ \delta _{\alpha }^{\beta
}\partial _{\beta }\phi ^{\alpha }\right] \left[ \delta _{\chi }^{\rho
}\partial _{\rho }\phi ^{\chi }\right] \right) \\
&=&\left( -\frac{1}{2}\eta ^{\lambda \alpha }\eta _{\gamma \beta }\partial
_{\alpha }\phi ^{\beta }-\frac{1}{2}\eta ^{\mu \lambda }\eta _{\nu \gamma
}\partial _{\mu }\phi ^{\nu }+\frac{1}{2}\left[ \delta _{\alpha }^{\beta }%
\frac{\partial }{\partial \partial _{\lambda }\phi ^{\gamma }}\partial
_{\beta }\phi ^{\alpha }\right] \left[ \delta _{\chi }^{\rho }\partial
_{\rho }\phi ^{\chi }\right] +\frac{1}{2}\left[ \delta _{\alpha }^{\beta
}\partial _{\beta }\phi ^{\alpha }\right] \left[ \delta _{\chi }^{\rho }%
\frac{\partial }{\partial \partial _{\lambda }\phi ^{\gamma }}\partial
_{\rho }\phi ^{\chi }\right] \right) \\
&=&-\partial ^{\lambda }\phi _{\gamma }+\frac{1}{2}\delta _{\alpha }^{\beta
}\delta _{\beta }^{\lambda }\delta _{\gamma }^{\alpha }\delta _{\chi }^{\rho
}\partial _{\rho }\phi ^{\chi }+\frac{1}{2}\delta _{\chi }^{\rho }\delta
_{\rho }^{\lambda }\delta _{\gamma }^{\chi }\delta _{\alpha }^{\beta
}\partial _{\beta }\phi ^{\alpha } \\
&=&-\partial ^{\lambda }\phi _{\gamma }+\delta _{\gamma }^{\lambda }\partial
_{\rho }\phi ^{\rho }
\end{eqnarray*}
y entonces,
\begin{eqnarray*}
-\partial _{\lambda }\frac{\partial \mathcal{L}}{\partial \partial _{\lambda
}\phi ^{\gamma }} &=&\partial _{\lambda }\partial ^{\lambda }\phi _{\gamma
}-\delta _{\gamma }^{\lambda }\partial _{\lambda }\partial _{\rho }\phi
^{\rho } \\
&=&\partial _{\lambda }\partial ^{\lambda }\phi _{\gamma }-\partial _{\gamma
}\partial _{\rho }\phi ^{\rho }
\end{eqnarray*}
y as'i, las ecuaciones de movimiento, son
\begin{equation}
\mu ^{2}\phi _{\gamma }+\partial _{\lambda }\partial ^{\lambda }\phi
_{\gamma }-\partial _{\gamma }\partial _{\rho }\phi ^{\rho }=0
\end{equation}

 y a que el campo $\phi ^{\alpha }(x) $ satisface la
condici'on de Lorentz,
\begin{equation}
\partial _{\alpha }\phi ^{\alpha }(x) =0
\end{equation}

\textbf{Soluci'on}:

Primero que nada, demostraremos el Teorema de Noether.

Bajo un cierto grupo de transformaciones, las coordenadas transforman
infinitesimalmente de la forma
\begin{equation}
x^{\mu }\rightarrow x^{\mu ^{\prime }}=x^{\mu }+\delta x^{\mu }
\end{equation}
y los campos transforman como
\begin{equation}
\phi ^{A}(x) \rightarrow \phi ^{A^{\prime }}\left( x^{\prime
}\right) =\phi ^{A}(x) +\delta _{T}\phi ^{A}(x) 
\end{equation}
Y as'i,
\begin{eqnarray*}
\delta _{T}\phi ^{A}(x) &=&\phi ^{A^{\prime }}\left( x^{\prime
}\right) -\phi ^{A}(x) \\
&=&\left[ \phi ^{A^{\prime }}\left( x^{\prime }\right) -\phi ^{A}\left(
x^{\prime }\right) \right] +\left[ \phi ^{A}\left( x^{\prime }\right) -\phi
^{A}(x) \right] \\
&=&\delta \phi ^{A}\left( x^{\prime }\right) +\delta x^{\mu }\partial _{\mu
}\phi ^{A}(x) \\
&=&\delta \phi ^{A}(x) +\delta x^{\mu }\partial _{\mu }\delta
\phi ^{A}(x) +\delta x^{\mu }\partial _{\mu }\phi ^{A}\left(
x\right) 
\end{eqnarray*}
y si tomamos en cuenta s'olo t'erminos de primer orden,
\begin{equation}
\delta _{T}\phi ^{A}(x) =\delta \phi ^{A}(x) +\delta
x^{\mu }\partial _{\mu }\phi ^{A}(x) 
\end{equation}
Si consideramos a la densidad Lagrangiana como un campo,
\begin{equation}
\delta _{T}\mathcal{L}(x) =\delta \mathcal{L}(x)
+\delta x^{\mu }\partial _{\mu }\mathcal{L}(x) 
\end{equation}
Ahora si queremos que bajo el grupo de transformaciones la densidad
Lagrangiana sea invariante,
\begin{equation}
\delta \mathcal{L}(x) +\delta x^{\mu }\partial _{\mu }\mathcal{L}%
(x) =0
\end{equation}
Y como
\begin{eqnarray*}
\delta \mathcal{L} &=&\frac{\partial \mathcal{L}}{\partial \phi ^{A}}\delta
\phi ^{A}+\frac{\partial \mathcal{L}}{\partial \partial _{\mu }\phi ^{A}}%
\delta \partial _{\mu }\phi ^{A} \\
&=&\frac{\partial \mathcal{L}}{\partial \phi ^{A}}\delta \phi ^{A}+\partial
_{\mu }\left( \frac{\partial \mathcal{L}}{\partial \partial _{\mu }\phi ^{A}}%
\delta \phi ^{A}\right) -\partial _{\mu }\frac{\partial \mathcal{L}}{%
\partial \partial _{\mu }\phi ^{A}}\delta \phi ^{A} \\
&=&\left( \frac{\partial \mathcal{L}}{\partial \phi ^{A}}-\partial _{\mu }%
\frac{\partial \mathcal{L}}{\partial \partial _{\mu }\phi ^{A}}\right)
\delta \phi ^{A}+\partial _{\mu }\left( \frac{\partial \mathcal{L}}{\partial
\partial _{\mu }\phi ^{A}}\delta \phi ^{A}\right) 
\end{eqnarray*}
Y as'i, sobre las ecuaciones de movimiento se cumple que
\begin{equation}
\delta \mathcal{L}=\partial _{\mu }\left( \frac{\partial \mathcal{L}}{%
\partial \partial _{\mu }\phi ^{A}}\delta \phi ^{A}\right) 
\end{equation}
Y como
\begin{eqnarray*}
\delta \mathcal{L}+\delta x^{\mu }\partial _{\mu }\mathcal{L} &=&0 \\
\partial _{\mu }\left( \frac{\partial \mathcal{L}}{\partial \partial _{\mu
}\phi ^{A}}\delta \phi ^{A}\right) +\delta x^{\mu }\partial _{\mu }\mathcal{L%
} &=&0 \\
\partial _{\mu }\left( \frac{\partial \mathcal{L}}{\partial \partial _{\mu
}\phi ^{A}}\delta \phi ^{A}\right) +\delta x^{\mu }\partial _{\mu }\mathcal{L%
} &=&0
\end{eqnarray*}
Como
\begin{equation}
\partial _{\mu }\left( \delta x^{\mu }\mathcal{L}\right) -\partial _{\mu
}\delta x^{\mu }\mathcal{L}=\delta x^{\mu }\partial _{\mu }\mathcal{L}
\end{equation}
entonces
\begin{equation}
\partial _{\mu }\left( \frac{\partial \mathcal{L}}{\partial \partial _{\mu
}\phi ^{A}}\delta \phi ^{A}+\delta x^{\mu }\mathcal{L}\right) -\partial
_{\mu }\delta x^{\mu }\mathcal{L}=0
\end{equation}
Y como
\begin{equation}
\delta \phi ^{A}(x) =\delta _{T}\phi ^{A}(x) -\delta
x^{\mu }\partial _{\mu }\phi ^{A}(x) 
\end{equation}
entonces
\begin{equation}
\partial _{\mu }\left( \frac{\partial \mathcal{L}}{\partial \partial _{\mu
}\phi ^{A}}\left[ \delta _{T}\phi ^{A}(x) -\delta x^{\lambda
}\partial _{\lambda }\phi ^{A}(x) \right] +\delta x^{\mu }%
\mathcal{L}\right) -\partial _{\mu }\delta x^{\mu }\mathcal{L}=0
\end{equation}
Y si consideramos s'olo t'erminos de primer orden,
\begin{equation}
\partial _{\mu }\delta x^{\mu }=0
\end{equation}
y as'i,
\begin{equation}
\partial _{\mu }F^{\mu }=0
\end{equation}
con
\begin{eqnarray*}
F^{\mu } &=&\frac{\partial \mathcal{L}}{\partial \partial _{\mu }\phi ^{A}}%
\delta _{T}\phi ^{A}-\frac{\partial \mathcal{L}}{\partial \partial _{\mu
}\phi ^{A}}\delta x^{\lambda }\partial _{\lambda }\phi ^{A}+\delta x^{\mu }%
\mathcal{L} \\
&=&\frac{\partial \mathcal{L}}{\partial \partial _{\mu }\phi ^{A}}\delta
_{T}\phi ^{A}-\left( \frac{\partial \mathcal{L}}{\partial \partial _{\mu
}\phi ^{A}}\partial _{\lambda }\phi ^{A}-\delta _{\lambda }^{\mu }\mathcal{L}%
\right) \delta x^{\lambda }
\end{eqnarray*}
Si definimos
\begin{equation}
\mathcal{T}_{\_\lambda }^{\mu }=\frac{\partial \mathcal{L}}{\partial
\partial _{\mu }\phi ^{A}}\partial _{\lambda }\phi ^{A}-\delta _{\lambda
}^{\mu }\mathcal{L}
\end{equation}
entonces
\begin{equation}
F^{\mu }=\frac{\partial \mathcal{L}}{\partial \partial _{\mu }\phi ^{A}}%
\delta _{T}\phi ^{A}-\mathcal{T}_{\_\lambda }^{\mu }\delta x^{\lambda }
\end{equation}

Ahora, supongamos que la acci'on es invariante de Poincar'e. Entonces,
bajo una transformaci'on infinitesimal,
\begin{equation}
x^{\mu }\rightarrow x^{\mu ^{\prime }}=x^{\mu }+\omega _{\_\nu }^{\mu
}x^{\nu }+\delta ^{\mu }
\end{equation}
con
\begin{equation}
\omega ^{\mu \nu }=-\omega ^{\nu \mu }
\end{equation}
Entonces,
\begin{eqnarray*}
\delta _{T}\phi ^{\alpha }(x) &=&\delta \phi ^{\alpha }\left(
x\right) +\delta x^{\mu }\partial _{\mu }\phi ^{\alpha }(x) \\
&=&\omega _{\_\beta }^{\alpha }\phi ^{\beta }+\delta ^{\alpha }+\left(
\omega _{\_\nu }^{\mu }x^{\nu }+\delta ^{\mu }\right) \partial _{\mu }\phi
^{\alpha } \\
&=&\omega _{\_\beta }^{\alpha }\phi ^{\beta }+\omega _{\_\nu }^{\mu }x^{\nu
}\partial _{\mu }\phi ^{\alpha }+\delta ^{\alpha }+\delta ^{\mu }\partial
_{\mu }\phi ^{\alpha } \\
&=&\omega _{\_\nu }^{\mu }\left( \delta _{\mu }^{\alpha }\delta _{\beta
}^{\nu }\phi ^{\beta }+x^{\nu }\partial _{\mu }\phi ^{\alpha }\right)
+\delta ^{\mu }\left( \delta _{\mu }^{\alpha }+\partial _{\mu }\phi ^{\alpha
}\right) \\
&=&\omega _{\_\nu }^{\lambda }\left( \delta _{\lambda }^{\alpha }\delta
_{\beta }^{\nu }\phi ^{\beta }+x^{\nu }\partial _{\lambda }\phi ^{\alpha
}\right) +\delta ^{\lambda }\left( \delta _{\lambda }^{\alpha }+\partial
_{\lambda }\phi ^{\alpha }\right) 
\end{eqnarray*}
y entonces,
\begin{eqnarray*}
F^{\mu } &=&\frac{\partial \mathcal{L}}{\partial \partial _{\mu }\phi
^{\alpha }}\delta _{T}\phi ^{\alpha }-\mathcal{T}_{\_\lambda }^{\mu }\delta
x^{\lambda } \\
&=&\frac{\partial \mathcal{L}}{\partial \partial _{\mu }\phi ^{\alpha }}%
\left[ \omega _{\_\nu }^{\lambda }\left( \delta _{\lambda }^{\alpha }\delta
_{\beta }^{\nu }\phi ^{\beta }+x^{\nu }\partial _{\lambda }\phi ^{\alpha
}\right) +\delta ^{\lambda }\left( \delta _{\lambda }^{\alpha }+\partial
_{\lambda }\phi ^{\alpha }\right) \right] -\mathcal{T}_{\_\lambda }^{\mu
}\left( \omega _{\_\nu }^{\lambda }x^{\nu }+\delta ^{\lambda }\right) \\
&=&\frac{\partial \mathcal{L}}{\partial \partial _{\mu }\phi ^{\alpha }}%
\omega _{\_\nu }^{\lambda }\left( \delta _{\lambda }^{\alpha }\delta _{\beta
}^{\nu }\phi ^{\beta }+x^{\nu }\partial _{\lambda }\phi ^{\alpha }\right) +%
\frac{\partial \mathcal{L}}{\partial \partial _{\mu }\phi ^{\alpha }}\delta
^{\lambda }\left( \delta _{\lambda }^{\alpha }+\partial _{\lambda }\phi
^{\alpha }\right) -\mathcal{T}_{\_\lambda }^{\mu }\omega _{\_\nu }^{\lambda
}x^{\nu }-\mathcal{T}_{\_\lambda }^{\mu }\delta ^{\lambda } \\
&=&\frac{\partial \mathcal{L}}{\partial \partial _{\mu }\phi ^{\alpha }}%
\omega _{\_\nu }^{\lambda }\left( \delta _{\lambda }^{\alpha }\delta _{\beta
}^{\nu }\phi ^{\beta }+x^{\nu }\partial _{\lambda }\phi ^{\alpha }\right) -%
\mathcal{T}_{\_\lambda }^{\mu }\omega _{\_\nu }^{\lambda }x^{\nu }+\frac{%
\partial \mathcal{L}}{\partial \partial _{\mu }\phi ^{\alpha }}\delta
^{\lambda }\left( \delta _{\lambda }^{\alpha }+\partial _{\lambda }\phi
^{\alpha }\right) -\mathcal{T}_{\_\lambda }^{\mu }\delta ^{\lambda } \\
&=&\omega _{\_\nu }^{\lambda }\left[ \frac{\partial \mathcal{L}}{\partial
\partial _{\mu }\phi ^{\alpha }}\left( \delta _{\lambda }^{\alpha }\delta
_{\beta }^{\nu }\phi ^{\beta }+x^{\nu }\partial _{\lambda }\phi ^{\alpha
}\right) -\mathcal{T}_{\_\lambda }^{\mu }x^{\nu }\right] +\delta ^{\lambda }%
\left[ \frac{\partial \mathcal{L}}{\partial \partial _{\mu }\phi ^{\alpha }}%
\left( \delta _{\lambda }^{\alpha }+\partial _{\lambda }\phi ^{\alpha
}\right) -\mathcal{T}_{\_\lambda }^{\mu }\right] \\
&=&\omega ^{\lambda \nu }\left[ \frac{\partial \mathcal{L}}{\partial
\partial _{\mu }\phi ^{\alpha }}\left( \delta _{\lambda }^{\alpha }\eta
_{\nu \beta }\phi ^{\beta }+x_{\nu }\partial _{\lambda }\phi ^{\alpha
}\right) -\mathcal{T}_{\_\lambda }^{\mu }x_{\nu }\right] +\delta ^{\lambda }%
\left[ \frac{\partial \mathcal{L}}{\partial \partial _{\mu }\phi ^{\alpha }}%
\left( \delta _{\lambda }^{\alpha }+\partial _{\lambda }\phi ^{\alpha
}\right) -\mathcal{T}_{\_\lambda }^{\mu }\right] \\
&=&\omega ^{\lambda \nu }\left[ \frac{\partial \mathcal{L}}{\partial
\partial _{\mu }\phi ^{\alpha }}\left( \delta _{\lambda }^{\alpha }\phi
_{\nu }+x_{\nu }\partial _{\lambda }\phi ^{\alpha }\right) -\mathcal{T}%
_{\_\lambda }^{\mu }x_{\nu }\right] +\delta ^{\lambda }\left[ \frac{\partial 
\mathcal{L}}{\partial \partial _{\mu }\phi ^{\alpha }}\left( \delta
_{\lambda }^{\alpha }+\partial _{\lambda }\phi ^{\alpha }\right) -\mathcal{T}%
_{\_\lambda }^{\mu }\right] 
\end{eqnarray*}
como
\begin{eqnarray*}
\partial _{\mu }\left\{ \delta ^{\lambda }\left[ \frac{\partial \mathcal{L}}{%
\partial \partial _{\mu }\phi ^{\alpha }}\left( \delta _{\lambda }^{\alpha
}+\partial _{\lambda }\phi ^{\alpha }\right) -\mathcal{T}_{\_\lambda }^{\mu }%
\right] \right\} &=&\delta ^{\lambda }\partial _{\mu }\left[ \frac{\partial 
\mathcal{L}}{\partial \partial _{\mu }\phi ^{\alpha }}\left( \delta
_{\lambda }^{\alpha }+\partial _{\lambda }\phi ^{\alpha }\right) -\mathcal{T}%
_{\_\lambda }^{\mu }\right] \\
&=&\delta ^{\lambda }\left[ \partial _{\mu }\frac{\partial \mathcal{L}}{%
\partial \partial _{\mu }\phi ^{\alpha }}\left( \delta _{\lambda }^{\alpha
}+\partial _{\lambda }\phi ^{\alpha }\right) +\frac{\partial \mathcal{L}}{%
\partial \partial _{\mu }\phi ^{\alpha }}\partial _{\mu }\partial _{\lambda
}\phi ^{\alpha }-\mathcal{T}_{\_\lambda }^{\mu }\right] \\
&=&\delta ^{\lambda }\left[ \partial _{\mu }\frac{\partial \mathcal{L}}{%
\partial \partial _{\mu }\phi ^{\lambda }}+\partial _{\mu }\frac{\partial 
\mathcal{L}}{\partial \partial _{\mu }\phi ^{\alpha }}\partial _{\lambda
}\phi ^{\alpha }+\frac{\partial \mathcal{L}}{\partial \partial _{\mu }\phi
^{\alpha }}\partial _{\mu }\partial _{\lambda }\phi ^{\alpha }-\mathcal{T}%
_{\_\lambda }^{\mu }\right] 
\end{eqnarray*}
\begin{equation}
\mathcal{T}_{\_\lambda }^{\mu }=\frac{\partial \mathcal{L}}{\partial
\partial _{\mu }\phi ^{\alpha }}\partial _{\lambda }\phi ^{\alpha }-\delta
_{\lambda }^{\mu }\mathcal{L}
\end{equation}

