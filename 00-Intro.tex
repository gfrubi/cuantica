\chapter{Introducci'on}

\section{Cuantizando el campo electromagn'etico}

Queremos cuantizar el campo electromagn'etico. \textquestiondown Por
qu'e?. B'asicamente, porque la teor'ia cl'asica de Maxwell  no describe
``fotones", sino ``ondas electromagn'eticas". Por otro lado, desde el
descrubrimiento del efecto fotoel'ectrico se tiene evidencia de las
propiedades corpusculares de la luz. Estas propiedades no son
descritas por la teor'ia cl'asica. As'i que hay que ``parchar" la
teor'ia de Maxwell para incluir aspectos corpusculares (cu'anticos!)
de la luz. Tenemos, por tanto, que ``cuantizar", de alg'un modo, la
teor'ia de Maxwell....

Einstein fue el primero que reconoci'o la impotancia de las
propiedades cu'anticas de la radiaci'on. En 1905 (en el mismo n'umero
de la revista \textit{Annalen der Physik} en el que public'o su
trabajo fundamental sobre la teor'ia de la relatividad especial)
Einstein propone una explicaci'on para el efecto fotoel'ectrico
(descubierto en 1887 por Heinrich Hertz). Philipp Lenard
(inesperadamente) descubri'o en 1902 que la energ'ia de los electrones
liberados de una placa met'alica irradiada por luz (ultravioleta)
\textit{no depende} de la intensidad de la luz, como se esperaba a
partir de la teor'ia electromagn'etica cl'asica (recuerde,
$E_{em}\propto \int dV (\vec{E}^2+\vec{B}^2)$), sino que
\textit{depende de la frecuencia} (e.d., del color) de la luz. La
intensidad de la luz determina s'olamente el \textit{n'umero} de
electrones emitidos. Einstein asumi'o que la luz est'a compuesta de
\textit{\underline{cuantos}} de energ'ia $h\nu$ ($=\hbar\omega$), es
decir, de \textit{fotones}, y que la emisi'on de un electr'on era
producida por su interacci'on con el fot'on, proceso en el cual el
primero absorb'ia la energ'ia del segundo. Con estos supuestos es
f'acil explicar que la energ'ia de los electrones emitidos aumenta con
la frecuencia de los fotones.

Otra evidencia de la necesidad de modificar la teor'ia
electromagn'etica cl'asica puede encontrarse en el \textit{efecto
Compton}. Cuando rayos X son dispersados por electrones, se observa
un corrimiento de frecuencia de la radiaci'on. Adem'as, la magnitud
del corrimiento depende del 'angulo de scattering. Este efecto fue
descubierto por Compton en 1923 y explicado (por Compton y Debye)
sobre la base de los supuestos de Einstein sobre el fot'on. La
electrodin'amica cl'asica, por otro lado, no predice alteraci'on alguna
de la frecuencia de las ondas electromagn'eticas dispersadas: s'olo
fotones con energ'ia $\hbar\omega$ y momentum $\hbar k$ posibilitan
este corrimiento.


Ya en los primeros años del desarrollo de la mec'anica cu'antica se
intent'o extender el m'etodo de cuantizaci'on a sistemas con infinitos
grados de libertad para obtener as'i reglas de cuantizaci'on para
\textit{campos}. En el caso de sistemas descritos por campos, es
posible definir cantidades an'alogas a las conocidas coordenadas de
posici'on y momentum can'onico usadas en la cuantizaci'on de sistemas
con finitos grados de libertad. Estas cantidades satisfacen
ecuaciones que son an'alogas a aquellas de la mec'anica cl'asica. Las
reglas de cuantizaci'on para los campos son entonces an'alogas a
aquellas usadas para sistemas mec'anicos. Al aplicar este
procedimiento de cuantizaci'on al campo electromagn'etico es posible
fundamentar el supuesto usado por Einstein de que los cuantos del
campos electromagn'etico (fotones) poseen energ'ia $\hbar\omega$,
adem'as de la propiedad (asumida por de Broglie) que los fotones
poseen un impulso $h/\lambda$ ($=\hbar k$). Naturalmente, de la
cuantizaci'on del campo electromagn'etico se desprenden muchos nuevos
fen'omenos como, por ejemplo, que los correspondientes operadores que
describen el campo electromagn'etico (en general) no conmutan, por lo
que no es posible medirlos simult'aneamente con infinita precisi'on.
En particular, un estado con $\vec{E}=\vec{0}$ y $\vec{B}=\vec{0}$
no es permitido. Como consecuencia, incluso una regi'on sin radiaci'on
electromagn'etica (vac'io fot'onico) posee propiedades no triviales. En
particular, 'este posee una energ'ia, la energ'ia del vac'io, o energ'ia
del punto cero.

La teor'ia del campo electromagn'etico cuantizado, la
\textit{electrodin'amica cu'antica} (QED) constituye el m'as exitoso
ejemplo de una \textit{teor'ia cu'antica de campos}. En el formalismo
de la QED fue necesario por primera vez la introducci'on de
operadores que describen la creaci'on (emisi'on) y destrucci'on
(absorci'on) de part'iculas. En forma an'aloga a QED, posteriormente
fue desarrollada la teor'ia del campo electr'on-positr'on (campo de
Dirac). Los cuantos de este campo describen electrones y sus
anti-part'iculas los positrones, tal como los fotones son los cuantos
del campo electromagn'etico. Actualmente, todas las teor'ia que
describen las interaciones fundamentales conocidas, con excepci'on de
la gravedad, son teor'ias cu'anticas de campos.
