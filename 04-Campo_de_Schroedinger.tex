\chapter{El campo de Schr\"{o}dinger}



AQUI HAY QUE CORREGIR VARIAS COSAS !!

\section{Teor'ia Cl'asica de Campos para el Campo de Schr\"{o}dinger}

El punto de partida es la conocida ecuaci'on de Schr\"{o}dinger:%
\begin{equation}
i\hbar \frac{\partial \varphi }{\partial t}=-\frac{\hbar ^{2}}{2m}\partial
_{i}^{2}\varphi +V(x,t)\varphi \label{EcSchrodinger}
\end{equation}%
para la funci'on de onda $\varphi (x,t)$ de una part%
'icula sujeta a un potencial externo arbitrario $V(x,t).
$ En el contexto de la teor'ia de campos, podr'iamos considerar a la
funci'on de onda como un campo complejo cl'asico, mientras que a la ec
de Schr\"{o}dinger como la ecuaci'on cl'asica de campo\footnote{%
En analog'ia con la ecuaci'on de ondas cl'asica para el campo EM, que al
ser cuantizada da origen a los fotones.} que debe satisfacer $\varphi \left(
x_{j},t\right) .$

Para poder hacer uso de la maquinaria conocida de la teor'ia de campos,
debemos conocer la densidad Lagrangeana que de origen al campo cl'asico
de Schr\"{o}dinger dado por la ecuaci'on (\ref{EcSchrodinger}).

La densidad Lagrangeana debe ser:%
\begin{equation}
{\cal L}\left( \psi ,\partial _{i}^{2},\dot{\psi}\right) =i\hbar \psi ^{\ast
}\dot{\psi}-\frac{\hbar ^{2}}{2m}\left( \partial _{i}\psi ^{\ast }\right)
\left( \partial _{i}\psi \right) -V(x,t)\psi ^{\ast }\psi ,
\label{DensidadLagrangeanadeSchrodinger}
\end{equation}%
donde $\psi (x,t)$ y $\psi ^{\ast }(x,t)$
son tratados como dos campos independientes.

Para verificar que efectivamente (\ref{DensidadLagrangeanadeSchrodinger}) es
la densidad Lagrangeana buscada, debemos calcular las conocidas ecuaciones
de movimiento de Euler-Lagrange\footnote{%
Recordemos que las ecuaci'on de Euler-Lagrange para un campo arbitrario $\varphi
\left( \vec{x},t\right) $ est'a dada por:%
\begin{equation*}
\frac{\partial {\cal L}}{\partial \varphi }-\partial _{i}\left( \frac{%
\partial {\cal L}}{\partial \left( \partial _{i}\varphi \right) }\right) -%
\frac{\partial }{\partial t}\left( \frac{\partial {\cal L}}{\partial \dot{%
\varphi}}\right) =0.
\end{equation*}%
} para los campos $\psi (x,t)$ y $\psi ^{\ast }\left(
x_{j},t\right) $. En efecto, considerando (\ref%
{DensidadLagrangeanadeSchrodinger}), obtenemos:%
\begin{eqnarray}
\frac{\partial {\cal L}}{\partial \psi }-\partial _{i}\left( \frac{\partial
{\cal L}}{\partial \left( \partial _{i}\psi \right) }\right) -\frac{\partial 
}{\partial t}\left( \frac{\partial {\cal L}}{\partial \dot{\psi}}\right)
&=&0 \notag \\
\frac{\partial }{\partial \psi }\left( -V(x,t)\psi ^{\ast
}\psi \right) -\partial _{i}\left( -\frac{\hbar ^{2}}{2m}\partial _{i}\psi
^{\ast }\right) -\frac{\partial }{\partial t}\left( i\hbar \psi ^{\ast
}\right) &=&0 \notag \\
-V(x,t)\psi ^{\ast }+\frac{\hbar ^{2}}{2m}\partial
_{i}^{2}\psi ^{\ast }-i\hbar \dot{\psi}^{\ast } &=&0 \notag \\
i\hbar \dot{\psi}^{\ast }+\frac{\hbar ^{2}}{2m}\partial _{i}^{2}\psi ^{\ast
}-V(x,t)\psi ^{\ast } &=&0, \label{EcSchrodingerdePhi*}
\end{eqnarray}%
y:%
\begin{eqnarray}
\frac{\partial {\cal L}}{\partial \psi ^{\ast }}-\partial _{i}\left( \frac{%
\partial {\cal L}}{\partial \left( \partial _{i}\psi ^{\ast }\right) }%
\right) -\frac{\partial }{\partial t}\left( \frac{\partial {\cal L}}{%
\partial \dot{\psi}^{\ast }}\right) &=&0 \notag \\
\frac{\partial }{\partial \psi ^{\ast }}\left( i\hbar \psi ^{\ast }\dot{\psi}%
-V(x,t)\psi ^{\ast }\psi \right) -\partial _{i}\left( -%
\frac{\hbar ^{2}}{2m}\partial _{i}\psi \right) &=&0 \notag \\
i\hbar \dot{\psi}+\frac{\hbar ^{2}}{2m}\partial _{i}^{2}\psi -V\left(
x_{j},t\right) \psi &=&0 \label{EcSchrodingerdePhi}
\end{eqnarray}%
que son las correspondientes ecuaci'on de Schr\"{o}dinger (\ref{EcSchrodinger})
para los campos $\psi (x,t)$ y $\psi ^{\ast }\left(
x_{j},t\right) .$

Una vez conocida la densidad Lagrangeana, la densidad de momentum conjugado
asociada a cada varible de campo, la calculamos de:%
\begin{equation}
\Pi (x,t)=\frac{\partial {\cal L}}{\partial \dot{\psi}},
\label{DensidaddeMomentumConjugadodePhi}
\end{equation}%
y:%
\begin{equation}
\Pi ^{\ast }(x,t)=\frac{\partial {\cal L}}{\partial \dot{%
\psi}^{\ast }}. \label{DensidaddeMomentumConjugadodePhi*}
\end{equation}

As'i, de (\ref{DensidaddeMomentumConjugadodePhi}) y (\ref%
{DensidaddeMomentumConjugadodePhi*}) obtenemos respectivamente:%
\begin{equation}
\Pi (x,t)=i\hbar \psi ^{\ast }(x,t)
\label{DensidaddeMomentumConjugadodeSchrodinger}
\end{equation}%
y:%
\begin{equation*}
\Pi ^{\ast }(x,t)=0.
\end{equation*}

Por lo tanto, tenemos s'olo dos varibles de campo independientes: $\psi
(x,t)$ y $\Pi (x,t)$ cuyas ecuaciones de
movimiento son respectivamente (\ref{EcSchrodingerdePhi}) y\footnote{%
Para 'esta última hemos ocupado la ecuaci'on (\ref{EcSchrodingerdePhi*}).}:%
\begin{equation*}
\dot{\Pi}-\frac{i\hbar }{2m}\partial _{i}^{2}\Pi +\frac{i}{\hbar }V\left(
x_{j},t\right) \Pi =0
\end{equation*}

Otra cantidad importante en la teor'ia de campos, que se deriva de la
Densidad Lagrangeana, es la Densidad Hamiltoniana:%
\begin{eqnarray}
{\cal H} (x,t) &=&\Pi (x,t)\dot{\psi}\left(
x_{j},t\right) -{\cal L}(x,t)\notag \\
&=&i\hbar \psi ^{\ast }\dot{\psi}-\left( i\hbar \psi ^{\ast }\dot{\psi}-%
\frac{\hbar ^{2}}{2m}\left( \partial _{i}\psi ^{\ast }\right) \left(
\partial _{i}\psi \right) -V(x,t)\psi ^{\ast }\psi \right)
\notag \\
&=&\frac{\hbar ^{2}}{2m}\left( \partial _{i}\psi ^{\ast }\right) \left(
\partial _{i}\psi \right) +V(x,t)\psi ^{\ast }\psi ,
\label{DensidadHamiltonianadeSchrodinger}
\end{eqnarray}%
la cual nos permite calcular el Hamiltoniano del campo. En efecto:%
\begin{eqnarray*}
H &=&\int d^{3}x{\cal H} (x,t) \\
&=&\int d^{3}x\left\{ \frac{\hbar ^{2}}{2m}\left( \partial _{i}\psi ^{\ast
}\right) \left( \partial _{i}\psi \right) +V(x,t)\psi
^{\ast }\psi \right\} .
\end{eqnarray*}

Ahora, integrando por partes el primer t'ermino del lado derecho,
obtenemos finalmente que el Hamiltoniano del campo es de la forma:%
\begin{equation}
H=\int d^{3}x\psi ^{\ast }\left\{ -\frac{\hbar ^{2}}{2m}\partial
_{i}^{2}+V(x,t)\right\} \psi .
\label{HamiltonianodeSchroginger}
\end{equation}

\section{Cuantizaci'on del Campo de Schr\"{o}dinger.}

Para poder cuantizar el campo de Schr\"{o}dinger, debemos reemplazar las
variables cl'asicas de campo $\psi (x,t)$ y $\Pi \left(
x_{j},t\right) ,$ por los operadores $\hat{\psi}(x,t),$ $%
\hat{\pi}(x,t).$ Y, a su vez, el campo cl'asico
conjugado $\psi ^{\ast }(x,t),$ debe ser reemplazado por el
operador herm'itico adjunto $\psi ^{\dagger }(x,t).$ De
esta manera, es directo que:%
\begin{equation}
\hat{{\cal L}}\left( \hat{\psi},\partial _{i}^{2},\overset{\cdot }{\hat{\psi}%
}\right) =i\hbar \hat{\psi}^{\dagger }\overset{\cdot }{\hat{\psi}}-\frac{%
\hbar ^{2}}{2m}\left( \partial _{i}\hat{\psi}^{\dagger }\right) \left(
\partial _{i}\hat{\psi}\right) -V(x,t)\hat{\psi}^{\dagger }%
\hat{\psi}, \label{OperadorDensidadLagrangeanadeSchrodinger}
\end{equation}%
\begin{equation}
\hat{\pi}(x,t)=i\hbar \psi ^{\dagger }(x,t)
, \label{OperadorDensidaddeMomentumConjugadodePhi}
\end{equation}%
\begin{equation}
\hat{H}=\int d^{3}x\hat{\psi}^{\dagger }d\hat{\psi}
\label{OperadorHamiltonianodeSchrodinger}
\end{equation}%
son los operadores de densidad Lagrangeana, densidad de momentum conjugado y
Hamiltoniano del campo, respectivamente. El operador:%
\begin{equation*}
d=-\frac{\hbar ^{2}}{2m}\partial _{i}^{2}+V(x,t)
\end{equation*}%
es llamado operador diferencial de Schr\"{o}dinger.

Por otro lado, postulamos que los operadores de campo deben satisfacer las
siguientes relaciones de conmutaci'on:%
\begin{eqnarray*}
\left[ \hat{\psi}(x,t),\hat{\Pi}\left( x'_{j},t\right) \right] &=&\delta
^{\left( 3\right) }\left( x_{j}-x'_{j}\right) , \\
\left[ \hat{\psi}(x,t),\hat{\Pi}\hat{\psi}\left( x'_{j},t\right) \right] &=&0,
\\
\left[ \hat{\pi}(x,t),\hat{\Pi}\left( x'_{j},t\right) \right] &=&0,
\end{eqnarray*}%
o, alternativamente ocupando (\ref{OperadorDensidaddeMomentumConjugadodePhi}%
), 'estas pueden ser escritas en la forma:%
\begin{eqnarray*}
\left[ \hat{\psi}(x,t),\psi ^{\dagger }\left( x'_{j},t\right) \right] &=&\delta
^{\left( 3\right) }\left( x_{j}-x'_{j}\right) , \\
\left[ \hat{\psi}(x,t),\hat{\psi}\left( x'_{j},t\right) \right] &=&0, \\
\left[ \psi ^{\dagger }(x,t),\psi ^{\dagger }\left( x'_{j},t\right) \right]
&=&0.
\end{eqnarray*}

Podemos justificar el reemplazo de las variables de campo cl'asicas por
operadores de campo, que satisfacen las relaciones de conmutaci'on
mostradas; demostrando que las ecuaci'ones de movimiento para los
operadores de campo, son las mismas que para sus an'alogos cl'asicos.
En efecto, sabemos que la din'amica de los operadores est'a gobernada
por la ecuaci'on de movimento de Heisenberg, por lo tanto tendremos:%
\begin{eqnarray*}
\overset{\cdot }{\hat{\psi}} &=&-\frac{i}{\hbar }\left[ \hat{\psi},\hat{H}%
\right] , \\
\overset{\cdot }{\hat{\Pi}} &=&-\frac{i}{\hbar }\left[ \hat{\Pi},\hat{H}%
\right] .
\end{eqnarray*}

Ocupando el Hamiltoniano (\ref{OperadorHamiltonianodeSchrodinger}), tenemos
de la primera de las ecuaciones anteriores que:%
\begin{eqnarray*}
i\hbar \overset{\cdot }{\hat{\psi}} &=&\left[ \hat{\psi}\left( x_{j}\right)
,\int d^{3}x'\hat{\psi}^{\dagger }\left( x'_{j}\right) d'\hat{\psi}\left(
x'_{j}\right) \right] \\
&=&\int d^{3}x'\left[ \hat{\psi}\left( x_{j}\right) ,\hat{\psi}^{\dagger }\left(
x'_{j}\right) d'\hat{\psi}\left( x'_{j}\right) \right] \\
&=&\int d^{3}x'\left\{ \left[ \hat{\psi}\left( x_{j}\right) ,\hat{\psi}^{\dagger
}\left( x'_{j}\right) \right] d'\hat{\psi}\left( x'_{j}\right)
+\hat{\psi}^{\dagger }\left( x'_{j}\right) d'\left[ \hat{\psi}\left(
x_{j}\right) ,\hat{\psi}\left( x'_{j}\right) \right] \right\} \\
&=&\int d^{3}x'\delta ^{\left( 3\right) }\left( x_{j}-x'_{j}\right)
d'\hat{\psi}\left( x'_{j}\right) \\
&=&d\hat{\psi}\left( x_{j}\right) \\
&=&-\frac{\hbar ^{2}}{2m}\partial _{i}^{2}\hat{\psi}+V(x,t)
\hat{\psi}
\end{eqnarray*}%
que corresponde a la ecuaci'on de Schr\"{o}dinger para el operador de campo
$\hat{%
\psi}(x,t) .$

La soluci'on para esta ecuaci'on de movimiento sabemos puede ser escrita como
una expansi'on en serie de Fourier de la forma:%
\begin{equation}
\hat{\psi}(x,t) =\sum_{n}\hat{a}_{n}(t)
u_{n}\left( x^{i}\right) , \label{SeriedeFourierparaPhi}
\end{equation}%
de donde es directo que:%
\begin{equation}
\hat{\psi}^{\dagger }(x,t) =\sum_{n}\hat{a}_{n}^{\dagger
}(t) u_{n}^{\ast }\left( x^{i}\right) .
\label{SeriedeFourierparaPhiDaga}
\end{equation}

Las funciones $u_{n}\left( x^{i}\right) $ forman un conjunto ortogonal
completo de funciones complejas, es decir:%
\begin{eqnarray}
\int d^{3}xu_{n}^{\ast }\left( x^{i}\right) u_{n'}\left( x^{i}\right) &=&\delta
_{nn'}, \label{Relaciones1} \\
\sum_{n}u_{n}\left( x^{i}\right) u_{n}^{\ast }\left( x'_{i}\right) &=&\delta
^{\left( 3\right) }\left( x^{i}-x'_{i}\right) \label{Relaciones2}
\end{eqnarray}%
mientras que $\hat{a}_{n}(t) $ son los coeficientes de la expansi%
'on. Mas tarde, reconoceremos a $\hat{a}_{n}^{\dagger }(t) $
y $\hat{a}_{n}(t) $ como los operadores de creaci'on y
destrucci'on respectivamente.

Al realizar la expansi'on en serie de Fourier, estamos haciendo un cambio
de variables de campo que nos permite describir el campo en t'erminos de
los operadores $\hat{a}_{n}(t) $ y $\hat{a}_{n}^{\dagger }\left(
t\right) ,$ reduciendo as'i los grados de libertad del sistema. Como
ahora tenemos nuevas variables de campo, debemos encontrar las nuevas
relaciones de conmutaci'on. Para esto, es necesario invertir las
ecuaciones(\ref%
{SeriedeFourierparaPhi}) y (\ref{SeriedeFourierparaPhiDaga}), y esto lo
logramos haciendo uso de las relaciones (\ref{Relaciones1}) y (\ref%
{Relaciones2}). En efecto, multiplicando ambos lados de (\ref%
{SeriedeFourierparaPhi}) por $\int d^{3}xu_{n}^{\ast }\left( x^{i}\right) $:%
\begin{eqnarray*}
\sum_{n}\hat{a}_{n}(t) \int d^{3}xu_{n}\left( x^{i}\right) u_{n'}^{\ast }\left(
x^{i}\right) &=&\int d^{3}x\hat{\psi}(x,t)u_{n'}^{\ast }\left( x^{i}\right) \\
\sum_{n}\hat{a}_{n}(t) \delta _{nn'} &=&\int d^{3}x\hat{\psi}(x,t) u_{n'}^{\ast
}\left( x^{i}\right) \\
\hat{a}_{n'}(t) &=&\int d^{3}xu_{n'}^{\ast }\left( x^{i}\right) \hat{\psi}(x,t)
,
\end{eqnarray*}%
y haciendo lo mismo con (\ref{SeriedeFourierparaPhiDaga}):%
\begin{eqnarray*}
\sum_{n}\hat{a}_{n}^{\dagger }(t) \int d^{3}xu_{n}^{\ast }\left(
x^{i}\right) u_{n'}\left( x^{i}\right) &=&\int d^{3}xu_{n'}\left( x^{i}\right)
\hat{\psi}^{\dagger }(x,t) \\
\sum_{n}\hat{a}_{n}^{\dagger }(t) \delta _{nn'} &=&\int d^{3}xu_{n'}\left(
x^{i}\right) \hat{\psi}^{\dagger }(x,t) \\
\hat{a}_{n'}^{\dagger }(t) &=&\int d^{3}xu_{n'}\left( x^{i}\right)
\hat{\psi}^{\dagger }(x,t) .
\end{eqnarray*}

Ahora, podemos calcular:%
\begin{eqnarray}
\left[ \hat{a}_{n}(t) ,\hat{a}_{n'}^{\dagger }(t) \right] &=&\hat{a}_{n}(t)
\hat{a}_{n'}^{\dagger }(t) -\hat{a}_{n'}^{\dagger }(t) \hat{a}_{n}(t) \notag \\
&=&\int d^{3}xd^{3}x'u_{n'}\left( x'_{i}\right) u_{n}^{\ast }\left( x^{i}\right)
\hat{\psi}(x,t)
\hat{\psi}^{\dagger }\left( x'_{i},t\right) -\int d^{3}xd^{3}x'u_{n'}\left(
x'_{i}\right) u_{n}^{\ast }\left( x^{i}\right) \hat{\psi}^{\dagger }\left(
x'_{i},t\right) \hat{\psi}(x,t) \notag \\
&=&\int d^{3}xd^{3}x'u_{n'}\left( x'_{i}\right) u_{n}^{\ast }\left( x^{i}\right)
\left\{ \hat{\psi}\left(
x^{i},t\right) \hat{\psi}^{\dagger }\left( x'_{i},t\right) -\hat{\psi}^{\dagger
}\left( x'_{i},t\right) \hat{\psi}(x,t) \right\} \notag \\
&=&\int d^{3}xd^{3}x'u_{n'}\left( x'_{i}\right) u_{n}^{\ast }\left( x^{i}\right)
\left[ \hat{\psi}\left(
x^{i},t\right) ,\hat{\psi}^{\dagger }\left( x'_{i},t\right) \right] \notag \\
&=&\int d^{3}xd^{3}x'u_{n'}\left( x'_{i}\right) u_{n}^{\ast }\left( x^{i}\right)
\delta ^{\left( 3\right)
}\left( x^{i}-x'_{i}\right) \notag \\
&=&\int d^{3}xu_{n'}\left( x^{i}\right) u_{n}^{\ast }\left( x^{i}\right) \notag
\\
&=&\delta _{nn'}. \label{RelConmutacion1}
\end{eqnarray}

Analogamente, los otros conmutadores dan:%
\begin{eqnarray}
\left[ \hat{a}_{n}(t) ,\hat{a}_{n'}(t) \right] &=&\hat{a}_{n}(t) \hat{a}_{n'}(t)
-\hat{a}_{n'}(t) \hat{a}_{n}(t) \notag \\
&=&\int d^{3}xd^{3}x'u_{n}^{\ast }\left( x^{i}\right) u_{n'}^{\ast }\left(
x'_{i}\right) \hat{\psi}(x,t) \hat{\psi}\left( x'_{i},t\right) -\int
d^{3}xd^{3}x'u_{n'}^{\ast }\left( x'_{i}\right) u_{n}^{\ast }\left( x^{i}\right)
\hat{\psi}\left( x'_{i},t\right) \hat{\psi}(x,t) \notag \\
&=&\int d^{3}xd^{3}x'u_{n}^{\ast }\left( x^{i}\right) u_{n'}^{\ast }\left(
x'_{i}\right) \left\{ \hat{\psi}(x,t) \hat{\psi}\left( x'_{i},t\right)
-\hat{\psi}\left( x'_{i},t\right) \hat{\psi}(x,t) \right\} \notag \\
&=&\int d^{3}xd^{3}x'u_{n}^{\ast }\left( x^{i}\right) u_{n'}^{\ast }\left(
x'_{i}\right) \left[ \hat{\psi}(x,t) ,\hat{\psi}\left( x'_{i},t\right) \right]
\notag \\
&=&0, \label{RelConmutacion2}
\end{eqnarray}%
y:%
\begin{eqnarray}
\left[ \hat{a}_{n}^{\dagger }(t) ,\hat{a}_{n'}^{\dagger }(t) \right]
&=&\hat{a}_{n}^{\dagger }(t) \hat{a}_{n'}^{\dagger }(t) -\hat{a}_{n'}^{\dagger
}(t) \hat{a}_{n}^{\dagger }(t) \notag \\
&=&\int d^{3}xd^{3}x'u_{n}\left( x^{i}\right) u_{n'}\left( x'_{i}\right)
\hat{\psi}^{\dagger }(x,t) \hat{\psi}^{\dagger}\left( x'_{i},t\right) -\int
d^{3}xd^{3}x'u_{n}\left( x^{i}\right) u_{n'}\left( x'_{i}\right)
\hat{\psi}^{\dagger }\left( x'_{i},t\right) \hat{\psi}^{\dagger }(x,t) \notag \\
&=&\int d^{3}xd^{3}x'u_{n}\left( x^{i}\right) u_{n'}\left( x'_{i}\right) \left\{
\hat{\psi}^{\dagger }(x,t) \hat{\psi}^{\dagger }\left( x'_{i},t\right)
-\hat{\psi}^{\dagger }\left( x'_{i},t\right) \hat{\psi}^{\dagger }(x,t) \right\}
\notag \\
&=&\int d^{3}xd^{3}x'u_{n}\left( x^{i}\right) u_{n'}\left( x'_{i}\right) \left[
\hat{\psi}^{\dagger }(x,t) ,\hat{\psi}^{\dagger }\left( x'_{i},t\right) \right]
\notag \\
&=&0. \label{RelConmutacion3}
\end{eqnarray}

Si el sistema no depende expl'icitamente del tiempo, es decir, el
potencial es del tipo $V=V\left( x^{i}\right) ,$ es natural hacer que las
funciones $u_{n}\left( x^{i}\right) $ que ocupamos como base para expadir el
operador de campo, sean las funciones de onda o eigenfunciones de la ecuaci%
'on estacionaria de Schr\"{o}dinger:%
\begin{equation*}
\left( -\frac{\hbar ^{2}}{2m}\partial _{i}^{2}+V\left( x_{j}\right) \right)
u_{n}\left( x_{j}\right) =E_{n}u_{n}\left( x_{j}\right) ,
\end{equation*}%
por lo tanto el Hamiltoniano (\ref{OperadorHamiltonianodeSchrodinger}) en t%
'erminos de las nuevas variables de campo se puede escribir como:

\begin{eqnarray}
\hat{H} &=&\int d^{3}x\hat{\psi}^{\dagger }\left( -\frac{\hbar ^{2}}{2m}%
\partial _{i}^{2}+V\left( x_{j}\right) \right) \hat{\psi} \notag \\
&=&\int d^{3}x\left( \sum_{n}\hat{a}_{n}^{\dagger }(t)
u_{n}^{\ast }\left( x_{j}\right) \right) \left( -\frac{\hbar ^{2}}{2m}%
\partial _{i}^{2}+V\left( x_{j}\right) \right) \left( \sum_{n'}\hat{a}_{n'}(t)
u_{n''}\left( x_{j}\right) \right) \notag \\
&=&\sum_{n}\sum_{n'}\hat{a}_{n}^{\dagger }(t) \hat{a}_{n'}(t) \int
d^{3}xu_{n}^{\ast }\left( x_{j}\right) \left( -\frac{%
\hbar ^{2}}{2m}\partial _{i}^{2}+V\left( x_{j}\right) \right) u_{n'}\left(
x_{j}\right) \notag \\
&=&\sum_{n}\sum_{n'}\hat{a}_{n}^{\dagger }(t) \hat{a}_{n'}(t) E_{n'}\int
d^{3}xu_{n}^{\ast }\left( x_{j}\right) u_{n'}\left( x_{j}\right) \notag \\
&=&\sum_{n}\sum_{n'}E_{n'}\hat{a}_{n}^{\dagger }(t) \hat{a}_{n'}(t) \delta
_{nn'} \notag \\
&=&\sum_{n}E_{n}\hat{a}_{n}^{\dagger }(t) \hat{a}_{n}\left(
t\right) \notag \\
&=&\hbar \sum_{n}\omega _{n}\hat{N}_{n},
\label{OperadorHamiltonianodeSchrodinger2}
\end{eqnarray}%
con $\hat{N}_{n}=\hat{a}_{n}^{\dagger }(t) \hat{a}_{n}\left(
t\right) $ como el operador de Número.

El Hamiltoniano anterior representa un sistema de infinitos osciladores arm%
'onicos desacoplados con energ'ias $E_{n}=\hbar \omega _{n}.$ Es por
eso y por las relaciones de conmutaci'on (\ref{RelConmutacion1}), (\ref{RelConmutacion2}) y (\ref{RelConmutacion3}), que, como ya hab'iamos
mencionado, podemos reconocer a $\hat{a}_{n}^{\dagger }(t) $ y $%
\hat{a}_{n}(t) $ como los operadores de creaci'on y destrucci'on.
