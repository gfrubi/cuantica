\appendix
\chapter{Teor'ia de Campos Cl'asica}
\section{Formulaci'on Funcional}

Uno de los problemas m'as recurrentes en f'isica es el intentar minimizar una
expresi'on que no s'olo depende de una variable continua sino que tambi'en de todos
los valores de una funci'on sobre dicha variable y, posiblemente, de derivadas de
dicha funci'on. Este tipo de problema jug'o un rol crucial en el desarrollo de la
F'isica Cl'asica en los siglos XVIII y XIX y su importancia se vi'o reforzada con
el advenimiento de las Teor'ias de Campo en el siglo pasado. La formulaci'on
matem'atica de este problema corresponde al c'alculo de variaciones.

Tres problemas famosos relacionados con el c'aculo de variaciones son:

\begin{itemize}
\item C'alculo de la curva de longitud m'as corta que une dos puntos.
\item Dados dos puntos en el plano vertical encontrar la trayectoria de tiempo
m'inimo que una part'icula sigue entre dichos puntos suponiendo que 
su aceleraci'on es debida a la gravedad. 
\item Encontrar la  superficie m'inima de revoluci'on que pasa por dos puntos
dados.
\end{itemize}

\subsubsection{Definici'on de derivada funcional}

Usualmente una funci'on $y(x)$ en F'isica asume valores reales sobre una variable
$x$ cuyos valores est'an comprendidos en un intervalo $I=\left[ a,b\right] $.
Consideremos ahora $N$ puntos en dicho intervalo y escojamos los puntos $x_n\in
I$ de la forma $x_n=a+\epsilon b$, con $N\epsilon=b-a$. Podemos ahora
representar la funci'on $y(x)$ por los valores de 'esta en los puntos $x_n$, es
decir por la funci'on $y_n=y(x_n)=y(a+\epsilon b)$. En el l'imite
$N\rightarrow\infty$, o equivalentemente $\epsilon\rightarrow 0 $, la funci'on
$y_n$ entrega la misma informaci'on que la funci'on $y(x)$.

Ahora, podemos definir una funci'on $F$ sobre todos los $y_n$, es decir
$F(\left\lbrace y_n\right\rbrace)$. En el l'imite $N\rightarrow\infty$, $F$ es
una funci'on de $y(x)$. \textit{Llamamos a $F$} funcional de $y(x)$ y lo
denotamos por $F\left[ y\right] $. $F$ es una funci'on de todos los valores de
$y(x)$ en el intervalo $\left[ a,b\right] $. Luego, $F$ depende de un n'umero
infinito de variables independientes y su valor es un n'umero.

Un ejemplo simple de funcional es aquel que selecciona el valor de una funci'on
en un punto determinado, por ejemplo $F\left[ y\right]=y(x=3)$, de modo que
$F\left[ y^3\right]=27 $ y $F\left[ y=\sin(\pi x)\right]=0 $. Otro ejemplo
simple de funcional es $F\left[ y\right]=\int_0^1y(x)dx$, de modo que $F$
depende de la forma completa de la funci'on $y(x)$. Obviamente, existen
funcionales m'as interesantes.

Veamos qu'e sucede con $F(\left\lbrace y_n\right\rbrace)$ si cambiamos el valor
de los $y_n$ alrededor del set de valores $\left\lbrace y^0\right\rbrace$ . El
cambio en $F$ est'a dado por 
\begin{equation}
dF=\sum_{n=1}^N\left( \frac{\partial F}{\partial y_n}\right)_{y^0} dy_n.
\end{equation}
Para ver c'omo se comporta el cambio en $F$ en el l'imite $\epsilon\rightarrow 0$
reescribimos la expresi'on anterior en la forma
\begin{equation}
dF=\sum_{n=1}^N \epsilon\left(\frac{1}{\epsilon} \frac{\partial F}{\partial
y_n}\right)_{y^0} dy_n,
\end{equation}
la cual claramente evoca la definici'on  de integral por Riemann, es decir,
\begin{equation}
\int_{a}^b f(x)dx=\lim_{\epsilon\rightarrow 0}\sum_{n=1}^N \epsilon f(x_n).
\end{equation}
Luego, tomando el l'imite $\epsilon\rightarrow 0$ con $x=a+n\epsilon $ y usando
la notaci'on $dy_n=\delta y(x)$ obtenemos
\begin{equation}
dF=\int_a^b dx \left( \frac{\delta F}{\delta y(x)}\right)_{y^0(x)}\delta y(x),
\end{equation}
donde la $y^0(x)$ es la funci'on particular $y(x)$ que sirve como punto de
partida para el cambio infinitesimal $\delta y(x)$, y el factor $1/\epsilon$ ha
sido incluido en el t'ermino $\delta F/\delta y(x)$.

El resultado anterior puede ser interpretado de la siguiente forma: El cambio en
$F$ es una suma de t'erminos proporcionales a los cambios infinitesimales $\delta
y(x)$, cuyas constantes de proporcionalidad son la derivada funcional $\delta
F/\delta y(x)$. Para fines pr'acticos podemos considerar este resultado como la
definici'on de derivada funcional.

\subsubsection{Ejemplo simple de c'alculo de derivada funcional}

Consideremos ahora el funcional $F$ definido como
\begin{equation}
F\left[ y\right]=\int_0^1y(x)^2dx.
\end{equation}
Para calcular la derivada funcional comenzamos calculando el cambio $dF$ debido
a un cambio infinitesimal $\delta y(x)$ en las variables independientes, esto es
\begin{eqnarray}
F\left[ y+\delta y\right]&=&\int_0^1(y(x)+\delta y(x))^2 dx 
\\
&=&\int_0^1(y(x)^2+2y(x)\delta y(x)+\delta y(x)^2) dx.
\end{eqnarray}
Considerando que tomaremos el l'imite $\delta y(x)\rightarrow 0$ podemos eliminar
el t'ermino cuadr'atico en $\delta y$, obteniendo de esta forma
\begin{equation}
F\left[ y+\delta y\right]=\int_0^1\delta y(x)^2 dx 
+\int_0^12y(x)\delta y(x) dx.
\end{equation}
De modo que tenemos
\begin{equation}
dF=F\left[ y+\delta y\right]-F\left[ y+\delta y\right]=\int_0^12y(x)\delta y(x)
dx.
\end{equation}
Comparando con la definici'on de derivada funcional obtenemos finalmente que
\begin{equation}
\frac{\delta F}{\delta y(x)}=2y(x).
\end{equation}

En general, un funcional puede depender de m'as de una funci'on, por ejemplo
\begin{equation}
F\left[y(x), z(x)\right]=\int_0^1y(x)^2z(x)^3dx.
\end{equation}
En este caso tenemos que
\begin{eqnarray}
\frac{\delta F}{\delta y(x)}&=&2y(x)z(x)^3,
\\
\frac{\delta F}{\delta z(x)}&=&3y(x)^2z(x)^2.
\end{eqnarray}
Es posible tambi'en que la o las funciones dependan de m'as de una variable, por
ejemplo
\begin{equation}
F\left[ f(x,y,z)\right]=\int_{V}dV \left[ f(x,y,z)\right] ^3, 
\end{equation}
de donde vemos que la derivada funcional
\begin{equation}
\frac{\delta F}{\delta f(x,y,z)}=3\left[ f(x,y,z)\right] ^2
\end{equation}
es tal como en el caso unidimensional.

\subsubsection{Ecuaciones de Euler I}

En algunos casos especiales es posible obtener expresiones simples para las
derivadas de funcionales gen'ericos con dependencias simples. Consideremos por
ejemplo el funcional
\begin{equation}
F\left[ y\right] = \int dx L(x, y(x)).
\end{equation}
Éste es una integral simple cuyo integrando es una funci'on de $x$ e $y(x)$.
Variando $y(x)$ tenemos
\begin{equation}
F\left[ y+\delta y\right] = \int dx L(x, y+\delta y)=\int \left(
L(x,y)+\frac{\partial L(x,y)}{\partial y}\delta y\right), 
\end{equation}
donde hemos omitido todos los t'erminos de orden superior a $\delta y$. Esto nos
conduce a
\begin{equation}
F\left[ y+\delta y\right]-F\left[ y\right]=\int \frac{\partial L(x,y)}{\partial
y}\delta y, 
\end{equation}
de donde obtenemos que
\begin{equation}
\frac{\delta F}{\delta y(x)}=\frac{\partial L(x,y(x))}{\partial y(x)}.
\end{equation}

Un caso un tanto m'as complicado es un funcional de la forma
\begin{equation}
F\left[ y\right] = \int dx L(x, y(x), y'(x)),
\end{equation}
donde $y'(x)=dy(x)/dx$. Una variaci'on de $y(x)$ nos conduce a
\begin{equation}
F\left[ y+\delta y\right] = \int dx L(x, y+\delta y, y'+\delta y'),
\end{equation}
donde $\delta y'$ corresponde a $d(\delta y)/dx$, es decir a la derivada de la
variaci'on. Expandiendo a primer orden en la variaci'on y sus derivadas obtenemos
\begin{equation}
F\left[ y+\delta y\right]=\int dx\left( 
L(x,y,y')+\frac{\partial L(x,y,y')}{\partial y}\delta y+\frac{\partial
L(x,y,y')}{\partial y'}\delta y'
\right).
\end{equation}
Para poder usar la definici'on de derivada funcional debemos escribir el 'ultimo
t'ermino en funci'on de la variaci'on $\delta y$. Esto puede ser hecho por medio de
una integraci'on por partes, es decir,
\begin{equation}
\int_a^b \frac{\partial L(x,y,y')}{\partial y'}\delta y'dx=
\left[ \frac{\partial L(x,y,y')}{\partial y'}\delta y\right]_a^b-
\int_a^b dx \frac{d}{dx}
\left( 
\frac{\partial L(x,y,y')}{\partial y'}
\right)  \delta y(x).
\end{equation} 
Luego, obtenemos que
\begin{equation}
F\left[ y+\delta y\right]-F\left[ y\right]= 
\int_a^b dx \frac{\partial L(x,y,y')}{\partial y}\delta y(x)
-\int_a^b dx \frac{d}{dx}
\left( 
\frac{\partial L(x,y,y')}{\partial y'}
\right)  \delta y(x)
+\left[ \frac{\partial L(x,y,y')}{\partial y'}\delta y\right]_a^b.
\end{equation}
Agrupando t'erminos la expresi'on anterior es
\begin{equation}
F\left[ y+\delta y\right]-F\left[ y\right]= 
\int_a^b dx \left[ \frac{\partial L(x,y,y')}{\partial y}
-\frac{d}{dx}
\left( 
\frac{\partial L(x,y,y')}{\partial y'}
\right)\right]   \delta y(x)
+\left[ \frac{\partial L(x,y,y')}{\partial y'}\delta y(x)\right]_a^b.
\end{equation}
Asumiendo que el segundo t'ermino se anula, obtenemos que la derivada funcional
est'a dada por
\begin{equation}
\frac{\delta F}{\delta y(x)}=\frac{\partial L(x,y,y')}{\partial y}
-\frac{d}{dx}
\left( 
\frac{\partial L(x,y,y')}{\partial y'}
\right).
\end{equation}
Éstas son las ecuaciones de Euler cuando la derivada funcional se iguala a cero.

Otro conjunto util de ecuaciones tipo Euler es obtenido al considerar un
funcional de la forma
\begin{equation}
F\left[ f(x,y,z)\right]=\int_Vdxdydz L(x, y, z, f, f_x, f_y, f_z),
\end{equation}
donde vemos que la funci'on $f$ depende ahora de tres variables independientes y
$L$ depende de las primeras derivadas de $f$ con respecto a las variables
independientes.
La variaci'on de la funci'on $f$ nos lleva a
\begin{equation}
F\left[ f+\delta f\right]=\int_Vdxdydz
L(x, y, z, f+\delta f, f_x+\delta f_x, f_y+\delta f_y, f_z+\delta f_z).
\end{equation}
Expandiendo hasta primer orden en las variaciones y las derivadas de las
variaciones obtenemos
\begin{equation}
F\left[ f+\delta f\right]=\int_Vdxdydz
\left( 
L+
\frac{\partial L}{\partial f}\delta f+
\frac{\partial L}{\partial x}\delta f_x+
\frac{\partial L}{\partial y}\delta f_y+
\frac{\partial L}{\partial z}\delta f_z
\right). 
\end{equation}
Los t'erminos proporcionales a las variaciones $\delta f_x$, $\delta f_y$ y
$\delta f_z$ pueden ser relacionados con la variaci'on $\delta f$ por medio de
integraci'on por partes,
esto es
\begin{equation}
\int_Vdxdydz\frac{\partial L}{\partial x}\delta f_x=
\left[ \frac{\partial L}{\partial f_x}\delta f\right]_{\partial V}-
\int_V dxdydz \frac{\partial}{\partial x}
\left( 
\frac{\partial L}{\partial f_x}
\right)  \delta f(x,y,z),
\end{equation}
\begin{equation}
\int_Vdxdydz\frac{\partial L}{\partial y}\delta f_y=
\left[ \frac{\partial L}{\partial f_y}\delta f\right]_{\partial V}-
\int_V dxdydz \frac{\partial}{\partial y}
\left( 
\frac{\partial L}{\partial f_y}
\right)  \delta f(x,y,z),
\end{equation}
y
\begin{equation}
\int_Vdxdydz\frac{\partial L}{\partial z}\delta f_z=
\left[ \frac{\partial L}{\partial f_z}\delta f\right]_{\partial V}-
\int_V dxdydz \frac{\partial}{\partial z}
\left( 
\frac{\partial L}{\partial f_z}
\right)  \delta f(x,y,z).
\end{equation}
Usando las ecuaciones anteriores, el cambio en $F$ debido a la variaci'on en $f$
puede ser escrito como
\begin{eqnarray}
F\left[ f+\delta f\right]-F\left[ f\right]&=&
\int_Vdxdydz \frac{\partial L}{\partial f}\delta f
\nonumber
\\
&+&
\left[ \frac{\partial L}{\partial f_x}\delta f\right]_{\partial V}
-
\int_V dxdydz \frac{\partial}{\partial x}
\left( 
\frac{\partial L}{\partial f_x}
\right)  \delta f(x,y,z)
\nonumber
\\
&+&
\left[ \frac{\partial L}{\partial f_y}\delta f\right]_{\partial V}
-
\int_V dxdydz \frac{\partial}{\partial y}
\left( 
\frac{\partial }{\partial f_y}
\right)  \delta f(x,y,z)
\nonumber
\\
&+&
\left[ \frac{\partial L}{\partial f_z}\delta f\right]_{\partial V}
-
\int_V dxdydz \frac{\partial}{\partial z}
\left( 
\frac{\partial L}{\partial f_z}
\right)  \delta f(x,y,z).
\end{eqnarray}
Agrupando t'erminos obtenemos
\begin{eqnarray}
F\left[ f+\delta f\right]-F\left[ f\right]&=&
\int_Vdxdydz \frac{\partial L}{\partial f}\delta f
-
\int_V dxdydz \frac{\partial}{\partial x}
\left( 
\frac{\partial L}{\partial f_x}
\right)  \delta f(x,y,z)
\nonumber
\\
&-&
\int_V dxdydz \frac{\partial}{\partial y}
\left( 
\frac{\partial }{\partial f_y}
\right)  \delta f(x,y,z)
-
\int_V dxdydz \frac{\partial}{\partial z}
\left( 
\frac{\partial L}{\partial f_z}
\right)  \delta f(x,y,z)
\nonumber
\\
&+&\left[ \frac{\partial L}{\partial f_x}\delta f\right]_{\partial V}
\nonumber
+\left[ \frac{\partial L}{\partial f_y}\delta f\right]_{\partial V}
+\left[ \frac{\partial L}{\partial f_z}\delta f\right]_{\partial V}.
\end{eqnarray}
Asumiendo que la variaci'on se anula en la frontera obtenemos finalmente que la
derivada funcional de $F$ est'a dada por
\begin{equation}
\frac{\delta F}{\delta f}=\frac{\partial L}{\partial f}
-\frac{\partial}{\partial x}\left( \frac{\partial L}{\partial f_x}\right)  
-\frac{\partial}{\partial y}\left( \frac{\partial L}{\partial f_y}\right)
-\frac{\partial}{\partial z}\left( \frac{\partial L}{\partial f_z}\right).
\end{equation}

\subsubsection{Ecuaciones de Euler II}

En esta secci'on obtendremos las ecuaciones de Euler por un procedimiento
ligeramente distinto. Nuestro objetivo es minimizar la integral
\begin{equation}
F=\int_{x_a}^{x_b} L(x,y,y')dx,
\end{equation}
donde $L$ es una funci'on fija de $x$, $y$ e $y'$. Los valores de $x_a$, $x_b$,
$y(x_a)=y_a$ e $y(x_a)=y_a$ son dados (fijos). 

Para resolver el problema presentado por la minimizaci'on de $F$ usaremos una
familia de funciones $y(x,\epsilon)$. Dicha familia est'a parametrizada por medio
de $\epsilon\in\left[0,1 \right] $ y sus miembros satisfacen las siguientes
propiedades:
\begin{itemize}
\item $y(x_a,\epsilon)=y_a$ y $y(x_b,\epsilon)=y_b$ para todo valor de
$\epsilon$.
\item $y(x,\epsilon=0)=y(x)$ minimiza $F$.
\item $y(x,\epsilon)$ y sus derivadas hasta el segundo orden son continuas en
$x$ y $\epsilon$.
\end{itemize}

Usaremos el siguiente funci'on de $\epsilon$ para comparar con $F$
\begin{equation}
F(\epsilon)=\int_{x_a}^{x_b} L(x,y(x,\epsilon),y'(x,\epsilon))dx.
\end{equation}
Dado que $y(x,\epsilon=0)$ minimiza a $F(\epsilon)$ y siendo $F(\epsilon)$ una
funci'on diferenciable y continua de $\epsilon$, podemos aplicar los m'etodos del
c'alculo elemental para obtener el m'inimo. Una condici'on necesaria para obtener
el m'inimo es
\begin{equation}
\left( \frac{dF}{d\epsilon}\right)_{\epsilon=0}=0. 
\end{equation}
Esta condici'on nos lleva a
\begin{equation}
\frac{dF}{d\epsilon}=\int_{x_a}^{x_b} 
\left( 
\frac{\partial L}{\partial y}\frac{dy}{d\epsilon}+
\frac{\partial L}{\partial y'}\frac{dy'}{d\epsilon}
\right)dx. 
\end{equation}
Esta expresi'on puede ser puesta en la forma
\begin{equation}
\frac{dF}{d\epsilon}=\int_{x_a}^{x_b} 
\left( 
\frac{\partial L}{\partial y}\frac{dy}{d\epsilon}+
\frac{\partial L}{\partial y'}\frac{d}{dx}\left( \frac{dy}{d\epsilon}\right)
\right)dx. 
\end{equation}
El segundo t'ermino puede ser integrado por partes, de modo que obtenemos
\begin{equation}
\frac{dF}{d\epsilon}=\int_{x_a}^{x_b}  
\frac{\partial L}{\partial y}\frac{dy}{d\epsilon}dx
-
\int_{x_a}^{x_b}\frac{dy}{d\epsilon}\frac{d}{dx}\left( \frac{\partial L}{
\partial y'}\right)dx
+
\left[ 
\frac{dy}{d\epsilon}\frac{\partial L}{\partial y'}
\right]_{x_a}^{x_b}.  
\end{equation}
Puesto que los extremos de la variaci'on son fijos se cumple que 
\begin{equation}
\left( \frac{dy}{d\epsilon}\right)_{x=x_a}=\left(
\frac{dy}{d\epsilon}\right)_{x=x_b}=0.
\end{equation}
Por lo tanto, obtenemos finalmente:
\begin{equation}
\frac{dF}{d\epsilon}=\int_{x_a}^{x_b}  
\frac{\partial L}{\partial y}\frac{dy}{d\epsilon}dx
-
\int_{x_a}^{x_b}\frac{dy}{d\epsilon}\frac{d}{dx}\left( \frac{\partial L}{
\partial y'}\right)dx.
\end{equation}
Aplicando la condici'on necesaria para obtener el m'inimo llegamos a
\begin{equation}
\int_{x_a}^{x_b} \left[  
\frac{\partial L}{\partial y}
-
\frac{d}{dx}\left( \frac{\partial L}{ \partial y'}\right)\right] 
\left( \frac{dy}{d\epsilon}\right)_{\epsilon=0} 
dx=0.
\end{equation}
Puesto que la funci'on
\begin{equation}
\eta(x)=\left( \frac{dy}{d\epsilon}\right)_{\epsilon=0}
\end{equation}
es arbitraria tenemos que el resto del integrando debe anularse, luego
\begin{equation}
\frac{\partial L}{\partial y}
-
\frac{d}{dx}\left( \frac{\partial L}{ \partial y'}\right)=0.
\end{equation}
\subsubsection{Conexi'on con Mec'anica Cl'asica}

En Mec'anica Cl'asica las trayectorias que sigue un sistema son soluciones de las
ecuaciones de Lagrange, las que a su vez son obtenidas como una extremizaci'on de
la acci'on. Ésta est'a definida de la siguiente forma:
\begin{equation}
S=\int_{t_1}^{t_2}dt L(t, x(t), x'(t)),
\end{equation}
donde $L(t, x(t), x'(t))$ es el Lagrangeano del sistema. Considerando las
ecuaciones de Euler antes deducidas con las equivalencias $x\rightarrow t$,
$y(x)\rightarrow x(t)$ e $y'(x)\rightarrow x'(t)$ obtenemos entonces que la
derivada funcional de la acci'on est'a dada por
\begin{equation}
\frac{\delta S}{\delta x(t)}=
\frac{\partial L(t,x(t),x'(t))}{\partial x}
-\frac{d}{dt}
\left( 
\frac{\partial L(t,x(t),x'(t))}{\partial y'}
\right).
\end{equation}
El principio (de m'inima acci'on) de Hamilton establece que las trayectorias
f'isicas son aquellas que extreman la acci'on (pueden ser m'inimos o m'aximos), es
decir aquellas que son soluciones de $\delta S/\delta x(t)$ o,equivalentemente,
\begin{equation}
\frac{\partial L(t,x(t),x'(t))}{\partial x}
-\frac{d}{dt}
\left( 
\frac{\partial L(t,x(t),x'(t))}{\partial x'}
\right)=0.
\end{equation}
N'otese que los extremos de la acci'on son obtenidos de la misma forma que cuando
se extrema una funci'on. Este conjunto de ecuaciones es conocido como las
ecuaciones de Euler-Lagrange. Dado que en el pricipio variacional de Hamilton
las condiciones de frontera $x(t_1)$ y $x(t_2)$ est'an fijas, no es necesario
considerar los t'erminos de frontera en la derivada funcional de la acci'on.

\subsubsection{Conexi'on con Teor'ia Cl'asica de Campos}



\chapter{Cuantizaci'on del oscilador arm'onico unidimensional.}

Sabemos que el Hamiltoniano de un oscilador arm'onico unidimensional de
masa $m$ y frcuencia $\omega$ tiene la forma:%
\begin{equation}
H=\frac{\hat{p}^{2}}{2m}+\frac{m}{2}\omega^{2}\hat{q}^{2},
\end{equation}
donde $\hat{p}$ y $\hat{q}$ son los operadores de momentum lineal y de posici'on, respectivamente.

\section{Energ'ia m'inima y niveles de energ'ia.}

Dos formas simples de estudiar al oscilador arm'onico y de encontrar sus
niveles de energ'ia permitidos son usando el principio de
indeterminaci'on de Heisenberg y la regla de cuantizaci'on de
Bohr-Sommerfeld. A pesar de su simplicidad, no nos permiten un estudio
completo del problema ya que no podemos por ejemplo obtener los estados
correspondientes a cada energ'ia. Tampoco suministran un m'etodo que
podamos utilizar para luego abordar problemas m'as complejos, como la
cuantizaci'on del campo electromagn'etico, que son modelados mediante el
oscilador arm'onico.

\subsection{Principio de Indeterminaci'on.}

En realidad este camino nos lleva directamente a encontrar s'olo la
energ'ia minima que puede tener el oscilador.

Para esto, resulta obvio calcular el valor esperado de la energ'ia
ocupando el Hamiltoniano del sistema:%
\begin{equation}
\left\langle E\right> =\left\langle \hat{H}\right> =\frac{1}%
{2m}\left\langle \hat{p}^{2}\right> +\frac{m}{2}\omega^{2}\left\langle
\hat{q}^{2}\right> . \label{ValEspEa}
\end{equation}


Las expresiones $\left\langle \hat{p}^{2}\right> $ y $\left\langle
\hat{q}^{2}\right> $ podemos obtenerlas de la relaci'on de
desviaci'on cuadr'atica para un operador arbitrario $\hat{O}$ :
\begin{equation}
\left( \Delta\hat{O}\right) ^{2}=\left\langle \hat{O}^{2}\right>
-\left\langle \hat{O}\right> ^{2} .
\end{equation}
luego, para nuestros operadores:%
\begin{equation}
\left\langle \hat{p}^{2}\right> =\left( \Delta\hat{p}\right)
^{2}+\left\langle \hat{p}\right> ^{2},\label{DesCuaP2}
\end{equation}
\begin{equation}
\left\langle \hat{q}^{2}\right> =\left( \Delta\hat{q}\right)
^{2}+\left\langle \hat{q}\right> ^{2}.\label{DesCuaQ2}
\end{equation}


Sabemos que
\begin{equation}
\left\langle \hat{q}\right> =\left\langle \psi\right| \hat
{q}\left| \psi\right>=\int_{-\infty}^{\infty}q\left| \psi\left( q\right) \right| ^{2}dq,
\end{equation} 
pero $q\left| \psi\left( q\right) \right| ^{2}$ es una funci'on
antisim'etrica que est'a siendo integrada sobre un intervalo sim'etrico,
y por lo tanto, su valor es cero.

Por otro lado, para el operador momentum tenemo que
\begin{equation}
\left\langle \hat{p}\right>
=\left\langle \psi\right| \hat{p}\left| \psi\right>
=-i\hbar\int_{-\infty}^{\infty}dq\psi^{\ast}\left( q\right) \frac
{\partial}{\partial q}\psi\left( q\right) .
\end{equation} 
Haciendo el cambio de variable $\zeta=\frac{\hat{q}}{\lambda}$ con
\begin{equation}
\lambda=\sqrt
{\frac{\hbar}{m\omega}},\label{sust}
\end{equation}
obtenemos
\begin{equation}
\left\langle \hat{p}\right> =-i\hbar\int_{-\infty}^{\infty}d\zeta
\psi^{\ast}\left( \zeta\right) \frac{\partial}{\partial\zeta}\psi\left(
\zeta\right) .\label{p}
\end{equation}
Pero
\begin{equation}
\frac{\partial}{\partial\zeta}\psi\left( \zeta\right) =\frac{e^{-\frac
{\zeta^{2}}{2}}}{N}\frac{\partial H\left( \zeta\right) }{\partial\zeta
}+\zeta\psi\left( \zeta\right) ,\label{zeta}%
\end{equation}
donde la funci'on $H$ son los polinomios de Hermite y $N$ es una constante de
normalizaci'on. La funci'on de onda $\psi\left( \zeta\right) $
est'a relacionada con los polinomios de Hermite mediante%
\begin{equation}
\psi\left( \zeta\right) =H\left( \zeta\right) e^{-\frac{\zeta^{2}}{2}},
\end{equation}
que corresponde a la soluci'on de la ecuaci'on diferencial
\begin{equation}
\frac{\partial^{2}}{\partial\zeta^{2}}\psi\left( \zeta\right) -\zeta^{2}%
\psi\left( \zeta\right) =0,
\end{equation}
que se deriva de la ecuaci'on de Schr\"{o}dinger para el oscilador
arm'onico en la representaci'on de coordenadas
\begin{eqnarray}
\hat{H}\left| \psi\right> &=& E\left| \psi\right>  \\
\frac{\hat{p}^{2}}{2m}+\frac{m}{2}\omega^{2}\hat{q}^{2}\left|
\psi\right> & = &E\left| \psi\right>
\end{eqnarray}
de modo que
\begin{equation}
-\frac{\hbar^{2}}{2m}\frac{\partial^{2}}{\partial q^{2}}\psi\left( q\right)
+\frac{m}{2}\omega^{2}q^{2}\psi\left( q\right) =E\psi\left( q\right).
\end{equation}

Cuando hacemos la sustituci'on dada por la ecuaci'on (\ref{sust}) donde
\begin{eqnarray}
\frac{\partial^{2}}{\partial q^{2}}\psi\left( q\right) & = &\frac{\partial
}{\partial q}\left( \frac{\partial\psi\left( q\right) }{\partial q}\right)
\\
& = &\frac{\partial}{\partial q}\left( \frac{\partial\psi}{\partial\zeta}
\frac{\partial\zeta}{\partial q}\right) \\
& = &\frac{\partial}{\partial q}\left( \sqrt{\frac{m\omega}{\hbar}}
\frac{\partial\psi}{\partial\zeta}\right) \\
& = &\sqrt{\frac{m\omega}{\hbar}}\left( \frac{\partial}{\partial\zeta}
\frac{\partial\zeta}{\partial q}\right) \left( \frac{\partial\psi}
{\partial\zeta}\right) \\
& = &\sqrt{\frac{m\omega}{\hbar}}\left( \sqrt{\frac{m\omega}{\hbar}}
\frac{\partial}{\partial\zeta}\right) \left( \frac{\partial\psi}
{\partial\zeta}\right) \\
& = &\frac{m\omega}{\hbar}\frac{\partial^{2}\psi}{\partial\zeta^{2}},
\end{eqnarray}
entonces
\begin{equation}
-\frac{\hbar\omega}{2}\frac{\partial^{2}\psi\left( \zeta\right) }
{\partial\zeta^{2}}+\frac{\hbar\omega}{2}\zeta^{2}\psi\left( \zeta\right)
=E\psi\left( \zeta\right),
\end{equation}
o, equivalentemente,
\begin{equation}
\frac{\partial^{2}\psi\left( \zeta\right) }{\partial\zeta^{2}}+\left(
\frac{2E}{\hbar\omega}-\zeta^{2}\right) \psi\left( \zeta\right) =0,
\end{equation} 
y considerando desplazamientos grandes, e.d., $\zeta$ $\gg\frac{2E}{\hbar\omega}$,
\begin{equation}
\frac{\partial^{2}\psi\left( \zeta\right) }{\partial\zeta^{2}}-\zeta^{2}%
\psi\left( \zeta\right) =0.\label{EcDifZeta}%
\end{equation}


La soluci'on de esta ecuaci'on diferencial la asumimos como
\begin{equation}
\psi\left( \zeta\right) =H\left( \zeta\right) e^{-\frac{\zeta^{2}}{2}}.
\end{equation}


La ecuaci'on (\ref{zeta}) la obtenemos f'acilmente sustituyendo la soluci'on
anterior en la ecuaci'on diferencial (\ref{EcDifZeta}) y luego integrando una vez
con respecto a $\zeta.$

Volviendo a lo nuestro, reemplazamos la ecuaci'on (\ref{zeta}) en la ecuaci'on (\ref{p}) del
valor esperado del momentum $p$ obtenemos que este 'ultimo puede ser
escrito como:%
\begin{eqnarray}
\left\langle \hat{p}\right> & = &-i\hbar\int_{-\infty}^{\infty}d\zeta
\psi^{\ast}\left( \zeta\right) \left( \frac{e^{-\frac{\zeta^{2}}{2}}}%
{N}\frac{\partial H\left( \zeta\right) }{\partial\zeta}+\zeta\psi\left(
\zeta\right) \right) \\
& = &-\frac{i\hbar}{N}\int_{-\infty}^{\infty}d\zeta\psi^{\ast}\left(
\zeta\right) e^{-\frac{\zeta^{2}}{2}}\frac{\partial H\left( \zeta\right)
}{\partial\zeta}-i\hbar\int_{-\infty}^{\infty}d\zeta\psi^{\ast}\left(
\zeta\right) \zeta\psi\left( \zeta\right) \\
& = &-\frac{i\hbar}{N}\int_{-\infty}^{\infty}\psi^{\ast}\left( \zeta\right)
e^{-\frac{\zeta^{2}}{2}}\frac{\partial H\left( \zeta\right) }{\partial\zeta
}d\zeta-i\hbar\int_{-\infty}^{\infty}\zeta\left| \psi\left( \zeta\right)
\right| ^{2}d\zeta\\
& = &-\frac{i\hbar}{N}\int_{-\infty}^{\infty}e^{-\zeta^{2}}H\left(
\zeta\right) \frac{\partial H\left( \zeta\right) }{\partial\zeta}%
d\zeta-i\hbar\int_{-\infty}^{\infty}\zeta\left| \psi\left( \zeta\right)
\right| ^{2}d\zeta,
\end{eqnarray}
pero, como los polinomios de Hermite son sim'etricos o antisim'etricos en su argumento, el
producto $H\left( \zeta\right) \frac{\partial H\left( \zeta\right)
}{\partial\zeta}$ es siempre antisim'etrico. Por lo tanto, la integral se anula.
Adem'as, la segunda integral tambi'en se anula por la misma raz'on que se
anulaba la correspondiente a $\left\langle \hat{q}\right> .$

Por lo tanto, podemos decir que por razones de simetr'ia en las integrales
se cumple:%
\begin{equation}
\left\langle \hat{p}\right> =\left\langle \hat{q}\right> =0.
\end{equation}


Esto tiene sentido ya que en una oscilaci'on completa, el valor medio de la
posici'on debe anularse al igual que su momentum.

Finalmente, volviendo a las ecuaciones(\ref{DesCuaP2}) y (\ref{DesCuaQ2}) tenemos:%
\begin{equation}
\left\langle \hat{p}^{2}\right> =\left( \Delta\hat{p}\right)
^{2},\label{DesCuaP2b}
\end{equation}
\begin{equation}
\left\langle \hat{q}^{2}\right> =\left( \Delta\hat{q}\right)
^{2}.
\end{equation}


Del principio de indeterminaci'on de Heisenberg tenemos que:%
\begin{equation}
\left( \Delta\hat{p}\right) ^{2}\left( \Delta\hat{q}\right) ^{2}\geq
\frac{\hbar^{2}}{4},
\end{equation}
y, puesto que para el estado fundamental es v'alida la igualdad, de las
ecuaciones (\ref{DesCuaP2}) y (\ref{DesCuaQ2}) obtenemos%
\begin{equation}
\left\langle \hat{p}^{2}\right> \left\langle \hat{q}^{2}\right>
=\frac{\hbar^{2}}{4}.
\end{equation}

Reemplazando esta 'ultima ecuaci'on en la ecuaci'on (\ref{ValEspEa}) tenemos%
\begin{eqnarray}
\left\langle E\right> & = &\frac{1}{2m}\frac{\hbar^{2}}{4\left\langle
\hat{q}^{2}\right> }+\frac{m}{2}\omega^{2}\left\langle \hat{q}%
^{2}\right> \\
& = &\frac{\hbar^{2}}{8m}\frac{1}{\left\langle \hat{q}^{2}\right> }%
+\frac{m}{2}\omega^{2}\left\langle \hat{q}^{2}\right> \label{ValEspEb}%
\end{eqnarray}

Ahora, minimizando $\left\langle E\right> $ con respecto a $\left\langle
\hat{q}^{2}\right> $
\begin{equation}
\frac{d\left\langle E\right> }{d\left\langle \hat{q}^{2}\right>
}=0,
\end{equation}
tenemos
\begin{equation}
-\frac{\hbar^{2}}{8m}\frac{1}{\left\langle \hat{q}^{2}\right> _{\min
}^{2}}+\frac{m}{2}\omega^{2} =0,
\end{equation}
de modo que
\begin{equation}
\left\langle \hat{q}^{2}\right> _{\min} =\sqrt{\frac{\hbar^{2}%
}{4m^{2}\omega^{2}}} =\frac{\hbar}{2m\omega}.
\end{equation}

Para terminar, s'olo queda reemplazar $\left\langle \hat{q}^{2}%
\right> _{\min}$ en la ecuaci'on (\ref{ValEspEb}). Haciendo esto obtendremos
que
\begin{eqnarray}
\left\langle E\right> _{\min} & = &\frac{\hbar^{2}}{8m}\frac{1}%
{\frac{\hbar}{2m\omega}}+\frac{m}{2}\omega^{2}\frac{\hbar}{2m\omega}\\
& = &\frac{1}{2}\hbar\omega .
\end{eqnarray}
que es la energ'ia m'inima posible del oscilador arm'onico.


\section{Niveles de energ'ia y estados cu'anticos de energ'ia.}

Ahora desarrollaremos un m'etodo que nos permitir'a encontrar; adem'as de
los niveles de energ'ia, los estados cu'anticos del sistema
correspondientes a esas energ'ias y nos dar'a directrices para poder
estudiar sistemas m'as complicados.

\subsection{Operadores de Creaci'on y de Destrucci'on.}

Sabemos que el hamiltoniano cu'antico del oscilador arm'onico esta dado
por
\begin{equation}
\hat{H} =\frac{\hat{p}^{2}}{2m}+\frac{m}{2}\omega^{2}\hat{q}^{2}
=-\frac{\hbar^{2}}{2m}\frac{d^{2}}{dq^{2}}+\frac{m}{2}\omega^{2}q^{2},
\end{equation}
donde utilizamos la regla de cuantizaci'on%
\begin{equation}
\hat{p}\quad\rightarrow\quad-i\hbar\frac{d}{dq} ,
\end{equation}
para el operador de momentum lineal en la representaci'on de coordenadas.
As'i, la ecuaci'on de Schr\"{o}dinger,
\begin{equation}
\hat{H}\left| \psi\right> =E\left| \psi\right> ,
\end{equation}
para el oscilador en esta representaci'on es%
\begin{eqnarray}
\hat{H}\left| \psi\right> & = &E\left| \psi\right> ,\\
\left( \frac{\hat{p}^{2}}{2m}+\frac{m}{2}\omega^{2}\hat{q}^{2}\right) \left|
\psi\right> & = &E\left| \psi\right> ,\\
\left( -\frac{\hbar^{2}}{2m}\frac{d^{2}}{dq^{2}}+\frac{m}{2}\omega^{2}%
q^{2}\right) \psi\left( q\right) & = &E\psi\left( q\right) .
\end{eqnarray}
Ahora introducimos el cambio de variable
\begin{equation}
q=\sqrt{\frac{\hbar}{m\omega}}\zeta ,
\end{equation}
que implica
\begin{equation}
\frac{d}{dq}=\frac{d}{d\zeta}\frac{d\zeta}{dq}=\sqrt{\frac{m\omega}{\hbar}}\frac{d}{d\zeta},
\end{equation}
\begin{equation}
\frac{d^{2}}{dq^{2}}=\frac{d}{dq}\left( \frac{d}{dq}\right) =\frac{m\omega}{\hbar}\frac{d^{2}}{d\zeta^{2}}.
\end{equation} 
De este modo, $\zeta$ ser'a nuestra nueva coordenada adimensional. La ecuaci'on de Schrödinger toma la forma:%
\begin{eqnarray}
\left( -\frac{\hbar^{2}}{2m}\frac{m\omega}{\hbar}\frac{d^{2}}%
{d\zeta^{2}}+\frac{m}{2}\omega^{2}\frac{\hbar}{m\omega}\zeta^{2}\right)
\psi\left( \zeta\right) & = &E\psi\left( \zeta\right) ,\\
\left( -\frac{\hbar}{2}\omega\frac{d^{2}}{d\zeta^{2}}+\frac{\hbar}%
{2}\omega\zeta^{2}\right) \psi\left( \zeta\right) & = &E\psi\left(
\zeta\right) ,\\
\hbar\omega\left[ \frac{1}{2}\left( -\frac{d^{2}}{d\zeta^{2}}%
+\zeta^{2}\right) \right] \psi\left( \zeta\right) & = &E\psi\left(
\zeta\right) ,\\
\hbar\omega \hat{N}\psi\left( \zeta\right) & = &E\psi\left( \zeta\right) ,
\end{eqnarray}
donde
\begin{equation}
\hat{H}=\hbar\omega\hat{N}\label{H(N)} ,
\end{equation}
y%
\begin{equation}
\hat{N}=\frac{1}{2}\left( -\frac{d^{2}}{d\zeta^{2}}+\zeta%
^{2}\right) \label{N1} ,
\end{equation}
es el operador Hamiltoniano en unidades de $\hbar\omega$, llamado operador de N'umero.

Veamos qu'e podemos hacer con el t'ermino $-\frac{d^{2}}{d\zeta^{2}%
}+\zeta^{2}$. Usando
\begin{equation}
[\zeta,\frac{d}{d\zeta}]  =-1 ,
\end{equation}
podemos escribir:%
\begin{equation}
-\frac{d^{2}}{d\zeta^{2}}+\zeta^{2}=\left( -\frac{d}%
{d\zeta}+\zeta\right) \left( \frac{d}{d\zeta}+\hat{\zeta
}\right) +1 ,
\end{equation}
o, de una manera m'as conveniente,
\begin{equation}
\hat{N}=\frac{1}{\sqrt{2}}\left( -\frac{d}{d\zeta}+\zeta\right)
\frac{1}{\sqrt{2}}\left( \frac{d}{d\zeta}+\zeta\right) +\frac
{1}{2}\label{N2}.
\end{equation}
En efecto:
\begin{eqnarray}
\hat{N} & = &\frac{1}{\sqrt{2}}\left( -\frac{d}{d\zeta}+\hat{\zeta
}\right) \frac{1}{\sqrt{2}}\left( \frac{d}{d\zeta}+\zeta\right)
+\frac{1}{2}\\
& = &-\frac{1}{2}\frac{d^{2}}{d\zeta^{2}}-\frac{1}{2}\frac{d}%
{d\zeta}\zeta+\frac{1}{2}\zeta\frac{d}{d\zeta}+\frac{1}%
{2}\zeta^{2}+\frac{1}{2}\\
& = &-\frac{1}{2}\frac{d^{2}}{d\zeta^{2}}+\frac{1}{2}\left( \hat{\zeta
}\frac{d}{d\zeta}-\frac{d}{d\zeta}\zeta\right) +\frac{1}%
{2}\zeta^{2}+\frac{1}{2}\\
& = &-\frac{1}{2}\frac{d^{2}}{d\zeta^{2}}+\frac{1}{2}[\zeta%
,\frac{d}{d\zeta}]+\frac{1}{2}\zeta^{2}+\frac{1}{2}\\
& = &-\frac{1}{2}\frac{d^{2}}{d\zeta^{2}}-\frac{1}{2}+\frac{1}{2}%
\zeta^{2}+\frac{1}{2}\\
& = &\frac{1}{2}\left( -\frac{d^{2}}{d\zeta^{2}}+\zeta^{2}\right) .
\end{eqnarray}

As'i, la ecuaci'on de Schr\"{o}dinger queda escrita como:%
\begin{eqnarray}
\hbar\omega\hat{N}\psi\left( \zeta\right) & = &E\psi\left( \zeta\right)\\
\hbar\omega\left[ \frac{1}{\sqrt{2}}\left( -\frac{d}{d\zeta}%
+\zeta\right) \frac{1}{\sqrt{2}}\left( \frac{d}{d\zeta}%
+\zeta\right) +\frac{1}{2}\right] \psi\left( \zeta\right) &
=&E\psi\left( \zeta\right) \\
\hbar\omega\left[ \frac{1}{\sqrt{2}}\left( -\frac{d}{d\zeta}%
+\zeta\right) \right] \left[ \frac{1}{\sqrt{2}}\left( \frac{d%
}{d\zeta}+\zeta\right) \right] \psi\left( \zeta\right) +\frac
{\hbar\omega}{2}\psi\left( \zeta\right) & = &E\psi\left( \zeta\right)\\
\hbar\omega\left[ \frac{1}{\sqrt{2}}\left( -\frac{d}{d\zeta}%
+\zeta\right) \right] \left[ \frac{1}{\sqrt{2}}\left( \frac{d%
}{d\zeta}+\zeta\right) \right] \psi\left( \zeta\right) &
=&E\psi\left( \zeta\right) -\frac{\hbar\omega}{2}\psi\left( \zeta\right)\\
\hbar\omega\left[ \frac{1}{\sqrt{2}}\left( -\frac{d}{d\zeta}%
+\zeta\right) \right] \left[ \frac{1}{\sqrt{2}}\left( \frac{d%
}{d\zeta}+\zeta\right) \right] \psi\left( \zeta\right) & = &\left(
E-\frac{\hbar\omega}{2}\right) \psi\left( \zeta\right) \\
\hbar\omega\hat{b}^{\dagger}\hat{b}\psi\left( \zeta\right) & = &E'
\psi\left( \zeta\right) \label{EcSchrodOscArmSimpUnid}%
\end{eqnarray}
donde hemos definido los nuevos operadores%
\begin{eqnarray}
\hat{b}^{\dagger} & = &\frac{1}{\sqrt{2}}\left( -\frac{d}{d\zeta}%
+\zeta\right) \label{bdag} ,\\
\hat{b} & = &\frac{1}{\sqrt{2}}\left( \frac{d}{d\zeta}+\zeta\right) \label{b} ,
\end{eqnarray}
y redefinido la energ'ia al valor%
\begin{equation}
E'=E-\frac{\hbar}{2}\omega .
\end{equation}

Como se puede ver, hemos encontrado una forma alternativa de escribir la
ecuaci'on de Schr\"{o}dinger para el oscilador arm'onico unidimensional
en funci'on de los nuevos operadores $\hat{b}^{\dagger}$ y $\hat{b}$ que
tienen relaciones de conmutaci'on%
\begin{eqnarray}
\lbrack\hat{b},\hat{b}^{\dagger}] & = &\hat{b}\hat{b}^{\dagger}-\hat
{b}^{\dagger}\hat{b}\\
& = &\frac{1}{\sqrt{2}}\left( \frac{d}{d\zeta}+\zeta\right)
\frac{1}{\sqrt{2}}\left( -\frac{d}{d\zeta}+\zeta\right)
-\frac{1}{\sqrt{2}}\left( -\frac{d}{d\zeta}+\zeta\right)
\frac{1}{\sqrt{2}}\left( \frac{d}{d\zeta}+\zeta\right)\\
& = &\frac{1}{2}\left( -\frac{d^{2}}{d\zeta^{2}}+\frac{d}{d\zeta
}\zeta-\zeta\frac{d}{d\zeta}+\zeta^{2}\right)
-\frac{1}{2}\left( -\frac{d^{2}}{d\zeta^{2}}-\frac{d}{d\zeta}%
\zeta+\zeta\frac{d}{d\zeta}+\zeta^{2}\right)\\
& = &-\frac{1}{2}\frac{d^{2}}{d\zeta^{2}}+\frac{1}{2}\frac{d}%
{d\zeta}\zeta-\frac{1}{2}\zeta\frac{d}{d\zeta}+\frac{1}%
{2}\zeta^{2}+\frac{1}{2}\frac{d^{2}}{d\zeta^{2}}+\frac{1}{2}%
\frac{d}{d\zeta}\zeta-\frac{1}{2}\zeta\frac{d}{d\zeta
}-\frac{1}{2}\zeta^{2}\\
& = &\frac{d}{d\zeta}\zeta-\zeta\frac{d}{d\zeta
}\\
& = &[\frac{d}{d\zeta},\zeta]\\
& = &1\label{[b,bdag]} ,
\end{eqnarray}
que corresponden a las relaciones de conmutaci'on bos'onicas. Por lo
tanto, $\hat{b}$ y $\hat{b}^{\dagger}$ son operadores de Bose.

Nos podemos preguntar c'omo es el operador N'umero en funci'on de los
operadores $\hat{b}^{\dagger}$ y $\hat{b}.$ Esto se ve directamente de la
ecuaci'on (\ref{N2}) ocupando las definiciones dadas por las ecuaciones(\ref{bdag}) y
(\ref{b}), en efecto:%
\begin{equation}
\hat{N}=\hat{b}^{\dagger}\hat{b}+\frac{1}{2}\label{N3} .
\end{equation}


Por lo tanto, de la ecuaci'on (\ref{H(N)}) vemos que el Hamiltoniano del oscilador
tambien puede ser escrito como funci'on de estos nuevos operadores:%
\begin{equation}
\hat{H}=\hbar\omega\hat{N}=\hbar\omega\left( \hat{b}^{\dagger}\hat{b}%
+\frac{1}{2}\right) =\hbar\omega\left( \hat{b}\hat{b}^{\dagger}-\frac{1}%
{2}\right) \label{H(bdagb)} .
\end{equation}


\subsection{Eigenestados y Eigenenerg'ias.}

Ahora, podemos empezar a construir los eigenestados del oscilador arm'onico
simple unidimensional. Recordemos que la ecuaci'on de Schr\"{o}dinger est'a
dada por%
\begin{equation}
\hat{H}\left| \psi\right> =E\left| \psi\right> ,
\end{equation}
lo que nos dice que $\left| \psi\right> $ es un estado cu'antico
del sistema con energ'ia bien definida $E.$

Usaremos el Hamiltoniano dado por la ecuaci'on (\ref{H(bdagb)}). Por lo tanto, es
conveniente considerar a los estados cu'anticos $\left| \psi
\right> $ del oscilador, como los eigenestados $\left|
n\right> $ del operador N'umero $\hat{N}.$ Entonces:%
\begin{eqnarray}
\hat{H}\left| n\right> & = &E_{n}\left| n\right> \\
\hbar\omega\hat{b}^{\dagger}\hat{b}\left| n\right> +\frac{\hbar}%
{2}\omega\left| n\right> & = &E_{n}\left| n\right> \\
\hbar\omega\hat{b}^{\dagger}\hat{b}\left| n\right> & = &E_{n}%
\left| n\right> -\frac{\hbar}{2}\omega\left| n\right> \\
\hbar\omega\hat{b}^{\dagger}\hat{b}\left| n\right> & = &\left(
E_{n}-\frac{\hbar}{2}\omega\right) \left| n\right> \\
\hbar\omega\hat{b}^{\dagger}\hat{b}\left| n\right> & = &E_{n}'
\left| n\right> ,
\end{eqnarray}
donde  $n=0,1,2,3,\dots $, y
\begin{equation}
E_{n}\acute{}=E_{n}-\frac{1}{2}\hbar\omega .
\end{equation}

Partamos operando con $\hat{b}$ sobre el estado base $\left|
0\right> .$ Para este estado tenemos%
\begin{equation}
\hbar\omega\hat{b}^{\dagger}\hat{b}\left| 0\right> =E'
_{0}\left| 0\right> , \qquad E_{0}'=E_{0}-\frac{1}{2}\hbar\omega .
\end{equation}
Ahora, aplicando el operador $\hat{b}$ obtenemos
\begin{eqnarray}
\hbar\omega\hat{b}\hat{b}^{\dagger}\hat{b}\left| 0\right> & = &E'
_{0}\hat{b}\left| 0\right> \\
\hbar\omega\left( \hat{b}\hat{b}^{\dagger}\right) \hat{b}\left|
0\right> & = &E'
_{0}\hat{b}\left| 0\right> \\
\hbar\omega\left( 1+\hat{b}^{\dagger}\hat{b}\right) \hat{b}\left|
0\right> & = &E'
_{0}\hat{b}\left| 0\right> \\
\hbar\omega\hat{b}^{\dagger}\hat{b}\hat{b}\left| 0\right> & = &E'
_{0}\hat{b}\left| 0\right> -\hbar\omega\hat{b}\left|
0\right> \\
\hbar\omega\hat{b}^{\dagger}\hat{b}\left( \hat{b}\left| 0\right>
\right) & = &\left( E'
_{0}-\hbar\omega\right) \left( \hat{b}\left| 0\right> \right) .
\end{eqnarray}
Esta ecuaci'on nos dice que $\hat{b}\left| 0\right> $ es un nuevo
eigenvector del Hamiltoniano con eigenvalor $E'_{0}-\hbar\omega$.
Al aplicar el operador $\hat{b}$ vemos que obtenemos otro
estado con energ'ia $\hbar\omega$ veces menor$.$ Pero la suposici'on
inicial fue que estabamos en el estado de m'as baja energ'ia, por lo
tanto para evitar esta contradicci'on debe ocurrir que%
\begin{equation}
\hat{b}\left| 0\right> =0\label{EstadoFundamental} ,
\end{equation}
es decir, si el sistema est'a en el estado fundamental, no puede bajar a un
estado de menor energ'ia. Por lo tanto, su energ'ia debe ser%
\begin{equation}
E'_{0}=0 ,
\end{equation}
de donde encontramos que:%
\begin{equation}
E_{0}-\frac{1}{2}\hbar\omega  =0 ,
\end{equation}
es decir,
\begin{equation}
E_{0}  =\frac{1}{2}\hbar\omega .
\end{equation}
Ésta, como sabemos, es la m'inima energ'ia en la cual se puede encontrar
el oscilador arm'onico unidimensional.

Ahora, apliquemos el operador $\hat{b}^{\dag}$ sobre el estado base:%
\begin{eqnarray}
\hbar\omega\hat{b}^{\dag}\left( \hat{b}^{\dagger}\hat{b}\right) \left|
0\right> & = &E_{0}\hat{b}^{\dag}\left| 0\right> \\
\hbar\omega\hat{b}^{\dag}\left( \hat{b}\hat{b}^{\dagger}-1\right) \left|
0\right> & = &E_{0}\hat{b}^{\dag}\left| 0\right> \\
\hbar\omega\hat{b}^{\dag}\hat{b}\left( \hat{b}^{\dagger}\left|
0\right> \right) -\hbar\omega\left( \hat{b}^{\dag}\left|
0\right> \right) & = &E_{0}\left( \hat{b}^{\dag}\left|
0\right> \right) \\
\hbar\omega\hat{b}^{\dag}\hat{b}\left( \hat{b}^{\dagger}\left|
0\right> \right) & = &\left( E_{0}+\hbar\omega\right) \left( \hat
{b}^{\dag}\left| 0\right> \right) \\
\hbar\omega\hat{b}^{\dag}\hat{b}\left| 1\right> & = &E_{1}\left|
1\right> .
\end{eqnarray}
Es decir, el estado $\left| 1\right> =\hat{b}^{\dagger}\left|
0\right> $ es un eigenestado de $\hat{H}$ con eigenenerg'ia
$E_{1}=E_{0}+\hbar\omega$.

Ahora aplicamos $\hat{b}^{\dag}$ por la izquierda a la igualdad $\hbar\omega\hat{b}^{\dag}\hat{b}\left| 1\right>  =E_{1}\left|
1\right>$. Obtenemos
\begin{eqnarray}
\hbar\omega\hat{b}^{\dagger}\left( \hat{b}^{\dag}\hat{b}\right) \left|
1\right> & = &E_{1}\hat{b}^{\dagger}\left| 1\right> \\
\hbar\omega\hat{b}^{\dagger}\left( \hat{b}\hat{b}^{\dagger}-1\right)
\left| 1\right> & = &E_{1}\hat{b}^{\dagger}\left| 1\right>
\\
\hbar\omega\hat{b}^{\dagger}\hat{b}\left( \hat{b}^{\dagger}\left|
1\right> \right) -\hbar\omega\left( \hat{b}^{\dagger}\left|
1\right> \right) & = &E_{1}\left( \hat{b}^{\dagger}\left|
1\right> \right) \\
\hbar\omega\hat{b}^{\dagger}\hat{b}\left( \hat{b}^{\dagger}\left|
1\right> \right) & = &\left( E_{1}+\hbar\omega\right) \left( \hat
{b}^{\dagger}\left| 1\right> \right) \\
\hbar\omega\hat{b}^{\dagger}\hat{b}\left| 2\right> & = &E_{2}%
\left| 2\right> .
\end{eqnarray}
Es decir, el estado $\left| 2\right> :=\frac{1}{\sqrt{2}}\hat{b}^{\dagger}\left|
1\right> =\frac{1}{\sqrt{2}}\left( \hat{b}^{\dagger}\right)
^{2}\left| 0\right> $ es un eigenestado de $\hat{H}$ con
eigenenerg'ia $E_{2}=E_{1}+\hbar\omega=E_{0}+2\hbar\omega$ . El factor $\frac{1}{\sqrt{2}}$ es introducido de modo que el estado $\left| 2\right>$ sea normalizado (ver m'as adelante).

Si volvemos a aplicar $\hat{b}^{\dagger},$ obtendremos%
\begin{eqnarray}
\hbar\omega\hat{b}^{\dagger}\left( \hat{b}^{\dagger}\hat{b}\right)
\left| 2\right> & = &E_{2}\hat{b}^{\dagger}\left| 2\right>
\\
\hbar\omega\hat{b}^{\dagger}\left( \hat{b}\hat{b}^{\dagger}-1\right)
\left| 2\right> & = &E_{2}\hat{b}^{\dagger}\left| 2\right>
\\
\hbar\omega\hat{b}^{\dagger}\hat{b}\left( \hat{b}^{\dagger}\left|
2\right> \right) -\hbar\omega\left( \hat{b}^{\dagger}\left|
2\right> \right) & = &E_{2}\left( \hat{b}^{\dagger}\left|
2\right> \right) \\
\hbar\omega\hat{b}^{\dagger}\hat{b}\left( \hat{b}^{\dagger}\left|
2\right> \right) & = &\left( E_{2}+\hbar\omega\right) \left( \hat
{b}^{\dagger}\left| 2\right> \right) \\
\hbar\omega\hat{b}^{\dagger}\hat{b}\left| 3\right> & = &\left(
E_{2}+\hbar\omega\right) \left| 3\right> ,
\end{eqnarray}
resultado an'alogo a los dos casos anteriores.

****** INSERTAR PRUEBA GENERAL! **********

Por lo tanto, podemos concluir que:
\begin{itemize}
\item Cualquier estado $\left| n\right> $ puede se construido
aplicando $n$ veces el operador de creaci'on $\hat{b}^{\dagger}$ sobre el
estado fundamental:%
\begin{equation}
\left| n\right> =\frac{1}{\sqrt{n!}}\left( \hat{b}^{\dagger
}\right) ^{n}\left| 0\right> \label{|n>} ,
\end{equation}
donde el factor $\frac{1}{\sqrt{n!}}$ es para que los estados $\left|
n\right> $ sean ortonormales.

\item El espectro de energ'ias esta dado por la expresi'on:%
\begin{eqnarray}
E_{n} & = &E_{0}+n\hbar\omega\\
& = &\frac{1}{2}\hbar\omega+n\hbar\omega\\
& = &\hbar\omega\left( n+\frac{1}{2}\right),\qquad n=0,1,2,3,\dots
\end{eqnarray}
es decir, cada vez que se aplica el operador de creaci'on sobre un estado,
estamos creando un quantum de energ'ia $\hbar\omega.$

\item La manera como operan $\hat{b}$ y $\hat{b}^{\dagger}$ sobre los estados
$\left| n\right> $ es la siguiente%
\begin{eqnarray}
\hat{b}\left| n\right> & = &\sqrt{n}\left| n-1\right>
\label{b|n>} ,\\
\hat{b}^{\dagger}\left| n\right> & = &\sqrt{n+1}\left|
n+1\right> \label{bdag|n>} ,\\
\hat{b}^{\dagger}\hat{b}\left| n\right> & = &n\left|
n\right> \label{bdagb|n>} .
\end{eqnarray}

\end{itemize}

Demostrar el 'ultimo punto es f'acil. Para demostrar la relaci'on
dada por (\ref{b|n>}), aplicamos $\hat{b}^{\dagger}$ sobre la ecuaci'on (\ref{|n>}):%
\begin{eqnarray}
\hat{b}^{\dagger}\left| n\right> & = &\frac{1}{\sqrt{n!}}\hat
{b}^{\dagger}\left( \hat{b}^{\dagger}\right) ^{n}\left| 0\right>
\\
& = &\frac{1}{\sqrt{n!}}\frac{\sqrt{\left( n+1\right) !}}{\sqrt{\left(
n+1\right) !}}\left( \hat{b}^{\dagger}\right) ^{n+1}\left|
0\right> \\
& = &\sqrt{\frac{n!\left( n+1\right) }{n!}}\left[ \frac{1}{\sqrt{\left(
n+1\right) !}}\left( \hat{b}^{\dagger}\right) ^{n+1}\left|
0\right> \right] \\
& = &\sqrt{\left( n+1\right) }\left| n+1\right> .
\end{eqnarray}

Por otro lado sabemos que:%
\begin{equation}
\left| n-1\right> \propto\hat{b}\left| n\right>
\qquad\Leftrightarrow\qquad\left| n-1\right> =c_{n-1}\hat{b}\left|
n\right> .
\end{equation}
Luego, si multiplicamos por el bra $\left\langle n-1\right| $, podremos
encontrar la constante de proporcionalidad $c_{n-1}.$ En efecto:
\begin{eqnarray}
\left\langle n-1\right| \left. n-1\right> & = &c_{n-1}\left\langle
n-1\right| \hat{b}\left| n\right> \\
& = &c_{n-1}\left[ \left\langle n\right| c_{n-1}^{\ast}\hat{b}^{\dag
}\right] \hat{b}\left| n\right> \\
& = &c_{n-1}c_{n-1}^{\ast}\left\langle n\right| \hat{b}^{\dag}\hat
{b}\left| n\right> \\
& = &c_{n-1}c_{n-1}^{\ast}\left\langle n\right| \hat{b}^{\dag}\left(
\hat{b}\left| n\right> \right) \\
& = &c_{n-1}c_{n-1}^{\ast}\left\langle n\right| \hat{b}^{\dag}\left(
\frac{1}{c_{n-1}}\left| n-1\right> \right) \\
& = &c_{n-1}c_{n-1}^{\ast}\frac{1}{c_{n-1}}\left\langle n\right| \hat
{b}^{\dag}\left| n-1\right> \\
& = &c_{n-1}^{\ast}\sqrt{n}\left\langle n\right| \left. n\right> \\
& = &c_{n-1}^{\ast}\sqrt{n} ,
\end{eqnarray}
pero el lado izquierdo es igual a $1$, debido a que $\left\{ \left| n\right>
\right\} $ es una base ortonormal. Por lo tanto,
\begin{equation}
c_{n-1}^{\ast}\sqrt{n}  =1, \qquad n=1,2,\dots .
\end{equation}
Ya que $n$ es real, tenemos que
\begin{equation}
c_{n-1}=c_{n-1}^{\ast}=\frac{1}{\sqrt{n}}, \qquad n=1,2,\dots ,
\end{equation}
con lo que demostramos la segunda relaci'on%
\begin{eqnarray}
\hat{b}\left| n\right> & = &\frac{1}{c_{n-1}}\left|
n-1\right> \\
& = &\sqrt{n}\left| n-1\right> .
\end{eqnarray}
Demostrar (\ref{bdagb|n>}) es trivial usando (\ref{b|n>}) y (\ref{bdag|n>}).
En efecto:
\begin{eqnarray}
\hat{b}^{\dag}\left( \hat{b}\left| n\right> \right) & = &\hat
{b}^{\dag}\left( \sqrt{n}\left| n-1\right> \right) \\
& = &\sqrt{n}\left( \hat{b}^{\dag}\left| n-1\right> \right) \\
& = &\sqrt{n}\sqrt{n}\left| n\right> \\
& = &n\left| n\right> .
\end{eqnarray}

\section{Estados Coherentes}

\section{Problemas.}

\begin{enumerate}
\item El campo de radiaci'on dentro de un recinto c'ubico, que no
contiene cargas, es especificado por el estado 
\begin{equation}
\left| c\right> =\exp \left( -\frac{1}{2}\left| c\right| ^{2}\right)
\sum_{n=0}^{\infty }\frac{c^{n}}{\sqrt{n!}}\left| n\right> 
\end{equation}
donde 
\begin{equation}
c=\left| c\right| e^{i\delta }
\end{equation}
es un n'umero complejo cualquiera, y $\left| n\right> $ es el
estado 
\begin{equation}
\left| n_{r}\left( {\bf k}\right) \right> =\frac{\left[
a_{r}^{\dagger }\left( {\bf k}\right) \right] ^{n_{r}\left( {\bf k}%
\right) }}{\sqrt{n_{r}\left( k\right) !}}\left| 0\right> 
\end{equation}
em el cual hay $n$ fotones con vector de onda ${\bf k}$, y vector de
polarizaci'on ${\bf \varepsilon }_{r}\left( {\bf k}\right) $
presentes, y no otros. Derive las siguientes propiedades del estado $\left|
c\right> $:

\begin{enumerate}
\item $\left| c\right> $ es normalizado: $\left\langle
c|c\right> =1$.

\textbf{Respuesta}: En efecto,
\begin{eqnarray*}
\left\langle c|c\right> &=&\sum_{m=0}^{\infty }\left\langle m\right| 
\frac{c^{\ast m}}{\sqrt{m!}}\exp \left( -\frac{1}{2}\left| c\right|
^{2}\right) \exp \left( -\frac{1}{2}\left| c\right| ^{2}\right)
\sum_{n=0}^{\infty }\frac{c^{n}}{\sqrt{n!}}\left| n\right> \\
&=&\exp \left( -\left| c\right| ^{2}\right) \sum_{m=0}^{\infty
}\sum_{n=0}^{\infty }\left\langle m\right| \frac{c^{\ast m}}{\sqrt{m!}}\frac{%
c^{n}}{\sqrt{n!}}\left| n\right> \\
&=&\exp \left( -\left| c\right| ^{2}\right) \sum_{m=0}^{\infty
}\sum_{n=0}^{\infty }\frac{c^{\ast m}}{\sqrt{m!}}\frac{c^{n}}{\sqrt{n!}}%
\left\langle m|n\right> 
\end{eqnarray*}
pero 
\begin{eqnarray*}
\left\langle m|n\right> &=&\left\langle 0\right| \frac{\left[ a\left( 
{\bf k}\right) \right] ^{m\left( {\bf k}\right) }}{\sqrt{m\left(
k\right) !}}\frac{\left[ a^{\dagger }\left( {\bf k}\right) \right]
^{n\left( {\bf k}\right) }}{\sqrt{n\left( k\right) !}}\left|
0\right> \\
&=&\frac{1}{\sqrt{m!n!}}\left\langle 0\right| a^{m}a^{\dagger n}\left|
0\right> 
\end{eqnarray*}
y como 
\begin{eqnarray*}
a\left| n\right> &=&\sqrt{n}\left| n-1\right> \\
a^{\dagger }\left| n\right> &=&\sqrt{n+1}\left| n+1\right> 
\end{eqnarray*}
entonces 
\begin{eqnarray*}
a^{\dagger }\left| 0\right> &=&\sqrt{1}\left| 1\right> \\
a^{\dagger 2}\left| 0\right> &=&\sqrt{1}a^{\dagger }\left|
1\right> =\sqrt{1}\sqrt{2}\left| 2\right> \\
&&\vdots \\
a^{\dagger n}\left| 0\right> &=&\sqrt{n!}\left| n\right> 
\end{eqnarray*}
y de la misma forma, 
\begin{eqnarray*}
a\left| n\right> &=&\sqrt{n}\left| n-1\right> \\
a^{2}\left| n\right> &=&\sqrt{n}a\left| n-1\right> =\sqrt{n}%
\sqrt{n-1}\left| n-2\right> \\
&&\vdots \\
a^{m}\left| n\right> &=&\sqrt{\frac{n!}{\left( n-m\right) !}}\left|
n-m\right> {\rm , si }m\leq n \\
a^{m}\left| n\right> &=&0{\rm , si }m>n
\end{eqnarray*}
y entonces, 
\begin{eqnarray*}
\left\langle m|n\right> &=&\frac{1}{\sqrt{m!n!}}\left\langle 0\right|
a^{m}\sqrt{n!}\left| n\right> \\
&=&\frac{1}{\sqrt{m!}}\left\langle 0\right| a^{m}\left| n\right> \\
&=&\frac{1}{\sqrt{m!}}\left\langle 0\right| \sqrt{\frac{n!}{\left(
n-m\right) !}}\left| n-m\right> \\
&=&\frac{1}{\sqrt{m!}}\sqrt{\frac{n!}{\left( n-m\right) !}}\left\langle
0|n-m\right> 
\end{eqnarray*}
Entonces 
\begin{equation}
\left\langle m|n\right> =\delta _{m}^{n}
\end{equation}
Entonces, 
\begin{eqnarray*}
\left\langle c|c\right> &=&\exp \left( -\left| c\right| ^{2}\right)
\sum_{m=0}^{\infty }\sum_{n=0}^{\infty }\frac{c^{\ast m}}{\sqrt{m!}}\frac{%
c^{n}}{\sqrt{n!}}\left\langle m|n\right> \\
&=&\exp \left( -\left| c\right| ^{2}\right) \sum_{m=0}^{\infty
}\sum_{n=0}^{\infty }\frac{c^{\ast m}}{\sqrt{m!}}\frac{c^{n}}{\sqrt{n!}}%
\delta _{m}^{n} \\
&=&\exp \left( -\left| c\right| ^{2}\right) \sum_{n=0}^{\infty }\frac{%
c^{\ast n}c^{n}}{n!} \\
&=&\exp \left( -\left| c\right| ^{2}\right) \sum_{n=0}^{\infty }\frac{\left|
c\right| ^{n}}{n!} \\
&=&\exp \left( -\left| c\right| ^{2}\right) \exp \left( +\left| c\right|
^{2}\right) \\
&=&1.\blacksquare 
\end{eqnarray*}

\item $\left| c\right> $ es un eigenestado del operador de destrucci%
'on $a_{r}\left( {\bf k}\right) $ con el eigenvalor complejo $c$: 
\begin{equation}
a_{r}\left( {\bf k}\right) \left| c\right> =c\left| c\right> 
\end{equation}

\textbf{Respuesta}: En efecto, recordemos\footnote{%
He preferido suprimir el sub'indice $r$ y la dependencia en ${\bf k}$
de todas las variables que pone el libro, para hacer la notaci'on un poco
m'as compacta y agradable.} que 
\begin{equation}
\left| c\right> =\exp \left( -\frac{1}{2}\left| c\right| ^{2}\right)
\sum_{n=0}^{\infty }\frac{c^{n}}{\sqrt{n!}}\left| n\right> 
\end{equation}
y 
\begin{equation}
\left| n\right> =\frac{\left[ a^{\dagger }\right] ^{n}}{\sqrt{n!}}%
\left| 0\right> 
\end{equation}
Primero que nada, sabemos que 
\begin{equation}
a\left| n\right> =\sqrt{n}\left| n-1\right> 
\end{equation}
as'i que 
\begin{eqnarray*}
a\left| c\right> &=&\exp \left( -\frac{1}{2}\left| c\right|
^{2}\right) \sum_{n=0}^{\infty }\frac{c^{n}}{\sqrt{n!}}a\left|
n\right> \\
&=&\exp \left( -\frac{1}{2}\left| c\right| ^{2}\right) \sum_{n=1}^{\infty }%
\frac{c^{n}}{\sqrt{n!}}\sqrt{n}\left| n-1\right> 
\end{eqnarray*}
pues 
\begin{equation}
a\left| 0\right> =0
\end{equation}
Y as'i, 
\begin{eqnarray*}
a\left| c\right> &=&\exp \left( -\frac{1}{2}\left| c\right|
^{2}\right) \sum_{n=1}^{\infty }\frac{c^{n}}{\sqrt{\left( n-1\right) !}}%
\left| n-1\right> \\
&=&c\exp \left( -\frac{1}{2}\left| c\right| ^{2}\right) \sum_{n=1}^{\infty }%
\frac{c^{n-1}}{\sqrt{\left( n-1\right) !}}\left| n-1\right> \\
&=&c\exp \left( -\frac{1}{2}\left| c\right| ^{2}\right) \sum_{n=0}^{\infty }%
\frac{c^{n}}{\sqrt{n!}}\left| n\right> \\
&=&c\left| c\right> .\blacksquare 
\end{eqnarray*}

\item El n'umero medio $\bar{N}$ de fotones en el recinto en el estado $%
\left| c\right> $, es dado por 
\begin{equation}
\bar{N}=\left\langle c\right| N\left| c\right> =\left| c\right| ^{2}
\end{equation}
donde $N$ es operador de n'umero total de fotones.

\textbf{Respuesta}: En efecto, como
\begin{equation}
N\equiv a^{\dagger }a
\end{equation}
entonces 
\begin{eqnarray*}
\bar{N} &=&\left\langle c\right| N\left| c\right> \\
&=&\left\langle c\right| a^{\dagger }a\left| c\right> 
\end{eqnarray*}
Como de la pregunta anterior 
\begin{equation}
a\left| c\right> =c\left| c\right> ,
\end{equation}
entonces 
\begin{eqnarray*}
\bar{N} &=&\left\langle c\right| c^{\ast }c\left| c\right> \\
&=&\left| c\right| ^{2}\left\langle c|c\right> \\
&=&\left| c\right| ^{2}.\blacksquare 
\end{eqnarray*}

\item La fluctuaci'on cuadr'atica media $\Delta N$ en el n'umero
de fotones en el recinto, en el estado $\left| c\right> $, es dado por 
\begin{equation}
\left( \Delta N\right) ^{2}=\left\langle c\right| N^{2}\left| c\right>
-\bar{N}^{2}=\left| c\right| ^{2} 
\end{equation}

\textbf{Respuesta}: En efecto,
\begin{eqnarray*}
\left( \Delta N\right) ^{2} &=&\left\langle c\right| N^{2}\left|
c\right> -\bar{N}^{2} \\
&=&\left\langle c\right| a^{\dagger }aa^{\dagger }a\left| c\right>
-\left| c\right| ^{4} \\
&=&\left\langle c\right| c^{\ast }aa^{\dagger }c\left| c\right>
-\left| c\right| ^{4} \\
&=&\left| c\right| ^{2}\left\langle c\right| aa^{\dagger }\left|
c\right> -\left| c\right| ^{4}
\end{eqnarray*}
Como 
\begin{eqnarray*}
\left[ a,a^{\dagger }\right] &=&1 \\
aa^{\dagger }-a^{\dagger }a &=&1 \\
aa^{\dagger } &=&1+a^{\dagger }a
\end{eqnarray*}
entonces 
\begin{eqnarray*}
\left( \Delta N\right) ^{2} &=&\left| c\right| ^{2}\left\langle c\right|
\left( 1+a^{\dagger }a\right) \left| c\right> -\left| c\right| ^{4} \\
&=&\left| c\right| ^{2}\left[ \left\langle c|c\right> +\left\langle
c\right| a^{\dagger }a\left| c\right> \right] -\left| c\right| ^{4} \\
&=&\left| c\right| ^{2}\left[ 1+\left| c\right| ^{2}\right] -\left| c\right|
^{4} \\
&=&\left| c\right| ^{2}+\left| c\right| ^{4}-\left| c\right| ^{4} \\
&=&\left| c\right| ^{2}.\blacksquare
\end{eqnarray*}

\item El valor de expectaci'on del campo el'ectrico $\vec{E}$, en
el estado $\left| c\right> $, est'a dado por 
\begin{equation}
\left\langle c\right| \vec{E}\left| c\right> =-2{\bf \varepsilon 
}_{r}\left( {\bf k}\right) \sqrt{\frac{\hbar \omega _{{\bf k}}}{2V}}%
\left| c\right| \sin\left( {\bf k\cdot x}-\omega _{{\bf k}%
}t+\delta \right) 
\end{equation}
donde $V$ es el volumen del recinto.

\textbf{Respuesta}: En efecto, como
\begin{equation}
\vec{A}\left( \vec{x},t\right) =\sum_{{\bf k}}\sum_{r}\sqrt{\frac{%
\hbar c^{2}}{2V\omega _{{\bf k}}}}{\bf \varepsilon }_{r}\left( {\bf %
k}\right) \left[ a_{r}\left( {\bf k}\right) e^{i\left( {\bf k}\cdot 
\vec{x}-\omega _{{\bf k}}t\right) }+a_{r}^{\dagger }\left( {\bf k}%
\right) e^{-i\left( {\bf k}\cdot \vec{x}-\omega _{{\bf k}}t\right) }%
\right] 
\end{equation}
Como no hay fuentes, podemos usar 
\begin{equation}
\phi =0 
\end{equation}
y entonces, 
\begin{equation}
\vec{E}\left( \vec{x},t\right) =-\frac{1}{c}\frac{\partial \vec{A}%
\left( \vec{x},t\right) }{\partial t}=\vec{E}_{T}\left( \vec{x}%
,t\right) 
\end{equation}
As'i, 
\begin{eqnarray*}
\vec{E}\left( \vec{x},t\right) &=&-\frac{1}{c}\frac{\partial }{%
\partial t}\sum_{{\bf k}}\sum_{r}\sqrt{\frac{\hbar c^{2}}{2V\omega _{%
{\bf k}}}}{\bf \varepsilon }_{r}\left( {\bf k}\right) \left[
a_{r}\left( {\bf k}\right) e^{i\left( {\bf k}\cdot \vec{x}-\omega _{%
{\bf k}}t\right) }+a_{r}^{\dagger }\left( {\bf k}\right) e^{-i\left( 
{\bf k}\cdot \vec{x}-\omega _{{\bf k}}t\right) }\right] \\
&=&-\frac{1}{c}\sum_{{\bf k}}\sum_{r}\sqrt{\frac{\hbar c^{2}}{2V\omega _{%
{\bf k}}}}{\bf \varepsilon }_{r}\left( {\bf k}\right) \left[
a_{r}\left( {\bf k}\right) \frac{\partial }{\partial t}e^{i\left( {\bf %
k}\cdot \vec{x}-\omega _{{\bf k}}t\right) }+a_{r}^{\dagger }\left( 
{\bf k}\right) \frac{\partial }{\partial t}e^{-i\left( {\bf k}\cdot 
\vec{x}-\omega _{{\bf k}}t\right) }\right] \\
&=&-\frac{1}{c}\sum_{{\bf k}}\sum_{r}\sqrt{\frac{\hbar c^{2}}{2V\omega _{%
{\bf k}}}}{\bf \varepsilon }_{r}\left( {\bf k}\right) \left[
a_{r}\left( {\bf k}\right) \left( -i\omega _{{\bf k}}\right)
e^{i\left( {\bf k}\cdot \vec{x}-\omega _{{\bf k}}t\right)
}+a_{r}^{\dagger }\left( {\bf k}\right) \left( i\omega _{{\bf k}%
}\right) e^{-i\left( {\bf k}\cdot \vec{x}-\omega _{{\bf k}}t\right)
}\right] \\
&=&-i\frac{1}{c}\sum_{{\bf k}}\sum_{r}\sqrt{\frac{\hbar c^{2}}{2V\omega _{%
{\bf k}}}}{\bf \varepsilon }_{r}\left( {\bf k}\right) \omega _{%
{\bf k}}\left[ a_{r}^{\dagger }\left( {\bf k}\right) e^{-i\left( 
{\bf k}\cdot \vec{x}-\omega _{{\bf k}}t\right) }-a_{r}\left( 
{\bf k}\right) e^{i\left( {\bf k}\cdot \vec{x}-\omega _{{\bf k}%
}t\right) }\right] \\
&=&-i\sum_{{\bf k}}\sum_{r}\sqrt{\frac{\hbar \omega _{{\bf k}}}{2V}}%
{\bf \varepsilon }_{r}\left( {\bf k}\right) \left[ a_{r}^{\dagger
}\left( {\bf k}\right) e^{-i\left( {\bf k}\cdot \vec{x}-\omega _{%
{\bf k}}t\right) }-a_{r}\left( {\bf k}\right) e^{i\left( {\bf k}%
\cdot \vec{x}-\omega _{{\bf k}}t\right) }\right] \\
&=&i\sum_{{\bf k}}\sum_{r}\sqrt{\frac{\hbar \omega _{{\bf k}}}{2V}}%
{\bf \varepsilon }_{r}\left( {\bf k}\right) \left[ a_{r}\left( {\bf %
k}\right) e^{i\left( {\bf k}\cdot \vec{x}-\omega _{{\bf k}}t\right)
}-a_{r}^{\dagger }\left( {\bf k}\right) e^{-i\left( {\bf k}\cdot 
\vec{x}-\omega _{{\bf k}}t\right) }\right]
\end{eqnarray*}
\begin{equation}
\vec{E}\left( \vec{x},t\right) =i\sum_{{\bf k}}\sum_{r}\sqrt{\frac{%
\hbar \omega _{{\bf k}}}{2V}}{\bf \varepsilon }_{r}\left( {\bf k}%
\right) \left[ a_{r}\left( {\bf k}\right) e^{i\left( {\bf k}\cdot 
\vec{x}-\omega _{{\bf k}}t\right) }-a_{r}^{\dagger }\left( {\bf k}%
\right) e^{-i\left( {\bf k}\cdot \vec{x}-\omega _{{\bf k}}t\right) }%
\right] 
\end{equation}
Como el estado $\left| c\right> $ s'olo contiene fotones con
s'olo \textbf{un} vector de onda ${\bf k}$ y \textbf{un} vector $%
{\bf \varepsilon }_{r}\left( {\bf k}\right) $, y no otros, entonces 
\begin{eqnarray*}
\left\langle c\right| \vec{E}\left| c\right> &=&\left\langle
c\right| i\sum_{{\bf k}}\sum_{r}\sqrt{\frac{\hbar \omega _{{\bf k}}}{2V%
}}{\bf \varepsilon }_{r}\left( {\bf k}\right) \left[ a_{r}\left( 
{\bf k}\right) e^{i\left( {\bf k}\cdot \vec{x}-\omega _{{\bf k}%
}t\right) }-a_{r}^{\dagger }\left( {\bf k}\right) e^{-i\left( {\bf k}%
\cdot \vec{x}-\omega _{{\bf k}}t\right) }\right] \left| c\right>
\\
&=&\left\langle c\right| i\sqrt{\frac{\hbar \omega _{{\bf k}}}{2V}}%
{\bf \varepsilon }_{r}\left( {\bf k}\right) \left[ a_{r}\left( {\bf %
k}\right) e^{i\left( {\bf k}\cdot \vec{x}-\omega _{{\bf k}}t\right)
}-a_{r}^{\dagger }\left( {\bf k}\right) e^{-i\left( {\bf k}\cdot 
\vec{x}-\omega _{{\bf k}}t\right) }\right] \left| c\right>
\end{eqnarray*}
ya que los otros estados posibles no est'an en la cavidad. Entonces, 
\begin{eqnarray*}
\left\langle c\right| \vec{E}\left| c\right> &=&i\sqrt{\frac{\hbar
\omega _{{\bf k}}}{2V}}{\bf \varepsilon }_{r}\left( {\bf k}\right)
\left\langle c\right| \left[ a_{r}\left( {\bf k}\right) e^{i\left( 
{\bf k}\cdot \vec{x}-\omega _{{\bf k}}t\right) }-a_{r}^{\dagger
}\left( {\bf k}\right) e^{-i\left( {\bf k}\cdot \vec{x}-\omega _{%
{\bf k}}t\right) }\right] \left| c\right> \\
&=&i\sqrt{\frac{\hbar \omega _{{\bf k}}}{2V}}{\bf \varepsilon }%
_{r}\left( {\bf k}\right) \left[ \left\langle c\right| a_{r}\left( 
{\bf k}\right) \left| c\right> e^{i\left( {\bf k}\cdot \vec{x}%
-\omega _{{\bf k}}t\right) }-\left\langle c\right| a_{r}^{\dagger }\left(
{\bf k}\right) \left| c\right> e^{-i\left( {\bf k}\cdot {\bf x%
}-\omega _{{\bf k}}t\right) }\right] \\
&=&i\sqrt{\frac{\hbar \omega _{{\bf k}}}{2V}}{\bf \varepsilon }%
_{r}\left( {\bf k}\right) \left[ \left\langle c\right| c\left|
c\right> e^{i\left( {\bf k}\cdot \vec{x}-\omega _{{\bf k}%
}t\right) }-\left\langle c\right| c^{\ast }\left| c\right> e^{-i\left(
{\bf k}\cdot \vec{x}-\omega _{{\bf k}}t\right) }\right] \\
&=&i\sqrt{\frac{\hbar \omega _{{\bf k}}}{2V}}{\bf \varepsilon }%
_{r}\left( {\bf k}\right) \left[ ce^{i\left( {\bf k}\cdot \vec{x}%
-\omega _{{\bf k}}t\right) }-c^{\ast }e^{-i\left( {\bf k}\cdot {\bf %
x}-\omega _{{\bf k}}t\right) }\right] \\
&=&i\sqrt{\frac{\hbar \omega _{{\bf k}}}{2V}}{\bf \varepsilon }%
_{r}\left( {\bf k}\right) \left[ ce^{i\left( {\bf k}\cdot \vec{x}%
-\omega _{{\bf k}}t\right) }-\left( ce^{i\left( {\bf k}\cdot \vec{x}%
-\omega _{{\bf k}}t\right) }\right) ^{\ast }\right]
\end{eqnarray*}
Si ahora escribimos 
\begin{equation}
c=\left| c\right| e^{i\delta } 
\end{equation}
entonces 
\begin{equation}
\left\langle c\right| \vec{E}\left| c\right> =i\sqrt{\frac{\hbar
\omega _{{\bf k}}}{2V}}{\bf \varepsilon }_{r}\left( {\bf k}\right)
\left| c\right| \left[ e^{i\left( {\bf k}\cdot \vec{x}-\omega _{%
{\bf k}}t+\delta \right) }-\left( e^{i\left( {\bf k}\cdot \vec{x}%
-\omega _{{\bf k}}t+\delta \right) }\right) ^{\ast }\right] 
\end{equation}
Como 
\begin{eqnarray*}
z &=&z_{R}+iz_{C} \\
z-z^{\ast } &=&z_{R}+iz_{C}-z_{R}+iz_{C} \\
&=&2iz_{C}
\end{eqnarray*}
entonces, 
\begin{eqnarray*}
\left\langle c\right| \vec{E}\left| c\right> &=&i\sqrt{\frac{\hbar
\omega _{{\bf k}}}{2V}}{\bf \varepsilon }_{r}\left( {\bf k}\right)
2i\left| c\right| \sin\left( {\bf k}\cdot \vec{x}-\omega _{%
{\bf k}}t\right) \\
&=&-2\sqrt{\frac{\hbar \omega _{{\bf k}}}{2V}}{\bf \varepsilon }%
_{r}\left( {\bf k}\right) \left| c\right| \sin\left( {\bf k}%
\cdot \vec{x}-\omega _{{\bf k}}t\right) .\blacksquare
\end{eqnarray*}

\item La fluctuaci'on cuadr'atica media del campo el'ectrico, $%
\Delta E$ en el estado $\left| c\right> $ est'a dada por 
\begin{equation}
\left( \Delta E\right) ^{2}=\left\langle c\right| \vec{E}^{2}\left|
c\right> -\left[ \left\langle c\right| \vec{E}\left| c\right> %
\right] ^{2}=\frac{\hbar \omega _{{\bf k}}}{2V} 
\end{equation}

\textbf{Respuesta}: En nuestro caso, s'olo nos interesa el campo
el'ectrico dado por 
\begin{equation}
\vec{E}=i\sqrt{\frac{\hbar \omega _{{\bf k}}}{2V}}{\bf \varepsilon }%
_{r}\left( {\bf k}\right) \left[ a_{r}\left( {\bf k}\right) e^{i\left( 
{\bf k}\cdot \vec{x}-\omega _{{\bf k}}t\right) }-a_{r}^{\dagger
}\left( {\bf k}\right) e^{-i\left( {\bf k}\cdot \vec{x}-\omega _{%
{\bf k}}t\right) }\right] 
\end{equation}
\begin{eqnarray*}
\vec{E} &=&i\sqrt{\frac{\hbar \omega _{{\bf k}}}{2V}}{\bf %
\varepsilon }\left[ ae^{i\left( {\bf k}\cdot \vec{x}-\omega _{{\bf k%
}}t\right) }-a^{\dagger }e^{-i\left( {\bf k}\cdot \vec{x}-\omega _{%
{\bf k}}t\right) }\right] \\
\vec{E}^{2} &=&-\frac{\hbar \omega _{{\bf k}}}{2V}\left[ ae^{i\left( 
{\bf k}\cdot \vec{x}-\omega _{{\bf k}}t\right) }-a^{\dagger
}e^{-i\left( {\bf k}\cdot \vec{x}-\omega _{{\bf k}}t\right) }\right]
\left[ ae^{i\left( {\bf k}\cdot \vec{x}-\omega _{{\bf k}}t\right)
}-a^{\dagger }e^{-i\left( {\bf k}\cdot \vec{x}-\omega _{{\bf k}%
}t\right) }\right] \\
&=&-\frac{\hbar \omega _{{\bf k}}}{2V}\left[ a^{2}e^{2i\left( {\bf k}%
\cdot \vec{x}-\omega _{{\bf k}}t\right) }+\left( a^{\dagger }\right)
^{2}e^{-2i\left( {\bf k}\cdot \vec{x}-\omega _{{\bf k}}t\right)
}-aa^{\dagger }-a^{\dagger }a\right]
\end{eqnarray*}
como 
\begin{eqnarray*}
\left[ a,a^{\dagger }\right] &=&1 \\
aa^{\dagger }-a^{\dagger }a &=&1 \\
aa^{\dagger } &=&1+a^{\dagger }a
\end{eqnarray*}
\begin{eqnarray*}
\vec{E}^{2} &=&-\frac{\hbar \omega _{{\bf k}}}{2V}\left[
a^{2}e^{2i\left( {\bf k}\cdot \vec{x}-\omega _{{\bf k}}t\right)
}+\left( a^{\dagger }\right) ^{2}e^{-2i\left( {\bf k}\cdot \vec{x}%
-\omega _{{\bf k}}t\right) }-aa^{\dagger }-a^{\dagger }a\right] \\
&=&-\frac{\hbar \omega _{{\bf k}}}{2V}\left[ a^{2}e^{2i\left( {\bf k}%
\cdot \vec{x}-\omega _{{\bf k}}t\right) }+\left( a^{\dagger }\right)
^{2}e^{-2i\left( {\bf k}\cdot \vec{x}-\omega _{{\bf k}}t\right)
}-1-a^{\dagger }a-a^{\dagger }a\right] \\
&=&-\frac{\hbar \omega _{{\bf k}}}{2V}\left[ a^{2}e^{2i\left( {\bf k}%
\cdot \vec{x}-\omega _{{\bf k}}t\right) }+\left( a^{\dagger }\right)
^{2}e^{-2i\left( {\bf k}\cdot \vec{x}-\omega _{{\bf k}}t\right)
}-1-2a^{\dagger }a\right]
\end{eqnarray*}
y entonces, 
\begin{eqnarray*}
\left\langle c\right| \vec{E}^{2}\left| c\right> &=&-\frac{\hbar
\omega _{{\bf k}}}{2V}\left\langle c\right| \left[ a^{2}e^{2i\left( 
{\bf k}\cdot \vec{x}-\omega _{{\bf k}}t\right) }+\left( a^{\dagger
}\right) ^{2}e^{-2i\left( {\bf k}\cdot \vec{x}-\omega _{{\bf k}%
}t\right) }-1-2a^{\dagger }a\right] \left| c\right> \\
&=&-\frac{\hbar \omega _{{\bf k}}}{2V}\left[ \left\langle c\right|
a^{2}\left| c\right> e^{2i\left( {\bf k}\cdot \vec{x}-\omega _{%
{\bf k}}t\right) }+\left\langle c\right| \left( a^{\dagger }\right)
^{2}\left| c\right> e^{-2i\left( {\bf k}\cdot \vec{x}-\omega _{%
{\bf k}}t\right) }-\left\langle c|c\right> -2\left\langle c\right|
a^{\dagger }a\left| c\right> \right] \\
&=&-\frac{\hbar \omega _{{\bf k}}}{2V}\left[ \left\langle c\right|
cc\left| c\right> e^{2i\left( {\bf k}\cdot \vec{x}-\omega _{%
{\bf k}}t\right) }+\left\langle c\right| c^{\ast }c^{\ast }\left|
c\right> e^{-2i\left( {\bf k}\cdot \vec{x}-\omega _{{\bf k}%
}t\right) }-\left\langle c|c\right> -2\left\langle c\right| c^{\ast
}c\left| c\right> \right] \\
&=&-\frac{\hbar \omega _{{\bf k}}}{2V}\left[ c^{2}e^{2i\left( {\bf k}%
\cdot \vec{x}-\omega _{{\bf k}}t\right) }+\left( c^{\ast }\right)
^{2}e^{-2i\left( {\bf k}\cdot \vec{x}-\omega _{{\bf k}}t\right)
}-1-2\left| c\right| ^{2}\right] \\
&=&-\frac{\hbar \omega _{{\bf k}}}{2V}\left[ \left| c\right|
^{2}e^{2i\delta }e^{2i\left( {\bf k}\cdot \vec{x}-\omega _{{\bf k}%
}t\right) }+\left| c\right| ^{2}e^{-2i\delta }e^{-2i\left( {\bf k}\cdot 
\vec{x}-\omega _{{\bf k}}t\right) }-1-2\left| c\right| ^{2}\right] \\
&=&-\frac{\hbar \omega _{{\bf k}}}{2V}\left[ \left| c\right|
^{2}e^{2i\left( {\bf k}\cdot \vec{x}-\omega _{{\bf k}}t+\delta
\right) }+\left| c\right| ^{2}e^{-2i\left( {\bf k}\cdot \vec{x}-\omega
_{{\bf k}}t+\delta \right) }-1-2\left| c\right| ^{2}\right] \\
&=&-\frac{\hbar \omega _{{\bf k}}}{2V}\left[ \left| c\right| ^{2}\left(
e^{2i\left( {\bf k}\cdot \vec{x}-\omega _{{\bf k}}t+\delta \right)
}+e^{-2i\left( {\bf k}\cdot \vec{x}-\omega _{{\bf k}}t+\delta
\right) }\right) -1-2\left| c\right| ^{2}\right] \\
&=&-\frac{\hbar \omega _{{\bf k}}}{2V}\left[ 2\left| c\right| ^{2}\cos
2\left( {\bf k}\cdot \vec{x}-\omega _{{\bf k}}t+\delta \right)
-1-2\left| c\right| ^{2}\right] \\
&=&-\frac{\hbar \omega _{{\bf k}}}{2V}\left[ 2\left| c\right| ^{2}\left(
1-2\sin^{2}\left( {\bf k}\cdot \vec{x}-\omega _{{\bf k}%
}t+\delta \right) \right) -1-2\left| c\right| ^{2}\right] \\
&=&-\frac{\hbar \omega _{{\bf k}}}{2V}\left[ 2\left| c\right|
^{2}-4\left| c\right| ^{2}\sin^{2}\left( {\bf k}\cdot \vec{x}%
-\omega _{{\bf k}}t+\delta \right) -1-2\left| c\right| ^{2}\right] \\
&=&-\frac{\hbar \omega _{{\bf k}}}{2V}\left[ -4\left| c\right| ^{2}\sin^{2}\left( {\bf k}\cdot \vec{x}-\omega _{{\bf k}}t+\delta
\right) -1\right] \\
&=&\frac{\hbar \omega _{{\bf k}}}{2V}\left[ 4\left| c\right| ^{2}\sin^{2}\left( {\bf k}\cdot \vec{x}-\omega _{{\bf k}}t+\delta \right)
+1\right]
\end{eqnarray*}
Por otra parte, 
\begin{eqnarray*}
\left\langle c\right| \vec{E}\left| c\right> &=&-2\sqrt{\frac{\hbar
\omega _{{\bf k}}}{2V}}{\bf \varepsilon }_{r}\left( {\bf k}\right)
\left| c\right| \sin\left( {\bf k}\cdot \vec{x}-\omega _{{\bf %
k}}t\right) \\
\left\langle c\right| \vec{E}\left| c\right> ^{2} &=&2\frac{\hbar
\omega _{{\bf k}}}{V}\left| c\right| ^{2}\sin^{2}\left( {\bf k}%
\cdot \vec{x}-\omega _{{\bf k}}t\right)
\end{eqnarray*}
As'i que 
\begin{eqnarray*}
\left( \Delta E\right) ^{2} &=&\left\langle c\right| \vec{E}^{2}\left|
c\right> -\left[ \left\langle c\right| \vec{E}\left| c\right> %
\right] ^{2} \\
&=&\frac{\hbar \omega _{{\bf k}}}{2V}\left[ 4\left| c\right| ^{2}\sin^{2}\left( {\bf k}\cdot \vec{x}-\omega _{{\bf k}}t+\delta \right)
+1\right] -2\frac{\hbar \omega _{{\bf k}}}{V}\left| c\right| ^{2}\sin^{2}\left( {\bf k}\cdot \vec{x}-\omega _{{\bf k}}t\right) \\
&=&2\frac{\hbar \omega _{{\bf k}}}{V}\left| c\right| ^{2}\sin%
^{2}\left( {\bf k}\cdot \vec{x}-\omega _{{\bf k}}t+\delta \right) +%
\frac{\hbar \omega _{{\bf k}}}{2V}-2\frac{\hbar \omega _{{\bf k}}}{V}%
\left| c\right| ^{2}\sin^{2}\left( {\bf k}\cdot \vec{x}-\omega _{%
{\bf k}}t\right) \\
&=&\frac{\hbar \omega _{{\bf k}}}{2V}
\end{eqnarray*}
\end{enumerate}
\end{enumerate}

 Finalmente, consideramos el estado "localizado" $\left| x\right> :=\sqrt{\frac{4\pi\lambda\omega_{k'}}{\hbar}}\hat{\phi}(x) \left| 0\right> $
\begin{eqnarray}
\left\langle x|  x'\right> & = &\left\langle 0\right| \left[ \sqrt{\frac{4\pi\lambda
\omega_{k}}{\hbar}}\hat{\phi}^{\dagger}(x) \right] \left[\sqrt{\frac{4\pi\lambda\omega_{k'}}{\hbar}}\hat{\phi}(x) \right] \left| 0\right> \\
& = &\left\langle 0\right| \left[ \sqrt{\frac{\hbar}{4\pi\lambda}}\sum_k\sqrt{\frac{4\pi\lambda\omega_{k}}{\hbar}} \left( \omega_{k}\right)
^{-\frac{1}{2}}\left\{ \hat{b}_{k}^{\dagger}e^{-ikx}+\hat{b}_{k}%
e^{ikx}\right\} \right] \nonumber \\
&&\qquad \left[ \sqrt{\frac{\hbar}{4\pi\lambda}}\sum_{k'}\sqrt{\frac{4\pi\lambda\omega_{k'}}{\hbar}}\left( \omega_{k'}\right) ^{-\frac{1}{2}}\left\{ \hat{b}_{k'}e^{ik'x'}+\hat{b}_{k'}^{\dagger}e^{-ik'x'}\right\} \right] \left| 0\right> \\
& = &\left\langle 0\right| \left[ \sum_k\left\{ \hat{b}_{k}^{\dagger
}e^{-ikx}+\hat{b}_{k}e^{ikx}\right\} \right] \left[ \sum_{k'}\left\{ \hat{b}_{k'}e^{ik'x'}+\hat{b}_{k'}^{\dagger}e^{-ik'x'}\right\} \right] \left| 0\right> \\
& = &\sum_{kk'}\left\{\left\langle 0\right| \hat{b}_{k}^{\dagger}\hat{b}_{k'}\left| 0\right> e^{-ikx+ik'x'}+\left\langle 0\right| \hat{b}_{k}^{\dagger}\hat{b}_{k'}^{\dagger}\left| 0\right> e^{-ikx-ik'x'}\right. \nonumber \\
&& \qquad \left.+\left\langle 0\right| \hat{b}_{k}\hat{b}_{k'}\left| 0\right> e^{ikx+ik'x'}
+\left\langle 0\right| \hat{b}_{k}\hat{b}_{k'}^{\dagger}\left| 0\right> e^{ikx-ik'x'}%
\right\} \\
& = &\sum_{kk'}\left\{\left\langle 0\right| \hat{b}_{k}^{\dagger}\left| 0,\dots,1_{k'},0,\dots\right> e^{-ikx-ik'x'} \right.\nonumber \\
&&\qquad \left.+\left\langle 0\right| \left[ \delta(k-k') +\hat{b}_{k'}^{\dagger}\hat{b}_{k}\right] \left| 0\right> e^{ikx-ik'x'}\right\} \\
& = &\sum_{kk'}\left\{\left\langle 0\right| \left. 0,\dots,1_{k},1_{k'},0,\dots\right> e^{-ikx-ik'x'}+\delta\left( k-k\right) \left\langle 0\right| \left. 0\right> e^{ikx-ik'x'}\right. \nonumber \\
&&\qquad\left.+\left\langle 0\right| \hat{b}_{k'}^{\dagger}\hat{b}_{k}\left| 0\right> e^{ikx-ik'x'}\right\} \\
& = &\sum_{kk'}e^{ikx-ik'x'}\delta(k-k') \\
& = &\sum_k e^{i(x-x') k}\\
& = &\delta(x-x') ,
\end{eqnarray}
\begin{eqnarray}
\left\langle x\right| \left. k\right> & = &\left\langle 0\right|
\left[ \sqrt{\frac{4\pi\lambda\omega_{k}}{\hbar}}\hat{\phi}^{\dagger}\left(
x\right) \right] \hat{b}_{k}^{\dagger}\left| 0\right> \\
& = &\left\langle 0\right| \left[ \sqrt{\frac{\hbar}{4\pi\lambda}}\sum_{k'}\sqrt{\frac{4\pi\lambda\omega_{k'}}{\hbar}}\left( \omega_{k'}\right) ^{-\frac{1}{2}}\left\{ \hat{b}_{k'}^{\dagger}e^{-ik'x}+\hat{b}_{k'}e^{ik'x}\right\} \right] \hat{b}_{k}^{\dagger}\left| 0\right> \\
& = &\left\langle 0\right| \left[ \sum_{k'}\left\{ \hat{b}_{k'}^{\dagger}\hat{b}_{k}^{\dagger}e^{-ik'x}+\hat{b}_{k'}\hat{b}_{k}^{\dagger}e^{ik'x}\right\} \right] \left| 0\right> \\
& = &\sum_{k'}\left\{ \left\langle 0\right| \hat{b}_{k'}^{\dagger}\hat{b}_{k}^{\dagger}\left| 0\right> e^{-ik'x}+\left\langle 0\right| \hat{b}_{k'}\hat{b}_{k}^{\dagger}\left| 0\right> e^{ik'x}\right\} \\
& = &\sum_{k'}\left\{ \left\langle 0\right| \hat{b}_{k'}^{\dagger}\hat{b}_{k}^{\dagger}\left| 0\right> e^{-ik'x}+\left\langle 0\right| \left[ \hat{b}_{k}^{\dagger}\hat{b}_{k'}+\delta(k-k') \right] \left| 0\right> e^{ik'x}\right\} \\
& = &\sum_{k'}\left\{ \left\langle 0\right| \hat{b}_{k'}^{\dagger}\hat{b}_{k}^{\dagger}\left| 0\right> e^{-ik'x}+\left\langle 0\right| \hat{b}_{k}^{\dagger}\hat{b}_{k'}\left| 0\right> e^{ik'x}+\left\langle 0\right| \delta(k-k') \left| 0\right> e^{ik'x}\right\} \\
& = &\sum_{k'}\left\{ \left\langle 0\right| \hat{b}_{k'}^{\dagger}\left| 0,\dots,1_{k},0,\dots\right> e^{-ik'x}+\delta(k-k') \left\langle 0\right| \left. 0\right> e^{ik'x}\right\} \\
& = &\sum_{k'}\left\{ \left\langle 0\right| \left. 0,\dots,1_{k},1_{k'},0,\dots\right> e^{-ik'x}+\delta(k-k') \left\langle 0\right| \left. 0\right> e^{ik'x}\right\} \\
& = &\sum_{k'}\delta(k-k') e^{ik'x}\\
& = &e^{ikx} .
\end{eqnarray}


\subsubsection{Interpretaci'on}
Luego, hacemos el siguiente cambio de variables:%
\begin{eqnarray}
a_{\vec{k}\sigma} & = &\sqrt{\frac{\omega_{k}}{2\hbar}}\left( \hat{q}%
_{\vec{k}\sigma}+i\frac{\hat{p}_{\vec{k}\sigma}}{\omega_{k}}\right)
\label{variables 1}\\
a_{\vec{k}\sigma}^{\dagger} & = &\sqrt{\frac{\omega_{k}}{2\hbar}}\left(
\hat{q}_{\vec{k}\sigma}-i\frac{\hat{p}_{\vec{k}\sigma}}{\omega_{k}}\right)
\label{variables 2}%
\end{eqnarray}
de manera que el Hamiltoniano se transforme en:%
\begin{eqnarray}
\hat{H} & = &\frac{\hbar}{2}\sum_{\vec{k}\sigma}\omega_{k}\frac{\omega
_{k}}{2\hbar}\left\{ \left( \hat{q}_{\vec{k}\sigma}+i\frac{\hat{p}_{\vec
{k}\sigma}}{\omega_{k}}\right) \left( \hat{q}_{\vec{k}\sigma}-i\frac
{\hat{p}_{\vec{k}\sigma}}{\omega_{k}}\right) +\left( \hat{q}_{\vec{k}%
\sigma}-i\frac{\hat{p}_{\vec{k}\sigma}}{\omega_{k}}\right) \left( \hat
{q}_{\vec{k}\sigma}+i\frac{\hat{p}_{\vec{k}\sigma}}{\omega_{k}}\right)
\right\} \\
& = &\sum_{\vec{k}\sigma}\left\{ \frac{1}{2}\hat{p}_{\vec{k}\sigma}%
^{2}+\frac{1}{2}\omega_{k}^{2}\hat{q}_{\vec{k}\sigma}^{2}\right\}
\end{eqnarray}
que corresponde al conocido Hamiltoniano de un conjunto de infinitos
osciladores arm'onicos desacoplados con polarizaci'on $\sigma,$ cuyas
variables "de campo" son.$\hat{p}_{\vec{k}\sigma}$ y $\hat{q}_{\vec{k}\sigma}%
$. Las relaci'ones de conmutaci'on entre estas variables tambien es bien
conocida, siendo:%
\begin{eqnarray}
\left[ \hat{q}_{\vec{k}\sigma},\hat{p}_{\vec{k}'\sigma '}\right] & = &i\hbar\delta_{\sigma\sigma '}\delta^{\left( 3\right) }\left( \vec{k}-\vec{k}'\right) \label{Conmutador Osciladores Desacoplados}\\
\left[ \hat{q}_{\vec{k}\sigma},\hat{q}_{\vec{k}'\sigma '}\right] & = &\left[ \hat{p}_{\vec{k}\sigma},\hat{p}_{\vec{k}'\sigma '}\right] =0\nonumber
\end{eqnarray}
de la cual podemos finalmente obtener la relaci'on de conmutaci'on que
tanto buscamos (la de nuestras nuevas variables del campo EM $\hat{a}_{\vec
{k}\sigma}$ y $\hat{a}_{\vec{k}\sigma}^{\dagger}$) simplemente invirtiendo las
ecuaciones(\ref{variables 1}), (\ref{variables 2}), y luego reemplazandolas en
(\ref{Conmutador Osciladores Desacoplados}). En efecto, al invertir las ecuaciones
mencionadas obtenemos:%
\begin{eqnarray}
\hat{q}_{\vec{k}\sigma} & = &\sqrt{\frac{\hbar}{2\omega_{k}}}\left(
a_{\vec{k}\sigma}+a_{\vec{k}\sigma}^{\dagger}\right) \\
\hat{p}_{\vec{k}\sigma} & = &-i\sqrt{\frac{\hbar \omega_{k}}{2}}\left(
a_{\vec{k}\sigma}-a_{\vec{k}\sigma}^{\dagger}\right)
\end{eqnarray}
y luego ocupando sus conmutadores:%
\begin{eqnarray}
\left[ \hat{q}_{\vec{k}\sigma},\hat{p}_{\vec{k}'\sigma'}\right] & = &i\hbar\delta_{\sigma\sigma'}\delta^{\left( 3\right) }\left( \vec{k}-\vec{k}'\right) \\
\left[ \sqrt{\frac{\hbar}{2\omega_{k}}}\left( a_{\vec{k}\sigma}+a_{\vec
{k}\sigma}^{\dagger}\right) ,-i\sqrt{\frac{\hbar \omega_{k'}}{2}}\left( a_{\vec{k}'\sigma '}-a_{\vec{k}'\sigma '}^{\dagger}\right) \right] & = &i\hbar\delta_{\sigma\sigma '}\delta^{\left( 3\right) }\left( \vec{k}-\vec{k}'\right) \\
-\frac{i\hbar}{2}\sqrt{\frac{\omega_{k'}}{\omega_{k}}}\left\{ -\left[ a_{\vec{k}\sigma},a_{\vec{k}'\sigma '}^{\dagger}\right] -\left[ a_{\vec{k}'\sigma '},a_{\vec{k}\sigma}^{\dagger}\right] \right\} & = &i\hbar\delta_{\sigma\sigma '}\delta^{\left( 3\right) }\left( \vec{k}-\vec{k}'\right)
\end{eqnarray}
pero, si calculamos%
\begin{eqnarray}
\left[ \hat{q}_{\vec{k}\sigma},\hat{q}_{\vec{k}'\sigma '}\right] & = &0\\
\left[ \sqrt{\frac{\hbar}{2\omega_{k}}}\left( a_{\vec{k}\sigma}+a_{\vec
{k}\sigma}^{\dagger}\right) ,\sqrt{\frac{\hbar}{2\omega_{k'}}}\left( a_{\vec{k}'\sigma '}+a_{\vec{k}'\sigma '}^{\dagger}\right) \right] & = &0\\
\left[ a_{\vec{k}\sigma},a_{\vec{k}'\sigma '}^{\dagger}\right] & = &\left[ a_{\vec{k}'\sigma '},a_{\vec{k}\sigma}^{\dagger}\right]
\end{eqnarray}
resultado que se corrobora con%
\begin{eqnarray}
\left[ \hat{p}_{\vec{k}\sigma},\hat{p}_{\vec{k}'\sigma '}\right] & = &0\\
\left[ -i\sqrt{\frac{\hbar \omega_{k}}{2}}\left( a_{\vec{k}\sigma}%
-a_{\vec{k}\sigma}^{\dagger}\right) ,-i\sqrt{\frac{\hbar \omega_{k}'}{2}}\left( a_{\vec{k}'\sigma '}-a_{\vec{k}'\sigma '}^{\dagger}\right) \right] & = &0\\
\left[ a_{\vec{k}'\sigma '},a_{\vec{k}\sigma}^{\dagger}\right] & = &\left[ a_{\vec{k}\sigma},a_{\vec{k}'\sigma '}^{\dagger}\right]
\end{eqnarray}
encontramos que:%
\begin{eqnarray}
-\frac{i\hbar}{2}\sqrt{\frac{\omega_{k'}}{\omega_{k}}}\left\{ -\left[ a_{\vec{k}\sigma},a_{\vec{k}'\sigma '}^{\dagger}\right] -\left[ a_{\vec{k}'\sigma '},a_{\vec{k}\sigma}^{\dagger}\right] \right\} & = &i\hbar\delta_{\sigma\sigma '}\delta^{\left( 3\right) }\left( \vec{k}-\vec{k}'\right) \\
i\hbar\sqrt{\frac{\omega_{k'}}{\omega_{k}}}\left[ a_{\vec{k}\sigma},a_{\vec{k}'\sigma'}^{\dagger}\right] & = &i\hbar\delta_{\sigma\sigma '}\delta^{\left( 3\right) }\left( \vec{k}-\vec{k}'\right) \\
\left[ a_{\vec{k}\sigma},a_{\vec{k}'\sigma'}^{\dagger}\right] & = &\sqrt{\frac{\omega_{k}}{\omega_{k'}}}\delta_{\sigma\sigma '}\delta^{\left( 3\right) }\left( \vec{k}-\vec{k}'\right)
\end{eqnarray}


Como estamos realizando la observaci'on (midiendo) en el mismo punto del
espacio de los $\vec{k}$ (de lo contrario sabemos que los operadores
conmutan), el resultado es no nulo solo cuando $\vec{k}=\vec{k}'$, por lo tanto $\omega_{k}=\omega_{k'}$ y:%
\begin{equation}
\left[ a_{\vec{k}\sigma},a_{\vec{k}'\sigma '}^{\dagger}\right] =\delta_{\sigma\sigma '}\delta^{\left( 3\right) }\left( \vec{k}-\vec{k}'\right)
\end{equation}


Los otros conmutadores son obvios:%
\begin{equation}
\left[ a_{\vec{k}\sigma},a_{\vec{k}'\sigma '}\right] =\left[ a_{\vec{k}\sigma}^{\dagger},a_{\vec{k}'\sigma '}^{\dagger}\right] =0
\end{equation}
por el razonamiento hecho anteriormente.

ara obtener la energ'ia y el momentum lineal del campo electromagn'etico en funci'on de nuestras nuevas variables de las amplitudes $a_{\vec{k}\sigma}(t)$ usaremos las
ecuaciones (\ref{Campo Electrico}), (\ref{Campo Magnetico}) para reemplazarlas en las
expresiones (\ref{Energia del Campo EM2}) y (\ref{Momentum Lineal Total EM2}).
En efecto\footnote{Otra manera alternativa, hubiese sido reemplazar
directamente (\ref{Ai(aksigma)}) en las expresiones
(\ref{Energia del Campo EM1}) y (\ref{Momentum Lineal Total EM1}). Pero la que
seguimos aqui es m'as did'actica.}:%

\begin{eqnarray} H & = &\frac{1}{2}\int_{V}\left( \varepsilon_{0}E^{2}+\frac{1}{\mu_{0}} B^{2}\right) dV\nonumber\\
& = &-\frac{1}{2}\sum_{\sigma,\sigma ' }\int d^{3}x\int d^{3}k\int d^{3}k ' N_{k}N_{k ' }\left[
\begin{array} [c]{c}
 \varepsilon_{0}\omega_{k}\omega_{k ' }\left( \hat{\varepsilon}_{\vec{k}\sigma}\cdot\hat{\varepsilon}_{\vec{k} ' \sigma ' }\right) \\ +\frac{1}{\mu_{0}}\left( \vec{k}\times\hat{\varepsilon}_{\vec{k}\sigma }\right) \cdot\left( \vec{k} ' \times\hat{\varepsilon}_{k ' \sigma ' }\right) \end{array} \right] \left\{ a_{\vec{k}\sigma}e^{i\vec{k}\cdot\vec{x}}-a_{\vec{k}\sigma }^{\ast}e^{-i\vec{k}\cdot\vec{x}}\right\} \left\{ a_{\vec{k} ' \sigma ' }e^{i\vec{k} ' \cdot\vec{x}}-a_{\vec{k} ' \sigma ' }^{\ast}e^{-i\vec{k} ' \cdot\vec{x}}\right\} \nonumber\\ & = &-\frac{1}{2}\sum_{\sigma,\sigma ' }\int d^{3}x\int d^{3}k\int d^{3}k ' N_{k}N_{k ' }\left[ \begin{array} [c]{c}%
 \varepsilon_{0}\omega_{k}\omega_{k ' }\left( \hat{\varepsilon}_{\vec{k}\sigma}\cdot\hat{\varepsilon}_{\vec{k} ' \sigma ' }\right) \\ +\frac{1}{\mu_{0}}\left\{ \left( \vec{k}\cdot\vec{k} ' \right) \left( \hat{\varepsilon}_{\vec{k}\sigma}\cdot\hat{\varepsilon}_{\vec{k} ' \sigma ' }\right) -\left( \hat{\varepsilon}_{\vec{k}\sigma}\cdot\vec{k} ' \right) \left( \hat{\varepsilon}_{\vec{k} ' \sigma ' }\cdot\vec{k}\right) \right\} \end{array} \right] \left\{ \begin{array} [c]{c}%
 a_{\vec{k}\sigma}a_{\vec{k} ' \sigma ' }e^{i\left( \vec{k}-\left( -\vec{k} ' \right) \right) \cdot\vec{x}}-a_{\vec{k}\sigma}a_{\vec{k} ' \sigma ' }^{\ast}e^{i\left( \vec{k}-\vec{k} ' \right) \cdot\vec{x}}\\ -a_{\vec{k}\sigma}^{\ast}a_{\vec{k} ' \sigma ' }e^{-i\left( \vec{k}-\vec{k} ' \right) \cdot\vec{x}}+a_{\vec{k}\sigma}^{\ast}a_{\vec{k} ' \sigma ' }^{\ast}e^{-i\left( \vec{k}-\left( -\vec{k} ' \right) \right) \cdot\vec{x}}%
\end{array} \right\} \nonumber\\ & = &-\frac{\left( 2\pi\right) ^{3}}{2}\sum_{\sigma,\sigma ' }\int d^{3}k\int d^{3}k ' N_{k}N_{k ' }\left[ \begin{array} [c]{c}%
\varepsilon_{0}\omega_{k}\omega_{k ' }\left( \hat{\varepsilon}_{\vec{k}\sigma}\cdot\hat{\varepsilon}_{\vec{k} ' \sigma ' }\right) \\ +\frac{1}{\mu_{0}}\left\{ \left( \vec{k}\cdot\vec{k} ' \right) \left( \hat{\varepsilon}_{\vec{k}\sigma}\cdot\hat{\varepsilon}_{\vec{k} ' \sigma ' }\right) -\left( \hat{\varepsilon}_{\vec{k}\sigma}\cdot\vec{k} ' \right) \left( \hat{\varepsilon}_{\vec{k} ' \sigma ' }\cdot\vec{k}\right) \right\} \end{array} \right] \left\{ \begin{array} [c]{c}%
\left( a_{\vec{k}\sigma}a_{\vec{k} ' \sigma ' }+a_{\vec{k}\sigma}^{\ast}a_{\vec{k} ' \sigma ' }^{\ast}\right) \delta^{\left( 3\right) }\left( \vec{k}-\left( -\vec{k} ' \right) \right) \\ -\left( a_{\vec{k}\sigma}a_{\vec{k} ' \sigma ' }^{\ast}+a_{\vec{k}\sigma}^{\ast}a_{\vec{k} ' \sigma ' }\right) \delta^{\left( 3\right) }\left( \vec{k}-\vec{k} ' \right) \end{array} \right\} \nonumber\\ & = &-\frac{\left( 2\pi\right) ^{3}}{2}\sum_{\sigma,\sigma ' }\int d^{3}kN_{k}^{2}\left[ \begin{array} [c]{c}%
\varepsilon_{0}\omega_{k}^{2}\left\{ \left( \hat{\varepsilon}_{\vec{k}%
\sigma}\cdot\hat{\varepsilon}_{-\vec{k}\sigma ' }\right) \left( a_{\vec{k}\sigma}a_{-\vec{k}\sigma ' }+a_{\vec{k}\sigma}^{\ast}a_{-\vec{k}\sigma ' }^{\ast}\right) -\left( \hat{\varepsilon}_{\vec{k}\sigma}\cdot \hat{\varepsilon}_{\vec{k}\sigma ' }\right) \left( a_{\vec{k}\sigma}a_{\vec{k}\sigma ' }^{\ast}+a_{\vec{k}\sigma}^{\ast}a_{\vec{k}\sigma ' }\right) \right\} \\ +\frac{1}{\mu_{0}}\left\{ \left( \vec{k}\cdot-\vec{k}\right) \left( \hat{\varepsilon}_{\vec{k}\sigma}\cdot\hat{\varepsilon}_{-\vec{k}\sigma ' }\right) -\left( \hat{\varepsilon}_{\vec{k}\sigma}\cdot-\vec{k}\right) \left( \hat{\varepsilon}_{-\vec{k}\sigma ' }\cdot\vec{k}\right) \right\} \left( a_{\vec{k}\sigma}a_{-\vec{k}\sigma ' }+a_{\vec{k}\sigma}^{\ast}a_{-\vec{k}\sigma ' }^{\ast}\right) \\ -\frac{1}{\mu_{0}}\left\{ \left( \vec{k}\cdot\vec{k}\right) \left( \hat{\varepsilon}_{\vec{k}\sigma}\cdot\hat{\varepsilon}_{\vec{k}\sigma ' }\right) -\left( \hat{\varepsilon}_{\vec{k}\sigma}\cdot\vec{k}\right) \left( \hat{\varepsilon}_{\vec{k}\sigma ' }\cdot\vec{k}\right) \right\} \left( a_{\vec{k}\sigma}a_{\vec{k}\sigma ' }^{\ast}+a_{\vec{k}\sigma}^{\ast}a_{\vec{k}\sigma ' }\right) \end{array} \right] \nonumber\\ & = &-\frac{\left( 2\pi\right) ^{3}}{2}\sum_{\sigma,\sigma ' }\int d^{3}kN_{k}^{2}\left[ \begin{array} [c]{c}%
\varepsilon_{0}\omega_{k}^{2}\left\{ -\left( -1\right) ^{\sigma}\left( a_{\vec{k}\sigma}a_{-\vec{k}\sigma ' }+a_{\vec{k}\sigma}^{\ast}a_{-\vec{k}\sigma ' }^{\ast}\right) -\left( a_{\vec{k}\sigma}a_{\vec{k}\sigma ' }^{\ast}+a_{\vec{k}\sigma}^{\ast}a_{\vec{k}\sigma ' }\right) \right\} \\ +\frac{k^{2}}{\mu_{0}}\left\{ \left( -1\right) ^{\sigma}\left( a_{\vec {k}\sigma}a_{-\vec{k}\sigma ' }+a_{\vec{k}\sigma}^{\ast}a_{-\vec{k}\sigma ' }^{\ast}\right) -\left( a_{\vec{k}\sigma}a_{\vec{k}\sigma ' }^{\ast}+a_{\vec{k}\sigma}^{\ast}a_{\vec{k}\sigma ' }\right) \right\} \end{array} \right] \delta_{\sigma\sigma ' }\nonumber\\ & = &\frac{\left( 2\pi\right) ^{3}\left( 2\varepsilon_{0}\right) }{2}%
\sum_{\sigma}\int d^{3}kN_{k}^{2}\omega_{k}^{2}\left\{ a_{\vec{k}\sigma }a_{\vec{k}\sigma}^{\ast}+a_{\vec{k}\sigma}^{\ast}a_{\vec{k}\sigma}\right\} \nonumber\\ & = &\frac{\left( 2\pi\right) ^{3}\left( 2\varepsilon_{0}\right) }{2}%
\sum_{\sigma}\int d^{3}k\left( \frac{\hbar}{\varepsilon_{0}\left( 2\pi\right) ^{3}2\omega_{k}}\right) \omega_{k}^{2}\left\{ a_{\vec{k}\sigma }a_{\vec{k}\sigma}^{\ast}+a_{\vec{k}\sigma}^{\ast}a_{\vec{k}\sigma}\right\} \nonumber\\ & = &\frac{\hbar}{2}\sum_{\sigma}\int d^{3}k\omega_{k}\left\{ a_{\vec{k}\sigma }^{\ast}a_{\vec{k}\sigma}+a_{\vec{k}\sigma}a_{\vec{k}\sigma}^{\ast}\right\} \label{Energia Campo EM}%
\end{eqnarray} donde ocupamos expl\'{\i}citamente que \begin{eqnarray} \left( \vec{k}\times\hat{\varepsilon}_{\vec{k}\sigma}\right) \cdot\left( \vec{k} ' \times\hat{\varepsilon}_{\vec{k} ' \sigma ' }\right) & = &\left( \vec{k}\cdot\vec{k} ' \right) \left( \hat{\varepsilon}_{\vec{k}\sigma}\cdot\hat{\varepsilon}_{\vec{k} ' \sigma ' }\right) -\left( \hat{\varepsilon}_{\vec{k}\sigma}\cdot\vec{k} ' \right) \left( \hat{\varepsilon}_{\vec{k} ' \sigma ' }\cdot\vec{k}\right) \label{rel1}\\ \omega_{-k} & = &\omega_{k}\label{rel2}\\ \hat{\varepsilon}_{\vec{k}\sigma}\cdot\vec{k} & = &\hat{\varepsilon}_{\vec {k}\sigma}\cdot\left( -\vec{k}\right) =0\label{rel3}\\ \vec{k}\cdot\vec{k} & = &k^{2}\label{rel4}\\ \vec{k}\cdot\left( -\vec{k}\right) & = &-k^{2}\label{rel5}\\ \hat{\varepsilon}_{\vec{k}\sigma}\cdot\hat{\varepsilon}_{\vec{k}\sigma ' } & = &\delta_{\sigma\sigma ' }\label{rel6}\\ \hat{\varepsilon}_{\vec{k}\sigma}\cdot\hat{\varepsilon}_{-\vec{k}\sigma ' } & = &-\left( -1\right) ^{\sigma}\delta_{\sigma\sigma ' }\label{rel7}\\ k^{2} & = &c^{2}\omega_{k}^{2}=\varepsilon_{0}\mu_{0}\omega_{k}^{2}\label{rel8}%
\end{eqnarray} y en especial, escogimos \begin{equation} N_{-k}=N_{k}=\sqrt{\frac{\hbar}{\varepsilon_{0}\left( 2\pi\right) ^{3}2\omega_{k}}}\qquad\qquad\left[ \frac{Nsm^{3/2}}{C}\right] \label{rel9}%
\end{equation} de manera que la energ\'{\i}a nos diera de la misma forma que la de un conjunto de osciladores arm\'{o}nicos tridimensionales desacoplados. Notar que esta elecci\'{o}n para $N_{k}$ tambi\'{e}n nos asegura que las dimensiones de las cantidades f\'{\i}sicas en juego sean las correctas.

As'i, cada modo normal del campo electromagn'etico, normalizado con
(\ref{rel9}), se comporta como un oscilador arm'onico cl'asico con
energ'ia $\hbar\omega_{k}$.

Analogamente, para el momentum lineal obtenemos\footnote{Nuevamente ocupamos
las relaciones (\ref{rel1})-(\ref{rel9}).}:%
\begin{eqnarray}
\vec{P} & = &\varepsilon_{0}\int_{V}\left( \vec{E}\times\vec{B}\right) dV\\
& = &-\varepsilon_{0}\sum_{\sigma,\sigma'}\int d^{3}x\int d^{3}k\int d^{3}k'N_{k}N_{k'}\omega_{k}\left\{ \hat{\varepsilon}_{\vec{k}\sigma}\times\left( \vec{k}'\times\hat{\varepsilon}_{\vec{k}'\sigma'}\right) \right\} \left\{ a_{\vec{k}\sigma}e^{i\vec{k}\cdot\vec{x}}%
-a_{\vec{k}\sigma}^{\ast}e^{-i\vec{k}\cdot\vec{x}}\right\} \left\{
a_{\vec{k}'\sigma'}e^{i\vec{k}'\cdot\vec{x}}-a_{\vec{k}'\sigma'}^{\ast}e^{-i\vec{k}'\cdot\vec{x}}\right\} \\
& = &-\varepsilon_{0}\sum_{\sigma,\sigma'}\int d^{3}x\int d^{3}k\int d^{3}k'N_{k}N_{k'}\omega_{k}\left\{ \vec{k}'\left( \hat{\varepsilon}_{\vec{k}\sigma}\cdot\hat{\varepsilon}_{\vec{k}'\sigma'}\right) -\hat{\varepsilon}_{\vec{k}'\sigma'}\left( \hat{\varepsilon}_{\vec{k}\sigma}\cdot\vec{k}'\right) \right\} \left\{
\begin{array}
[c]{c}%
a_{\vec{k}\sigma}a_{\vec{k}'\sigma'}e^{i\left( \vec{k}-\left( -\vec{k}'\right) \right) \cdot\vec{x}}-a_{\vec{k}\sigma}a_{\vec{k}'\sigma'}^{\ast}e^{i\left( \vec{k}-\vec{k}'\right) \cdot\vec{x}}\\
-a_{\vec{k}\sigma}^{\ast}a_{\vec{k}'\sigma'}e^{-i\left( \vec{k}-\vec{k}'\right) \cdot\vec{x}}+a_{\vec{k}\sigma}^{\ast}a_{\vec{k}'\sigma'}^{\ast}e^{-i\left( \vec{k}-\left( -\vec{k}'\right) \right) \cdot\vec{x}}%
\end{array}
\right\} \\
& = &-\left( 2\pi\right) ^{3}\varepsilon_{0}\sum_{\sigma,\sigma'}\int d^{3}k\int d^{3}k'N_{k}N_{k'}\omega_{k}\left\{ \vec{k}'\left( \hat{\varepsilon}_{\vec{k}\sigma}\cdot\hat{\varepsilon}_{\vec{k}'\sigma'}\right) -\hat{\varepsilon}_{\vec{k}'\sigma'}\left( \hat{\varepsilon}_{\vec{k}\sigma}\cdot\vec{k}'\right) \right\} \left\{
\begin{array}
[c]{c}%
\left( a_{\vec{k}\sigma}a_{\vec{k}'\sigma'}+a_{\vec{k}\sigma}^{\ast}a_{\vec{k}'\sigma'}^{\ast}\right) \delta^{\left( 3\right) }\left( \vec{k}-\left( -\vec{k}'\right) \right) \\
-\left( a_{\vec{k}\sigma}^{\ast}a_{\vec{k}'\sigma'}+a_{\vec{k}\sigma}a_{\vec{k}'\sigma'}^{\ast}\right) \delta^{\left( 3\right) }\left( \vec{k}-\vec{k}'\right)
\end{array}
\right\} \\
& = &\left( 2\pi\right) ^{3}\varepsilon_{0}\sum_{\sigma,\sigma'}\int d^{3}kN_{k}^{2}\omega_{k}\vec{k}\left\{
\begin{array}
[c]{c}%
\left( \hat{\varepsilon}_{\vec{k}\sigma}\cdot\hat{\varepsilon}_{-\vec{k}\sigma'}\right) \left( a_{\vec{k}\sigma}a_{-\vec{k}\sigma'}+a_{\vec{k}\sigma}^{\ast}a_{-\vec{k}\sigma'}^{\ast}\right) \\
+\left( \hat{\varepsilon}_{\vec{k}\sigma}\cdot\hat{\varepsilon}_{\vec{k}\sigma'}\right) \left( a_{\vec{k}\sigma}^{\ast}a_{\vec{k}\sigma'}+a_{\vec{k}\sigma}a_{\vec{k}\sigma'}^{\ast}\right)
\end{array}
\right\} \\
& = &\left( 2\pi\right) ^{3}\varepsilon_{0}\sum_{\sigma,\sigma'}\int d^{3}k\left( \frac{\hbar}{\varepsilon_{0}\left( 2\pi\right)^{3}2\omega_{k}}\right) \omega_{k}\vec{k}\left\{ -\left( -1\right)
^{\sigma}\left( a_{\vec{k}\sigma}a_{-\vec{k}\sigma'}+a_{\vec{k}\sigma}^{\ast}a_{-\vec{k}\sigma'}^{\ast}\right) +a_{\vec{k}\sigma}^{\ast}a_{\vec{k}\sigma'}+a_{\vec{k}\sigma}a_{\vec{k}\sigma'}^{\ast}\right\} \delta_{\sigma\sigma'}\\
& = &\frac{\hbar}{2}\sum_{\sigma}\int d^{3}k\vec{k}\left\{ a_{\vec{k}\sigma
}a_{\vec{k}\sigma}^{\ast}+a_{\vec{k}\sigma}^{\ast}a_{\vec{k}\sigma}\right\}
-2\pi\varepsilon_{0}\sum_{\sigma}\left( -1\right) ^{\sigma}\int d^{3}%
kN_{k}^{2}\omega_{k}\vec{k}\left\{ a_{\vec{k}\sigma}a_{-\vec{k}\sigma
}+a_{\vec{k}\sigma}^{\ast}a_{-\vec{k}\sigma}^{\ast}\right\}
\end{eqnarray}
pero analizemos un poco mejor la primera integral del segundo t'ermino:%
\begin{eqnarray}
\int d^{3}kN_{k}^{2}\omega_{k}\vec{k}a_{\vec{k}\sigma}a_{-\vec{k}\sigma} &
=\int d^{3}kN_{k}^{2}\omega_{k}\vec{k}a_{\vec{k}\sigma}a_{-\vec{k}\sigma}\\
& = &\int_{-\infty}^{\infty}d^{3}kN_{k}^{2}\omega_{k}\vec{k}a_{\vec{k}\sigma
}a_{-\vec{k}\sigma}\\
& = &\int_{-\infty}^{0}d^{3}kN_{k}^{2}\omega_{k}\vec{k}a_{\vec{k}\sigma}%
a_{-\vec{k}\sigma}+\int_{0}^{\infty}d^{3}kN_{k}^{2}\omega_{k}\vec{k}a_{\vec
{k}\sigma}a_{-\vec{k}\sigma}\\
& = &\int_{-\infty}^{0}d^{3}kN_{k}^{2}\omega_{k}\vec{k}a_{\vec{k}\sigma}%
a_{-\vec{k}\sigma}+\int_{0}^{-\infty}d^{3}( -k) N_{-k}^{2}%
\omega_{-k}\left( -\vec{k}\right) a_{-\vec{k}\sigma}a_{\vec{k}\sigma}\\
& = &\int_{-\infty}^{0}d^{3}kN_{k}^{2}\omega_{k}\vec{k}a_{\vec{k}\sigma}%
a_{-\vec{k}\sigma}+\int_{0}^{-\infty}d^{3}kN_{k}^{2}\omega_{k}\vec{k}%
a_{\vec{k}\sigma}a_{-\vec{k}\sigma}\\
& = &\int_{-\infty}^{0}d^{3}kN_{k}^{2}\omega_{k}\vec{k}a_{\vec{k}\sigma}%
a_{-\vec{k}\sigma}-\int_{-\infty}^{0}d^{3}kN_{k}^{2}\omega_{k}\vec{k}%
a_{\vec{k}\sigma}a_{-\vec{k}\sigma}\\
& = &0
\end{eqnarray}
y an'alogamente sucede con la segunda integral: $\int d^{3}kN_{k}^{2}%
\omega_{k}\vec{k}a_{\vec{k}\sigma}^{\ast}a_{-\vec{k}\sigma}^{\ast}$ del
segundo t'ermino. Por lo tanto, el momentum lineal del campo
electromagn'etico es:%
\begin{equation}
\vec{P}=\frac{\hbar}{2}\sum_{\sigma}\int d^{3}k\vec{k}\left\{ a_{\vec
{k}\sigma}a_{\vec{k}\sigma}^{\ast}+a_{\vec{k}\sigma}^{\ast}a_{\vec{k}\sigma
}\right\} \label{Momentum Lineal EM}%
\end{equation}
que siguiendo con la analog'ia del sistema de osciladores arm'onicos
tridimensionales desacoplados, tiene la misma forma matem'atica.

Finalmente, gracias a las relaciones de conmutaci'on encontradas, podemos
modificar la forma de los operadores Hamiltoniano y momentum lineal, de modo
que nos sean m'as 'utiles en los futuros c'alculos. Entonces:%
\begin{eqnarray}
\hat{H} & = &\frac{\hbar}{2}\sum_{\sigma}\int d^{3}k\omega_{k}\left\{ \hat
{a}_{\vec{k}\sigma}^{\dagger}\hat{a}_{\vec{k}\sigma}+\hat{a}_{\vec{k}\sigma
}\hat{a}_{\vec{k}\sigma}^{\dagger}\right\} \nonumber\\
& = &\frac{\hbar}{2}\sum_{\sigma}\int d^{3}k\omega_{k}\left\{ \hat{a}_{\vec
{k}\sigma}^{\dagger}\hat{a}_{\vec{k}\sigma}+\left[ \hat{a}_{\vec{k}\sigma
},\hat{a}_{\vec{k}\sigma}^{\dagger}\right] +\hat{a}_{\vec{k}\sigma}^{\dagger
}\hat{a}_{\vec{k}\sigma}\right\} \nonumber\\
& = &\hbar\sum_{\sigma}\int d^{3}k\omega_{k}\left\{ \hat{a}_{\vec{k}\sigma
}^{\dagger}\hat{a}_{\vec{k}\sigma}+\frac{1}{2}\delta^{\left( 3\right)
}(0) \right\} \label{Operador Hamiltoniano EM}%
\end{eqnarray}
y de (\ref{Momentum Lineal EM}):%
\begin{eqnarray}
\overrightarrow{\hat{P}} & = &\frac{\hbar}{2}\sum_{\sigma}\int d^{3}k\vec
{k}\left\{ \hat{a}_{\vec{k}\sigma}^{\dagger}\hat{a}_{\vec{k}\sigma}+\hat
{a}_{\vec{k}\sigma}\hat{a}_{\vec{k}\sigma}^{\dagger}\right\} \nonumber\\
& = &\frac{\hbar}{2}\sum_{\sigma}\int d^{3}k\vec{k}\left\{ \hat{a}_{\vec
{k}\sigma}^{\dagger}\hat{a}_{\vec{k}\sigma}+\left[ \hat{a}_{\vec{k}\sigma
},\hat{a}_{\vec{k}\sigma}^{\dagger}\right] +\hat{a}_{\vec{k}\sigma}^{\dagger
}\hat{a}_{\vec{k}\sigma}\right\} \nonumber\\
& = &\hbar\sum_{\sigma}\int d^{3}k\vec{k}\hat{a}_{\vec{k}\sigma}^{\dagger}%
\hat{a}_{\vec{k}\sigma}+\frac{\hbar}{2}\delta^{\left( 3\right) }\left(
0\right) \sum_{\sigma}\int\limits_{-\infty}^{\infty}d^{3}k\vec{k}\nonumber\\
& = &\hbar\sum_{\sigma}\int d^{3}k\vec{k}\hat{a}_{\vec{k}\sigma}^{\dagger}%
\hat{a}_{\vec{k}\sigma}\label{Operador Momentum Lineal EM}%
\end{eqnarray}
ya que la segunda integral del lado derecho se anula por que el integrando es
una funci'on impar integrada sobre un intervalo sim'etrico.


\subsection{Estados de Glauber: Estados Coherentes del Campo
Electromagn'etico.}

Sabemos que el operador campo el'ectrico est'a determinado por:%
\begin{equation}
\overrightarrow{\hat{E}}=i\sqrt{\frac{\hbar}{2\varepsilon_{0}\left(
2\pi\right) ^{3}}}\sum_{\sigma}\int d^{3}k\left( \omega_{k}\right)
^{\frac{1}{2}}\check{\varepsilon}_{\vec{k}\sigma}\left\{ \hat{a}_{\vec
{k}\sigma}(t) e^{i\vec{k}\cdot\vec{x}}-\hat{a}_{\vec{k}\sigma
}^{\dagger}(t) e^{-i\vec{k}\cdot\vec{x}}\right\}
\end{equation}


Generalmente el campo electromagn'etico contiene muchos fotones, y est'a
caracterizado por el estado cu'antico de $n$-fot'ones $\left|
\dots,n_{\vec{k}\sigma},\dots,n_{\vec{k}'\sigma '},\dots\right> .$ Si s'olamente est'an presentes los fotones
correspondientes al modo $k$ y polarizaci'on $\sigma,$ sabemos que el
estado cu'antico queda como $\left| n_{\vec{k}\sigma}\right>
=\left| 0_{11},0_{12},\dots,n_{\vec{k}\sigma},0_{\vec{k}\sigma'},\dots\right> .$ Para determinar el valor cl'asico del campo $\vec
{E},$ calculamos el valor de expectaci'on del operdaor $\overrightarrow
{\hat{E}}$ en el estado de polarizaci'on $\left| n_{\vec{k}\sigma
}\right> .$ En efecto:%
\begin{eqnarray}
\left\langle n_{\vec{k}\sigma}\right| \overrightarrow{\hat{E}}\left|
n_{\vec{k}\sigma}\right> & = &\left\langle n_{\vec{k}\sigma}\right|
\left[ i\sqrt{\frac{\hbar}{2\varepsilon_{0}\left( 2\pi\right) ^{3}}}%
\sum_{\sigma}\int d^{3}k\left( \omega_{k}\right) ^{\frac{1}{2}}%
\check{\varepsilon}_{\vec{k}\sigma}\left\{ \hat{a}_{\vec{k}\sigma}\left(
t\right) e^{i\vec{k}\cdot\vec{x}}-\hat{a}_{\vec{k}\sigma}^{\dagger}\left(
t\right) e^{-i\vec{k}\cdot\vec{x}}\right\} \right] \left| n_{\vec
{k}\sigma}\right> \\
& = &i\sqrt{\frac{\hbar}{2\varepsilon_{0}\left( 2\pi\right) ^{3}}}\sum
_{\sigma}\int d^{3}k\left( \omega_{k}\right) ^{\frac{1}{2}}\check
{\varepsilon}_{\vec{k}\sigma}\left\{ \left\langle n_{\vec{k}\sigma
}\right| \hat{a}_{\vec{k}\sigma}\left| n_{\vec{k}\sigma}\right>
e^{i\vec{k}\cdot\vec{x}}-\left\langle n_{\vec{k}\sigma}\right| \hat
{a}_{\vec{k}\sigma}^{\dagger}\left| n_{\vec{k}\sigma}\right>
e^{-i\vec{k}\cdot\vec{x}}\right\} \\
& = &i\sqrt{\frac{\hbar}{2\varepsilon_{0}\left( 2\pi\right) ^{3}}}\sum
_{\sigma}\int d^{3}k\left( \omega_{k}\right) ^{\frac{1}{2}}\check
{\varepsilon}_{\vec{k}\sigma}\left\{ \sqrt{n_{\vec{k}\sigma}}\left\langle
n_{\vec{k}\sigma}\right| \left. n_{\vec{k}\sigma}-1\right>
e^{i\vec{k}\cdot\vec{x}}-\sqrt{n_{\vec{k}\sigma}+1}\left\langle n_{\vec
{k}\sigma}\right| \left. n_{\vec{k}\sigma}+1\right> e^{-i\vec
{k}\cdot\vec{x}}\right\} \\
& = &0
\end{eqnarray}
debido a la ortogonalidad de los estados $\left\{ \left| n_{\vec{k}%
\sigma}\right> \right\} .$ Por otro lado, el valor esperado de la
densidad de energ'ia del campo $\vec{E}$ est'a dado por:%
\begin{eqnarray}
\frac{\varepsilon_{0}}{2V}\left\langle n_{\vec{k}\sigma}\right|
\overrightarrow{\hat{E}}\cdot\overrightarrow{\hat{E}}^{\dagger}\left|
n_{\vec{k}\sigma}\right> & = &\frac{\varepsilon_{0}}{2V}\left\langle
n_{\vec{k}\sigma}\right| \left[ i\sqrt{\frac{\hbar}{2\varepsilon
_{0}\left( 2\pi\right) ^{3}}}\sum_{\sigma}\int d^{3}k\hat{\varepsilon}%
_{\vec{k}\sigma}\left( \omega_{k}\right) ^{\frac{1}{2}}\left\{ \hat
{a}_{\vec{k}\sigma}(t) e^{i\vec{k}\cdot\vec{x}}-\hat{a}_{\vec
{k}\sigma}^{\dagger}(t) e^{-i\vec{k}\cdot\vec{x}}\right\}
\right] \left[ -i\sqrt{\frac{\hbar}{2\varepsilon_{0}\left( 2\pi\right)
^{3}}}\sum_{\sigma'}\int d^{3}k'\left( \omega_{k'}\right) ^{\frac{1}{2}}\hat{\varepsilon}_{\vec{k}'\sigma '}\left\{ \hat{a}_{\vec{k}'\sigma '}^{\dagger}(t) e^{-i\vec{k}'\cdot\vec{x}}-\hat{a}_{\vec{k}' \sigma '}(t) e^{i\vec{k}'\cdot\vec{x}}\right\} \right] \left| n_{\vec{k}\sigma}\right> \\
& = &\frac{\hbar}{4V\left( 2\pi\right) ^{3}}\sum_{\sigma}\sum_{\sigma '}\int d^{3}k\int d^{3}k'\left( \omega_{k}\right) ^{\frac{1}{2}}\left( \omega_{k'}\right) ^{\frac{1}{2}}\left( \hat{\varepsilon}_{\vec{k}\sigma}\cdot\hat{\varepsilon}_{\vec{k}'\sigma '}\right) \left\{
\begin{array}
[c]{c}%
\left\langle n_{\vec{k}\sigma}\right| \hat{a}_{\vec{k}\sigma}\hat{a}_{\vec{k}'\sigma '}^{\dagger}\left| n_{\vec{k}\sigma}\right> e^{i\left( \vec{k}-\vec{k}'\right) \cdot\vec{x}}-\left\langle n_{\vec{k}\sigma}\right| \hat{a}_{\vec{k}\sigma}\hat{a}_{\vec{k}'\sigma '}\left| n_{\vec{k}\sigma}\right> e^{i\left( \vec{k}-\left( -\vec{k}'\right) \right) \cdot\vec{x}}\\
-\left\langle n_{\vec{k}\sigma}\right| \hat{a}_{\vec{k}\sigma}^{\dagger}\hat{a}_{\vec{k}'\sigma '}^{\dagger}\left| n_{\vec{k}\sigma}\right> e^{-i\left( \vec{k}-\left( -\vec{k}'\right) \right) \cdot\vec{x}}+\left\langle n_{\vec{k}\sigma}\right|\hat{a}_{\vec{k}\sigma}^{\dagger}\hat{a}_{\vec{k}'\sigma '}\left| n_{\vec{k}\sigma}\right> e^{-i\left( \vec{k}-\vec{k}'\right) \cdot\vec{x}}%
\end{array}
\right\} \\
& = &\frac{\hbar}{4V\left( 2\pi\right) ^{3}}\sum_{\sigma}\sum_{\sigma '}\int d^{3}k\int d^{3}k'\left( \omega_{k}\right) ^{\frac{1}{2}}\left( \omega_{k'}\right) ^{\frac{1}{2}}\left[ \delta_{\sigma\sigma '}\delta^{\left( 3\right) }\left( \vec{k}-\vec{k}'\right) \right] \left\{
\begin{array}
[c]{c}%
\left\langle n_{\vec{k}\sigma}\right| \hat{a}_{\vec{k}\sigma}\hat{a}_{\vec{k}'\sigma'}^{\dagger}\left| n_{\vec{k}\sigma}\right> e^{i\left( \vec{k}-\vec{k}'\right) \cdot\vec{x}}-\left\langle n_{\vec{k}\sigma}\right| \hat{a}%
_{\vec{k}\sigma}\hat{a}_{\vec{k}'\sigma '}\left| n_{\vec{k}\sigma}\right> e^{i\left( \vec{k}-\left(
-\vec{k}'\right) \right) \cdot\vec{x}}\\
-\left\langle n_{\vec{k}\sigma}\right| \hat{a}_{\vec{k}\sigma}^{\dagger
}\hat{a}_{\vec{k}'\sigma '}^{\dagger}\left| n_{\vec{k}\sigma}\right> e^{-i\left( \vec
{k}-\left( -\vec{k}'\right) \right) \cdot\vec{x}}+\left\langle n_{\vec{k}\sigma}\right|
\hat{a}_{\vec{k}\sigma}^{\dagger}\hat{a}_{\vec{k}'\sigma '}\left| n_{\vec{k}\sigma}\right> e^{-i\left( \vec{k}-\vec{k}'\right) \cdot\vec{x}}%
\end{array}
\right\} \\
& = &\frac{\hbar}{4V\left( 2\pi\right) ^{3}}\sum_{\sigma}\int d^{3}k\omega
_{k}\left\{
\begin{array}
[c]{c}%
\left\langle n_{\vec{k}\sigma}\right| \hat{a}_{\vec{k}\sigma}\hat{a}%
_{\vec{k}\sigma}^{\dagger}\left| n_{\vec{k}\sigma}\right>
-\left\langle n_{\vec{k}\sigma}\right| \hat{a}_{\vec{k}\sigma}\hat
{a}_{\vec{k}\sigma}\left| n_{\vec{k}\sigma}\right> e^{i2\vec{k}%
\cdot\vec{x}}\\
-\left\langle n_{\vec{k}\sigma}\right| \hat{a}_{\vec{k}\sigma}^{\dagger
}\hat{a}_{\vec{k}\sigma}^{\dagger}\left| n_{\vec{k}\sigma}\right>
e^{-i2\vec{k}\cdot\vec{x}}+\left\langle n_{\vec{k}\sigma}\right| \hat
{a}_{\vec{k}\sigma}^{\dagger}\hat{a}_{\vec{k}\sigma}\left| n_{\vec
{k}\sigma}\right>
\end{array}
\right\} \\
& = &\frac{\hbar}{4V\left( 2\pi\right) ^{3}}\sum_{\sigma}\int d^{3}k\omega
_{k}\left\{
\begin{array}
[c]{c}%
\sqrt{n_{\vec{k}\sigma}+1}\left\langle n_{\vec{k}\sigma}\right| \hat
{a}_{\vec{k}\sigma}\left| n_{\vec{k}\sigma}+1\right> -\sqrt
{n_{\vec{k}\sigma}}\left\langle n_{\vec{k}\sigma}\right| \hat{a}_{\vec
{k}\sigma}\left| n_{\vec{k}\sigma}-1\right> e^{i2\vec{k}\cdot\vec
{x}}\\
-\sqrt{n_{\vec{k}\sigma}+1}\left\langle n_{\vec{k}\sigma}\right| \hat
{a}_{\vec{k}\sigma}^{\dagger}\left| n_{\vec{k}\sigma}+1\right>
e^{-i2\vec{k}\cdot\vec{x}}+\sqrt{n_{\vec{k}\sigma}}\left\langle n_{\vec
{k}\sigma}\right| \hat{a}_{\vec{k}\sigma}^{\dagger}\left| n_{\vec
{k}\sigma}-1\right>
\end{array}
\right\} \\
& = &\frac{\hbar}{4V\left( 2\pi\right) ^{3}}\sum_{\sigma}\int d^{3}k\omega
_{k}\left\{
\begin{array}
[c]{c}%
\left( n_{\vec{k}\sigma}+1\right) \left\langle n_{\vec{k}\sigma}\right|
\left. n_{\vec{k}\sigma}\right> -\sqrt{n_{\vec{k}\sigma}\left(
n_{\vec{k}\sigma}-1\right) }\left\langle n_{\vec{k}\sigma}\right| \left.
n_{\vec{k}\sigma}-2\right> e^{i2\vec{k}\cdot\vec{x}}\\
-\sqrt{\left( n_{\vec{k}\sigma}+1\right) \left( n_{\vec{k}\sigma}+2\right)
}\left\langle n_{\vec{k}\sigma}\right| \left. n_{\vec{k}\sigma
}+2\right> e^{-i2\vec{k}\cdot\vec{x}}+n_{\vec{k}\sigma}\left\langle
n_{\vec{k}\sigma}\right| \left. n_{\vec{k}\sigma}\right>
\end{array}
\right\} \\
& = &\frac{\hbar}{2V\left( 2\pi\right) ^{3}}\sum_{\sigma}\int d^{3}k\omega
_{k}\left\{ n_{\vec{k}\sigma}+\frac{1}{2}\right\}
\end{eqnarray}


Estos son resultados sorpresivos ya que el valor esperado del campo
el'ectrico en un estado de $n$ fotones es nulo, mientras que la densidad de
energ'ia es, como era de esperar, igual a la mitad de $n$ fotones cada uno
con energ'ia $\hbar\omega_{k}$ por unidad de volumen. La otra mitad
proviene de la contribuci'on del campo magn'etico: $\frac{1}{2\mu_{0}%
}B^{2}.$ La 'unica manera de entender esto es que los $n$ fotones del campo
poseen fases y que ellas est'an distribuidas estad'isticamente, de
manera que el promedio sobre las fases del valor esperado de $\vec{E}$ se
anule. En la construcci'on de la densidad de energ'ia, las fases quedan
fuera: la densidad de energ'ia es, en cada punto del espacio-tiempo, un
n'umero real y positivo.

Fu'e Glauber el que resolvi'o el problema introduciendo nuevos estados
del campo electromagn'etico, para los cuales el valor de expectaci'on
del operador $\overrightarrow{\hat{E}}$ coincide con el campo el'ectrico
cl'asico $\vec{E}.$ El precio que se paga es que se introduce una incerteza
absoluta en el n'umero de fotones que tiene el campo, pero esto nos permite
definir la fase del campo de manera mas precisa.

Los estados de Glauber, o llamados tambi'en estados coherentes, se escogen
como los estados base del operador $\hat{a}_{\vec{k}\sigma}:$%
\begin{equation}
\hat{a}_{\vec{k}\sigma}\left| c_{\vec{k}\sigma}\right> =c_{\vec
{k}\sigma}\left| c_{\vec{k}\sigma}\right>
\label{eigenecuacion de aksigma}%
\end{equation}
donde $\left| c_{\vec{k}\sigma}\right> =\left| 0_{11}%
,0_{12},\dots,c_{\vec{k}\sigma},0_{\vec{k}\sigma'},\dots\right> ,$ o sea, el estado de $\hat{a}_{\vec{k}\sigma}$ donde
s'olo el eigenvalor de $\hat{a}_{\vec{k}\sigma}$ no es nulo.

Estos estados, los podemos escribir en la base de estados del operador
n'umero para un modo y una polorizaci'on en particular\footnote{S'olo
cambia el n'umero de fotones presentes en ese modo y polarizaci'on
escogiga.}: $\left| n_{\vec{k}\sigma}\right> ,$ de la siguiente
manera:%
\begin{eqnarray}
\left| c_{\vec{k}\sigma}\right> & = &\left( \sum_{n_{\vec{k}\sigma
}=0}^{\infty}\left| n_{\vec{k}\sigma}\right> \left\langle n_{\vec
{k}\sigma}\right| \right) \left| c_{\vec{k}\sigma}\right>
\nonumber\\
& = &\sum_{n_{\vec{k}\sigma}=0}^{\infty}c_{n_{_{\vec{k}\sigma}}}\left|
n_{\vec{k}\sigma}\right> \label{glauber en |n>}%
\end{eqnarray}
donde $c_{n_{_{\vec{k}\sigma}}}$ son los coeficientes de la expansi'on, y
que todav'ia hay que determinarlos. Para eso, exigimos que los
coeficientes satisfagan (\ref{eigenecuacion de aksigma}), es decir:%
\begin{eqnarray}
\hat{a}_{\vec{k}\sigma}\left| c_{\vec{k}\sigma}\right> &
=\sum_{n_{\vec{k}\sigma}=0}^{\infty}c_{n_{_{\vec{k}\sigma}}}\hat{a}_{\vec
{k}\sigma}\left| n_{\vec{k}\sigma}\right> \label{glauber 1}\\
& = &c_{0}\hat{a}_{\vec{k}\sigma}\left| 0\right> +\sum_{n_{\vec
{k}\sigma}=0}^{\infty}c_{n_{_{\vec{k}\sigma}}}\hat{a}_{\vec{k}\sigma
}\left| n_{\vec{k}\sigma}\right> \nonumber\\
& = &\sum_{n_{\vec{k}\sigma}=0}^{\infty}c_{n_{_{\vec{k}\sigma}}}\hat{a}_{\vec
{k}\sigma}\left| n_{\vec{k}\sigma}\right> \nonumber\\
& = &\sum_{n_{\vec{k}\sigma}=0}^{\infty}c_{n_{_{\vec{k}\sigma}}}\sqrt{n_{\vec
{k}\sigma}}\left| n_{\vec{k}\sigma}-1\right> \nonumber
\end{eqnarray}
ya que $\hat{a}_{\vec{k}\sigma}\left| 0\right> =0$. Tener en cuenta
que en el t'ermino que qued'o, s'olo estan presentes los estados
$\left| n_{\vec{k}\sigma}\right> $ distintos al del vac'io, es
decir, para $n_{\vec{k}\sigma}=1,2,\dots$

Ahora, cambiando $n_{\vec{k}\sigma}-1\rightarrow n_{\vec{k}\sigma}$ de manera
que est'e presente nuevamente el estado del vac'io:%
\begin{equation}
\hat{a}_{\vec{k}\sigma}\left| c_{\vec{k}\sigma}\right>
=\sum_{n_{\vec{k}\sigma}=0}^{\infty}c_{n_{_{\vec{k}\sigma}}+1}\sqrt{n_{\vec
{k}\sigma}+1}\left| n_{\vec{k}\sigma}\right>
\end{equation}
e igualando con (\ref{eigenecuacion de aksigma}), obtenemos:%
\begin{equation}
c_{\vec{k}\sigma}\left| c_{\vec{k}\sigma}\right> =\sum_{n_{\vec
{k}\sigma}=0}^{\infty}c_{n_{_{\vec{k}\sigma}}+1}\sqrt{n_{\vec{k}\sigma}%
+1}\left| n_{\vec{k}\sigma}\right>
\end{equation}
pero $\left| c_{\vec{k}\sigma}\right> $ est'a dado por
(\ref{glauber en |n>}), entonces:%
\begin{equation}
c_{\vec{k}\sigma}\sum_{n_{\vec{k}\sigma}=0}^{\infty}c_{n_{_{\vec{k}\sigma}}%
}\left| n_{\vec{k}\sigma}\right> =\sum_{n_{\vec{k}\sigma}=0}%
^{\infty}c_{n_{_{\vec{k}\sigma}}+1}\sqrt{n_{\vec{k}\sigma}+1}\left|
n_{\vec{k}\sigma}\right>
\end{equation}
de donde se obtiene la ecuaci'on de recurrencia para los coeficientes:%
\begin{equation}
c_{n_{_{\vec{k}\sigma}}+1}=\frac{c_{\vec{k}\sigma}c_{n_{_{\vec{k}\sigma}}}%
}{\sqrt{n_{\vec{k}\sigma}+1}}\qquad\qquad,n_{_{\vec{k}\sigma}}=0,1,2,\dots
\end{equation}


Para:%
\begin{eqnarray}
n_{_{\vec{k}\sigma}} & = &0\qquad\rightarrow\qquad c_{1}=\frac{c_{\vec{k}%
\sigma}c_{0}}{\sqrt{1}}\\
n_{_{\vec{k}\sigma}} & = &1\qquad\rightarrow\qquad c_{2}=\frac{c_{\vec{k}%
\sigma}c_{1}}{\sqrt{2}}=\frac{c_{\vec{k}\sigma}^{2}c_{0}}{\sqrt{1\cdot2}}\\
n_{_{\vec{k}\sigma}} & = &2\qquad\rightarrow\qquad c_{3}=\frac{c_{\vec{k}%
\sigma}c_{2}}{\sqrt{3}}=\frac{c_{\vec{k}\sigma}^{3}c_{0}}{\sqrt{1\cdot2\cdot
3}}\\
& \vdots\\
n_{_{\vec{k}\sigma}} & = &n_{\vec{k}\sigma}\qquad\rightarrow\qquad
c_{n_{\vec{k}\sigma}}=\frac{c_{\vec{k}\sigma}^{n_{\vec{k}\sigma}}c_{0}}%
{\sqrt{n_{\vec{k}\sigma}!}}%
\end{eqnarray}
por lo tanto:%
\begin{eqnarray}
\left| c_{\vec{k}\sigma}\right> & = &\sum_{n_{\vec{k}\sigma}%
=0}^{\infty}c_{n_{_{\vec{k}\sigma}}}\left| n_{\vec{k}\sigma}\right>
\\
& = &c_{0}\sum_{n_{\vec{k}\sigma}=0}^{\infty}\frac{c_{\vec{k}\sigma}^{n_{\vec
{k}\sigma}}}{\sqrt{n_{\vec{k}\sigma}!}}\left| n_{\vec{k}\sigma
}\right> \qquad\qquad,%
\genfrac{.}{.}{0pt}{}{c\in\protect\rule{0.1in}{0.1in} }{c_{0}\in
\protect\rule{0.1in}{0.1in} }%
\end{eqnarray}


Para determinar el valor de $c_{0}$ ocupamos la relaci'on de
normalizaci'on:%
\begin{eqnarray}
\left\langle c_{\vec{k}\sigma}\right| \left. c_{\vec{k}\sigma
}\right> & = &\left| c_{0}\right| ^{2}\sum_{n_{\vec{k}\sigma
},m_{\vec{k}\sigma}=0}^{\infty}\frac{\left( c_{\vec{k}\sigma}\right)
^{n_{\vec{k}\sigma}}\left( c_{\vec{k}\sigma}^{\ast}\right) ^{m_{\vec
{k}\sigma}}}{\sqrt{n_{\vec{k}\sigma}!m_{\vec{k}\sigma}!}}\left\langle
m_{\vec{k}\sigma}\right| \left. n_{\vec{k}\sigma}\right> \\
& = &\left| c_{0}\right| ^{2}\sum_{n,m=0}^{\infty}\frac{\left(
c_{\vec{k}\sigma}\right) ^{n_{\vec{k}\sigma}}\left( c_{\vec{k}\sigma}^{\ast
}\right) ^{m_{\vec{k}\sigma}}}{\sqrt{n_{\vec{k}\sigma}!m_{\vec{k}\sigma}!}%
}\delta_{m_{\vec{k}\sigma},n_{\vec{k}\sigma}}\\
& = &\left| c_{0}\right| ^{2}\sum_{n_{\vec{k}\sigma}=0}^{\infty}%
\frac{\left( \left| c_{\vec{k}\sigma}\right| ^{2}\right)
^{n_{\vec{k}\sigma}}}{n_{\vec{k}\sigma}!}\\
& = &\left| c_{0}\right| ^{2}e^{\left| c_{\vec{k}\sigma}\right|^{2}}\equiv 1
\end{eqnarray}


Elegimos%
\begin{equation}
c_{0}=e^{-\frac{\left| c_{\vec{k}\sigma}\right| ^{2}}{2}}%
\end{equation}
por lo tanto:%
\begin{equation}
\left| c_{\vec{k}\sigma}\right> =e^{-\frac{\left| c_{\vec
{k}\sigma}\right| ^{2}}{2}}\sum_{n_{\vec{k}\sigma}=0}^{\infty}%
\frac{\left( c_{\vec{k}\sigma}\right) ^{n_{\vec{k}\sigma}}}{\sqrt{n_{\vec
{k}\sigma}!}}\left| n_{\vec{k}\sigma}\right>
\label{Estado de Glauber}%
\end{equation}


De la expansi'on en estados de n'umero de fotones para un modo y
polarizaci'on determinada, vemos claramente que los estados $\left|
c_{\vec{k}\sigma}\right> $ no tienen un n'umero definido de fotones.

El valor esperado de $\hat{a}_{\vec{k}\sigma}$ en los estados de Glauber es:%
\begin{eqnarray}
\left\langle c_{\vec{k}\sigma}\right| \hat{a}_{\vec{k}\sigma}\left|
c_{\vec{k}\sigma}\right> & = &e^{-\left| c_{\vec{k}\sigma}\right|
^{2}}\sum_{n_{\vec{k}\sigma},m_{\vec{k}\sigma}=0}^{\infty}\frac{\left(
c_{\vec{k}\sigma}^{\ast}\right) ^{m_{\vec{k}\sigma}}\left( c_{\vec{k}\sigma
}\right) ^{n_{\vec{k}\sigma}}}{\sqrt{m_{\vec{k}\sigma}!n_{\vec{k}\sigma}!}%
}\left\langle m_{\vec{k}\sigma}\right| \hat{a}_{\vec{k}\sigma}\left|
n_{\vec{k}\sigma}\right> \\
& = &e^{-\left| c_{\vec{k}\sigma}\right| ^{2}}\sum_{n_{\vec{k}\sigma
},m_{\vec{k}\sigma}=0}^{\infty}\frac{\left( c_{\vec{k}\sigma}^{\ast}\right)
^{m_{\vec{k}\sigma}}\left( c_{\vec{k}\sigma}\right) ^{n_{\vec{k}\sigma}}%
}{\sqrt{m_{\vec{k}\sigma}!n_{\vec{k}\sigma}!}}\sqrt{n_{\vec{k}\sigma}%
}\left\langle m_{\vec{k}\sigma}\right| \left. n_{\vec{k}\sigma
}-1\right> \\
& = &e^{-\left| c_{\vec{k}\sigma}\right| ^{2}}\sum_{n_{\vec{k}\sigma
},m_{\vec{k}\sigma}=0}^{\infty}\frac{\left( c_{\vec{k}\sigma}^{\ast}\right)
^{m_{\vec{k}\sigma}}\left( c_{\vec{k}\sigma}\right) ^{n_{\vec{k}\sigma}}%
}{\sqrt{m_{\vec{k}\sigma}!n_{\vec{k}\sigma}!}}\sqrt{n_{\vec{k}\sigma}}%
\delta_{m_{\vec{k}\sigma},n_{\vec{k}\sigma}-1}\\
& = &e^{-\left| c_{\vec{k}\sigma}\right| ^{2}}\sum_{m_{\vec{k}\sigma}%
=0}^{\infty}\frac{\left( c_{\vec{k}\sigma}^{\ast}\right) ^{m_{\vec{k}\sigma
}}\left( c_{\vec{k}\sigma}\right) ^{m_{\vec{k}\sigma}+1}}{\sqrt{m_{\vec
{k}\sigma}!\left( m_{\vec{k}\sigma}+1\right) !}}\sqrt{m_{\vec{k}\sigma}+1}\\
& = &c_{\vec{k}\sigma}e^{-\left| c_{\vec{k}\sigma}\right| ^{2}}%
\sum_{m_{\vec{k}\sigma}=0}^{\infty}\frac{\left( c_{\vec{k}\sigma}^{\ast
}\right) ^{m_{\vec{k}\sigma}}\left( c_{\vec{k}\sigma}\right) ^{m_{\vec
{k}\sigma}}}{\sqrt{m_{\vec{k}\sigma}!m_{\vec{k}\sigma}!\left( m_{\vec
{k}\sigma}+1\right) }}\sqrt{m_{\vec{k}\sigma}+1}\\
& = &c_{\vec{k}\sigma}e^{-\left| c_{\vec{k}\sigma}\right| ^{2}}%
\sum_{m_{\vec{k}\sigma}=0}^{\infty}\frac{\left( \left| c_{\vec{k}\sigma
}\right| ^{2}\right) ^{m_{\vec{k}\sigma}}}{m_{\vec{k}\sigma}!}\\
& = &c_{\vec{k}\sigma}e^{-\left| c_{\vec{k}\sigma}\right| ^{2}%
}e^{\left| c_{\vec{k}\sigma}\right| ^{2}}\\
& = &c_{\vec{k}\sigma}%
\end{eqnarray}
mientras que para $\hat{a}_{\vec{k}\sigma}^{\dagger}$ se tiene:%
\begin{eqnarray}
\left\langle c_{\vec{k}\sigma}\right| \hat{a}_{\vec{k}\sigma}^{\dagger
}\left| c_{\vec{k}\sigma}\right> & = &e^{-\left| c_{\vec{k}%
\sigma}\right| ^{2}}\sum_{n_{\vec{k}\sigma},m_{\vec{k}\sigma}=0}^{\infty
}\frac{\left( c_{\vec{k}\sigma}^{\ast}\right) ^{m_{\vec{k}\sigma}}\left(
c_{\vec{k}\sigma}\right) ^{n_{\vec{k}\sigma}}}{\sqrt{m_{\vec{k}\sigma
}!n_{\vec{k}\sigma}!}}\left\langle m_{\vec{k}\sigma}\right| \hat{a}%
_{\vec{k}\sigma}^{\dagger}\left| n_{\vec{k}\sigma}\right> \\
& = &e^{-\left| c_{\vec{k}\sigma}\right| ^{2}}\sum_{n_{\vec{k}\sigma
},m_{\vec{k}\sigma}=0}^{\infty}\frac{\left( c_{\vec{k}\sigma}^{\ast}\right)
^{m_{\vec{k}\sigma}}\left( c_{\vec{k}\sigma}\right) ^{n_{\vec{k}\sigma}}%
}{\sqrt{m_{\vec{k}\sigma}!n_{\vec{k}\sigma}!}}\sqrt{n_{\vec{k}\sigma}%
+1}\left\langle m_{\vec{k}\sigma}\right| \left. n_{\vec{k}\sigma
}+1\right> \\
& = &e^{-\left| c_{\vec{k}\sigma}\right| ^{2}}\sum_{n_{\vec{k}\sigma
},m_{\vec{k}\sigma}=0}^{\infty}\frac{\left( c_{\vec{k}\sigma}^{\ast}\right)
^{m_{\vec{k}\sigma}}\left( c_{\vec{k}\sigma}\right) ^{n_{\vec{k}\sigma}}%
}{\sqrt{m_{\vec{k}\sigma}!n_{\vec{k}\sigma}!}}\sqrt{n_{\vec{k}\sigma}+1}%
\delta_{m_{\vec{k}\sigma},n_{\vec{k}\sigma}+1}\\
& = &e^{-\left| c_{\vec{k}\sigma}\right| ^{2}}\sum_{n_{\vec{k}\sigma}%
=0}^{\infty}\frac{\left( c_{\vec{k}\sigma}^{\ast}\right) ^{n_{\vec{k}\sigma
}+1}\left( c_{\vec{k}\sigma}\right) ^{n_{\vec{k}\sigma}}}{\sqrt{\left(
n_{\vec{k}\sigma}+1\right) !n_{\vec{k}\sigma}!}}\sqrt{n_{\vec{k}\sigma}+1}\\
& = &c_{\vec{k}\sigma}^{\ast}e^{-\left| c_{\vec{k}\sigma}\right| ^{2}%
}\sum_{n_{\vec{k}\sigma}=0}^{\infty}\frac{\left( c_{\vec{k}\sigma}^{\ast
}\right) ^{n_{\vec{k}\sigma}}\left( c_{\vec{k}\sigma}\right) ^{n_{\vec
{k}\sigma}}}{\sqrt{n_{\vec{k}\sigma}!\left( n_{\vec{k}\sigma}+1\right)
n_{\vec{k}\sigma}!}}\sqrt{n_{\vec{k}\sigma}+1}\\
& = &c_{\vec{k}\sigma}^{\ast}e^{-\left| c_{\vec{k}\sigma}\right| ^{2}%
}\sum_{n_{\vec{k}\sigma}=0}^{\infty}\frac{\left( \left| c_{\vec{k}\sigma
}\right| ^{2}\right) ^{n_{\vec{k}\sigma}}}{n_{\vec{k}\sigma}!}\\
& = &c_{\vec{k}\sigma}^{\ast}e^{-\left| c_{\vec{k}\sigma}\right| ^{2}%
}e^{\left| c_{\vec{k}\sigma}\right| ^{2}}\\
& = &c_{\vec{k}\sigma}^{\ast}%
\end{eqnarray}


Ahora, si calculamos el valor esperado de un modo del campo $\vec{E}$ con
polarizaci'on fijada en el estado de Glauber respectivo, obtenemos el
resultado esperado:%
\begin{eqnarray}
\left\langle c_{\vec{k}\sigma}\right| \overrightarrow{\hat{E}}\left|
c_{\vec{k}\sigma}\right> & = &\left\langle c_{\vec{k}\sigma}\right|
\left[ i\sqrt{\frac{\hbar}{2\varepsilon_{0}\left( 2\pi\right) ^{3}}}%
\sum_{\sigma}\int d^{3}k\left( \omega_{k}\right) ^{\frac{1}{2}}%
\check{\varepsilon}_{\vec{k}\sigma}\left\{ \frac{1}{V}a_{\vec{k}^{\prime
}\sigma^{\prime}}\delta_{\sigma\sigma^{\prime}}\delta^{\left( 3\right)
}\left( \vec{k}-\vec{k}^{\prime}\right) e^{i\vec{k}\cdot\vec{x}}-\frac{1}%
{V}a_{\vec{k}^{\prime}\sigma^{\prime}}\delta_{\sigma\sigma^{\prime}}%
\delta^{\left( 3\right) }\left( \vec{k}-\vec{k}^{\prime}\right)
e^{-i\vec{k}\cdot\vec{x}}\right\} \right] \left| c_{\vec{k}\sigma
}\right> \\
& = &\left\langle c_{\vec{k}\sigma}\right| \left[ \frac{i}{V}\sqrt
{\frac{\hbar\omega_{k}}{2\varepsilon_{0}\left( 2\pi\right) ^{3}}}%
\check{\varepsilon}_{\vec{k}\sigma}\left\{ a_{\vec{k}\sigma}e^{i\vec{k}%
\cdot\vec{x}}-a_{\vec{k}\sigma}e^{-i\vec{k}\cdot\vec{x}}\right\} \right]
\left| c_{\vec{k}\sigma}\right> \\
& = &\frac{i}{V}\sqrt{\frac{\hbar\omega_{k}}{2\varepsilon_{0}\left(
2\pi\right) ^{3}}}\check{\varepsilon}_{\vec{k}\sigma}\left\{ \left\langle
c_{\vec{k}\sigma}\right| \hat{a}_{\vec{k}\sigma}\left| c_{\vec
{k}\sigma}\right> e^{i\vec{k}\cdot\vec{x}}-\left\langle c_{\vec{k}%
\sigma}\right| \hat{a}_{\vec{k}\sigma}^{\dagger}\left| c_{\vec
{k}\sigma}\right> e^{-i\vec{k}\cdot\vec{x}}\right\} \\
& = &-\frac{2}{V}\sqrt{\frac{\hbar\omega_{k}}{2\varepsilon_{0}\left(
2\pi\right) ^{3}}}\check{\varepsilon}_{\vec{k}\sigma}\left\{ \frac
{c_{\vec{k}\sigma}e^{i\vec{k}\cdot\vec{x}}-c_{\vec{k}\sigma}^{\ast}%
e^{-i\vec{k}\cdot\vec{x}}}{2i}\right\} \qquad\qquad,c_{\vec{k}\sigma
}=\left| c_{\vec{k}\sigma}\right| e^{i\theta_{\vec{k}\sigma}}\\
& = &-\frac{2}{V}\left| c_{\vec{k}\sigma}\right| \sqrt{\frac{\hbar
\omega_{k}}{2\varepsilon_{0}\left( 2\pi\right) ^{3}}}\check{\varepsilon
}_{\vec{k}\sigma}\left\{ \frac{e^{i\left( \vec{k}\cdot\vec{x}+\theta
_{\vec{k}\sigma}\right) }-e^{-i\left( \vec{k}\cdot\vec{x}+\theta_{\vec
{k}\sigma}\right) }}{2i}\right\} \\
& = &-\frac{2}{V}\left| c_{\vec{k}\sigma}\right| \sqrt{\frac{\hbar
\omega_{k}}{2\varepsilon_{0}\left( 2\pi\right) ^{3}}}\check{\varepsilon
}_{\vec{k}\sigma}\sin\left\{ \vec{k}\cdot\vec{x}+\theta_{\vec{k}\sigma
}\right\}
\end{eqnarray}


Esta es la forma de una onda electromagn'etica cl'asica, cuya amplitud
es dada por $|c_{\vec{k}\sigma}|$ y cuya fase es determinada por la fase de
$c_{\vec{k}\sigma}=|c_{\vec{k}\sigma}|e^{i\theta_{\vec{k}\sigma}}.$

Definimos\footnote{Es necesario bajar a la normalizaci'on en una caja, para
evitar la poca claridad que puede provocar al considerar la delta de Dirac en
el conmutador:%
\begin{equation}
\left[ a_{\vec{k}\sigma},a_{\vec{k}'\sigma '}^{\dagger}\right] =\delta_{\sigma\sigma '}\delta^{\left( 3\right) }\left( \vec{k}-\vec{k}'\right)
\end{equation}
} la indeterminaci'on del n'umero de fotones en el estado coherente
$\left| c_{\vec{k}\sigma}\right> ,$ para un modo y una
polarizaci'on espec'ifica, por:%
\begin{eqnarray}
\Delta n_{\vec{k}\sigma} & = &\sqrt{\left\langle c_{\vec{k}\sigma}\right|
\left( \hat{N}_{_{\vec{k}\sigma}}-\left\langle \hat{N}_{_{\vec{k}\sigma}%
}\right> \right) ^{2}\left| c_{\vec{k}\sigma}\right> }\\
& = &\sqrt{\left\langle c_{\vec{k}\sigma}\right| \left( \hat{N}_{_{\vec
{k}\sigma}}^{2}-2\hat{N}_{_{\vec{k}\sigma}}\left\langle \hat{N}_{_{\vec
{k}\sigma}}\right> -\left\langle \hat{N}_{_{\vec{k}\sigma}}\right>
^{2}\right) \left| c_{\vec{k}\sigma}\right> }\\
& = &\sqrt{\left\langle c_{\vec{k}\sigma}\right| \hat{N}_{_{\vec{k}\sigma}%
}^{2}\left| c_{\vec{k}\sigma}\right> -2\left\langle c_{\vec{k}%
\sigma}\right| \hat{N}_{_{\vec{k}\sigma}}\left| c_{\vec{k}\sigma
}\right> \left\langle \hat{N}_{_{\vec{k}\sigma}}\right>
-\left\langle \hat{N}_{_{\vec{k}\sigma}}\right> ^{2}\left\langle
c_{\vec{k}\sigma}\right| \left. c_{\vec{k}\sigma}\right> }\\
& = &\sqrt{\left\langle \hat{N}_{_{\vec{k}\sigma}}^{2}\right>
-2\left\langle \hat{N}_{_{\vec{k}\sigma}}\right> \left\langle \hat
{N}_{_{\vec{k}\sigma}}\right> -\left\langle \hat{N}_{_{\vec{k}\sigma}%
}\right> ^{2}}\\
& = &\sqrt{\left\langle \hat{N}_{_{\vec{k}\sigma}}^{2}\right>
-2\left\langle \hat{N}_{_{\vec{k}\sigma}}\right> ^{2}-\left\langle
\hat{N}_{_{\vec{k}\sigma}}\right> ^{2}}\\
& = &\sqrt{\left\langle \hat{N}_{_{\vec{k}\sigma}}^{2}\right>
-\left\langle \hat{N}_{_{\vec{k}\sigma}}\right> ^{2}}%
\end{eqnarray}
donde:%
\begin{eqnarray}
\left\langle \hat{N}_{_{\vec{k}\sigma}}\right> & = &\left\langle
c_{\vec{k}\sigma}\right| \hat{N}_{_{\vec{k}\sigma}}\left| c_{\vec
{k}\sigma}\right> \\
& = &\left\langle c_{\vec{k}\sigma}\right| \hat{a}_{_{\vec{k}\sigma}%
}^{\dagger}\hat{a}_{\vec{k}\sigma}\left| c_{\vec{k}\sigma}\right> \\
& = &\left( \left\langle c_{\vec{k}\sigma}\right| \hat{a}_{_{\vec{k}\sigma
}}^{\dagger}\right) \left( \hat{a}_{_{\vec{k}\sigma}}\left| c_{\vec
{k}\sigma}\right> \right) \\
& = &\left( \hat{a}_{_{\vec{k}\sigma}}\left| c_{\vec{k}\sigma}\right>
\right) ^{\dagger}\left( \hat{a}_{_{\vec{k}\sigma}}\left| c_{\vec
{k}\sigma}\right> \right) \\
& = &\left( c_{_{\vec{k}\sigma}}\left| c_{\vec{k}\sigma}\right>
\right) ^{\dagger}\left( c_{_{\vec{k}\sigma}}\left| c_{\vec{k}\sigma
}\right> \right) \\
& = &\left( \left\langle c_{\vec{k}\sigma}\right| c_{_{\vec{k}\sigma}%
}^{\ast}\right) \left( c_{_{\vec{k}\sigma}}\left| c_{\vec{k}\sigma
}\right> \right) \\
& = &c_{_{\vec{k}\sigma}}^{\ast}c_{_{\vec{k}\sigma}}\left\langle c_{\vec
{k}\sigma}\right| \left. c_{\vec{k}\sigma}\right> \\
& = &|c_{_{\vec{k}\sigma}}|^{2}%
\end{eqnarray}


Por otro lado:%
\begin{eqnarray}
\left\langle \hat{N}_{_{\vec{k}\sigma}}^{2}\right> & = &\left\langle
c_{\vec{k}\sigma}\right| \hat{N}_{_{\vec{k}\sigma}}^{2}\left|
c_{\vec{k}\sigma}\right> \\
& = &\left\langle c_{\vec{k}\sigma}\right| \hat{a}_{_{\vec{k}\sigma}%
}^{\dagger}\hat{a}_{\vec{k}\sigma}^{\dagger}\hat{a}_{\vec{k}\sigma}\hat
{a}_{\vec{k}\sigma}\left| c_{\vec{k}\sigma}\right> \\
& = &\left\langle c_{\vec{k}\sigma}\right| \hat{a}_{_{\vec{k}\sigma}%
}^{\dagger}\left( \hat{a}_{_{\vec{k}\sigma}}^{\dagger}\hat{a}_{\vec{k}\sigma
}+[\hat{a}_{_{\vec{k}\sigma}},\hat{a}_{_{\vec{k}\sigma}}^{\dagger}]\right)
\hat{a}_{_{\vec{k}\sigma}}\left| c_{\vec{k}\sigma}\right> \\
& = &\left\langle c_{\vec{k}\sigma}\right| \hat{a}_{_{\vec{k}\sigma}%
}^{\dagger}\left( \hat{a}_{_{\vec{k}\sigma}}^{\dagger}\hat{a}_{\vec{k}\sigma
}+1\right) \hat{a}_{_{\vec{k}\sigma}}\left| c_{\vec{k}\sigma
}\right> \\
& = &\left( \left\langle c_{\vec{k}\sigma}\right| \hat{a}_{_{\vec{k}\sigma
}}^{\dagger}\hat{a}_{_{\vec{k}\sigma}}^{\dagger}\right) \left( \hat
{a}_{_{\vec{k}\sigma}}\hat{a}_{_{\vec{k}\sigma}}\left| c_{\vec{k}\sigma
}\right> \right) +\left( \left\langle c_{\vec{k}\sigma}\right|
\hat{a}_{_{\vec{k}\sigma}}^{\dagger}\right) \left( \hat{a}_{_{\vec{k}\sigma
}}\left| c_{\vec{k}\sigma}\right> \right) \\
& = &c_{_{\vec{k}\sigma}}^{\ast}c_{_{\vec{k}\sigma}}^{\ast}c_{_{\vec{k}\sigma}%
}c_{_{\vec{k}\sigma}}\left\langle c_{\vec{k}\sigma}\right| \left.
c_{\vec{k}\sigma}\right> +c_{_{\vec{k}\sigma}}^{\ast}c_{_{\vec{k}\sigma
}}\left\langle c_{\vec{k}\sigma}\right| \left. c_{\vec{k}\sigma
}\right> \\
& = &|c_{_{\vec{k}\sigma}}|^{4}+|c_{_{\vec{k}\sigma}}|^{2}%
\end{eqnarray}
por lo tanto:%
\begin{eqnarray}
\Delta n_{\vec{k}\sigma} & = &\sqrt{\left\langle \hat{N}_{_{\vec{k}\sigma}}%
^{2}\right> -\left\langle \hat{N}_{_{\vec{k}\sigma}}\right> ^{2}%
}\\
& = &\sqrt{|c_{_{\vec{k}\sigma}}|^{4}+|c_{_{\vec{k}\sigma}}|^{2}-|c_{_{\vec
{k}\sigma}}|^{4}}\\
& = &|c_{_{\vec{k}\sigma}}|\\
& = &\sqrt{\left\langle \hat{N}_{_{\vec{k}\sigma}}\right> }%
\end{eqnarray}
es decir, la incertidumbre aumenta al aumentar el valor de $c_{_{\vec{k}%
\sigma}},$ o tambi'en, al aumentar el n'umero de fotones presentes en el campo.

Es conveniente calcular la incertidumbre relativa;%
\begin{eqnarray}
\frac{\Delta n_{\vec{k}\sigma}}{\left\langle \hat{N}_{_{\vec{k}\sigma}%
}\right> } & = &\frac{|c_{_{\vec{k}\sigma}}|}{|c_{_{\vec{k}\sigma}}|^{2}%
}\\
& = &\frac{1}{|c_{_{\vec{k}\sigma}}|}%
\end{eqnarray}
la que se hace peque\~{n}a a medida que el promedio del n'umero de fotones
aumenta en los estados de Glauber. Entonces, para estados coherentes con
muchos fotones presentes, el campo el'ectrico cu'antico se comporta como
el campo cl'asico. Los estados coherentes son estados con
indeterminaci'on m'inima, donde es v'alido el signo de la igualdad
en la relaci'on de incertidumbre de Heisenberg.
 
